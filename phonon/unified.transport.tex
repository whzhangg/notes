\documentclass{article}
\usepackage{amsmath}
\usepackage[margin=0.8in]{geometry}
\usepackage{verbatim}
\usepackage{graphicx}

\begin{document}

\title{Unified Theory of Thermal Transport}
\author{Wenhao}
\date{\today}
\maketitle
\section{Transport equation}
We consider that the system of phonons are governed by the equation:
\begin{equation}
    \frac{\partial \rho(t)}{\partial t} + \frac{i}{\hbar} \left[H_0, \rho(t)\right] = \left. \frac{\partial\rho(t) }{\partial t} \right|_{coll} \label{master}
\end{equation}
Define the creation and annhiliation operator $a_{qb}$ and $a^{\dagger}_{qb}$ with 
$b = (b,\alpha)$, which is related 
to phonon creation and annhiliation operator $a_{qv}$ and $a^{\dagger}_{qv}$ by:
\begin{align}
    a_{qb} = \sum_v e^b_{qv} a_{qv} \\
    a^{\dagger}_{qb} = \sum_v e^{*b}_{qv} a^{\dagger}_{qv}
\end{align}
$e^b_{qv}$ gives the transformation between the two set of operators. 
The Harmonic Hamiltonian written using $a_{qb}$ and $a^{\dagger}_{qb}$ is:
\begin{equation}
    H_0 = \sum_{q} \sum_{b,b'} \hbar \sqrt{D_q}_{bb'} \left( a^{\dagger}_{qb}a_{qb'} + \frac{1}{2} \delta_{bb'} \right)
\end{equation}
$\sqrt{D_q}$ is the square root of matrix $D_q$ with matrix 
elements $\Phi_{q,bb'} (m_b m_{b'})^{-\frac{1}{2}}$.
Taking $e^b_{qv}$ to be the $v^{th}$ orthonormal eigenvector of the 
Dynamic matrix $D_q e^b_{qv} = \omega^2_{qv}e^b_{qv}$. $e^b_{qv}$ then also
is the eigenvector of matrix $\sqrt{D_q}$ with eigenvalue $\omega_{qv}$. We can then recovery
the harmonic Hamiltonian in its usual form:
\begin{equation}
    H_0 = \sum_{q,v} \hbar \omega_{qv} \left( a^{\dagger}_{qv}a_{qv} + \frac{1}{2} \right)
\end{equation}

The one body density matrix $\rho_1(q,q',t)$ is defined as:
\begin{equation}
    \rho_1(q,q',t)_{b,b'} = \text{Tr}[\rho(t)a^{\dagger}_{q'b'}a_{qb}]
\end{equation}
% here, the reversed $a^{\dagger}_{q'b'}a_{qb}$ is not very natural but 
% with this we can write the equation in matrix form 
We insert $H_0$ into Eq.\ref{master} and multiply on both side $a^{\dagger}_{q'b'}a_{qb}$ and take the trace:
\begin{align}
    \text{Tr}\left[ \frac{\partial \rho(t) a^{\dagger}_{q'b'}a_{qb} }{\partial t}  \right] &= \frac{\partial \rho_1(q,q',t)_{b,b'} }{\partial t} \\
    \text{Tr}\left[ \left( \frac{\partial\rho(t) a^{\dagger}_{q'b'}a_{qb} }{\partial t} \right) _{coll} \right] 
        &= \left. \frac{\partial \rho_1(q,q',t)_{b,b'} }{\partial t} \right|_{coll}
\end{align} 
For the term $\text{Tr}\left[\frac{i}{\hbar} \left[H_0, \rho(t)\right]a^{\dagger}_{q'b'}a_{qb}\right]$, we can derive:
\begin{align}
    i &\sum_{q_1} \sum_{b_1b_2} \sqrt{D_{q_1}}_{b_1b_2} \text{Tr}
    \left[ \rho (a^{\dagger}_{q'b'}a_{qb}a^{\dagger}_{q_1b_1}a_{q_1b_2} - a^{\dagger}_{q_1b_1}a_{q_1b_2}a^{\dagger}_{q'b'}a_{qb} ) \right] \notag \\
    &= i\sum_{q_1} \sum_{b_1b_2} \sqrt{D_{q_1}}_{b_1b_2} \text{Tr}
    \left[ \rho ( \delta_{q,q_1}\delta_{b,b_1} a^{\dagger}_{q'b'}a_{q_1b_2} - \delta_{q',q_1}\delta_{b',b_2} a^{\dagger}_{q_1b_1}a_{qb} ) \right] \notag \\
    &= i \left( \sum_{q_1} \sum_{b_1b_2} \sqrt{D_{q_1}}_{b_1b_2} \text{Tr} \right.
       [ \rho \delta_{q,q_1}\delta_{b,b_1} a^{\dagger}_{q'b'}a_{q_1b_2} ] -
    \sum_{q_1} \sum_{b_1b_2} \sqrt{D_{q_1}}_{b_1b_2} \text{Tr} 
    \left. [ \rho \delta_{q',q_1}\delta_{b',b_2} a^{\dagger}_{q_1b_1}a_{qb} ] \right) \notag \\
    &= i \left( \sum_{b_2} \sqrt{D_{q}}_{bb_2} \text{Tr} \right.
       [ \rho a^{\dagger}_{q'b'}a_{qb_2} ] - \sum_{b_1} \sqrt{D_{q'}}_{b_1b'} \text{Tr} 
        \left. [ \rho a^{\dagger}_{q'b_1}a_{qb} ] \right) \notag \\
    &= i \left( \sum_{b_2} \sqrt{D_{q}}_{bb_2} \rho_1(q,q',t)_{b_2,b'} - \sum_{b_1} \sqrt{D_{q'}}_{b_1b'} \rho_1(q,q',t)_{b,b_1} \right) \notag \\
    &= i \left[ \sqrt{D_q}\cdot \rho_1(q,q',t) - \rho_1(q,q',t) \cdot \sqrt{D_{q'}} \right] _{bb'} 
\end{align}
So that we obtain the equation:
\begin{equation}
    \frac{\partial \rho_1(q,q',t)_{b,b'} }{\partial t} + i \left[ \sqrt{D_q}\cdot \rho_1(q,q',t) - \rho_1(q,q',t) \cdot \sqrt{D_{q'}} \right] _{bb'}
     = \left. \frac{\partial \rho_1(q,q',t)_{b,b'} }{\partial t} \right|_{coll}  \label{derived1}
\end{equation}
We perform the Weyl transformation to $\partial \rho_1(q,q',t)_{b,b'}$:
\begin{equation}
    N(R,q,t)_{b,b'} = \sum_{q''} \rho_1(q+q'',q-q'',t)_{b,b'} e^{2iq''R}
\end{equation}
to Eq.\ref{derived1}, we will have:
\begin{align}
    \frac{\partial \rho_1(q+q'',q-q'',t)_{b,b'} }{\partial t} + & + i \left[ \sqrt{D_{q+q''}}\cdot \rho_1(q+q'',q-q'',t) - \rho_1(q+q'',q-q'',t) \cdot \sqrt{D_{q-q''}} \right] _{bb'}
    \\ &= \left. \frac{\partial \rho_1(q+q'',q-q'',t)_{b,b'} }{\partial t} \right|_{coll}
\end{align}
Assume the one particle density $\rho_1(q+q'',q-q'',t)_{b,b'}$ is sharply peaked at $q$, $q''$ will be small,
we can then replace frequency $\sqrt{D_{q+q''}}$ and $\sqrt{D_{q-q''}}$ by:
\begin{align}
    \sqrt{D_{q+q''}} &= \sqrt{D_q} + \frac{\partial\sqrt{D_q}}{\partial q''} q''\\
    \sqrt{D_{q-q''}} &= \sqrt{D_q} - \frac{\partial\sqrt{D_q}}{\partial q''} q''
%    \omega_{q+q''v} &= \omega_{qv} + \frac{\partial \omega_{qv}}{\partial q''} q'' \\
%    \omega_{q-q''v'} &= \omega_{qv'} - \frac{\partial \omega_{qv'}}{\partial q''} q'' 
\end{align}
Multiply both side with $e^{2iq''R}$ and integrate, we have:
\begin{align}
    \frac{\partial N(R,q,t)_{bb'}}{\partial t} + &i \left[ \sqrt{D_q}\cdot N(R,q,t) - N(R,q,t) \cdot \sqrt{D_q} \right] _{bb'} \notag \\
    + &\frac{1}{2} \left[ \nabla_q\sqrt{D_q} \cdot \nabla_R N(R,q,t) + \nabla_R N(R,q,t) \cdot \nabla_q\sqrt{D_q}  \right]_{bb'} = \left. \frac{\partial N(R,q,t)_{bb'} }{\partial t} \right|_{coll}
\end{align}
which can be simplified a bit:
\begin{equation}
    \frac{\partial N(R,q,t)_{bb'}}{\partial t} + i \left[ \sqrt{D_q}, N(R,q,t) \right] _{bb'} 
    + \frac{1}{2} \left\{ \nabla_q\sqrt{D_q} , \nabla_R N(R,q,t) \right\}_{bb'} = \left. \frac{\partial N(R,q,t)_{bb'} }{\partial t} \right|_{coll}
\end{equation}
%\begin{align}
%    \frac{\partial N(R,q,t)_{vv'}}{\partial t} + i \left(\omega_{qv'} N(R,q,t)_{vv'} - \omega_{qv} N(R,q,t)_{vv'}\right) + 
%        \frac{1}{2} \left( \nabla_q \omega_{qv'} \nabla_R N(R,q,t)_{vv'} + \nabla_q \omega_{qv} \nabla_R N(R,q,t)_{vv'}  \right) 
%        = \left. \frac{\partial N(R,q,t)_{vv'} }{\partial t} \right|_{coll} \label{derived2}
%\end{align}
%In the form of $ n_v \times n_v$ matrix, we can rewrite Eq.\ref{derived2} as:
%\begin{align}
%    \frac{\partial N(R,q,t)}{\partial t} + i \left[ N(R,q,t), \omega_{q} \right] + 
%        \frac{1}{2} \left\{ \nabla_R N(R,q,t), \nabla_q \omega_{q} \right\} 
%        = \left. \frac{\partial N(R,q,t) }{\partial t} \right|_{coll} \label{derived3}
%\end{align}
% the difference in sequency in communicator come from the different definition of one partial density matrix $\rho_1(q,q',t)$
Finally, we apply the transformation from $(qb)$ to phonon coordinate $(qv)$, obtaining:
\begin{align}
    \frac{\partial N(R,q,t)}{\partial t} + i \left[ \Omega_{q},N(R,q,t) \right] + 
        \frac{1}{2} \left\{ V_{q}, \nabla_R N(R,q,t) \right\} 
        = \left. \frac{\partial N(R,q,t) }{\partial t} \right|_{coll} \label{derived3}
\end{align}
where $\Omega_q$ is a diagonal matrix with diagonal element the frequency of phonon mode $\omega_{qv}$, and $V_{qbb'}$ is 
the velocity matrix containing off-diagonal elements:
\begin{equation}
    V_{q,vv'} = \sum_{bb'} e^{*b}_{qv} ( \nabla_q\sqrt{D_q})_{bb'} e^{b'}_{qv'}
\end{equation}

\section{Solving the equation}
The scattering term on the right of Eq.\ref{derived3} is given:
\begin{align}
    \left. \frac{\partial N(R,q,t)_{vv'} }{\partial t} \right|_{coll} 
    = - (1-\delta_{vv'}) \frac{\Gamma_{qv} + \Gamma_{qv'}}{2} N(R,q,t)_{vv'} 
      - \frac{\delta_{vv'}}{VN} \sum_{q''v''} A_{qv}^{q''v''}  (N(R,q'',t)_{v''v''} - \bar{N}_{q''v''} )
\end{align}
We aim to solve the Eq.\ref{derived3} under a temperature field $T_l(R)$, $l$ indicate local temperature 
as opposed to the equilibrium temperature $T$. In an steady state, $N(R,q,t)$ will be time independent, we 
linearize $N(R,q)$ as:
\begin{equation}
    N(R,q)_{vv'} = \delta_{vv'} \left[ \bar{N}(qv) + \partial \bar{N}(qv)/\partial T (T_l(R)-T) \right] + n^{(1)}_{q,vv'} \cdot \nabla T \label{linear}
\end{equation}
the first term of the right hand side depend only on equilibrium temperature, the second term accounts for the correction 
due to the local temperature, and the third term is the linear response (vector) correspond to a temperature grident.
Putting Eq.\ref{linear} into Eq.\ref{derived3}


\pagebreak
\section*{Appendix A. Wigner function}
%The expectation value of an operator $A$ of a quantum 
%state $|\psi\rangle $ is given by:
%\begin{equation}
%    \langle A \rangle = \int \psi^*(x) A \psi(x) dx
%\end{equation}
%with $\psi(x) = \langle x | \psi \rangle$. 

Define the transformation, called \emph{Weyl transformation} from an operator $A$ to a function $A(x,p)$:
\begin{align}
    \tilde{A}(x,p)  &= \int e^{-ipy/\hbar} \langle x + \frac{y}{2} | A | x - \frac{y}{2} \rangle dy \\
                    &= \int e^{ ixu/\hbar} \langle p + \frac{u}{2} | A | p - \frac{u}{2} \rangle du
\end{align}
where $\langle x | A | x' \rangle$ and $\langle p | A | p' \rangle$ denotes the matrix element of $A$ 
in position or momentum base, and both integral give the same expression $\tilde{A}(x,p)$. 
Suppose the operator $A$ is only a function of $x$, than the Weyl transformationwill give:
\begin{align}
    \tilde{A} &= \int e^{-ipy/\hbar} \langle x + \frac{y}{2} | A | x - \frac{y}{2} \rangle dy \\
              &= \int e^{-ipy/\hbar} \langle x + \frac{y}{2} | A | x - \frac{y}{2} \rangle \delta_{y=0} dy \\
              &= \langle x | A | x \rangle = A(x)
\end{align}
The same will be true if an operator is purely a function of momentum $p$. However, this is not true 
if an operator is a function of $x,p$ at the same time.
It can be shown that:
\begin{equation}
    \text{Tr}[AB] = \frac{1}{\hbar} \int \int \tilde{A}(x,p) \tilde{B}(x,p) dx dp
\end{equation}
define the density operator $\rho$ so that $\text{Tr}[\rho A] = \langle A \rangle$, we thus have:
\begin{equation}
    \langle A \rangle = \frac{1}{\hbar} \int \int \tilde{\rho}(x,p) \tilde{A}(x,p) dx dp
\end{equation}
It is therefore convenient to define a function:
\begin{align}
    W(x,p)  &= \frac{1}{\hbar} \int e^{-ipy/\hbar} \langle x + \frac{y}{2} | \rho | x - \frac{y}{2} \rangle dy \\
            &= \frac{1}{\hbar} \int e^{ ixu/\hbar} \langle p + \frac{u}{2} | \rho | p - \frac{u}{2} \rangle du
\end{align}
This is called \emph{Wigner function}. Now, we can find expectation value of an operator by 
integrating over phase space $x,p$, similar to classical statistic mechanics:
\begin{align}
    \langle A \rangle = \int \int W(x,p) \tilde{A}(x,p) dx dp
\end{align}
Intergrating over one phase space coordinates gives the probability distribution of another:
\begin{equation}
    \langle A \rangle (x) = \int W(x,p) \tilde{A}(x,p) dp
\end{equation}
Wigner function is real and normalized:
\begin{equation}
    \int\int W(x,p) dx dp = 1
\end{equation}
But it is not always positive, therefore, it cannot be interpreted as a classical probability density.

\end{document}