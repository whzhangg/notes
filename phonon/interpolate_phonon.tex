\documentclass{article}
\usepackage{amsmath}
\usepackage[margin=0.8in]{geometry}
\usepackage{verbatim}
\usepackage{graphicx}
\usepackage{hyperref} % \url \href

\newcommand{\pfrac}[2]{\frac{\partial #1}{\partial #2}}
% \renewcommand{\H}{\mathcal{H}}

\begin{document}

\title{Interpolation of phonons}
\author{Wenhao}
\date{\today}
\maketitle

\section{Transformation of phonon green's function}
The phonon green's function in terms of 
phonon creation and annihilation operator can be written as
\begin{align}
    G(qvv';t) = \langle i | \mathcal{T}[A_{qv}(t)A_{qv'}^{\dagger}(0)]| i \rangle \label{green_A}
\end{align}
where $A_{qv} = a_{qv} + a^{\dagger}_{-qv}$. The atomic displacement operator 
is connected with the phonon operator by transformation
\begin{align}
    \eta_{lb} &= \sum_{qv} \left(\frac{\hbar}{2N\omega_{qv}m_b}\right)^{\frac{1}{2}} 
        e_{qv}^b e^{iql} A_{qv} \label{displacement} \\
    A_{qv} &= \sum_{lb} \left( \frac{2\omega_{qv} m_b}{N\hbar} \right)^{\frac{1}{2}}
        e_{qv}^{b*} e^{-iql} \eta_{lb}
\end{align}
The phonon green's function in terms of atomic displacement operator is 
written as:
\begin{equation}
    D_{ll'bb'}(t) =  \langle i | \mathcal{T}[\eta_{lb}(t)\eta_{l'b'}^{\dagger}(0)]| i \rangle
\end{equation}
substituting Eq.\ref{displacement}, we obtain the relationship:
\begin{align}
    D_{ll'bb'}(t) &= \sum_{qq'vv'} \langle i | \mathcal{T}[
        \left(\frac{\hbar}{2N\omega_{qv}m_b}\right)^{\frac{1}{2}} 
        e_{qv}^b e^{iql} A_{qv}(t)
         \left(\frac{\hbar}{2N\omega_{q'v'}m_b'}\right)^{\frac{1}{2}} 
        e_{q'v'}^{b'*} e^{-iq'l'} A_{q'v'}^{\dagger}(0)]| i \rangle \\
        &= \frac{\hbar}{2N}\sum_{qq'vv'} (\omega_{qv}m_b\omega_{q'v'}m_b')^{-\frac{1}{2}} 
        \langle i | \mathcal{T}[
            e_{qv}^b e^{iql} A_{qv}(t)
            e_{q'v'}^{b'*} e^{-iq'l} A_{q'v'}^{\dagger}(0)]| i \rangle \\
        & = \frac{\hbar}{2}\sum_{qvv'} (\omega_{qv}m_b\omega_{q'v'}m_b')^{-\frac{1}{2}} 
            e^{iq(l-l')} e_{qv}^b e_{q'v'}^{b'*}
            \langle i | \mathcal{T}[A_{qv}(t) A_{q'v'}^{\dagger}(0)]| i \rangle \\
\end{align}
where we used the result $\sum_{qq'}e^{i(ql-q'l')} = N\delta_{qq'} e^{iq(l-l')}$. The 
reverse of the above relationship is:
\begin{align}
    G_{qvv'}(t) = \sum_{ll'bb'} \frac{2}{N\hbar} (\omega_{qv}\omega_{qv'}m_bm_b')^{\frac{1}{2}}
                e^{b*}_{qv}e_{qv'}^{b'} e^{iq(l-l')} D_{ll'bb'}(t)
\end{align}

\section{Phonon green's function} 
We write the phonon Green's function as:
\begin{equation}
    G(qvv';t) = -i \langle \mathcal{T}[A_{qv}(t)A_{qv'}^{\dagger}(0)] \rangle \label{green_phonon}
\end{equation}
The time dependence of the operator is:
\begin{equation}
    A(t) = e^{i\frac{t}{\hbar}H} A e^{-i\frac{t}{\hbar}H}
\end{equation}
$\mathcal{T}$ is the Wick's time ordering operator and average is taken at a finite temperature:
\begin{align}
    iG(qvv';t) = 
    \begin{cases} 
        \frac{1}{Z} \text{Tr} \sum_n \langle n | e^{-\beta H} e^{i\frac{t}{\hbar}H} (a_{qv} + a^{\dagger}_{-qv}) e^{-i\frac{t}{\hbar}H} (a^{\dagger}_{qv'} + a_{-qv'}) | n \rangle & t > 0 \\
        \frac{1}{Z} \text{Tr} \sum_n \langle n | e^{-\beta H} (a^{\dagger}_{qv'} + a_{-qv'}) e^{i\frac{t}{\hbar}H} (a_{qv} + a^{\dagger}_{-qv}) e^{-i\frac{t}{\hbar}H} | n \rangle & t < 0 \\
    \end{cases}
\end{align}
We can obtain the result:
\begin{align}
    G(qvv';t > 0) = -i \delta_{vv'}  ( e^{i\omega_{qv}t}\langle n_{qv} \rangle + e^{-i\omega_{qv}t}\langle n_{qv}+1 \rangle ) \\
    G(qvv';t < 0) = -i \delta_{vv'}  ( e^{-i\omega_{qv}t}\langle n_{qv} \rangle + e^{i\omega_{qv}t}\langle n_{qv}+1 \rangle )
\end{align}

Define the fourier transformation into frequency space:
\begin{align}
    G(qvv';\omega) &= \int_{-\infty}^{\infty} G(qvv';t) e^{-i\omega t} dt \\
            & = \int_{-\infty}^{0} G(qvv';t < 0) e^{-i\omega t} dt + \int_{0}^{\infty} G(qvv';t > 0) e^{-i\omega t} dt \\
\end{align}
and we have:
\begin{align}
    \int_{0}^{\infty} G(qv;t > 0) e^{-i\omega t} dt 
    &= -i \int_{0}^{\infty} \left( e^{i(\omega_{qv}-\omega)t}\langle n_{qv} \rangle + e^{-i(\omega_{qv}+\omega)t}\langle n_{qv}+1 \rangle \right) dt \notag \\
    &= -i \lim_{\eta \to 0^{+}} \left( \frac{\langle n_{qv} \rangle}{i(\omega - \omega_{qv}) + \eta} + \frac{\langle n_{qv}+1 \rangle}{i(\omega + \omega_{qv}) + \eta} \right) \notag \\
    &= - \lim_{\eta \to 0^{+}} \left( \frac{\langle n_{qv} \rangle}{\omega - \omega_{qv} - i\eta} + \frac{\langle n_{qv}+1 \rangle}{\omega + \omega_{qv} - i\eta} \right)
\end{align}
Similarly, for the second term in the integral, we have:
\begin{align}
    \int_{-\infty}^{0} G(qvv';t < 0) e^{-i\omega t} dt
    & = -i \int_{-\infty}^{0} \left( e^{-i(\omega_{qv}+\omega)t}\langle n_{qv} \rangle + e^{i(\omega_{qv}-\omega)t}\langle n_{qv}+1 \rangle \right) dt \notag \\
    & = i \int_{0}^{\infty} \left( e^{i(\omega_{qv}+\omega)t}\langle n_{qv} \rangle + e^{-i(\omega_{qv}-\omega)t}\langle n_{qv}+1 \rangle \right) dt \notag \\
    & = i \lim_{\eta \to 0^{+}} \left( \frac{\langle n_{qv} \rangle}{-i(\omega + \omega_{qv}) + \eta} + \frac{\langle n_{qv}+1 \rangle}{i(\omega_{qv} - \omega) + \eta} \right) \notag \\
    & = \lim_{\eta \to 0^{+}} \left( - \frac{\langle n_{qv} \rangle}{\omega + \omega_{qv} -i\eta} + \frac{\langle n_{qv}+1 \rangle}{\omega_{qv} - \omega -i\eta} \right)
\end{align}
So that the complete green's function is then:
\begin{align}
    G(qvv';\omega) = \delta_{vv'} \lim_{\eta \to 0^{+}} 
    \left( - \frac{\langle n_{qv} \rangle}{\omega + \omega_{qv} -i\eta} + \frac{\langle n_{qv}+1 \rangle}{\omega_{qv} - \omega -i\eta} - \frac{\langle n_{qv} \rangle}{\omega - \omega_{qv} - i\eta} - \frac{\langle n_{qv}+1 \rangle}{\omega + \omega_{qv} - i\eta} \right)
\end{align}

% TODO this form looks correct, notice that in RMP, the phonon green function is defined at a certain temperature
% we should reorganize this part and clearly verify the equations.

\newpage
\section*{Appendix}
This section summarized the formulation of paper \emph{Review of Modern Physics, 89 Electron-phonon from first principle}.
In that work, the displacement-displacement correlation function of atomic motion is given 
by the function:
\begin{equation}
    D_{k\alpha p, k'\alpha'p'}(t) = -\frac{i}{\hbar} \langle \mathcal{T} [\tau_{k\alpha p}(t) \tau_{k'\alpha'p'}(0)] \rangle 
\end{equation}


\end{document}