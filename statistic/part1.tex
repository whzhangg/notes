\documentclass{article}
\usepackage{amsmath}
\usepackage[margin=1in]{geometry}
\usepackage{verbatim}
\usepackage{graphicx}

\begin{document}

\title{Statistic Physics}
\author{Wenhao}
\date{\today}
\maketitle

\section{The fundamental postulation and Liouville's theorem}

\subsection{The Fundamental postulate}
An isolated system in equilibrium is equally likely to be found in any
of the microstates accessible to it.

\begin{itemize}
    \item \textbf{System} an part of universe that is only weakly coupled to the rest of the universe 
    so that its dynamic is dominated by internal interactions.
    \item \textbf{Equilibrium} the measurement of quantities are time independent.
    \item \textbf{Microstate} a complete microscope specification of coordinates of every particles (position and velocity).
    \item \textbf{Ensemble} a collection of the system that are macroscopically the same but microscopically different. 
\end{itemize}

\subsection{Motivation for the fundamental postulation}
For $N$ particles we have in total $6N$ coordinates $(q_1, q_2, \cdots , q_{3N}, p_1, p_2, \cdots , p_{3N})$ which completely
define a microscope state. We define the "phase density" as:
\begin{align}
    &\rho(q_1, q_2, \cdots , q_{3N}, p_1, p_2, \cdots , p_{3N}, t) \\
     &\ \ \   \to \text{Probability of finding a system near}\ (q_1, q_2, \cdots , q_{3N}, p_1, p_2, \cdots , p_{3N}) \text{at time} t
\end{align}
If property of this system is given by a function $f(q,p)$, then the ensemble average of $f$ at time $t$ will be 
given as:
\begin{align}
    \langle f(t) \rangle = \frac{\int\int\cdots\int f(q,p)\rho(q,p,t)dq^{3N}dp^{3N}}{\int\int\cdots\int \rho(q,p,t)dq^{3N}dp^{3N}}
\end{align}
With the above definition, $\rho(q,p,t)dq^{3N}dp^{3N}$ gives the number of states (points) that are included in the phase space 
volume $dq^{3N}dp^{3N}$ near $(q,p)$.
\subsection{Liouville's theorem}
Liouville's theorem states that the evolution of $\rho$ is given by:
\begin{equation}
    \frac{d\rho}{dt} = 0
\end{equation}
which is to say that if we follow the trajectory of a state $(q,p)$ as it 
evolve over time, its phase space density will not change (total derivative): $\rho(q(0),p(0),t=0) = \rho(q(t'),p(t'), t = t')$. 

\textbf{Proof 1} In this proof, we consider the phase space points inclosed by a volume at $t = 0$ at $(q_1,p_1)$, at a later time $\delta t$, 
we locate those phase space points agian and we show that the volume of phase space that enclose these points are the same. This will
thus mean the phase (point) density do not change following the trajectory.

let's consider an area in a two dimensional phase space that is a rectangle specified by its 2 diagonal points $(q_1,p_1),(q_2,p_2)$ at some 
initial time $t$, then at time $t+\delta t$, the points changed to $(q_1 + \dot{q_1}\delta t, p_1 + \dot{p_1}\delta t)$ and 
$(q_2 + \dot{q_2}\delta t, p_2 + \dot{p_2}\delta t)$. The volume difference, to first order in $\delta t$ is:
\begin{align}
    \Delta V &= (q_2 + \dot{q_2}\delta t - q_1 - \dot{q_1}\delta t)(p_2 + \dot{p_2}\delta t - p_1 - \dot{p_1}\delta t) - (q_2 - q_1)(p_2 - p_1) \notag \\
             &= (\dot{q_2} - \dot{q_1}) (p_2 - p_1) + (\dot{p_2} - \dot{p_1}) (q_2 - q_1) \notag \\
             &= \frac{1}{V} \left( \frac{\dot{q_2} - \dot{q_1}}{q_2-q_1} + \frac{\dot{p_2} - \dot{p_1}}{p_2-p_1} \right) \notag \\
             &= \frac{1}{V} \left( \frac{\partial \dot{q}}{\partial q} + \frac{\partial \dot{p}}{\partial p} \right) \delta t
\end{align} 
If a system envolve under Hamiltonian dynamics:
\begin{align}
    \dot{q_i} &= \frac{\partial H}{\partial p_i} ; \ \ \dot{p_i} = -\frac{\partial H}{\partial q_i}
\end{align}
then 
\begin{equation}
    \frac{\partial \dot{q}}{\partial q} + \frac{\partial \dot{p}}{\partial p} 
    = \frac{\partial^2 H}{\partial p \partial q} - \frac{\partial^2 H}{\partial q\partial p} = 0
\end{equation}
which shows that the enclosing volume of those phase space points do not change as the system evolve, and therefore, the phase space density
in this volume do not change over time, giving the result:
\begin{equation}
    \frac{d\rho}{dt} = \frac{\partial \rho}{\partial t} + 
        \sum_i \left( \frac{\partial \rho}{\partial q_i}\frac{\partial q_i}{\partial t} + \frac{\partial \rho}{\partial p_i}\frac{\partial p_i}{\partial t} \right)
        = 0
\end{equation}
Where the first equality is given merely by the definition of total derivative.

\textbf{Proof 2} In this proof, we compute the partial derivatives first and show that they result in the result of Liouville's theorem.

We first compute $\partial \rho / \partial t$. Consider the flow the phase space points in and out of a cubic volume element 
in the phase space arount $(q,p)$, The net flow of phase space points is given by:
\begin{align}
    \frac{\partial N}{\partial t} 
    &= -\sum_i \left( \frac{\partial (\rho \dot{q_i})}{\partial q_i} + \frac{\partial (\rho \dot{p_i})}{\partial p_i} \right) dq \cdots dp \notag \\
    \frac{\partial \rho}{\partial t} 
    &= -\sum_i \left( \frac{\partial (\rho \dot{q_i})}{\partial q_i} + \frac{\partial (\rho \dot{p_i})}{\partial p_i} \right)
\end{align}
the total derivatve is then:
\begin{align}
    \frac{d\rho}{dt} &= \frac{\partial \rho}{\partial t}
    + \sum_i \left( \frac{\partial \rho}{\partial q_i}\frac{\partial q_i}{\partial t} + \frac{\partial \rho}{\partial p_i}\frac{\partial p_i}{\partial t} \right) \notag \\
    &= -\sum_i \left( \frac{\partial (\rho \dot{q_i})}{\partial q_i} + \frac{\partial (\rho \dot{p_i})}{\partial p_i} \right) 
    + \sum_i \left( \frac{\partial \rho}{\partial q_i}\frac{\partial q_i}{\partial t} + \frac{\partial \rho}{\partial p_i}\frac{\partial p_i}{\partial t} \right) \notag \\
    &= - \sum_i \left( \rho\frac{\partial \dot{q_i}}{\partial q_i} + \rho\frac{\partial \dot{p_i}}{\partial p_i}  \right) = 0
\end{align}
thus proving the Liouville's theorem.

\subsection{Ergodicity}
In a complex system in equilibrium, over a sufficiently long time, all part of the phase space will be visited by a trajectory in the phase space. Since we have shown that the 
phase space density on the trajectory will be the same, and we consider the system in equilibrium, which implies $\partial \rho / \partial t = 0$, We can conclude that 
\begin{equation}
    \rho(q,p,t) = \rho_0
\end{equation}
That is, we have equal phase space density over all accessible phase space.

\section{Intensive and extensive properties}
\subsection{Entropy}
We define entropy $S$ for an isolated macroscopic system of $N$ particles in volume $V$ and energy $E$ to be:
\begin{equation}
    S(E,N,V,x) = k_b \ln\Omega(E,N,V,x)
\end{equation}
where $\Omega(E,N,V,x)$ is the number of accessible states at a given value of $E, N, V$, and $x$ is some constraints which 
influence the number of accessible states.
As a non-equilibrium isolated system allow to relax to equilibrium, entropy will increase monotonically and eventually maximize 
at equilibrium.
%imagine that the system will attempt to visit as many configurations as possible. 

\textbf{Temperature} 
consider an isolated system with two subsystem in weak contact but only energy is allowed to follw between the two subsystem.
The total number of accessible states (configurations) are given by the product of the number of configurations of the 
two subsystem. Entropy will be additive. Consider the energy of one of the subsystem $E_1$
as the constrain for the configurations.
\begin{align}
    \Omega(E,E_1) = \Omega_1(E_1) \Omega_2(E_2) \\
    S(E,E_1) = S_1(E_1) S_2(E_2)   
\end{align}
Relaxation process will increase entropy by changing $E_1$, at equilibrium, we have:
\begin{gather}
    \frac{\partial S}{\partial E_1} = 0 \ \Rightarrow \ 
    \left. \frac{\partial S_1(E_1)}{\partial E_1} \right|_{N_1,V_1} = \left. \frac{\partial S_2(E_2)}{\partial E_2}\right|_{N_2,V_2} = \frac{1}{T}
\end{gather}
where the final equality gives the definition of temperature, thus if two subsystem reaches equilibrium in terms of energy follow, their 
temperature will be equal.

\textbf{Chemical potential}
now, we fix only volume $V$ of each subsystem and allow both energy and particles to exchange, then:
\begin{equation}
    \left. \frac{\partial S_1(N_1)}{\partial N_1} \right|_{E_1,V_1} = \left. \frac{\partial S_2(N_2)}{\partial N_2}\right|_{E_2,V_2} = -\frac{\mu}{T}
\end{equation}
the last equality defines the chemical potential $\mu$. 
% minus sign the divide by temperature is because of the tradition

\textbf{Pressure}
finally, we allow the volume of the system to exchange, and we can similar define pressure:
\begin{equation}
    \left. \frac{\partial S_1(V_1)}{\partial V_1} \right|_{E_1,N_1} = \left. \frac{\partial S_2(V_2)}{\partial V_2}\right|_{E_2,N_2} = \frac{P}{T}
\end{equation}

Thus, we can see that if we set an initial system not in equilibrium, the two subsystem will
start to exchange energy, particles and volume until $T$, $P$ and $\mu$ become the same 
for the two subsystem. 

We can separate the macroscopic properties of a system into two different catagory:
\begin{itemize}
    \item \textbf{Extensive properties} that will increase proportional to the system size, such as $N,E,V$
    \item \textbf{Intensive properties} that will be same for any of the subsystem, such as  $P,\mu,T$
\end{itemize}

Writing out the definition of  $T$, $P$ and $\mu$
\begin{gather}
    \frac{1}{T} = \left(\frac{\partial S}{\partial E}\right)_{N,V};\  
    \frac{\mu}{T} = \left(\frac{\partial S}{\partial N}\right)_{E,V};\ 
    \frac{P}{T} = \left(\frac{\partial S}{\partial V}\right)_{E,N};\ 
\end{gather}
we can organize:
\begin{align}
    dS &= \frac{1}{T} dE - \frac{\mu}{T}dN + \frac{P}{T}dV \\
    dE &= TdS + \mu dN - P dV \label{partial_extensive}
\end{align}
For the Eq.\ref{partial_extensive}, we can integrate the subsystems parts by parts into 
the whole system, since $T, \mu$ and $P$ will be the same for all subsystems, the integration
gives:
\begin{equation}
    E = TS + \mu N - PV
\end{equation}

%\subsection{Carnot cycle}
\section{Ensembles}
We have already seem an ensemble containing possible configurations of an isolated system where energy
is known and all configurations are of equal probability. 
This ensemble is known as the \textbf{Microcanonical ensemble}. For a microcanonical ensemble, particle 
number, volume and energy are specified at the same time, thus it is also called $NVE$ ensemble.

Now consider a system that can exchange energy through a contact with a temperature bath and eventually 
come to an equilibrium. This ensemble is called \textbf{Canonical ensemble}. For canonical ensemble, 
it's dynamic is still governed by its internal interaction but now its energy may vary. Since now 
the energy of this system can change, it is more appropriate to describe it in terms of 
temperature $T$, from which we can find its energy. 

Now we want to find the probability distribution of its microstate (probability of finding the system 
to be in a specific microstate),

\end{document}