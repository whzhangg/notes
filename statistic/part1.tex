\documentclass{article}
\usepackage{amsmath}
\usepackage[margin=1in]{geometry}
\usepackage{verbatim}
\usepackage{graphicx}

\begin{document}

\title{Statistic Physics}
\author{Wenhao}
\date{\today}

\section{Part 1}
\subsection{The fundamental postulation and terms}
\textbf{The Fundamental postulate} An isolated system in equilibrium is equally likely to be found in any
of the microstates accessible to it.

\begin{itemize}
    \item \textbf{System} an part of universe that is only weakly coupled to the rest of the universe 
    so that its dynamic is dominated by internal interactions.
    \item \textbf{Equilibrium} the measurement of quantities are time independent.
    \item \textbf{Microstate} a complete microscope specification of coordinates of every particles (position and velocity).
    \item \textbf{Ensemble} a collection of the system that are macroscopically the same but microscopically different. 
\end{itemize}

\subsection{Motivation for the fundamental postulation}
For $N$ particles we have in total $6N$ coordinates $(q_1, q_2, \cdots , q_{3N}, p_1, p_2, \cdots , p_{3N})$ which completely
define a microscope state.

\end{document}