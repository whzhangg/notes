\documentclass{article}
\usepackage{amsmath}
\usepackage[margin=0.8in]{geometry}
\usepackage{verbatim}
\usepackage{graphicx}
\usepackage{hyperref} % \url \href

\newcommand{\pfrac}[2]{\frac{\partial #1}{\partial #2}}
\newcommand{\rms}{\text{rms}}
% \renewcommand{\H}{\mathcal{H}}

\begin{document}

\title{Kinetic theory}
\author{Wenhao}
\date{\today}
\maketitle

\section{Kinetic theory of ideal gas}
We describe here the problem of motion of molecules in a gas, 
ignoring its rotational and vibration degrees of freedom, with 
energy: 
\begin{equation}
    E_k = \frac{1}{2}mv^2 = \frac{1}{2}m(v_x^2 + v_y^2 + v_z^2)
\end{equation}
we consider some assumption: 1) the molecules do not interact 
with each other most of the time (only through collision) and 2) the collision is 
not frequent. Through the collision, molecules exchange energy with each other,
but every think remain in equilibrium. We also assume the motion before and 
after the collision is independent, thus their kinetic energies are 
independent variables. With the above assumption, we can consider 
each molecules as an independent system connected to the heat reservior at 
temperature $T$, where heat reservior is "all other molecules in the gas"
\footnote{Blundell, p48}.

\subsection*{Velocity distribution}
The velocity distribution $g(\mathbf{v})$ is defined to the fraction of molecules with 
velocity $\mathbf{v}$. 
\begin{align}
    g(\mathbf{v}) &\propto e^{-mv^2/2k_BT} dv_xdv_ydv_z \\
    &\propto e^{-m(v_x^2 + v_y^2 + v_z^2)/2k_BT} dv_xdv_ydv_z \\
    &= g(v_x)g(v_y)g(v_z) \label{g_product}
\end{align} 
we find the normalization factor for $g(v_x)$:
\begin{gather}
    \int_{-\infty}^{\infty} e^{-mv_x^2/2k_BT} dv_x = \sqrt{\frac{2\pi k_BT}{m}} \\
    g(v_x) = \sqrt{\frac{m}{2\pi k_BT}} e^{-mv_x^2/2k_BT}
\end{gather}
with $g(v_i)$ normalized to 1, $g(v)$ from Eq.\ref{g_product} is then also normalized to 1.
we can calculate the following values:
\begin{align}
    \langle v_x \rangle &= \int_{-\infty}^{\infty} v_x g(v_x) dv_x = 0 \notag \\
    \langle v_x^2 \rangle &= \int_{-\infty}^{\infty} v_x^2 g(v_x) dv_x = \frac{k_BT}{m} \notag 
\end{align}

To find the distribution with respect to speed $v$ instead of velocity $\mathbf{v}$, we 
consider the volume in velocity phase space corresponding to velocity $v$:
\begin{equation}
    dV = 4\pi v^2 dv
\end{equation}
which correspond to a thin spherical shell at radius $v$. The speed distribution function is then:
\begin{equation}
    f(v) dv \propto v^2 dv e^{-mv^2/2k_BT}
\end{equation}
normalize $\int_0^{\infty}f(v)dv = 1$, we find the distribution function:
\begin{equation}
    f(v) = \frac{4}{\sqrt{\pi}}\left(\frac{m}{2k_BT}\right)^{3/2} v^2 e^{-mv^2/2k_BT}
\end{equation}
which is known as the \textbf{Maxwell-Boltzmann distribution}. 
We can find the following expectation values:
\begin{align}
    \langle v \rangle &= \int_{0}^{\infty} v f(v) dv = \sqrt{\frac{8k_BT}{\pi m}} \notag \\
    \langle v^2 \rangle &= \int_{0}^{\infty} v^2 g(v) dv_x = \frac{3k_BT}{m} \notag  \\
    v_{\rms} &= \sqrt{\langle v^2 \rangle} = \sqrt{\frac{3k_BT}{m}}
\end{align}

The mean kinetic energy of a gas molecule is then:
\begin{equation}
    \langle E_k \rangle = \frac{1}{2}m \langle v^2 \rangle = \frac{3}{2}k_BT
\end{equation}

\subsection*{Pressure}
Since all molecules are equally likely to travel in any direction, the fraction of these 
whose velocity lie in solid angle $d\Omega$ is therefore $d\Omega/4\pi$. We consider the 
molecules traveling at angle between $\theta$ and $\theta + d\theta$ with some specific direction ($z$),
the solid angle is given by:
\begin{equation}
    d\Omega = 2\pi \sin\theta d\theta
\end{equation}
So the fraction of molecules having speed between $v$ and $v_dv$ and angel between $\theta$ 
and $\theta + d\theta$ with direction $z$ is then:
\begin{equation}
    f(v) dv \frac{2\pi \sin\theta d\theta}{4\pi} = \frac{1}{2} f(v) dv \sin\theta d\theta 
\end{equation}

In a time interval $dt$, 
a molecules with speed $v$ travelling at an angle $\theta$ to the normal of the wall
will hit the container wall of area $A$ if it is $vdt\cos \theta$ away from the wall. 
the number of such molecules hitting the wall is thus 
\begin{equation}
    A vdt\cos \theta n \cdot \frac{1}{2} f(v) dv \sin\theta d\theta 
\end{equation}
where the first part gives the total number of molecules in the volume $A vdt\cos \theta$ and 
the second part is the fraction of the molecule in the velocity and angle range.

Pressure of a gas on its container can be calculated by $dP = F dt$ where $dP$ is the momentum 
change of the particle reflecting from the wall: $dP = 2 mv \cos\theta$ (prependicular to wall). 
Pressure is defined to be $p = F/A$, so that 
We integrate over all possible velocity and angle $\theta$ to obtain the expression of $p$:
\begin{align}
    p &= \int_0^{\infty} dv \int_0^{\pi/2} d\theta  ( 2 mv \cos\theta ) 
    \left( v \cos \theta n \frac{1}{2} f(v) \sin\theta \right) \\
    &= \frac{1}{3} n m \langle v^2 \rangle
\end{align}
using the relationship $n = N/V$ and $\langle v^2 \rangle = 3k_BT / m$, we 
can obtain the ideal gas law:
\begin{equation}
    pV = N k_B T
\end{equation}

\subsection*{Partial pressure}
If we have a mixture of gases in equilibrium, the total pressure will be the 
sum of pressures of each component:
\begin{gather}
    n = \sum_i n_i \\
    p = \left( \sum_i n_i \right) k_B T = \sum_i p_i
\end{gather}
where $n_i = N_i / V$

\subsection*{Effusion}
Effusion is the process in which gas escape from a small hole\footnote{Blundell, p64}. we first define 
the flux $\Phi$ as the number of molecules going through unit area per unit time,
therefore, $\Phi$ has the unit of $m^{-2}s^{-1}$. we use the above result of particles 
hitting the wall to find flux:
\begin{align}
    \Phi &= \int_0^{\infty} dv \int_0^{\pi/2} d\theta 
            \left( v \cos \theta n \frac{1}{2} f(v) \sin\theta \right) \label{flux} \\
        &= \frac{1}{4} n \langle v \rangle
\end{align}
with the result:
\begin{align}
    n &= p / k_B T \notag \\
    \langle v \rangle &= \sqrt{\frac{8k_BT}{\pi m}} \notag \\
    \Phi &= \frac{p}{\sqrt{2\pi m k_BT}}
\end{align}

Now, consider a we have a small hole (small enough that the gas escaping does not 
change the equilibrium distribution of gas near the hole), the number of molecules
escaping through the hole is given as:
\begin{equation}
    \Phi A = \frac{pA}{\sqrt{2\pi m k_BT}}
\end{equation}
which is linear to pressure, inversely proportion to $T$ and $m$. It should be 
noted that the speed distribution of the gas through the hole is no longer the 
Maxwell-Boltzmann distribution, since the amount of gas that effuse through the 
hole depend on velocity $v$ (in Eq.\ref{flux}), therefore the speed distribution
of those that go through the hole as an extra factor $v$, compare to the speed 
distribution of gas molecules inside the box.

\subsection*{Average lifetime}
We start by considering a molecule moving at speed $v$ with other molecules in
the gas stationary. Within a time $dt$, the molecule sweep through an area $\sigma v dt$,
where $\sigma$ is the collision cross-section of the molecule. We denote $P(t)$ as 
the probability of this molecule travel without any collision up to time $t$. We have:
\begin{align}
    P(t + dt) &= P(t) + \frac{dP}{dt}dt \\
    P(t + dt) &= P(t) (1 - n\sigma v dt) 
\end{align}
with $n$ the number density of gas and therefore $(1 - n\sigma v dt)$ gives the 
probability of having no other gas molecule in the volume the moving molecule is going 
to sweep over from time $t$ to $t + dt$. 
Withe the above two equation, we obtain the result:
\begin{equation}
    P(t) = e^{-n\sigma vt}
\end{equation}
We have the boundary condition $P(0) = 1$ and $P(\infty) = 0$

The probability that the molecule collide in time interval from $t$ to $t+dt$ is given by
the probability that molecule does not collide at time $t$ but do collide at time $t + dt$:
$P(t) - P(t + dt)$, we have:
\begin{equation}
    - dP(t) = n\sigma v e^{-n\sigma vt} dt
\end{equation}
(since P(t) decay with time, $dP(t) < 0$). We can calculate the mean scattering time:
\begin{align}
    \tau &= \int_{0}^{\infty} t \sigma v e^{-n\sigma vt} dt \notag \\
         &= \frac{1}{n\sigma v} \int_{0}^{\infty} x e^{-x} dx \notag \\
         &= \frac{1}{n\sigma v}
\end{align}
where $x = n\sigma v t$ represent a change of variable.

In the case of hard sphere approximation in a gas consist of molecules with
equal radius, the collision cross-section is given as:
\begin{equation}
    \sigma = \pi (r_1 + r_2) ^ 2 = 4 \pi a^2
\end{equation}

\subsection*{Mean free path of gas}
With the previous result, we have, for the mean free path
\begin{equation}
    \lambda = \langle v \rangle \tau = \langle v \rangle \frac{1}{n\sigma \langle v_r\rangle }
\end{equation}
$\langle v \rangle$ is the average speed of gas molecules,
where for the lifetime, instead of using $v$, we recognize that we should use 
relative speed $v_r$ in a real system when all particles are in motion. It is not attempted to derive
the value of $\langle v_r\rangle$ here (refer to Blundell, p73). But the final result
is: 
\begin{equation}
    \lambda = \frac{1}{\sqrt{2}n\sigma}
\end{equation}

\pagebreak
\section*{Appendix A}


\end{document}
