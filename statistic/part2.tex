\documentclass{article}
\usepackage{amsmath}
\usepackage[margin=0.8in]{geometry}
\usepackage{verbatim}
\usepackage{graphicx}
\usepackage{hyperref} % \url \href

\newcommand{\pfrac}[2]{\frac{\partial #1}{\partial #2}}
\newcommand{\rms}{\text{rms}}
% \renewcommand{\H}{\mathcal{H}}

\begin{document}

\title{Kinetic theory}
\author{Wenhao}
\date{\today}
\maketitle

\section{Kinetic theory of ideal gas}
We describe here the problem of motion of molecules in a gas, 
ignoring its rotational and vibration degrees of freedom, with 
energy: 
\begin{equation}
    E_k = \frac{1}{2}mv^2 = \frac{1}{2}m(v_x^2 + v_y^2 + v_z^2)
\end{equation}
we consider some assumption: 1) the molecules do not interact 
with each other most of the time (only through collision) and 2) the collision is 
not frequent. Through the collision, molecules exchange energy with each other,
but every thing remain in equilibrium. We also assume the motion before and 
after the collision is independent, thus their kinetic energies are 
independent variables. With the above assumption, we can consider 
each molecules as an independent system connected to the heat reservior at 
temperature $T$, where heat reservior is "all other molecules in the gas"
\footnote{Blundell, p48} and energy is exchanged with collisions
\footnote{This point
is not stated very clear in the book, but collision is the only way a gas molecule
can interact with other molecules. The motion of the molecule after the collision is
independent of their motion before the collision, therefore, the motion after the 
collision is entirely determined by the heat bath.}.

\subsection*{Velocity distribution}
The velocity distribution $g(\mathbf{v})$ is defined to the fraction of molecules with 
velocity $\mathbf{v}$ 
($\mathbf{v}$ is the vector velocity while $v$ is the absulote value). 
\begin{align}
    g(\mathbf{v}) &\propto e^{-mv^2/2k_BT} dv_xdv_ydv_z \\
    &\propto e^{-m(v_x^2 + v_y^2 + v_z^2)/2k_BT} dv_xdv_ydv_z \\
    &= g(v_x)g(v_y)g(v_z) \label{g_product}
\end{align} 
we find the normalization factor for $g(v_x)$:
\begin{gather}
    \int_{-\infty}^{\infty} e^{-mv_x^2/2k_BT} dv_x = \sqrt{\frac{2\pi k_BT}{m}} \\
    g(v_x) = \sqrt{\frac{m}{2\pi k_BT}} e^{-mv_x^2/2k_BT}
\end{gather}
with $g(v_i)$ normalized to 1, $g(v)$ from Eq.\ref{g_product} is then also normalized to 1.
we can calculate the following values:
\begin{align}
    \langle v_x \rangle &= \int_{-\infty}^{\infty} v_x g(v_x) dv_x = 0 \notag \\
    \langle v_x^2 \rangle &= \int_{-\infty}^{\infty} v_x^2 g(v_x) dv_x = \frac{k_BT}{m} \notag 
\end{align}

To find the distribution with respect to speed $v$ instead of velocity $\mathbf{v}$, we 
consider the volume in velocity phase space corresponding to velocity $v$:
\begin{equation}
    dV = 4\pi v^2 dv
\end{equation}
which correspond to a thin spherical shell at radius $v$. The speed distribution function is then:
\begin{equation}
    f(v) dv \propto v^2 dv e^{-mv^2/2k_BT}
\end{equation}
normalize $\int_0^{\infty}f(v)dv = 1$, we find the distribution function:
\begin{equation}
    f(v) = \frac{4}{\sqrt{\pi}}\left(\frac{m}{2k_BT}\right)^{3/2} v^2 e^{-mv^2/2k_BT}
\end{equation}
which is known as the \textbf{Maxwell-Boltzmann distribution}. 
We can find the following expectation values:
\begin{align}
    \langle v \rangle &= \int_{0}^{\infty} v f(v) dv = \sqrt{\frac{8k_BT}{\pi m}} \notag \\
    \langle v^2 \rangle &= \int_{0}^{\infty} v^2 g(v) dv_x = \frac{3k_BT}{m} \notag  \\
    v_{rms} &= \sqrt{\langle v^2 \rangle} = \sqrt{\frac{3k_BT}{m}}
\end{align}
where $v_{rms}$ is the "root mean square" value. 
The mean kinetic energy of a gas molecule is then:
\begin{equation}
    \langle E_k \rangle = \frac{1}{2}m \langle v^2 \rangle = \frac{3}{2}k_BT
\end{equation}

\subsection*{Pressure}
Since all molecules are equally likely to travel in any direction, the fraction of these 
whose velocity lie in solid angle $d\Omega$ is therefore $d\Omega/4\pi$. We consider the 
molecules traveling at angle between $\theta$ and $\theta + d\theta$ with some specific direction ($z$),
the solid angle is given by:
\begin{equation}
    d\Omega = 2\pi \sin\theta d\theta
\end{equation}
So the fraction of molecules having speed between $v$ and $v_dv$ and angel between $\theta$ 
and $\theta + d\theta$ with direction $z$ is then:
\begin{equation}
    f(v) dv \frac{2\pi \sin\theta d\theta}{4\pi} = \frac{1}{2} f(v) dv \sin\theta d\theta 
\end{equation}

In a time interval $dt$, 
a molecules with speed $v$ travelling at an angle $\theta$ to the normal of the wall
will hit the container wall if it is $vdt\cos \theta$ away from the wall. 
the number of such molecules (speed $v$ at an angle $\theta$) 
hitting the wall of area $A$ is thus 
\begin{equation}
    A vdt\cos \theta n \cdot \frac{1}{2} f(v) dv \sin\theta d\theta 
\end{equation}
where the first part gives the total number of molecules in the volume $A vdt\cos \theta$ and 
the second part is the fraction of the molecule in the velocity and angle range.

Pressure of a gas on its container can be calculated by $dP = F dt$ where $dP$ is the momentum 
change of the particle reflecting from the wall: $dP = 2 mv \cos\theta$ (prependicular to wall). 
Pressure is defined to be $p = F/A$, so that 
We integrate over all possible velocity and angle $\theta$ to obtain the expression of $p$:
\begin{align}
    p &= \int_0^{\infty} dv \int_0^{\pi/2} d\theta  ( 2 mv \cos\theta ) 
    \left( v \cos \theta n \frac{1}{2} f(v) \sin\theta \right) \\
    &= \frac{1}{3} n m \langle v^2 \rangle
\end{align}
using the relationship $n = N/V$ and $\langle v^2 \rangle = 3k_BT / m$, we 
can obtain the ideal gas law:
\begin{equation}
    pV = N k_B T
\end{equation}

\subsection*{Partial pressure}
If we have a mixture of gases in equilibrium, the total pressure will be the 
sum of pressures of each component:
\begin{gather}
    n = \sum_i n_i \\
    p = \left( \sum_i n_i \right) k_B T = \sum_i p_i
\end{gather}
where $n_i = N_i / V$

\subsection*{Effusion}
Effusion is the process in which gas escape from a small hole\footnote{Blundell, p64}. we define 
the flux $\Phi$ as the number of molecules going through unit area per unit time,
therefore, $\Phi$ has the unit of $m^{-2}s^{-1}$. we use the above result of particles 
hitting the wall to find flux:
\begin{align}
    \Phi &= \int_0^{\infty} dv \int_0^{\pi/2} d\theta 
            \left( v \cos \theta n \frac{1}{2} f(v) \sin\theta \right) \label{flux} \\
        &= \frac{1}{4} n \langle v \rangle
\end{align}
with the result:
\begin{align}
    n &= p / k_B T \notag \\
    \langle v \rangle &= \sqrt{\frac{8k_BT}{\pi m}} \notag \\
    \Phi &= \frac{p}{\sqrt{2\pi m k_BT}}
\end{align}

Now, consider a we have a small hole (small enough that the gas escaping does not 
change the equilibrium distribution of gas near the hole), the number of molecules
escaping through the hole is given as:
\begin{equation}
    \Phi A = \frac{pA}{\sqrt{2\pi m k_BT}}
\end{equation}
which is linear to pressure, inversely proportion to $T$ and $m$. 

It should be noted that the speed distribution of the gas through the hole is no longer the 
Maxwell-Boltzmann distribution, since the amount of gas that effuse through the 
hole depend on velocity $v$ (in Eq.\ref{flux}), therefore the speed distribution
of those that go through the hole as an extra factor $v$, compare to the speed 
distribution of gas molecules inside the box.

\subsection*{Average lifetime}
We first considering a molecule moving at speed $v$ with other molecules in
the gas stationary. Within a time $dt$, the molecule sweep through an area $\sigma v dt$,
where $\sigma$ is the collision cross-section of the molecule. We denote $P(t)$ as 
the probability of this molecule travel without any collision up to time $t$. We have:
\begin{align}
    P(t + dt) &= P(t) + \frac{dP}{dt}dt \\
    P(t + dt) &= P(t) (1 - n\sigma v dt) 
\end{align}
with $n$ the number density of gas and therefore $(1 - n\sigma v dt)$ gives the 
probability of having no other gas molecule in the volume the moving molecule is going 
to sweep over from time $t$ to $t + dt$. 
Withe the above two equation, we obtain the result:
\begin{equation}
    P(t) = e^{-n\sigma vt}
\end{equation}
We have the boundary condition $P(0) = 1$ and $P(\infty) = 0$

The probability that the molecule collide in time interval from $t$ to $t+dt$ is given by
the probability that molecule does not collide at time $t$ but do collide at time $t + dt$:
$P(t) - P(t + dt)$ (since P(t) decay with time, $dP(t) < 0$), we have:
\begin{equation}
    - dP(t) = n\sigma v e^{-n\sigma vt} dt
\end{equation}
We can calculate the mean scattering time:
\begin{align}
    \tau &= \int_{0}^{\infty} t n \sigma v e^{-n\sigma vt} dt \notag \\
         &= \frac{1}{n\sigma v} \int_{0}^{\infty} x e^{-x} dx \notag \\
         &= \frac{1}{n\sigma v}
\end{align}
where $x = n\sigma v t$ represent a change of variable.

In the case of hard sphere approximation in a gas consist of molecules with
equal radius, the collision cross-section is given as:
\begin{equation}
    \sigma = \pi (r_1 + r_2) ^ 2 = 4 \pi a^2
\end{equation}

\subsection*{Mean free path of gas}
With the previous result, we have, for the mean free path
\begin{equation}
    \lambda = \langle v \rangle \tau = \langle v \rangle \frac{1}{n\sigma \langle v_r\rangle }
\end{equation}
$\langle v \rangle$ is the average speed of gas molecules,
where for the lifetime, instead of using $v$, we recognize that we should use 
relative speed $v_r$ in a real system when all particles are in motion. It is not attempted to derive
the value of $\langle v_r\rangle$ here (refer to Blundell, p73). But the final result
is: 
\begin{equation}
    \lambda = \frac{1}{\sqrt{2}n\sigma}
\end{equation}

\section{Transport properties in Ideal gas}

\subsection*{Viscosity}
Viscosity $\eta$ measures th resistance of a liquid (gas) to deform by shear stress. Suppose we sandwich fluid between 
two infinitly large plate. We fix the bottom plate and apply some force $F$ to the top plate. The fraction between the fluid and 
the top plate will accelerate the fluid near it, the the momentum is passed down inside the liquid through internal interaction (collision)

Now, we wait for long enough time until the whole system is in equilibrium. The top plate will reach some final speed $u$ and 
the fluid near the top plate will have the same macroscopic speed (condition of equilibrium). The bottom plate and the fluid near the bottom 
will be stationary (An external force is necessary to keep the bottom plate stationary). 
Therefore, a velocity gradient will be set up in the fluid between the two plates (in $z$ direction). We define 
the viscosity by equation:
\begin{equation}
    \tau_{shear} = \frac{F}{A} = \eta \frac{d \langle u_x \rangle}{dz}
\end{equation}

With force $F$ constantly applied on the top plate, we are inputing momentum $Fdt$ per unit area in time $dt$. In equilibrium, 
These momentum will be transported completely to the bottom plate. Therefore, we have momentum flux through $z$ direction 
(note that the momentum itself is along the plate, $x$ dirction, but the flux is along $z$):
\begin{equation}
    \Pi_z = - F / A = - \eta \frac{d \langle u_x \rangle}{dz}
\end{equation}
We have a negative sign because the momentum flow from high velocity area to low velocity area, which is opposite the velocity gradient.

We can calculate the momentum flux by considering the microscopic motion of gas molecules. We consider that the motion of free ideal gas superimposed 
with the collective drifting motion with velocity $\langle u_z \rangle$. A molecules travelling along $z$ direction will
change their momentum by collision with other molecules in the final positon $z_2$ ($z_1 \to z_2$). The number of particle with velocity $v$ travelling
at an angle $\theta$ with $z$ direction is $ v\cos \theta n \cdot \frac{1}{2} f(v) dv \sin\theta d\theta $ per unit area $A$ and time $dt$. How far they 
travel will be given by the mean free path $\lambda$. The momentum difference of a molecule just after collision at $z_1$ and just after collision at $z_2$ is 
given by:
\begin{equation}
    - m \left( \frac{\partial \langle u_x \rangle}{\partial z} \right) \lambda \cos\theta
\end{equation}
which is the momentum flux created by this single molecule. Summing over all molecules with different speed and angle, we have:
\begin{align}
    \Pi_z &= \int_0^{\infty} dv \int_0^{\pi/2} d\theta 
     v \cos \theta n \frac{1}{2} f(v) \sin\theta \cdot m \left( \frac{\partial \langle u_x \rangle}{\partial z} \right) \lambda \cos\theta \\
     &= -\frac{1}{3} n m \lambda \langle v \rangle \frac{\partial \langle u_x \rangle}{\partial z} \label{momentum_transfer}
\end{align}
We obtain the viscosity:
\begin{equation}
    \eta = \frac{1}{3} n m \lambda \langle v \rangle
\end{equation}
Using the previous result $\lambda = (\sqrt{2}n\sigma)^{-1}$ and $\langle v \rangle = ()\frac{8k_BT}{\pi m})^{1/2}$, we can also write:
\begin{equation}
    \eta = \frac{2}{3\sigma}\left( \frac{mk_BT}{\pi} \right)^{1/2}
\end{equation}
We have the following observations:
\begin{enumerate}
    \item $\eta$ is independent of pressure.
    \item $\eta \propto T^{1/2}$.
    \item For the above approximation of momentum transfer to be correct, we require $ L \ll \lambda \ll d $, where $d$ is the size of the molecule and $L$ is the size scale of the container.
\end{enumerate}

\subsection*{Thermal conductivity}
We define the heat flux in the $z$ direction
\begin{gather}
    J_z = -\kappa \left( \frac{\partial T}{\partial z} \right) \\
    \mathbf{J} = -\kappa \nabla T
\end{gather}
The gas molecules carrier heat through their kinetic energy, which depend on temperature $T$ via $\langle E_k \rangle = 3k_BT / 2$. Defining heat capacity of a 
molecule as $C$, we can calculate the total heat flux:
\begin{align}
    J_z &= \int_0^{\infty} dv \int_0^{\pi/2} d\theta 
    v \cos \theta n \frac{1}{2} f(v) \sin\theta \cdot - C \left( \frac{\partial T}{\partial z} \right) \lambda \cos\theta \\
    &= -\frac{1}{3} n C \lambda \langle v \rangle \frac{\partial T}{\partial z} \label{heat_transfer}
\end{align}
Using $C_V = n C$, the thermal conductivity of gas is therefore
\begin{equation}
    \kappa = \frac{1}{3} C_V \lambda \langle v \rangle
\end{equation}

We observe that Eq.\ref{heat_transfer} is very similar to Eq.\ref{momentum_transfer}, and we have:
\begin{equation}
    \frac{\kappa}{\eta} = \frac{C}{m}
\end{equation} 

\subsection*{Particle Diffusion}
We consider a gas of molecules in which some of them is labelled, if those labelled molecules are initially confined in 
certain area and the confinement is removed, they will start to diffuse (Self-diffuse). Suppose that the diffusion is 
along $z$ direction and we use $n^*(z)$ to denote the density of those labelled particles, we can define the 
diffusion coefficient: 
\begin{equation}
    \Phi_z = -D \left( \frac{\partial n^*}{\partial z} \right)
\end{equation}
Following the above microscopic picture, we have:
\begin{align}
    \Phi_z &= \int_0^{\infty} dv \int_0^{\pi/2} d\theta 
    v \cos \theta \frac{1}{2} f(v) \sin\theta \cdot - \left( \frac{\partial n^*}{\partial z} \right) \lambda \cos\theta \\
    &= -\frac{1}{3} \lambda \langle v \rangle \frac{\partial T}{\partial z} 
\end{align}
giving the self-diffusion coefficient
\begin{equation}
    D = \frac{1}{3} \langle v \rangle
\end{equation}

We have the following relationship:
\begin{enumerate}
    \item $ D \propto T^{3/2}$
    \item $D \rho = \eta $, where $rho$ is the density $\rho = nm$
\end{enumerate}

\subsection*{Heat diffusion equation}
Given the heat flux $J = - \kappa \nabla T$, the total heat flow out of a closed surface $S$ is $\int_S J \cdot dS$.
This value should equal to the loss of total thermal energy $\int_V CT dV$, where $C$ here is the volume heat capacity. 
We have the Result:
\begin{equation}
    \int_S J \cdot dS = \int_V \nabla \cdot J dV = -\frac{\partial}{\partial t} \int_V CT dV
\end{equation}
where we obtain the first equality through divergence theorem. We have
\begin{align}
    \nabla \cdot J = -C \frac{\partial T}{\partial t} \\
    \frac{\partial T}{\partial t} = D \nabla^2 T \label{thermaldiffusion}
\end{align}
with $D = \kappa/C$ is the thermal diffusivity.
Eq.\ref{thermaldiffusion} is called \textbf{Thermal diffusion equation}.

In a steady state, we have $\partial T / \partial t = 0 $ so that the 
diffusion equation is reduced to 
\begin{equation}
    \nabla^2 T = 0
\end{equation}

Suppose we have a gas between two hot plates separate with a distance $L$, 
one maintained at temperature $T_1$ and the other at $T_2 < T_1$. By integrating
the equation $\partial^2 T / \partial x^2 = 0$ twice 
(note that this equation imply a linear temperature distribution)
and using the boundary condition,
we have:
\begin{equation}
    T = \frac{(T_2-T_1)x}{L} + T_1 
\end{equation}
The heat flux is given by:
\begin{equation}
    J = - \kappa \left( \frac{\partial T}{\partial x} \right) = \frac{\kappa}{L} (T_1 - T_2)
\end{equation}
The value $\kappa / L$ is called \textbf{thermal conductance}.

If heat is generated at a rate $H$ per unit volume, the divergence of $J$ will 
be:
\begin{equation}
    \nabla \cdot J = -C \frac{\partial T}{\partial t} + H
\end{equation}
and the thermal diffusion equation will be modified to be:
\begin{equation}
    \frac{\partial T}{\partial t} = D \nabla^2 T + \frac{H}{C}
\end{equation}

\subsection*{Thermal diffusion equation in 1D}
We want to solve the equation:
\begin{equation}
    \frac{\partial T}{\partial t} = D \frac{\partial^2T}{\partial x^2}
\end{equation}
to obtain the temperature profile.

Since this equation is a second order linear partial equation, we can look for 
wave-like solutions
\begin{equation}
    T(x,t) \propto \exp(i(kx - \omega t))
\end{equation}
with $k = 2\pi/\lambda $ and $\omega = 2\pi/f$ the wave vector and angular frquency. Solution is 
given by:
\begin{gather}
    - i \omega = -D k^2 \\
    k = \pm (1 + i) \sqrt{\frac{\omega}{2D}}
\end{gather}
Noting that with $ k = -(1 + i) \sqrt{\frac{\omega}{2D}}$, $T \to \infty$ as $x \to \infty$. 
Therefore, the solution for the temperature profile can be written as a summation of frequency:
\begin{equation}
    T(x,t) = \sum_{\omega} A(\omega)\exp(-i\omega t) \exp((i-1)\sqrt{\frac{\omega}{2D}}x) \label{thermaldiffusionsolution}
\end{equation}

As an application, consider solving the 1D problem of heat diffusion into the earth ground. The 
boundary profile is given by
\begin{equation}
    T(0,t) = T_0 + \Delta T \cos(\Omega t) = T_0 + \frac{\Delta T}{2} e^{i\Omega t} + \frac{\Delta T}{2} e^{-i\Omega t} \label{thermaldiffusionboundary}
\end{equation}
where the period is given by the alternation of day and night. 

Requiring Eq.\ref{thermaldiffusionsolution} to give the boundary condition Eq.\ref{thermaldiffusionboundary}, we obtain
the solution:
\begin{equation}
    T(x,t) = T_0 + \Delta T e^{-x/\delta} \cos(\Omega t - \frac{x}{\delta})
\end{equation}
where $\delta = \sqrt{2D/\Omega}$ is called the skin depth, and temperature fall off exponentially as $e^{-x/\delta}$.

\subsection*{Newton's law of cooling}
Newton's low of cooling states that the heat loss of a surface is proportional to the area of the surface
multiplied by the temperature difference. The heat flux is given by:
\begin{equation}
    J = h \Delta T
\end{equation}
with $h$ the heat transfer coefficient of the surface. As an example, suppose a cup of tea at temperature $T_{hot}$ 
is placed in a room at temperature $T_{air}$ and the heat loss is through the surface area $A$. Suppose the air temperature
near the surface is maintained at $T_{air}$ with convection, we have:
\begin{equation}
    -C \frac{\partial T}{\partial t} = J A  = h A (T - T_{air})
\end{equation}
$T(t)$ is the temperature of the tea. We have the solution
\begin{equation}
    T(t) = T_{air} + (T_{hot} - T_{air})e^{-\lambda t}
\end{equation}
with $\lambda = A h / C $

\subsection*{Prandtl number}
Apart from thermal diffusion, convection also play a part in the heat transfer in solid and gas. Convection will
dominate if momentum diffusion dominates. We can thus compare the magnititude of the two mechanism by
\begin{equation}
    \sigma_p = \frac{v}{D} = \frac{\eta c_p}{\kappa}
\end{equation}
where $v = \eta / \rho$ is the kinematic viscosity and $D$ is the thermal diffusivity $D = \kappa/(\rho c_p)$ ($c_p$ is the specific heat).
This value of called \textbf{Prandtl number}. For $\sigma_p \gg 1$, the convection is the dominant mode of heat transport. 
For gas, $\sigma_p$ can be found to be $2/3$ with the previous results. For liquid, $\sigma_p \gg 1$.

\pagebreak
\section*{Appendix A}


\end{document}
