\documentclass{article}
\usepackage{amsmath}
\usepackage[margin=0.8in]{geometry}
\usepackage{verbatim}
\usepackage{graphicx}

\begin{document}

\title{Linear Response Theory}
\author{Wenhao}
\date{\today}
\maketitle

\section{Response functions}
We consider a time dependent Hamiltonian $H$, which is acted upon by an external time dependent field
$F(t)$ which couple linearly to an observable $B$ of the system:
\begin{equation}
    H(t) = H_0 - AF(t)
\end{equation} 
at $t \le t_0$, the system is in its groud state, at $t = t_0$ the external field is turned on and 
the system begin to evolve adiabatically. 
In the classical theory, the phase space probability $\rho$ evolve as:
\begin{eqnarray}
    \frac{\partial \rho}{\partial t} = \{ H(t), \rho\}
\end{eqnarray}
where $\{,\}$ is the poisson bracket. 
In linear approximation $\rho(t) = \rho_0 + \Delta \rho$, the equation of motion is:
\begin{gather}
    \frac{\partial \rho(t)}{\partial t} = \{ H_0, \rho(t)\} + \{H', \rho_0\} \notag \\
    \frac{\partial \Delta \rho(t)}{\partial t} = \{ H_0, \Delta \rho(t)\} - \{A, \rho_0\}F(t) \notag \\
\end{gather}

The change of the observed quantity $B$ is given by:
\begin{align}
    \Delta B(t) &= \int dqdp \Delta \rho(t) B(q,p) \notag \\
               &= -\int dqdp \int_{-\infty}^{t} \{A, \rho_0\} B(t-t') F(t') dt'
\end{align}
and the time dependence of operator $B$ is given by the Heisenburg equation:
\begin{equation}
    \dot{B}(p,q) = \{B,H_0\}
\end{equation}
The response function is defined by the reponse of an observable after time $t$ of a unit pulse:
\begin{align}
    \phi_{BA} (t) &= -\int dqdp \int_{-\infty}^{t} \{A, \rho_0\} B(t-t') \delta(t') dt' \notag \\
                &= -\int dqdp \{A, \rho_0\} B(t) \label{def_phi_t}
\end{align}
so that 
\begin{equation}
    \Delta B(t) = \int_{-\infty}^{t} \phi_{BA} (t-t') F(t') dt'
\end{equation}
which is summed over all the past time.

We define the frequency components of the response function as
\footnote{this follows the definition of \emph{Kubo 1957} Eq.2.21, in terms
of the more conventional way, we have:
\begin{equation}
    \chi_{BA}(\omega) = \lim_{\eta\to 0^+} \int_{0}^{\infty} \phi_{BA} (t) e^{i(\omega+i\eta)\tau} dt
\end{equation}
}
:
\begin{equation}
    \chi_{BA}(\omega) = \lim_{\eta\to 0^+} \int_{0}^{\infty} \phi_{BA} (t) e^{-\eta t-i\omega t} dt
\end{equation}

Let's consider the case where a (constant pertrubation) $F$ is applied continuously from $t = -\infty$ to 
$t = 0$ and stops. The system then relax throught internal interaction. The observable will
follow:
\begin{align}
    \Delta B(t) &= \int_{-\infty}^{0} \phi_{BA} (t-t') dt' F \notag \\
                &= \int_{t}^{\infty} \phi_{BA} (t') dt' F \notag \\
                &= \Phi_{BA} (t) F
\end{align}
and
\begin{equation}
    \Phi_{BA} (t) = \lim_{\eta\to 0^+} \int_{t}^{\infty} \phi_{BA} (t') e^{-\eta t'} dt'
\end{equation}

\section{Response functions in Quantum case}
In quantum case, we wish to find:
\begin{align}
    \langle B \rangle(t) - \langle B \rangle_0 
    &= \sum_n \langle n, t | \rho B | n, t \rangle - \sum_n \langle n, -\infty | \rho B | n, -\infty \rangle \notag \\
    &= \sum_n \langle n, t | \rho B | n, t \rangle - \sum_n \langle n, -\infty | \rho_{0} B | n, -\infty \rangle
\end{align}
where in the second equality, we use the adabatic approximation to assume that the probability of 
the states remain the same as in the unperturbed case. The states evolve according to the Schrodinger equation:
\begin{align}
    U(t,-\infty) &= e^{-i\frac{H_0}{h}t} U_I(t,-\infty) \notag \\
                 &= e^{-i\frac{H_0}{h}t} \exp\left[ -\frac{i}{\hbar} \int_{-\infty}^{t} H_I'(t') dt'  \right] \notag \\
                 &\approx e^{-i\frac{H_0}{h}t} \left( 1 + \frac{i}{\hbar} \int_{-\infty}^{t} A(t')f(t') dt' \right)
\end{align}
so the expectation value of $B$ at time t is given by:
\begin{align}
    \langle B \rangle(t) &=
    \sum_n \langle n, -\infty | \left( 1 - \frac{i}{\hbar} \int_{-\infty}^{t} A(t')f(t') dt' \right) e^{i\frac{H_0}{h}t} \rho_{eq} B
    e^{-i\frac{H_0}{h}t} \left( 1 + \frac{i}{\hbar} \int_{-\infty}^{t} A(t')f(t') dt' \right) | n, -\infty \rangle \notag \\
    &\approx \langle B \rangle_0  + \frac{i}{\hbar} \int_{-\infty}^{t} dt' 
    \left[ \sum_n \langle n, -\infty | e^{i\frac{H_0}{h}t} \rho_{eq} B e^{-i\frac{H_0}{h}t} A(t')f(t') - A(t')f(t') e^{i\frac{H_0}{h}t} \rho_{eq} B e^{-i\frac{H_0}{h}t} A(t')f(t') | n, -\infty \rangle \right] \notag \\
    & = \langle B \rangle_0  + \frac{i}{\hbar} \int_{-\infty}^{t} dt' 
    \text{Tr} \rho_{eq} [B(t),A(t')]
\end{align}
So that the difference is given by:
\begin{equation}
    \Delta B(t) = \frac{i}{\hbar} \int_{-\infty}^{t} dt' \text{Tr} \rho_{eq} [B(t),A(t')]
\end{equation}
The response function is given by
\begin{align}
    \phi_{BA} (t) &=\frac{i}{\hbar} \langle [B(t),A] \rangle_0 \\
                &= -\frac{i}{\hbar} \text{Tr} \rho_0 [A, B(t)] \\
                &= \frac{1}{i\hbar} \text{Tr} [\rho_0, A] B(t)
\end{align}
%In quantum case, the evolution of density matrix $\rho$ is given by:
%\begin{align}
%    \rho(t) &= U^{-1}(t,-\infty) \rho U(t,-\infty) \notag \\
%            &= U_{I}^{-1}(t,-\infty) e^{it/\hbar H_0}\rho e^{-it/\hbar H_0}  U_{I}(t,-\infty) \notag \\
%            &= U_{I}^{-1}(t,-\infty) \rho_{eq} U_{I}(t,-\infty)
%\end{align}
%where $U$ is the time evolution operator and $U_{I}$ is the time evolution operator due to the perturbation.
%\begin{equation}
%    U_I(t,-\infty) = exp\left( -\frac{i}{\hbar} \int_{-\infty}^{t} H'(t') dt'  \right)
%\end{equation}
%Now, $H'(t')$ is given by $H'(t') = -A(t')f(t')$. To linear approximation, we have:
%\begin{equation}
%    U_I(t,-\infty) \approx 1 + \frac{i}{\hbar} \int_{-\infty}^{t} A(t')f(t') dt' 
%\end{equation}
%And the density matrix is therefore:
%\begin{align}
%    \rho(t) &\approx \left( 1 - \frac{i}{\hbar} \int_{-\infty}^{t} A(t')f(t') dt' \right) \rho_{eq} 
%                    \left( 1 + \frac{i}{\hbar} \int_{-\infty}^{t} A(t')f(t') dt' \right) \notag \\
%            &\approx \rho_{eq} + \frac{i}{\hbar} \int_{-\infty}^{t} [\rho_{eq} , A(t') ]f(t') dt' 
%\end{align}
%The evolution of density matrix $\rho$ is given by:
%\begin{gather}
%    \frac{d}{dt} \rho_I(t) = \frac{1}{i\hbar} [ H_I'(t), \rho_I(t) ] 
%\end{gather}
%which is solved by, in linear order:
%\begin{equation}
%    \rho_I(t) \approx \rho_0 + \frac{i}{\hbar} \int_{-\infty}^{t}  [A(t'),\rho_{eq} ] f(t') dt'
%\end{equation}
%The change in expectation value of operator $B$ is then:
%\begin{align}
%    \Delta \langle B \rangle (t) &= \text{Tr} d\rho_I(t)B_I(t) \notag \\
%                &= \frac{i}{\hbar} \text{Tr}\int_{-\infty}^{t} [A(t'),\rho_{eq} ] B(t) f(t') dt' \notag \\
%                &= \frac{i}{\hbar} \text{Tr}\int_{-\infty}^{t} [A(t'),\rho_{eq} ] B(t-t') f(t') dt'
%\end{align}
%and 
%\begin{equation}
%    \phi_{BA} (t) = - \frac{1}{i\hbar} \text{Tr} [A, \rho_0] B(t-t') 
%\end{equation}
%with the time dependence of $B$ given by:
%\begin{equation}
%    \dot{B_I}(t) = \frac{1}{i\hbar} [B(t), H_0]
%\end{equation}
%with $B(0) = B$ in the Schrodinger's picture.
%\begin{align}
%    \phi_{BA} (t) &=\frac{1}{i\hbar} \text{Tr} [\rho_0, A] B(t) \\
%                &= \frac{1}{i\hbar} \text{Tr} \rho_0 [A, B(t)] \\
%                &= \frac{1}{i\hbar} \langle [A, B(t)] \rangle_0
%\end{align}
Using the identity:

\begin{gather}
    [A, e^{-\beta H_0}] = e^{-\beta H_0} \int_0^{\beta} e^{\lambda H_0} [H_0, A] e^{-\lambda H_0} d\lambda 
    = e^{-\beta H_0} \int_0^{\beta} e^{\lambda H_0} (-i\hbar) \dot{A} e^{-\lambda H_0} d\lambda 
\end{gather}
\begin{align}
    [\rho_{eq}, A] &=  i\hbar \rho_{eq} \int_0^{\beta} e^{\lambda H_0} \dot{A} e^{-\lambda H_0} d\lambda \notag \\
                &=  i\hbar  \int_0^{\beta} \rho_{eq} \dot{A}(-i\hbar\lambda) d\lambda
\end{align}
so that $\exp(-iH_0t/\hbar) \to \exp( -H_0\lambda ) $
%We can find the following relationship:
%\begin{equation}
%    [\rho_0, A] = i\hbar \int_0^{\beta} \rho_0 \dot{A}(-i\hbar\lambda) d\lambda
%\end{equation}
So that we can arrive at the formula given by Kubo:
\begin{align}
    \phi_{BA} (t) &= \int_0^{\beta} \text{Tr} \rho_0 \dot{A}(-i\hbar\lambda) B(t) d\lambda \\
    \chi_{BA}(\omega) &= 
    \lim_{\eta\to 0^+} \int_0^{\beta} d\lambda \int_{0}^{\infty} e^{-\eta t-i\omega t} dt \text{Tr} \rho_0 \dot{A}(-i\hbar\lambda) B(t)
\end{align}

%For the function $\Phi_{BA}(t)$, we can obtain:
%\begin{align}
%    \Phi_{BA} (t) = \int_{t}^{\infty} \phi_{BA} (t') dt' \notag \\
%\end{align}
\pagebreak
\section{Linear response formula of conductivity}
We consider an uniform external electric field (potential zero is arbitrary):
\begin{gather}
    H'(t) = -e \sum_i x_i E(t) = - A E(t) \\
    \dot{A} = e \sum_i \dot{x}_i = J
\end{gather}

where $x_i$ is the position operator of the $i^{th}$ particle. The current operator is 
defined to be:
\begin{equation}
    J_{\mu} = e \sum_i \dot{x}_i
\end{equation}
The response function is given by:
\begin{align}
    \phi_{\mu\nu} (t) &= \int_0^{\beta} \langle J_{\nu}(-i\hbar\lambda) J_{\mu}(t) \rangle_0 d\lambda \\
    \chi_{\mu\nu}(\omega)&= 
    \lim_{\eta\to 0^+} \int_0^{\beta} d\lambda \int_{0}^{\infty} e^{-\eta t-i\omega t} dt  \langle J_{\nu}(-i\hbar\lambda) J_{\mu}(t) \rangle_0
\end{align}
and the conductivity is given by:
\begin{equation}
    \sigma_{\mu\nu} = \frac{1}{V} \int_0^{\beta} d\lambda \int_{0}^{\infty} dt \langle J_{\nu}(-i\hbar\lambda) J_{\mu}(t) \rangle_0
\end{equation}

\section{Linear response formula of thermal conductivity}
Derivation of the expression of thermal conductivity is provided by \emph{Allen and Feldman, 1993}. 
The total current operator is given by:
\begin{equation}
    J_{\alpha} = \sum_{ij\beta\gamma} (R_{i\alpha} - R_{j\alpha}) \Phi_{ij}^{\beta\gamma} u_{i\beta} \dot{u}_{j\gamma}
\end{equation}
We consider the Hamiltonian of the system to be:
\begin{equation}
    H = \int h(x)  d^3x
\end{equation}
where $h(x) = \sum_i h_i \delta(x - x_i)$ consists of the vibration energy of each atom $i$. 
The local current operator $S(x)$ is related to $h(x)$ by the continuity equation:
\begin{gather}
    \frac{\partial h(x)}{\partial t} + \nabla\cdot S(x) = 0 \\
    J = \int S(x) d^3x
\end{gather}
The density matrix can be written as:
\begin{equation}
    \rho = \frac{1}{Z} e^{- \int \beta(x) h(x) d^3x}
\end{equation}
and $\beta(x) \approx \beta[1-\delta T(x)/T]$ with $T$ the average temperature, then
\begin{equation}
    \rho = \frac{1}{Z} e^{- \int \beta[1-\delta T(x)/T] h(x) d^3x} = \frac{1}{Z} e^{- \beta (H + H')}
\end{equation}
with $H'$:
\begin{align}
    H' &= -\frac{1}{T} \int \delta T(x)h(x) d^3x \notag \\
        & = \frac{1}{T} \int d^3x \int _{-\infty}^{0} dt \delta T(x) \nabla\cdot S(x,t) \notag \\
        & = - \left( \frac{1}{T} \int d^3x \int _{-\infty}^{0} dt S(x,t) \right) \nabla T \notag \\
        & = - \left( \frac{1}{T} \int _{-\infty}^{0} dt J(t) \right) \nabla T 
\end{align}
where we take $\nabla T$ to be uniform. Now, we can apply the Kubo's formula:
\begin{align}
    \kappa_{\mu\nu} (t) & = \frac{1}{VT} \int_0^{\beta} \langle J_{\nu}(-i\hbar\lambda) J_{\mu}(t) \rangle d\lambda \\
    \kappa_{\mu\nu} (\omega) & = \frac{1}{VT} \int_0^{\beta} d\lambda \int_{0}^{\infty} e^{-i\omega t} dt \langle J_{\nu}(-i\hbar\lambda) J_{\mu}(t) \rangle 
\end{align}

\pagebreak
\section{From Q.T.E.L.}
The time evolution of the system is given by:
\begin{align}
    | \Psi_n(t) \rangle &= U(t,t_0) | \Psi_n(t_0) \rangle \notag \\
                     &= e^{-\frac{i}{\hbar}H_0(t-t_0)} U_I(t,t_0) | \Psi_n(t_0) \rangle \notag 
\end{align}
$U_I(t,t_0)$ is given by the equation of motion:
\begin{equation}
    i\hbar \frac{\partial}{\partial t} U_I(t,t_0) = f(t) B(t-t_0) U_I(t,t_0) \label{eqmotion}
\end{equation}
with the initial condition $U_I(t_0,t_0) = 1$. Making the approximation that $U_I(t,t_0) = 1$ on the 
right hand side of the Eq.\ref{eqmotion} and integrate from time $t_0 \to t$, we obtain the first order approximation:
\begin{equation}
    U_I^{(1)}(t,t_0) = 1 - \frac{i}{\hbar} \int_{t_0}^{t} f(t') B(t'-t_0) dt'
\end{equation}

The thermal average of another observable of the system $A$ is given by:
\begin{equation}
    \langle A\rangle_0 = \frac{1}{Z} \sum_n e^{-\beta E_n} \langle \Psi_n(t_0) | A | \Psi_n(t_0) \rangle 
\end{equation}
and since we consider adiabatic evolution, at a later time $t$, the thermal average is 
given by:
\begin{equation}
    \langle A\rangle(t) = \frac{1}{Z} \sum_n e^{-\beta E_n} \langle \Psi_n(t) | A | \Psi_n(t) \rangle 
\end{equation}
where the probability of finding each state is kept constants (instead of $e^{-\beta H}$ as an operator).
Now, inserting the evolution of states, we obtain:
\begin{equation}
    \langle A\rangle(t) - \langle A\rangle_0 = -\frac{i}{\hbar} \int_{t_0}^{t} \langle [ A(t), B(t') ] \rangle_0 f(t') dt'
\end{equation} 
where is thermal average is taken at the time independent equilibrium ensemble, and the time dependence of
the operator $A(t)$ and $B(t)$ is given by the Heisenburg form:
\begin{equation}
    B(t) = e^{\frac{i}{\hbar}H_0t} B e^{-\frac{i}{\hbar}H_0t}
\end{equation}
taking $\tau = t - t' > 0$, we can define:
\begin{equation}
    \chi_{AB}(\tau) = -\frac{i}{\hbar} \theta(\tau) \langle [ A(\tau), B ] \rangle_0 \label{responsefunction}
\end{equation}
so that 
\begin{equation}
    \langle A\rangle(t) = \langle A\rangle_0 + \int_{0}^{t-t_0} \chi_{AB}(\tau) f(t - \tau) d\tau
\end{equation} 
where we changed the integral variable from $t'$ to $\tau = t - t'$. 

Because of the $\theta(\tau)$ in Eq.\ref{responsefunction}, $\chi_{AB}(\tau)$ is called the retarded response function.
We have previously taken the external field to turn on instantly at $t_0$, however, if we allow the field to switch on
slowly, as $f(t) \to 0$ as $t_0 \to -\infty$, the result at time $t$ should remain the same, so we 
can also take:
\begin{equation}
    \langle A\rangle(t) = \langle A\rangle_0 + \int_{0}^{\infty} \chi_{AB}(\tau) f(t - \tau) d\tau \label{response}
\end{equation} 

\section{Frequency domain}
Consider a switching on (real) periodic perturbation, which vanish for $t \to -\infty$:
\begin{equation}
    f(t) = f_{\omega}e^{-i(\omega + i\eta) t} + c.c.
\end{equation}
$\eta$ is a positive and $\eta^{-1}$ give a time scale that longer than period of the perturbation.
We can thus apply the linear response formalism and take the limit $\eta \to 0^+$ at the 
end of the calculation. If this limit exist, it should describe the reponse of the 
system to a steady periodic field that has been applied long enough so that initial 
condition is ignorable.

Inserting the periodic perturbation into Eq.\ref{response} gives:
\begin{equation}
    \langle A\rangle(t) - \langle A\rangle_0 = \chi_{AB}(\omega) f_{\omega} e^{-i\omega t} + c.c.
\end{equation}
where 
\begin{equation}
    \chi_{AB}(\omega) = -\frac{i}{\hbar} \lim_{\eta\to 0^+} \int_{0}^{\infty} \langle [ A(\tau), B ] \rangle_0 e^{i(\omega+i\eta)\tau} d\tau
\end{equation}
For any general perturbation, we fourier transform between $t$ and $\omega$ is given by:
\begin{gather}
    f(t) = \frac{1}{2\pi} \int_{-\infty}^{\infty} f(\omega) e^{i\omega t} d\omega \\
    f(\omega) = \int_{-\infty}^{\infty} f(t) e^{i\omega t} dt
\end{gather}

\end{document}