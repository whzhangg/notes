\documentclass{article}
\usepackage{amssymb, amsmath, amsthm}
\usepackage[margin=1in]{geometry}
\usepackage{verbatim}
\usepackage{graphicx}
\usepackage{hyperref} % \url \href
\usepackage{docmute}

\newtheorem{definition}{Definition}
\newtheorem{theorem}{Theorem}
\newcommand{\heff}{\mathbb{H}^{\text{eff}}}
\newcommand{\pfrac}[2]{\frac{\partial #1}{\partial #2}}

\newcommand{\MO}{\textbf{MO}}
\newcommand{\AO}{\textbf{AO}}

\newcommand{\huptb}{\text{H}_0}
\newcommand{\order}[2]{#1^{(#2)}}
\newcommand{\statebra}[1]{\langle #1 |}
\newcommand{\stateket}[1]{| #1 \rangle}

\begin{document}

\section{Spin Polarization}
\subsection{Many body states}
Multi-electron systems are fundamentally manybody problem, so single particle 
solutions are often not enough to present the true electronic structure of the 
system. We should have the following hierarchy:
\begin{enumerate}
    \item single particle states
    \item many-body hartree fock state using slater determinant
    \item mixture of hartree fock state
\end{enumerate}
Suppose that we have a set of orthogonal single particle states $\psi_i(r_{\mu})$, where $i$
index the single particle states and $r_{\mu}$ is the position of the $\mu$th particle. 
Spins are included in the wavefunction. Then a slater determinant is given as
\begin{equation}
    \Phi(r_1, \cdots, r_n) = \frac{1}{\sqrt{N!}} 
    \left| \begin{matrix}
        \psi_1(r_1) & \cdots & \psi_1(r_n) \\ 
        \psi_2(r_1) & \cdots & \psi_2(r_n) \\
        \vdots &  & \vdots \\ 
        \psi_n(r_1) & \cdots & \psi_n(r_n)
    \end{matrix} \right|
\end{equation}
which satisfy the antisymmetry under exchange of particle. 

Using these slater states as the initial ``unperturbed'' states, we can find their 
energy expectation values:
\begin{equation}
    E_{\Phi_a} = \statebra{\Phi_a(r_1, \cdots, r_n)} H \stateket{\Phi_a(r_1, \cdots, r_n)}
\end{equation}
Furthermore, the off-diagonal elements are also non-zero, leading to a treatment similar 
to the perturbation treatment. The true eigenstates can thus be written as a linear combination:
\begin{equation}
    \Psi_g = \sum_{i} c_{ig} \Phi_i
\end{equation}
by solving the Hamiltonian matrix with element $H_{ij} = \statebra{\Phi_i} H \stateket{\Phi_j}$. 

\subsection{Hartree Fock}


\subsection{Exchange Interaction and Spin polarization}


\end{document}
