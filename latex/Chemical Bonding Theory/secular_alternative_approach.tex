\documentclass{article}
\usepackage{amssymb, amsmath, amsthm}
\usepackage[margin=1in]{geometry}
\usepackage{verbatim}
\usepackage{graphicx}
\usepackage{hyperref} % \url \href
\usepackage{docmute}

\newtheorem{definition}{Definition}
\newtheorem{theorem}{Theorem}
\newcommand{\heff}{\mathbb{H}^{\text{eff}}}
\newcommand{\pfrac}[2]{\frac{\partial #1}{\partial #2}}

\begin{document}

\section{Another approach to obtain the Secular Equation}
Following the notation 
\begin{equation}
    \psi_i = \mathbf{\chi}\cdot \mathbf{C}_i = \sum_{\mu} C_{\mu i} \chi_{\mu}
\end{equation}
where the coefficients are assumed to be real and unknown. We denote:
\begin{align}
    H_{ij} &= \sum_{\mu\nu} c_{\mu i} c_{\nu i} H_{\mu \nu} \\ 
    S_{ij} &= \sum_{\mu\nu} c_{\mu i} c_{\nu i} S_{\mu \nu} 
\end{align}
let's seek to minimize the orbital energies:
\begin{equation}
    \varepsilon_i = \frac{\langle \psi_i | H | \psi_j \rangle}{\langle \psi_i | \psi_j \rangle} = \frac{B}{A}
\end{equation}
with respect to the coefficients $c_{\mu i}$:
\begin{equation}
    \pfrac{\varepsilon_i}{c_{\mu i}} = \pfrac{(B/A)}{c_{\mu i}} = \frac{1}{A} \pfrac{B}{c_{\mu i}} - \frac{B}{A^2} \pfrac{A}{c_{\mu i}} = 0
\end{equation}
The terms are given:
\begin{align}
    \pfrac{B}{c_{\mu i}} &= 2 \sum_{\nu} c_{\nu i} H_{\mu \nu} \\
    \pfrac{A}{c_{\mu i}} &= 2 \sum_{\nu} c_{\nu i} S_{\mu \nu} \\
\end{align}
so that we obtain the \emph{secular equation}:
\begin{equation}
    \label{E:secular}
    \sum_{\nu} \left[ H_{\mu\nu} - \varepsilon_i S_{\mu\nu} \right] c_{\nu i} = 0
\end{equation}

\end{document}
