\documentclass{article}
\usepackage{amssymb, amsmath, amsthm}
\usepackage[margin=1in]{geometry}
\usepackage{verbatim}
\usepackage{graphicx}
\usepackage{hyperref} % \url \href
\usepackage{docmute}

\newtheorem{definition}{Definition}
\newtheorem{theorem}{Theorem}
\newcommand{\heff}{\mathbb{H}^{\text{eff}}}
\newcommand{\pfrac}[2]{\frac{\partial #1}{\partial #2}}

\newcommand{\MO}{\textbf{MO}}
\newcommand{\AO}{\textbf{AO}}

\newcommand{\huptb}{\text{H}_0}
\newcommand{\order}[2]{#1^{(#2)}}
\newcommand{\statebra}[1]{\langle #1 |}
\newcommand{\stateket}[1]{| #1 \rangle}

\begin{document}

\section{Solution to the Molecular Orbtials}
In this section, we write out explicitly the solution of the molecular orbitals. In the next section,
we will focus on the perturbation treatment. 
Let's write the 
\begin{align}
    &\mathbf{\chi} = \left( \begin{matrix}
        \chi_1& \chi_2& \chi_3& \cdots& \chi_m
    \end{matrix} \right) \\
    &\mathbf{C}_i = \left( \begin{matrix}
        C_{1i} \\ C_{2i} \\ \vdots \\ C_{mi}
    \end{matrix} \right)
\end{align}
so that the \MO wavefunction can be written as dot product:
\begin{equation}
    \psi_i = \mathbf{\chi}\cdot \mathbf{C}_i = \sum_{\mu} C_{\mu i} \chi_{\mu}
\end{equation}
\MO s satisfy the eigenequation 
\begin{gather}
    \text{H}^{MO} \stateket{\psi_i} = \varepsilon_i \stateket{\psi_i} \\ 
    \sum_{\mu} \statebra{\chi_{\nu}}\text{H}\stateket{\chi_{\mu}}\statebra{\chi_{\mu}} \psi_i \rangle  = \sum_{\mu'} \varepsilon_i \statebra{\chi_{\nu}} \chi_{\mu'} \rangle \statebra{\chi_{\mu'}} \psi_i \rangle \\ 
    \sum_{\mu} \text{H}_{\nu\mu} C_{\mu i} =  \varepsilon_i S_{\nu \mu'} C_{\mu' i} \\ 
    \text{H}^{AO} \mathbf{C}_i = \varepsilon_i \mathbf{S}^{AO} \mathbf{C}_i
\end{gather}
in matrix form, we have:
\begin{equation}
    \mathbf{C} = \left( \begin{matrix} \mathbf{C}_1 & \mathbf{C}_2 & \cdots & \mathbf{C}_m \end{matrix} \right) 
    = \left( \begin{matrix}
        C_{11} & \cdots & C_{1m} \\
        \vdots & & \vdots \\ 
        C_{m1} & \cdots & C_{mm}  
    \end{matrix}\right)
\end{equation}
so that we can discard the index $i$:
\begin{equation}
    \textbf{H}^{AO} \mathbf{C} = \varepsilon \mathbf{S}^{AO} \mathbf{C}
\end{equation}
which is similar to an ordinary eigenequation except that the Hamiltonian 
is in a set of basis that are not orthogonal. The normalization of $\mathbf{C}$
is given with respect to overlap matrix $S$:
\begin{equation}
    \statebra{ \psi_i } \psi_j \rangle = \left( \sum_{\mu} C_{\mu i} \statebra{\chi_{\mu}} \right)  \left( \sum_{\nu} C_{\nu i} \stateket{\chi_{\nu}} \right) = \sum_{\mu\nu} c_{\mu i} c_{\nu i} S_{\mu \nu} = \delta_{ij}
\end{equation}
with real coefficients $\mathbf{C}$


\section{Molecular Perturbation Theory}
Let's consider the effect of perturbation on the molecular orbitals and their energies. The 
treatment is similar to that of the ordinary perturbation theory except that the basis functions 
are no longer orthogonal when the perturbation is introduced. For example, the change in geometry
by bringing two structure fragment together necessary introduce change in overlap so that the 
original wavefunctions are no longer orthogonal. This can be compared to text book perturbation
theory in quantum mechanics where only perturbation in Hamiltonian are introduced. As an example, 
let's consider the case of bringing the isolated molecular fragments $A$ and $B$ together to 
form molecular $AB$. In the original system, suppose we already solved the molecular orbitals on
each fragment. We can write down the system as
$H^0 C^0 = S^0 C^0 \varepsilon^0$ where
\begin{align}
    C^0 &= \left( \begin{matrix} C_A^0 & \mathbf{0} \\ \mathbf{0} & C_B^0 \end{matrix} \right) &
    S^0 &= \left( \begin{matrix} S_A^0 & \mathbf{0} \\ \mathbf{0} & S_B^0 \end{matrix} \right) \\
    H^0 &= \left( \begin{matrix} H_A^0 & \mathbf{0} \\ \mathbf{0} & H_B^0 \end{matrix} \right) &
    \varepsilon^0 &= \left( \begin{matrix} \varepsilon_A^0 & \mathbf{0} \\ \mathbf{0} & \varepsilon_B^0 \end{matrix} \right) \\
\end{align}
This is the block diagonal form composed of solutions of the molecular orbital coefficients $C_A^0$ and $C_B^0$
for each isolated fragments. In the case of the solution, the Hamiltonian is diagonalized and 
the overlap matrix is simply identity matrix. 

Now, the molecular orbital coefficients solved on the perturbed Hamiltonian can be written as 
\begin{equation}
    \label{perturbation_coefficients}
    \stateket{\psi_i} = \sum_j T_{ji} \stateket{\psi^0_j}
\end{equation}
we write down the perturbation as:
\begin{align}
    H = H^0 + \delta H \\
    S = S^0 + \delta S
\end{align}
We have the secular equation:
\begin{equation}
    (H - \varepsilon_i S) C_i = \left[ (H^0 + \delta H) - \varepsilon_i (S^0 + \delta S) \right] C_i = 0
\end{equation}
Using equation \eqref{perturbation_coefficients} and the fact that $\stateket{\psi^0_j}$ diagonalize $S^0$ and $H^0$
\footnote{is the solution of the secular equation spontaneously diagonalize H and S?},
we have:
\begin{align}
    (\varepsilon^0 + \delta H) T_i = \varepsilon_i (1 + \delta S) T_i
\end{align} 
which we simplify to $(h - \varepsilon_i S) T_i = 0$, where:
\begin{align}
    h &= \order{h}{0} + \lambda \order{h}{1} \\ 
    s &= \order{s}{0} + \lambda \order{s}{1} \\ 
    \varepsilon_i &= \order{\varepsilon_i}{0} + \lambda \order{\varepsilon_i}{1} + \lambda^2 \order{\varepsilon_i}{2} + \cdots \\
    T_i &= \order{T_i}{0} + \lambda \order{T_i}{1} + \lambda^2 \order{T_i}{2} + \cdots \\ 
\end{align}
Substituting these quantity into $(h - \varepsilon_i S) T_i = 0$ and matching the order up to second order, we find:
\begin{align}
    \left[(\order{h}{0} - \order{\varepsilon_i}{0}\order{S}{0}) + \lambda (\order{h}{1} - \order{\varepsilon_i}{1} \order{S}{0} - \order{\varepsilon_i}{0} \order{S}{1}) - \lambda^2 (\order{\varepsilon_i}{2} \order{S}{0} + \order{\varepsilon_i}{1} \order{S}{1})\right] \\ 
    \times \left( \order{T_i}{0} + \lambda \order{T_i}{1} + \lambda^2 \order{T_i}{2} \right) = 0
\end{align}
The zeroth order simply is the unperturbed solution, to first order, writting $\order{T_i}{1} = \sum_{j} a_{ji} \order{T_j}{0}$, we have:
\begin{equation}
    \label{E:mop_firstorder}
    (\order{h}{1} - \order{\varepsilon_i}{1} \order{S}{0} - \order{\varepsilon_i}{0} \order{S}{1}) \order{T_i}{0} + (\order{h}{0} - \order{\varepsilon_i}{0}\order{S}{0}) \sum_{k} a_{ki} \order{T_k}{0} = 0
\end{equation}
We use the following relationship:
\begin{align}
    (\order{T_j}{0})^T \order{T_i}{0} &= \delta_{ij}  & &\\ 
    (\order{T_j}{0})^T \order{h}{1} \order{T_i}{0} &= \tilde{H}_{ij} & (\order{T_j}{0})^T \order{h}{0} \order{T_i}{0} &= \varepsilon_i \delta_{ij}\\ 
    (\order{T_j}{0})^T \order{S}{1} \order{T_i}{0} &= \tilde{S}_{ij} & (\order{T_j}{0})^T \order{S}{0} \order{T_i}{0} &= \delta_{ij}\\
\end{align}
we multiply equation \eqref{E:mop_firstorder} with $(\order{T_i}{0})^T$ on the left to obtain:
\begin{gather}
    ( \tilde{H}_{ii} - \order{\varepsilon_i}{1} - \order{\varepsilon_i}{0} \tilde{S}_{ii} ) + a_{ii} (\order{\varepsilon_i}{0} - \order{\varepsilon_i}{0}) = 0 \\ 
    \order{\varepsilon_i}{1} = \tilde{H}_{ii} - \order{\varepsilon_i}{0} \tilde{S}_{ii}
\end{gather}
multiply \eqref{E:mop_firstorder} with $(\order{T_j}{0})^T; (j\neq i)$ on the left, we obtain:
\begin{gather}
    ( \tilde{H}_{ij}- \order{\varepsilon_i}{0} \tilde{S}_{ij} ) + \sum_k \delta_{jk} a_{ki} (\order{\varepsilon_j}{0} - \order{\varepsilon_i}{0}) = 0 \\ 
    a_{ji}(j\neq i) = \frac{\tilde{H}_{ij}- \order{\varepsilon_i}{0} \tilde{S}_{ij}}{\order{\varepsilon_i}{0} - \order{\varepsilon_j}{0}}
\end{gather}
For second order, we have:
\begin{equation}
    (\order{h}{0} - \order{\varepsilon_i}{0}\order{S}{0}) \order{T_i}{2} + (\order{h}{1} - \order{\varepsilon_i}{1} \order{S}{0} - \order{\varepsilon_i}{0} \order{S}{1}) \order{T_i}{1} 
    - (\order{\varepsilon_i}{2} \order{S}{0} + \order{\varepsilon_i}{1} \order{S}{1}) \order{T_i}{0} = 0
\end{equation} 
using the results for the first order perturbation and multiplying $(\order{T_i}{0})^T$ or  $(\order{T_j}{0})^T; (j\neq i)$ on the left, we find the 
result:
\begin{align}
    \order{\varepsilon_i}{2} &= - \order{\varepsilon_i}{0} \tilde{S}_{ii} + \sum_{j\neq i} \frac{(\tilde{H}_{ij}- \order{\varepsilon_i}{0} \tilde{S}_{ij})^2}{\order{\varepsilon_i}{0} - \order{\varepsilon_j}{0}} \\ 
    b_{ji} (j\neq i) &= - \frac{(\tilde{H}_{ii}- \order{\varepsilon_i}{0}\tilde{S}_{ii} )\tilde{S}_{ij}}{\order{\varepsilon_i}{0} - \order{\varepsilon_j}{0}}
    + \sum_{k\neq i} \frac{(\tilde{H}_{ik}- \order{\varepsilon_i}{0} \tilde{S}_{ik})(\tilde{H}_{kj}- \order{\varepsilon_i}{0} \tilde{S}_{kj})}{(\order{\varepsilon_i}{0} - \order{\varepsilon_j}{0})(\order{\varepsilon_i}{0} - \order{\varepsilon_k}{0})}
\end{align}
so that $\order{T_i}{2} = \sum_j b_{ji} \order{T_j}{0}$.

Now, we note that the coefficient $a_{ii}$ and $b_{ii}$ are not yet obtained. They are given by the normalization condition as well as to the previous results, we have:
\begin{align}
    \order{T_{ii}}{1} &= - \frac{1}{2} \tilde{S}_{ii} \\ 
    \order{T_{ii}}{2} &= 
    - \sum_{j\neq i} \frac{(\tilde{H}_{ij} - \order{\varepsilon_i}{0}\tilde{S}_{ij})\tilde{S}_{ij}}{\order{\varepsilon_i}{0} - \order{\varepsilon_j}{0}}
    - \frac{1}{2} \sum_{j\neq i} \left(\frac{\tilde{H}_{ij} - \order{\varepsilon_i}{0}\tilde{S}_{ij}}{\order{\varepsilon_i}{0} - \order{\varepsilon_j}{0}} \right)^2
\end{align}

Finally, we obtain the complete result up to second order:
\begin{align}
    \varepsilon_i &= \order{\varepsilon_i}{0} + \underbrace{(\tilde{H}_{ii} - \order{\varepsilon_i}{0} \tilde{S}_{ii})}_{1^{th} order} 
    - \underbrace{\order{\varepsilon_i}{0} \tilde{S}_{ii} + \sum_{j\neq i} \frac{(\tilde{H}_{ij}- \order{\varepsilon_i}{0} \tilde{S}_{ij})^2}{\order{\varepsilon_i}{0} - \order{\varepsilon_j}{0}} }_{2^{nd} order}  \\ 
    T_{ii} &= 1 - \underbrace{\frac{1}{2} \tilde{S}_{ii}}_{1^{th} order} - \underbrace{\sum_{j\neq i} \frac{(\tilde{H}_{ij} - \order{\varepsilon_i}{0}\tilde{S}_{ij})\tilde{S}_{ij}}{\order{\varepsilon_i}{0} - \order{\varepsilon_j}{0}}
    - \frac{1}{2} \sum_{j\neq i} \left(\frac{\tilde{H}_{ij} - \order{\varepsilon_i}{0}\tilde{S}_{ij}}{\order{\varepsilon_i}{0} - \order{\varepsilon_j}{0}} \right)^2}_{2^{nd} order} \\ 
    T_{ji} &= \underbrace{\frac{\tilde{H}_{ij}- \order{\varepsilon_i}{0} \tilde{S}_{ij}}{\order{\varepsilon_i}{0} - \order{\varepsilon_j}{0}} }_{1^{th} order}
    - \underbrace{\frac{(\tilde{H}_{ii}- \order{\varepsilon_i}{0}\tilde{S}_{ii} )\tilde{S}_{ij}}{\order{\varepsilon_i}{0} - \order{\varepsilon_j}{0}}
    + \sum_{k\neq i} \frac{(\tilde{H}_{ik}- \order{\varepsilon_i}{0} \tilde{S}_{ik})(\tilde{H}_{kj}- \order{\varepsilon_i}{0} \tilde{S}_{kj})}{(\order{\varepsilon_i}{0} - \order{\varepsilon_j}{0})(\order{\varepsilon_i}{0} - \order{\varepsilon_k}{0})} }_{2^{nd} order}
\end{align}

Slight simplification can be given if we assume that $\delta H$ and $\delta S$ separately for the two fragments are zero:
\begin{align}
    \delta H = \left( \begin{matrix} \mathbf{0} & \delta H_{AB} \\ \delta H_{BA}  & \mathbf{0} \end{matrix} \right); \quad
    \delta S = \left( \begin{matrix} \mathbf{0} & \delta S_{AB} \\ \delta S_{BA}  & \mathbf{0} \end{matrix} \right)
\end{align}
in which case, the terms in the expansion like $(\tilde{H}_{ii} - \order{\varepsilon_i}{0} \tilde{S}_{ii})$ will be zero, and the summation 
are restricted with respect to the fragments where those states come from. 
The degenerate case can proceed similar to the non-degenerate theory, with the coefficients $T_{ji}^0$ no longer simply given by $\delta_{ji}$. 
We omit the degenerate treatment here.


\end{document}
