\documentclass{article}
\usepackage{amssymb, amsmath, amsthm}
\usepackage[margin=1in]{geometry}
\usepackage{verbatim}
\usepackage{graphicx}
\usepackage{hyperref} % \url \href
\usepackage{docmute}

\newtheorem{definition}{Definition}
\newtheorem{theorem}{Theorem}
\newcommand{\heff}{\mathbb{H}^{\text{eff}}}
\newcommand{\pfrac}[2]{\frac{\partial #1}{\partial #2}}

\newcommand{\MO}{\textbf{MO}}
\newcommand{\AO}{\textbf{AO}}

\newcommand{\huptb}{\text{H}_0}
\newcommand{\order}[2]{#1^{(#2)}}
\newcommand{\statebra}[1]{\langle #1 |}
\newcommand{\stateket}[1]{| #1 \rangle}

\begin{document}

\section{Through-bond Interaction}
In this section, we provide example of how the presense of some bonds 
affect other bonds in molecular. As we can see, such bond-bond interaction 
can be understood following the consideration of molecular orbital 
perturbation treatment. 

\subsection{Bond-bond Interaction}
Let's consider the molecular orbital of C$_3$H$_6$ and its variant to illustrate the 
how the presense of orbitals change the interaction between other bonds and eventually the 
geometry of the molecular structure. First, let's construct the molecular orbital diagram 
of the C$_3$H$_6$ molecular itself. Instead of starting from the entire molecular, we first 
consider the MO diagram of the CH$_2$ fragments.  


\end{document}
