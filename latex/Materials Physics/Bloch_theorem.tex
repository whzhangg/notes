\documentclass{article}

\usepackage{amssymb, amsmath, amsthm}
\usepackage[margin=1in]{geometry}
\usepackage{verbatim}
\usepackage{graphicx}
\usepackage{hyperref}
\usepackage{docmute}

\newcommand{\bm}[1]{\mathbf{#1}}

\begin{document}

\part{Bloch Theorem}

\section{Bloch Theorm}
It is apparent that in the lattice, the Hamiltonian is periodic and commute with any 
translation of lattice\footnote{ J\"{u}rgen K\"{u}bler, \emph{Theory of Itinerant Electronic Magnetism} page 79}:
\begin{equation}
    [T_j,H]=0
\end{equation}
Where operator $T_j$ translate the state by $\bm{R_j}$. Therefore, any eigenstate of the Hamiltonian should be also the eigenstate of translation:
\begin{equation}
    T_j \psi_i(\bm{r})=\psi_i(\bm{r}+\bm{R_j})=\lambda_j \psi_i(\bm{r})
\end{equation}
This is satisfied if the eigenvalue $\lambda_j$ is:
\begin{equation}
    \lambda_j=e^{i\bm{k} \cdot \bm{R_j}}
\end{equation}
And the state is indicated by the number k.
Finally, Bloch Theorm states that eigenstate of the Hamiltonian in a periodic system need to satisfy such condition:
\begin{equation}
    \psi_{n\bm{k}}(\bm{r} + \bm{R_j})=e^{i\bm{k} \cdot \bm{R_j}} \psi_{n\bm{k}}(\bm{r})
\end{equation}
Function that satisfy such condition can be written as:
\begin{equation}
    \psi_{n\bm{k}}(r)=e^{i\bm{k} \cdot \bm{r}} u_{n\bm{k}}(\bm{r})
\end{equation}
where $u_{n\bm{k}}(\bm{r})$ has the periodicity of the lattice.
now $\bm{k}$ is restricted only in the unit cell of the reciprocal lattice 
if we define reciprocal lattice vector $\bm{K_s}$ by $\bm{K_s} \cdot \bm{R_j} = 2\pi n $: 
\begin{equation}
    \bm{k'}=\bm{k}+\bm{K_s}
\end{equation}
gives
\begin{equation}
    e^{i\bm{k} \cdot \bm{R_j}} = e^{i\bm{k'} \cdot \bm{R_j}} 
\end{equation}
Consequently, both wavefunctions $\psi_{n\bm{k}}(r)$ and $\psi_{n\bm{k'}}(r)$ prossess the same translation eigenvalue. 
The wavevector $\bm{k}$ and $\bm{k'}$ are said to be equivalent and we have:
\begin{align}
    \psi_{n\bm{k}}(\bm{r}) = \psi_{n\bm{k}+\bm{K_s}}(r)
    \varepsilon_{n\bm{k}}  = \varepsilon_{n\bm{k}+\bm{K_s}}
\end{align}
where eigenvalue $\varepsilon_{n\bm{k}}$ is for the Schr\"{o}dinger equation:
\begin{equation}
    H \psi_{n\bm{k}}(\bm{r}) = \varepsilon_{n\bm{k}} \psi_{n\bm{k}}(\bm{r})
\end{equation}


\end{document}
