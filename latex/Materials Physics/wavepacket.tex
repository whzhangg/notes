\documentclass{article}
\usepackage{amsmath}
\usepackage[margin=0.8in]{geometry}
\usepackage{verbatim}
\usepackage{graphicx}

\begin{document}

\title{Electronic wave packet}
\author{Wenhao}
\date{\today}
\maketitle

This short notes summarize some content in \emph{Ashcroft and Mermin} Chapter 2 and 12.
\section{Sommerfeld Theory}

Sommerfeld's theory improves the Drude theory of electron conduction by applying the fermi-dirac 
distribution for electrons in solids with an free electron approximation. The number of 
electrons per unit volume having velocity within $dv$ of $v$ is given by 
($v = \hbar k/m$, thus $dv = (\hbar/m)^3 dk$):
\begin{gather}
    n(k) = \frac{V}{4\pi^3} f(\varepsilon_k) dk \\
    n(v) = \frac{(m/\hbar)^3}{4\pi^3} f(\varepsilon_k) dv = f(v) dv
\end{gather}

However, this correspond to the result of free electron gas, where electrons have infinite
spatial extention. In solids with finite size and defects, the above free electron theory
will no longer apply, but in the limit that the electrons in solid approach free electron, 
we should obtain the same result as in the free electron case. Therefore, it implies that 
there are conditions within which we can apply the free electron approximation. 

If we are able to describe the momentum of the electron accurately (i.e. $k$ are still good index
for electronic states), then we expect the above semiclassical theory to apply. 
The uncertainty in momentum $\Delta p$ must then be small compared to $\hbar k$. Taking $k_F \sim 2/r_s$
and $r_s \sim 4 a_0$ for Na metal (Eq. 2.22 and Table 2.1),
we have:
\begin{equation}
    \Delta x \sim \frac{\hbar}{\Delta p} \gg \frac{1}{k_F} \sim 2a_0
\end{equation}

Thus, the free electron approximation is possible if the uncertainty in electron's position is much 
larger than the lattice parameter $a_0$ in simple metals. The requirement of uncertainty in position is ensured 
if our theory do not require describing electrons' positions, or electrons' positions are not 
constrained, for example, by some trapping potential.
\begin{enumerate}
    \item When we apply spatially varying potential energy from electromagnetic field, or a temperature gradient, we 
            need to specify the position of the electrons. But if the applied field are small enough and do not vary
            much on the scale $\sim a_0$, as is in most case, the requirement for electron's position are 
            less important and the above condition applies. However, if the wave length of the external field is small, such 
            as in the case of X-ray, the above condition fail and we need a full quantum description. 
    \item The semiclassical theory also assumes that electrons propagate freely between the scattering events, implying that 
            $\Delta x \ll l_{mfp}$. Therefore, the semiclassical theory would break down if mean free path $l \to a_0$.
\end{enumerate}

with the semiclassical theory, we can consider that we are describing the motion of a wave package, consist of 
free electron states around $k$:
\begin{equation}
    \varPsi(r,t) = \sum_{k'} g(k') e^{i(k'r - \frac{\hbar k'^2}{2m}t)} \label{free electron}
\end{equation}
where $g(k')$ is weight function and $g(k') \approx 0$ if $|k'-k| > \Delta k$. 
this wave package has uncertainty $\hbar \Delta k$ and $\Delta r$
for momentum and position satisfying the uncertainty principle ($\Delta r \Delta k \approx 1$), but both are small enough so that the 
position and velocity are relatively accurately described (as a semiclassical theory). 

\section{Wave packet of Bloch electron}
For Bloch electron, the same argument applies, except that the free electron wavefunction are 
replaced by Bloch function $\psi_{nk}$, the wave packet is written:
\begin{equation}
    \varPsi(r,t) = \sum_{k'} g(k') \psi_{nk} e^{-i\frac{\varepsilon_{nk}}{\hbar}t}
\end{equation}
The mean velocity is now given by the group velocity $v_{nk} = 1/\hbar \partial \varepsilon_{nk}/\partial k$.

The spatial extention of the Bloch wavepackage can be given by inspecting:
\begin{equation}
    \varPsi(r+R,t) = \sum_{k'} [ g(k') \psi_{nk} ] e^{i(k'R - \frac{\varepsilon_{nk}}{\hbar}t)}
\end{equation}
where we used the Bloch's theorem. This equation resemble Eq.\ref{free electron} with $r$ in Eq.\ref{free electron}
replaced by discrete lattice translation $R$. We have the relationship $\Delta R \Delta k \approx 1$. 
If $\Delta k$ is small compared to the Brillouin Zone size $\pi / a_0$, we need therefore $\Delta R \gg a_0$. The 
spatial dimension of the wave package of bloch electrons need to be over a few unit cell. The condition for 
the semiclassical theory follow the one for Sommerfeld case.


\end{document}