\documentclass{amsart}
%\usepackage{amssymb, amsmath, amsthm}
\usepackage[margin=1in]{geometry}
\usepackage{verbatim}
\usepackage{graphicx}
\usepackage{hyperref} % \url \href
\usepackage{docmute}

\newcommand{\pfrac}[2]{\frac{\partial #1}{\partial #2}}
\newcommand{\setM}{\mathcal{M}}
\newcommand{\bfx}{\mathbf{x}}
\newcommand{\bft}{\mathbf{t}}

\newtheorem*{theorem}{Theorem}
\newtheorem*{lemma}{Lemma}

\theoremstyle{remark}
\newtheorem*{remark}{remark}
\theoremstyle{remark}
\newtheorem*{example}{example}

\theoremstyle{definition}
\newtheorem*{definition}{Definition}

\DeclareMathOperator{\Aut}{Aut}
\DeclareMathOperator{\Image}{Im}
\DeclareMathOperator{\AO}{AO}
\DeclareMathOperator{\E}{E}
\DeclareMathOperator{\Sym}{Sym}
\DeclareMathOperator{\GL}{GL}
\DeclareMathOperator{\Hom}{Hom}
\DeclareMathOperator{\Tr}{trace}
\DeclareMathOperator{\Bij}{Bij}
\DeclareMathOperator{\Orb}{Orb}

% the highest level is part
% in each part, different section give different topics that are lossly connected
% subsection* should be used for giving subsequent definitions that are less important
\begin{document}

\part{Crystal Structure}
\section*{Lattice}
We consider a \emph{crystal structure} as the following object:
\begin{definition}
    [Crystal Pattern]
    The infinite, three-dimensional periodic array corresponding to a crystal is called \emph{crystal pattern}.
    The lengths of the periodicity may not be arbitrarily small. Crystal pattern is also known 
    as \emph{infinite ideal crystal}.
\end{definition}
We further distinguish two other concepts:
\begin{itemize}
    \item \emph{Macroscopic crystal} is a finite block of a crystal pattern,
    \item \emph{real crystal} has a finite size, but also defects.
\end{itemize}

\subsection*{Translation}
A shift which bring the crystal structure to superposition with itself is called 
a \emph{symmetry translation} or simply translation for this crystal structure. 
It is specified by a translation vector.

\begin{definition}
    [Vector lattice]
    The infinite set of all translation vectors $\bft_i$ of a crystal pattern is its \emph{vector lattice} $\mathbf{T}$, 
    or simply the \emph{lattice}. 
    The vectors are called \emph{lattice vectors}
\end{definition}
We distinguish vector lattice from the following two concepts:

\subsection*{Point lattice}
Choosing a starting point $\bfx_0$, the set of all the end points $\{\bfx_i\mid\bfx_i = \bfx_0 + \bft_i\}$
is call the \emph{point lattice} belongs to $\bfx_0$ and $\mathbf{T}$.
\subsection*{Particle lattice}
If the chosen point of a point lattice correspond to center of gravity of particles. Then we call this point lattice \emph{particle lattice}

The points that are related by translations are called \emph{translation equivalent}.

\vspace{10pt}
\section*{Basis and unit cell}

\subsection*{Basis}
To describe position in three-dimensional space, we need to choose an origin and a basis. 
A basis that consists of three lattice vectors of a crystal pattern is called a \emph{crystallograhic basis}
or a \emph{lattice basis} of the crystal structure. 

We further define the following terms:
\begin{itemize}
    \item \emph{Conventional basis} are the crystallograhic basis used in the International Tables A.
    \item \emph{Primitive basis} are the basis $a$, $b$ and $c$ with which every lattice vector $\bft$ can be expressed 
            as a linear combination with integral coefficients: $\bft = t_1 a + t_2 b + t_3 c$
\end{itemize}
A lattice is called \emph{primitive} if its conventional basis is primitive,  otherwise, we call it \emph{centred}. 

\subsection*{Unit cell}
A region in which all the points have coordinates \[0 \leq x,y,z < 1\]
with respect to its chosen basis and origin is called a \emph{unit cell} of the crystal structure.

\vspace{10pt}
\section*{Symmetry operations}

\begin{definition}
    [Symmetry operation]
    A symmetry operation is a mapping of an object such as 1) all distances remain unchanged and 2) the obejct is mapped onto itself or its mirror image.
    If the obejct is a crystal structure, the mapping is called a \emph{crystallographic symmetry operation}.
\end{definition}

\subsection*{Space group}
The set of all symmetry operations of a crystal structure is called the space group of the crystal structure.

\vspace{10pt}

\begin{definition}
    [Affine mapping]
    A mapping of space which maps parallel straight lines onto parallel straight lines is called an affine mapping. They do not necessary perserve 
    distance or angles.
\end{definition}

After choosing a coordination system, an affine mapping can always be represented as:
\begin{equation*}
    \left( \begin{matrix}
        a'\\b'\\c'
    \end{matrix} \right)  = \left( \begin{matrix}
        w_{11} & w_{12} & w_{13} \\
        w_{21} & w_{22} & w_{23} \\
        w_{31} & w_{32} & w_{33}
    \end{matrix} \right) \left( \begin{matrix}
        a\\b\\c
    \end{matrix} \right)+ \left( \begin{matrix}
        w_1\\w_2\\w_3
    \end{matrix} \right)
\end{equation*}
or simply, in matrix notation $\bfx' = W\bfx + w$. Affine mapping are Linear transformations.

\subsection*{Fixed point} A point $\bfx_F$ that is mapped onto itself is called a fixed point of the mapping.

\vspace{10pt}

\begin{definition}
    [Isometry]
    An affine mapping that leaves all distances and angles unchanged is called \emph{isometry}
\end{definition}
For an affine mapping to be an isometry, we require: 
\begin{enumerate}
    \item $\det W = \pm 1$ and 
    \item the matric tensor $G$ remain unchanged.
\end{enumerate}
where the matric tensor are a set of coefficients:
\begin{equation*}
    G = \left(\begin{matrix}
        a^2 & ab\cos\gamma & ac\cos\beta \\
        ab\cos\gamma & b^2 & bc\cos\alpha \\
        ac\cos\beta & bc\cos\alpha & c^2 
    \end{matrix}\right)
\end{equation*}
from the \emph{lattice parameter} $a$, $b$, $c$, $\alpha$, $\beta$ and $\gamma$

\vspace{10pt}

\subsection*{Types if isometries}
We can distinguish the following kinds of isometries in space:
\begin{enumerate}
    \item identity $\mathbf{I}$
    \item translation $\mathbf{T}$
    \item rotation and screw rotation $\mathbf{R}$
    \item inversion $\bar{\mathbf{I}}$
    \item rotoinversion $\overline{\mathbf{R}} = \bar{\mathbf{I}} \mathbf{R} = \mathbf{R} \bar{\mathbf{I}}$
    \item Reflection or glide reflections
\end{enumerate}
The symmetry operations satisfying \[\det W = -1\] are called the symmetry operation of the second kind.

\vspace{10pt}
\section*{Space groups and point groups of molecular}
\subsection*{Molecular symmetry}
The symmetry of a molecular (finite cluster of atoms) froms a group which is called the \emph{point group} $P_M$ of the molecule. $P_M$
can be infinite if the molecular is linear (infinite rotations). 

If we consider an ideal molecular to have translational symmetry in one direction, its symmetry group is called \emph{Rod group}, 
For translational symmetry in two dimension, it is called \emph{layer group}.

Two molecular point group belong to the same point group type if, after a change of basis, the transformation matrix of the two point group
coincide. 

\vspace{10pt}

\begin{definition}
    [Site symmetry]
    A point in a molecule has a definite site symmetry $S$ (site symmetry group) consists of all symmetry operations of the point group
    of the molecule that leave the point fixed. They correspond to stabilizer of the point in $G$.
\end{definition}

\subsection*{General position}
A set of symmetrically equivalent points $X$ of a molecular is in a \emph{general position} if the site symmetry $S$ of the point
consists of only identity. 
Otherwise, the points are in a \emph{special position}

For a point $X_s$ having site symmetry $S$, there exist $|P_M|/|S|$ symmetrically equivalent points. 
The length of the orbital is called its multiplicity.
The multiplicity of a point in the general position is the order of the molecular group $P_M$.

\begin{definition}
    [length of the orbital]
    For a finite group $G$ and stabilizer of the element $m$, then $L = |G|/|S|$ is the length of the orbital $G_m$
\end{definition}

\begin{definition}
    [Wyckoff position]
    Two orbitals of $P_M$ $O_1$ and $O_2$ belong to the same Wyckoff position if, after having selected two arbitrary points $X_1 \in O_1$ 
    and $X_2 \in O_2$, Their site symmetry $S_1$, $S_2$ are conguate in $P_M$: there exist a symmetry operation $g \in P_M$ so that:
    \[S_2 = g^{-1}S_1 g\]
\end{definition}

\vspace{10pt}
\section*{Space groups and point groups of crystal structure}

\begin{definition}
    [Point group of crystal structure]
    The point group of a crystal structure is the symmetry group of the bundle of vectors that is normal to the face of crystal.
\end{definition}
Here, the face refer to the surface of idealy grown crystal (crystal growth involves parallel advancement of crystal faces). 
The point group of crystal structure is therefore the symmetry of vectors, and the symmetry operation involve only 
the matrix part $W$ and is finite

\begin{lemma}
    Translation groiup $T$, the set of all translations of a space group $G$,sis a normal subgroup of $G$.
\end{lemma}
The coset decomposition of $G$ with respect to $T$ contain exactly those elements which has the same matrix part. 
Therefore, every matrix $W$ is characteristic for the corresponding cosets. 

\begin{lemma}
    The factor group $G/T$ is isomorphic to the point group $P$ of the crystal.
\end{lemma}

\vspace{10pt}
\subsection*{Site symmetry group}
Site symmetry group $S_x$ for space group $G$ has the same definition of that in molecular: it is 
a subgroup of $G$ that leave $x$ unchanged. 
If $g\in G$ is a site symmetry of $x$, then $g' = tgt^{-1}$ is a site symmetry of $tx$:
\begin{equation*}
    g'tx = tgx = tx
\end{equation*}

The number of symmetry equivalent points of a general position in a primitive cell is equal to the point 
group of the crystal structure $P$ because point group is isomorphic to the factor group $G/T$

Every site-symmetry group $S$ of a space group $G$ is isomorphic to a subgroup of the point group $P$ of $G$. 
For general position, $S$ is isomorphic to $P$, their multiplicity is given by $|P|/|S|$ or equivalently $|G/T|/|S|$.


\vspace{10pt}
\section*{Classification of space groups}
\begin{definition}
    [Crystal class]
    Two crystallograhic point groups $P_1$ and $P_2$ belong to the same point group type, if a basis can be found 
    so that the matrix part of $P_1$ and $P_2$ coincide. We say that they belong to the same \emph{Crystal class}
\end{definition}
There are 32 crystal class in space and 10 in plane.
Amoung the 32 crystal class, there are seven that belong to the point groups of the lattice. The seven point groups 
are called \emph{holohedries}. We can assign the points groups to holohedries as:
\begin{enumerate}
    \item The point group of the crystal class $P$ is a subgroup of that of the holohedry $H$,
    \item the index $H/P$ is as small as possible
\end{enumerate}
The seven holohedries assigned to the space groups are called the \emph{crystal systems} of space groups

\vspace{10pt}

\begin{definition}
    [Affine space group types]
    Two space groups $G_1$ and $G_2$ is said to belong to the same affine space group types if 
    all their matrix-column pairs $\{(W,w)\}$ coincide with an appropriately chosen basis vectors. 
\end{definition}
They belong to the same \emph{crystallograhic space group types} if 
all their matrix-column pairs $\{(W,w)\}$ coincide with an appropriately chosen right-handed coordination system.

There are 219 affine space group types and 230 space group types. 
We distinguish space groups and space group types. Therefore 230 different space group types, but there are 
infinite space groups in a space group types, one for each crystal structure.

\begin{definition}
    [Bravais type]
    Two point lattice belong to the same bravais type if their space groups belong to the same space-group type.
\end{definition}

\vspace{10pt}
\section*{Subgroups and Supergroups of point and space group}
\begin{definition}
    [Translationengleiche subgroup]
    $H$ is called an \emph{Translationengleiche subgroup} of $G$ if $H$ and $G$ have the same group of translation. 
    Therefore, $H$ belongs to a crystal class of lower symmetry than $G$: $P_H < P_G$
\end{definition}

\begin{definition}
    [Klassengleiche subgroup]
    $H$ is called an \emph{Klassengleiche subgroup} of $G$ if $H$ and $G$ belong to the same crystal class. 
    Therefore, $H$ has fewer translations than $G$: $T_H < T_G$
\end{definition}

\subsection*{General subgroup}
$H$ is called a general subgroup if $T_H < T_G$ and $P_H < P_G$

A klassengleiche subgroup is called an isomorphic subgroup if $G$ and $H$ belong to the same affine space-group type. 
For an isomorphic subgroup, we have the following:
\begin{enumerate}
    \item isomorphic subgroup $H$ in general belong to the same space group types with $G$, or its enantiomorphic partner (left-handed correspondence).
    \item The basis vectors are changed so that all symmetry elements are changed,
    \item Symmetry are lowered, since symmetry operations (translation) are lost, however, the two space groups $G$ and $H$ are isomorphic.
\end{enumerate}

\begin{theorem}
    [Theorem of Hermann]
    A maximal subgroup of a space group is either translationengleiche or klassengleiche.
\end{theorem}

\vspace{10pt}

\begin{definition}
    [Supergroup]
    If $H$ is a maximal translationengleiche, klassengleiche or isomorphic subgroup of $G$, then $G$
    is the minimum translationengleiche, klassengleiche or isomorphic supergroup of $H$.
\end{definition}


\newpage

\end{document}
