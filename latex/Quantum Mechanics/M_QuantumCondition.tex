\documentclass{article}

\usepackage{amssymb, amsmath, amsthm}
\usepackage[margin=1in]{geometry}
\usepackage{verbatim}
\usepackage{graphicx}
\usepackage{hyperref}
\usepackage{docmute}

\begin{document}


\section{Quantum condition}
In quantum mechanics, observables do not obey the commutative law of multiplication so
it is necessary for us to study the value of $\zeta\eta - \eta\zeta$ when $\zeta$ and $\eta$
are two observables. These new relationships are called \textbf{quantum conditions} or
\textbf{commutation relationships} 
\footnote{Principles of Quantum mechanics, Paul Dirac, 1958, Chapter IV}.

Suppose $p$ and $q$ are a set of canonical momentum and coordinates, we can define the classic
\textbf{Poisson bracket} of any two variables $u$ and $v$:
\begin{equation}
    \{ u,v \}_C = \sum_r \left\{ \frac{\partial u}{\partial q_r}\frac{\partial v}{\partial p_r} - \frac{\partial u}{\partial p_r}\frac{\partial v}{\partial q_r} \right\}
\end{equation}
Poisson brackets satisfy the following properties:
\begin{gather}
    \{u,v\}_C = - \{v,u\}_C \\
    \{u,c\}_C = 0 \\
    \begin{cases}
        \{ u_1 + u_2, v\}_C = \{ u_1, v\}_C + \{u_2,v\}_C \notag \\
        \{ u, v_1 + v_2\}_C = \{ u, v_1\}_C + \{u,v_2\}_C 
    \end{cases}
\end{gather}
\begin{align}
    \{u_1u_2,v\}_C 
    &= \sum_r \left\{ \left( \frac{\partial u_1}{\partial q_r} u_2 + u_1\frac{\partial u_2}{\partial q_r}\right)\frac{\partial v}{\partial p_r} \right. 
                \left.- \left( \frac{\partial u_1}{\partial p_r} u_2 + u_1\frac{\partial u_2}{\partial p_r}\right)\frac{\partial v}{\partial q_r} \right\} \notag \\
    &= \{u_1,v\}_C u_2 + u_1 \{ u_2,v \}_C \notag \\
    \{u,v_1v_2\}_C &= \{u,v_1\}_C v_2 + v_1 \{ u,v_2 \}_C
\end{align}
\begin{equation}
    \{u \{v,w\}_C\}_C + \{v \{w,u\}_C\}_C + \{w \{u,v\}_C\}_C = 0
\end{equation}

Now we introduce \textbf{Quantum Poisson bracket}, which we assume to satisfy all the above properties with the same order. 
These requirements sufficiently determine the form of the quantum poisson bracket. 
writting down:
\begin{align}
    \{u_1u_2,v_1v_2\}_Q &= \{u_1,v_1v_2\}_Qu_2 + u_1\{u_2,v_1v_2\}_Q \notag \\
                      &= \{u_1,v_1\}_Qv_2u_2 + v_1\{u_1,v_2\}_Qu_2 + u_1\{u_2,v_1\}_Qv_2 + u_1v_1\{ u_2,v_2 \}_Q \\
    \{u_1u_2,v_1v_2\}_Q &= \{u_1u_2,v_1\}_Qv_2 + v_1\{u_1u_2,v_2\}_Q \notag \\
                      &= \{u_1,v_1\}_Qu_2v_2 + u_1\{u_2,v_1\}_Qv_2 + v_1\{u_1,v_2\}_Qu_2 + v_1u_1\{ u_2,v_2 \}_Q        
\end{align} 
equating the above two results, we have the relationships:
\begin{equation}
    \{u_1,v_1\}_Q(u_2v_2 - v_2u_2) = (u_1v_1 - v_1u_1)\{u_2,v_2\}_Q
\end{equation}
since this equation should hold for $u_1,v_1$ independent of $u_2,v_2$, while 
the commutative property of multiplication is no longer true ($uv-vu\neq 0$) in general. we thus require:
\begin{align}
    u_2v_2 - v_2u_2 \equiv [u_2, v_2] = c\{u_2,v_2\}_Q \\
    u_1v_1 - v_1u_1 \equiv [u_1, v_1] = c\{u_1,v_1\}_Q
\end{align}
with $c$ some universial numerical factor independent of the observable $u$ and $v$. 

we want the poisson bracket of two observables to be real, but $uv - vu$ will be 
imaginary for two operators of real observable. Therefore, numerical factor $c$ must
be purly imaginary.  In order that the theory agree with experiment, we must take 
$c = i\hbar$, where $\hbar$ has the unit of action, giving the final result:
\begin{equation}
    [u,v] = i\hbar\{u,v\}_Q
\end{equation}
which gives the connection between quantum theory and classic theory.
We observe that the commutator $uv - vu$ is of the order of $\hbar$. Approximating $\hbar \to 0$
leads to $uv \approx vu$ and thus the classic limit.

We can check that the quantum poisson bracket of two variables given by their commutation relationship
satisfy all the properties of the classic poisson bracket. Therefore, we further assume that
\emph{the quantum poisson bracket has the same value as the classic poisson bracket} 
$\{u,v\}_Q = \{u,v\}_C$.

In the classical mechanics, the poisson bracket of canonical coordinates and momenta having the value:
\begin{gather}
    \{q_r,q_s\} = \{p_r,p_s\} = 0\\
    \{q_r,p_s\} = \delta_{rs}
\end{gather}
which gives the quantum commutation result:
\begin{gather}
    [q_r,q_s] = [p_r,p_s] = 0\\
    [q_r,p_s] = i\hbar\delta_{rs}
\end{gather}


\end{document}