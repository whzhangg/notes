\documentclass{article}

\usepackage{amssymb, amsmath, amsthm}
\usepackage[margin=1in]{geometry}
\usepackage{verbatim}
\usepackage{graphicx}
\usepackage{hyperref}
\usepackage{docmute}

\begin{document}


\section{Response functions}
\subsection{Linear response}
we wish to find the change of observable $B$ with a perturbation term in addition to the equilibrium Hamiltonian 
that is turned on at 
\begin{equation}
     H = H_0 - Af(t)
\end{equation}
The perturbation is turned on from $t_0$, 
In Schrodinger picture, the time dependence of a state is given by:
\begin{align}
    | \Psi_n(t) \rangle &= U(t,t_0) | \Psi_n(t_0) \rangle 
    %\notag \\ &= e^{-\frac{i}{\hbar}H_0(t-t_0)} U_I(t,t_0) | \Psi_n(t_0) \rangle \notag 
\end{align}
with
\begin{align}
    U(t,t_0) &= e^{-i\frac{H_0}{h}(t-t_0)} U_I(t,t_0) \notag \\
                 &= e^{-i\frac{H_0}{h}(t-t_0)} \exp\left[ -\frac{i}{\hbar} \int_{t_0}^{t} H_I'(t') dt'  \right] \notag \\
                 &\approx e^{-i\frac{H_0}{h}(t-t_0)} \left( 1 + \frac{i}{\hbar} \int_{t_0}^{t} A_I(t')f(t') dt' \right)
\end{align}
and $A_I(t) = \exp(iH_0t/\hbar) A \exp(-iH_0t/\hbar)$. We find the difference:
\begin{align}
    \langle B \rangle(t) - \langle B \rangle(t_0) 
    &= \sum_n \langle \Psi_n(t) | \rho_{0} B | \Psi_n(t) \rangle - \sum_n \langle \Psi_n(t_0) | \rho B | \Psi_n(t_0) \notag \\
    &\approx \sum_n \langle \Psi_n(t) | \rho_{0} B | \Psi_n(t) \rangle - \sum_n \langle \Psi_n(t_0) | \rho_{0} B | \Psi_n(t_0) \rangle
\end{align}
where $\rho_0 = e^{-\beta H_0} / Z$ and we use the adabatic approximation to assume that the probability of 
the states remain the same as in the unperturbed case. 
%The states evolve according to the Schrodinger equation:
%\begin{align}
%    U(t,-\infty) &= e^{-i\frac{H_0}{h}t} U_I(t,-\infty) \notag \\
%                 &= e^{-i\frac{H_0}{h}t} \exp\left[ -\frac{i}{\hbar} \int_{-\infty}^{t} H_I'(t') dt'  \right] \notag \\
%                 &\approx e^{-i\frac{H_0}{h}t} \left( 1 + \frac{i}{\hbar} \int_{-\infty}^{t} A(t')f(t') dt' \right)
%\end{align}
so the expectation value of $B$ at time t is given by:
\begin{align}
    \langle B \rangle(t) &=
    \sum_n \langle \Psi_n(t_0) | \left( 1 - \frac{i}{\hbar} \int_{t_0}^{t} A(t')f(t') dt' \right) e^{i\frac{H_0}{h}(t-t_0)} \rho_{0} B
    e^{-i\frac{H_0}{h}(t-t_0)} \left( 1 + \frac{i}{\hbar} \int_{t_0}^{t} A(t')f(t') dt' \right) | \Psi_n(t_0) \rangle \notag \\
    %&\approx \langle B \rangle_0  + \frac{i}{\hbar} \int_{t_0}^{t} dt' 
    %\left[ \sum_n \langle \Psi_n(t_0) | e^{i\frac{H_0}{h}(t-t_0)} \rho_{0} B e^{-i\frac{H_0}{h}(t-t_0)} A(t')f(t') - A(t')f(t') e^{i\frac{H_0}{h}(t-t_0)} \rho_{eq} B e^{-i\frac{H_0}{h}(t-t_0)} A(t')f(t') | \Psi_n(t_0) \rangle \right] \notag \\
    &\approx \langle B \rangle_0  + \frac{i}{\hbar} \int_{t_0}^{t} dt' 
    \text{Tr} \rho_{0} [B(t-t_0),A(t')] f(t')
\end{align}
So that the difference is given by:
\begin{align}
    \Delta B(t) &= \frac{i}{\hbar} \int_{t_0}^{t} dt' \text{Tr} \rho_{0} [B(t-t_0),A(t')] f(t') 
\end{align}
The response function is defined to be the response after a unit unit impulse at $t = 0$:
\begin{align}
    \phi_{BA} (t) &= \frac{i}{\hbar} \int_{t_0}^{t} dt' \text{Tr} \rho_{0} [B(t-t_0),A(t')] \delta(t'=0) \notag \\
            &= \frac{i}{\hbar} \text{Tr} \rho_{0} [B(t),A(0)] 
\end{align} 
\begin{align}
    \phi_{BA} (t) &=\frac{i}{\hbar} \langle [B(t),A] \rangle_0 = \frac{1}{i\hbar} \text{Tr} \rho_0 [A, B(t)]
                = \frac{1}{i\hbar} \text{Tr} [\rho_0, A] B(t)
\end{align}

Using the identity:
\begin{gather}
    [A, e^{-\beta H_0}] = e^{-\beta H_0} \int_0^{\beta} e^{\lambda H_0} [H_0, A] e^{-\lambda H_0} d\lambda 
    = e^{-\beta H_0} \int_0^{\beta} e^{\lambda H_0} (-i\hbar) \dot{A} e^{-\lambda H_0} d\lambda 
\end{gather}
\begin{align}
    [\rho_{0}, A] &=  i\hbar \rho_{0} \int_0^{\beta} e^{\lambda H_0} \dot{A} e^{-\lambda H_0} d\lambda \notag \\
                &=  i\hbar  \int_0^{\beta} \rho_{0} \dot{A}(-i\hbar\lambda) d\lambda
\end{align}
so that $\exp(-iH_0t/\hbar) \to \exp( -H_0\lambda ) $
and we can arrive at the formula given by Kubo:
\begin{align}
    \phi_{BA} (t) &= \int_0^{\beta} \text{Tr} \rho_0 \dot{A}(-i\hbar\lambda) B(t) d\lambda 
    = \int_0^{\beta} \langle \dot{A}(-i\hbar\lambda) B(t) \rangle_0 d\lambda 
\end{align}

We define the frequency components of the response function as
\footnote{this follows the definition of \emph{Kubo 1957} Eq.2.21, in terms
of the more conventional way, we have:
\begin{equation}
    \chi_{BA}(\omega) = \lim_{\eta\to 0^+} \int_{0}^{\infty} \phi_{BA} (t) e^{i(\omega+i\eta)\tau} dt
\end{equation}
}
:
\begin{equation}
    \chi_{BA}(\omega) = \lim_{\eta\to 0^+} \int_{0}^{\infty} \phi_{BA} (t) e^{-\eta t-i\omega t} dt
\end{equation}
We obtain the frequency response function:
\begin{equation}
    \chi_{BA}(\omega) = 
    \lim_{\eta\to 0^+} \int_0^{\beta} d\lambda \int_{0}^{\infty} dt e^{-\eta t-i\omega t} \langle \dot{A}(-i\hbar\lambda) B(t) \rangle_0 
\end{equation}

Let's also consider the case where a (constant pertrubation) $F$ is applied continuously from $t = -\infty$ to 
$t = 0$ and stops. The system then relax throught internal interaction. The observable will
follow:
\begin{align}
    \Delta B(t) &= \int_{-\infty}^{0} \phi_{BA} (t-t') dt' F \notag \\
                &= \int_{t}^{\infty} \phi_{BA} (t') dt' F \notag \\
                &= \Phi_{BA} (t) F
\end{align}
and
\begin{equation}
    \Phi_{BA} (t) = \lim_{\eta\to 0^+} \int_{t}^{\infty} \phi_{BA} (t') e^{-\eta t'} dt'
\end{equation}
is called the relaxation function.

\subsection{Linear response formula of electrical conductivity}
We consider an uniform external electric field (potential zero is arbitrary) ($V(x) = E(t)x $):
\begin{gather}
    H'(t) = -e \sum_i x_i E(t) = - A E(t) \\
    \dot{A} = e \sum_i \dot{x}_i = J
\end{gather}

where $x_i$ is the position operator of the $i^{th}$ particle and $e$ is the charge associated with that
particle. The current operator is 
defined to be:
\begin{equation}
    J_{\mu} = e \sum_i \dot{x}_i
\end{equation}
The response function is given by:
\begin{align}
    \phi_{\mu\nu} (t) &= \int_0^{\beta} \langle J_{\nu}(-i\hbar\lambda) J_{\mu}(t) \rangle_0 d\lambda \\
    \chi_{\mu\nu}(\omega)&= 
    \lim_{\eta\to 0^+} \int_0^{\beta} d\lambda \int_{0}^{\infty} e^{-\eta t-i\omega t} dt  \langle J_{\nu}(-i\hbar\lambda) J_{\mu}(t) \rangle_0
\end{align}
and the conductivity is given by:
\begin{equation}
    \sigma_{\mu\nu} = \frac{1}{V} \int_0^{\beta} d\lambda \int_{0}^{\infty} dt \langle J_{\nu}(-i\hbar\lambda) J_{\mu}(t) \rangle_0
\end{equation}

\subsection{Linear response formula of thermal conductivity}
Derivation of the expression of thermal conductivity is provided by \emph{Allen and Feldman, 1993}. 
The total current operator is given by:
\begin{equation}
    J_{\alpha} = \sum_{ij\beta\gamma} (R_{i\alpha} - R_{j\alpha}) \Phi_{ij}^{\beta\gamma} u_{i\beta} \dot{u}_{j\gamma}
\end{equation}
We consider the Hamiltonian of the system to be:
\begin{equation}
    H_0 = \int h(x)  d^3x
\end{equation}
where $h(x) = \sum_i h_i \delta(x - x_i)$ consists of the vibration energy of each atom $i$. 
The local current operator $S(x)$ is related to $h(x)$ by the continuity equation:
\begin{gather}
    \frac{\partial h(x)}{\partial t} + \nabla\cdot S(x) = 0 \\
    J = \int S(x) d^3x
\end{gather}
The density matrix can be written as:
\begin{equation}
    \rho = \frac{1}{Z} e^{- \int \beta(x) h(x) d^3x}
\end{equation}
and $\beta(x) \approx \beta[1-\delta T(x)/T]$ with $T$ the average temperature, then
\begin{equation}
    \rho = \frac{1}{Z} e^{- \int \beta[1-\delta T(x)/T] h(x) d^3x} = \frac{1}{Z} e^{- \beta (H_0 + H')}
\end{equation}
with $H'$:
\begin{align}
    H' &= -\frac{1}{T} \int \delta T(x)h(x) d^3x \notag \\
        & = \frac{1}{T} \int d^3x \int _{-\infty}^{0} dt \delta T(x) \nabla\cdot S(x,t) \notag \\
        & = - \left( \frac{1}{T} \int d^3x \int _{-\infty}^{0} dt S(x,t) \right) \nabla T \notag \\
        & = - \left( \frac{1}{T} \int _{-\infty}^{0} dt J(t) \right) \nabla T 
\end{align}

Using the relationship (\emph{Allen 1993}):
\begin{equation}
    e^{-\beta (H_0 + H')} \approx e^{-\beta H_0} \left[ 1 + \int_0^{\beta} d\lambda e^{\lambda H_0} H' e^{-\lambda H_0} \right]
        \approx e^{-\beta H_0} \left[ 1 - \frac{1}{T} \int_0^{\beta} d\lambda e^{\lambda H_0} \int _{-\infty}^{0} dt J(t) e^{-\lambda H_0} \right]
\end{equation} 
The expectation value of current $J_{\mu}$ is then:
\begin{equation}
    \text{Tr}\rho J_{\mu} = \text{Tr}\rho_0 J_{\mu} - \text{Tr}\rho_0\frac{1}{T}\int_0^{\beta} d\lambda \int _{-\infty}^{0} dt 
    e^{\lambda H_0} (J(t) \nabla T) e^{-\lambda H_0} J_{\mu}
\end{equation}
The first term is zero under equilibrium, the second term gives $ - V \kappa_{\mu\nu} \nabla T$ and 
\begin{align}
    \kappa_{\mu\nu} & = \text{Tr}\rho_0 \frac{1}{VT} \int_0^{\beta} d\lambda \int _{-\infty}^{0} dt 
    e^{\lambda H_0} J_{\nu}(t) e^{-\lambda H_0} J_{\mu} \notag \\
    & = \frac{1}{VT}\int_0^{\beta} d\lambda \int _{-\infty}^{0} dt \langle J_{\nu}(t-i\hbar\lambda) J_{\mu}(0) \rangle_0 \notag \\
    & = \frac{1}{VT}\int_0^{\beta} d\lambda \int _{0}^{\infty} dt \langle J_{\nu}(-i\hbar\lambda) J_{\mu}(t) \rangle_0 
\end{align}
which agrees with the Kubo's formula for electrical conductivity. 
\footnote{in the case of electrical conductivity of Kubo's derivation, 
the pertrubation is in the Hamiltonian, but for heat conductivity, pertrubation 
is in the density matrix, thus the process of derivation is different, i.e. The
Hamiltonian is time dependent because of $T(t)$. Specially,
we previously assumed $\rho = \rho_{0}$, which is no longer the approximation. 
If I just directly apply Kubo's form with $H'$, we have an negative sign of conductivity }
%where we take $\nabla T$ to be uniform. Now, we can apply the Kubo's formula:
%The final result take the same form as the Kubo's formula for electrical conductivity.
%\footnote{ I have not yet finish the derivation, since this case seems to be different from the above form,
%where the perturbation is in the Hamiltonian, using the above form of $H' = -Af$ leads to a minus sign, since $J = -\kappa \nabla T$, suggesting
%the form as used in electrical case may not apply here}
%\begin{align}
%    \kappa_{\mu\nu} (t) & = \frac{1}{VT} \int_0^{\beta} \langle J_{\nu}(-i\hbar\lambda) J_{\mu}(t) \rangle d\lambda \\
%    \kappa_{\mu\nu} (\omega) & = \frac{1}{VT} \int_0^{\beta} d\lambda \int_{0}^{\infty} e^{-i\omega t} dt \langle J_{\nu}(-i\hbar\lambda) J_{\mu}(t) \rangle 
%\end{align}

\end{document}