\documentclass{article}

\usepackage{amssymb, amsmath, amsthm}
\usepackage[margin=1in]{geometry}
\usepackage{verbatim}
\usepackage{graphicx}
\usepackage{hyperref}
\usepackage{docmute}

\newcommand{\pfrac}[2]{\frac{\partial #1}{\partial #2}}
\newcommand{\ddt}[1]{\frac{d}{dt}\left( #1 \right)}
\renewcommand{\L}{\mathcal{L}}
\renewcommand{\S}{\mathcal{S}}
\renewcommand{\H}{\mathcal{H}}
\newcommand{\dotx}{\dot{x}}
\newcommand{\dotq}{\dot{q}}
\newcommand{\dotp}{\dot{p}}
\newtheorem{theorem}{Theorem}
\newtheorem{example}{Example}


\begin{document}

\title{Quantum Mechanics}
\author{Wenhao Zhang}
\date{\today}
\maketitle

\tableofcontents
\newpage

\documentclass{article}

\usepackage{amssymb, amsmath, amsthm}
\usepackage[margin=1in]{geometry}
\usepackage{verbatim}
\usepackage{graphicx}
\usepackage{hyperref}
\usepackage{docmute}

\newcommand{\pfrac}[2]{\frac{\partial #1}{\partial #2}}
\newcommand{\ddt}[1]{\frac{d}{dt}\left( #1 \right)}
\renewcommand{\L}{\mathcal{L}}
\renewcommand{\S}{\mathcal{S}}
\renewcommand{\H}{\mathcal{H}}
\newcommand{\dotx}{\dot{x}}
\newcommand{\dotq}{\dot{q}}
\newcommand{\dotp}{\dot{p}}

\newtheorem{theorem}{Theorem}
\newtheorem{example}{Example}

\begin{document}


\section{Lagrange method}

\subsection{Equation of motion}
We define the Lagrangian\footnote{From David Morin's chapter in \emph{Introduction to Classical Mechanics}} from the kinetic energy $T$ and potential energy $V$ as:
\begin{equation}
    \L(x,\dot{x},t) = T - V 
\end{equation}
and action $\S$ with the unit of energy $\times$ time:
\begin{equation}
    \S = \int_{t_1}^{t_2} \L(x,\dot{x},t) dt
\end{equation}
with the initial condition $x(t_1) = x_0, \dot{x}(t_0) = \dot{x}_0$. 
The \textbf{equation of motion} can be determined by the 
\textbf{Principle of least action} which states that $\S$ will be 
a extrema (stationary point) for the dynamic of system from time
$t_1$ to time $t_2$. 

The equation of motion is obtained as:
\begin{align}
    \delta\S &= \int_{t_1}^{t_2} \L(x+\delta x,\dotx + \delta \dotx,t) dt - \int_{t_1}^{t_2} \L(x,\dot{x},t) dt \notag \\
            &= \int_{t_1}^{t_2} \left( \L + \frac{\delta \L}{\delta x} \delta x + \frac{\delta \L}{\delta \dotx} \delta \dotx \right) dt - \int_{t_1}^{t_2} \L(x,\dot{x},t) dt \notag \\
            &= \int_{t_1}^{t_2}\left( \frac{\delta \L}{\delta x} \delta x + \frac{\delta \L}{\delta \dotx} \delta \dotx \right) dt \notag \\
            &= \int_{t_1}^{t_2}\left( \frac{\delta \L}{\delta x} \delta x + \frac{\delta \L}{\delta \dotx} \delta \dotx \right) dt \notag \\
            &= \left. \frac{\delta \L}{\delta \dotx} \delta x \right|_1^2
            + \int_{t_1}^{t_2}\left( \frac{\delta \L}{\delta x} \delta x - \frac{d}{dt}\left( \frac{\delta \L}{\delta \dotx} \right) \delta x \right) dt \label{eom} \\
\end{align}
where we used integral by parts to obtain the final equation. the boundary condition for choosing $\delta x$ is that they are $0$ at the initial coordinate and 
final coordinate, therefore, $ (\delta \L / \delta \dotx) \delta x |_1^2 = 0$. $\delta x$ at other time can be choosen arbitrary. The requirement that
$\S = 0$ thus lead to:
\begin{equation}
    \frac{\delta \L}{\delta x} - \frac{d}{dt}\left( \frac{\delta \L}{\delta \dotx} \right) = 0
\end{equation}
In terms of multiply coordinates, we have:
\begin{equation}
    \frac{\partial \L}{\partial x_i} - \frac{d}{dt}\left( \frac{\partial \L}{\partial \dotx_i} \right) = 0
\end{equation}
since the deviation of $\L$ is to first order in each coordiante.

\subsection{Change of coordinates}
We consider changing coordinate ${x_i}$ to ${q_i}$ as:
\begin{equation}
    q_i = \mathbf{q_i} (x_1, x_2 , \cdots, x_N, t)
\end{equation}
which does not depend on $\dotx$. Using the relationship:
\begin{gather}
    \dotx_i = \sum_{i=1}^N \pfrac{x_i}{q_m} \dotq_m + \pfrac{x_i}{t} \label{xq_relation}\\
    \pfrac{\dotx_i}{\dotq_m} = \pfrac{x_i}{q_m}
\end{gather}
The equation of motion is
given by:
\begin{align}
    \frac{d}{dt}\left( \frac{\partial \L}{\partial \dotq_m} \right)
    &=  \frac{d}{dt}\left( \sum_{i=1}^N \pfrac{\L}{\dotx_i} \pfrac{\dotx_i}{\dotq_m} \right) \notag \\
    &=  \sum_{i=1}^N \left[ \ddt{\pfrac{\L}{\dotx_i}} \pfrac{x_i}{q_m} + \pfrac{\L}{\dotx_i}\ddt{\pfrac{x_i}{q_m}} \right] \notag \\
    &=  \sum_{i=1}^N \left[ \pfrac{\L}{x_i} \pfrac{x_i}{q_m} + \pfrac{\L}{\dotx_i} \pfrac{\dotx_i}{q_m} \right] \notag \\
    &=  \sum_{i=1}^N \left[ \pfrac{\L}{x_i} \pfrac{x_i}{q_m} \right] = \pfrac{\L}{q_m}
\end{align}
where in the final step, $\partial \dotx_i / \partial q_m = 0$ is from Eq.\ref{xq_relation}. We can see that the 
equation of motion still hold after the change of coordinate
\footnote{
we show that $\ddt{\pfrac{x_i}{q_m}} = \pfrac{\dotx_i}{q_m}$ is true:
\begin{align}
    \ddt{\pfrac{x_i}{q_m}} 
    &= \sum_{k=i}^N \pfrac{\ }{q_k} \left( \pfrac{x_i}{q_m} \right) \dotq_k + \pfrac{}{t} \left( \pfrac{x_i}{q_m} \right) \notag \\
    &= \sum_{k=i}^N \pfrac{\ }{q_k} \left( \pfrac{x_i}{q_m} \right) \dotq_k + \pfrac{}{q_m} \left( \pfrac{x_i}{t} \right) \notag \\
    &= \sum_{k=i}^N \pfrac{\ }{q_m} \left( \pfrac{x_i}{q_k} \right) \dotq_k + \pfrac{}{q_m} \left( \pfrac{x_i}{t} \right) \notag \\
    &= \pfrac{\ }{q_m} \left[ \sum_{k=i}^N  \left( \pfrac{x_i}{q_k} \right) \dotq_k + \left( \pfrac{x_i}{t} \right) \right] = \pfrac{\dotx_i}{q_m}\notag \\
\end{align}
}

\subsection{Conservation law}
If $\L$ does not explicitly depend on coordinate $q_k$, then
\begin{equation}
    \ddt{\pfrac{ \L}{\dotq_k}} = \pfrac{\L}{q_k} = 0
\end{equation}
thus, $\partial \L / \partial \dotq_k $ is a constant of motion.

Now, we define a quantity $E$ as:
\begin{equation}
    E = \sum_{i=1}^N \pfrac{\L}{\dotq_i} \dotq_i - \L
\end{equation}
we can show that:
\begin{align}
    \frac{dE}{dt} &= \sum_{i=1}^N \ddt{\pfrac{\L}{\dotq_i} \dotq_i} - \frac{d\L}{dt} \notag \\
    &= \sum_{i=1}^N \left[ \ddt{\pfrac{\L}{\dotq_i}}\dotq_i + \pfrac{\L}{\dotq_i}\ddot{q}_i \right]
    - \left[ \sum_{i=1}^N\left( \pfrac{\L}{q_i}\dotq_i + \pfrac{\L}{\dotq_i}\ddot{q}_i \right) + \pfrac{\L}{t}\right] \notag \\
    &= - \pfrac{\L}{t}
\end{align}
so that if $\L$ does not explicitly depend on time $t$ as $(\partial \L / \partial t = 0)$, than $E$ is a constant of motion.

\subsection{$E$ and total energy}
The quantity $E$ is a constant of motion if $\L$ does not explicitly depend on time, but $E$ is not 
necessarily the total energy of the system.

\begin{theorem}
A necessary and sufficient condition for $E$ to the the 
total energy of a system whose $\L$ is written in terms of a set of coordinates
$q_i$ is that these $q_i$ are related to a cartesian set of coordinates $x_i$ by:
\begin{equation}
    x_i = \mathbf{x}_i (q_1, q_2, \cdots, q_N)
\end{equation}
which does not include $t$ or $\dotq$ dependence.
\end{theorem}

We use three example as argument to see this is indeed correct:

\begin{example}
consider a particle in a horizontal plane connected to the origin by a spring. the potential 
energy is $V = k(x^2+y^2) / 2$ and the kinetic energy is $T = m(\dotx^2 + \dot{y}^2)/2$. 
\begin{align}
    \L &= m(\dotx^2 + \dot{y}^2)/2 - k(x^2+y^2) / 2 \notag \\
     &= m(\dot{r}^2 + r^2\dot{\theta}^2) - kr^2/2 \notag
\end{align}
with $x = r\cos\theta,\ y = r\sin\theta$. 
In the coordinate $(r,\theta)$, we have $E = m(\dot{r}^2 + r^2\dot{\theta}^2) + kr^2/2$ is the total energy.
\end{example}

\begin{example}
consider similar to the above case, but now we have the coordinate transformation 
$x = r\cos(\omega t),\ y = r\sin(\omega t)$ depend on time $t$. The quantity
$E$ is still conserved but is no longer the total energy:
\begin{align}
    \L = m(\dot{r}^2 + r^2\omega^2) - kr^2/2 \notag \\
    E = m(\dot{r}^2 - r^2\omega^2) + kr^2/2 \notag
\end{align}
\end{example}

\begin{example}
A particle is fixed on a rod accelerating in $y$ direction: $y = at^2 / 2$. The 
Lagrangian and $E$ is:
\begin{equation}
    \L = m(\dotx^2 + (at)^2 ) - (mg)at^2 / 2 \notag \\
    E = m(\dotx^2 - (at)^2 ) + (mg)at^2 / 2 \notag 
\end{equation}
Still, $E$ is conserved but $E$ is not total energy since coordinate $y$ depend on 
time $t$: $(x = x;\ y = a t)$
\end{example}

We note that the for the later two examples, we have an accelerating frame of reference (rotating, accelerating) and 
$E$ is not the total energy of the system.


\subsection{Noether's theorem}
Noether's theorem states that if $\L$ is invariant under transformation 
$q_i \to q_i + \varepsilon K_i(q) $ ($\L$ does not explicitly depend on $\varepsilon$), 
then some quantity will be conserved.

\begin{align}
    \pfrac{\L}{\varepsilon} 
    &= \sum_{i=1}^{N} \left[ \pfrac{\L}{q_i}\pfrac{q_i}{\varepsilon} + \pfrac{\L}{\dotq_i}\pfrac{\dotq_i}{\varepsilon} \right] \notag \\
    &= \sum_{i=1}^{N} \left[ \pfrac{\L}{q_i} K_i(q) + \pfrac{\L}{\dotq_i} \dot{K}_i(q) \right] \notag \\
    &= \sum_{i=1}^{N} \left[ \pfrac{\L}{\dotq_i} \dot{K}_i(q) + \ddt{\pfrac{\L}{\dotq_i}}K_i(q) \right] \notag \\
    &= \sum_{i=1}^{N} \left[ \pfrac{\L}{\dotq_i} \dot{K}_i(q) + \ddt{\pfrac{\L}{\dotq_i}K_i(q)} - \pfrac{\L}{\dotq_i}\dot{K}_i(q)  \right] \notag \\
    &= \ddt{ \sum_{i=1}^{N} \pfrac{\L}{\dotq_i}K_i(q) }
\end{align}
So that if $\partial \L / \partial \varepsilon = 0$, the quantity $\sum_{i=1}^{N} (\partial \L / \partial \dotq_i) K_i(q) $ is constant of motion.

As an example, suppose for $\L$ depend on coordinates $x$ that is invariant under translation $x \to x + \varepsilon$,
then the quantity $\partial \L / \partial \dotx$ is conserved, which is momentum if $\L = m\dotx^2 / 2 - V(x)$


\end{document}

\documentclass{article}

\usepackage{amssymb, amsmath, amsthm}
\usepackage[margin=1in]{geometry}
\usepackage{verbatim}
\usepackage{graphicx}
\usepackage{hyperref}
\usepackage{docmute}

\newcommand{\pfrac}[2]{\frac{\partial #1}{\partial #2}}
\newcommand{\ddt}[1]{\frac{d}{dt}\left( #1 \right)}
\renewcommand{\L}{\mathcal{L}}
\renewcommand{\S}{\mathcal{S}}
\renewcommand{\H}{\mathcal{H}}
\newcommand{\dotx}{\dot{x}}
\newcommand{\dotq}{\dot{q}}
\newcommand{\dotp}{\dot{p}}

\newtheorem{theorem}{Theorem}
\newtheorem{example}{Example}

\begin{document}

\section{Hamiltanion method}
\subsection{Hamiltonian}
The quantity $E$ is given:
\begin{equation}
    E(q,\dotq) = \sum_{i=1}^N \pfrac{\L(q,\dotq)}{\dotq_i} \dotq_i - \L(q,\dotq) 
\end{equation}
Define the conjugate momentum and Hamiltonian:
\begin{gather}
    p_i = \pfrac{\L(q,\dotq)}{\dotq_i} \\
    \H = \sum_{i=1}^N p_i \dotq_i - \L(q,\dotq)
\end{gather}
where $\dotq$ are implicit function of $(q,p)$
we can find:
\begin{equation}
    \begin{cases}
        \pfrac{\H}{p_i} = \dotq_i - \pfrac{\L}{p_i} = \dotq_i \notag \\
        \pfrac{\H}{q_i} = 0 - \pfrac{\L}{q_i} = - \dotp_i \notag \\
    \end{cases}
\end{equation}
and thus the Hamiltanion equation:
\begin{align}
        \dotq_i &= \pfrac{\H}{p_i} \\
        \dotp_i &= - \pfrac{\H}{q_i}
\end{align}

The cyclic coordinates are still the ones which $\H$ does not explicitly depend on.

\subsection{Legendre transformation}
Legendre transformation states that if $Z(x) = Y(X) - xX$, then we should have the 
relationship:
\begin{equation}
    \pfrac{Z}{x} = -X
\end{equation}
Using the relationship for Lagrangian and Hamiltanion, we can identify, for one specific coordinate i:
\begin{gather}
    \H(p_i) = p_i \dotq_i - \L(\dotq_i) \notag \\
    Z(x) = -H(p_i) \notag \\
    Y(X) = L(\dotq_i) \notag \\
    \text{leading to}\ \ \pfrac{H(p_i)}{p_i} = \dotq_i
\end{gather}


\end{document}

\documentclass{article}

\usepackage{amssymb, amsmath, amsthm}
\usepackage[margin=1in]{geometry}
\usepackage{verbatim}
\usepackage{graphicx}
\usepackage{hyperref}
\usepackage{docmute}

\begin{document}


\section{Quantum condition}
In quantum mechanics, observables do not obey the commutative law of multiplication so
it is necessary for us to study the value of $\zeta\eta - \eta\zeta$ when $\zeta$ and $\eta$
are two observables. These new relationships are called \textbf{quantum conditions} or
\textbf{commutation relationships} 
\footnote{Principles of Quantum mechanics, Paul Dirac, 1958, Chapter IV}.

Suppose $p$ and $q$ are a set of canonical momentum and coordinates, we can define the classic
\textbf{Poisson bracket} of any two variables $u$ and $v$:
\begin{equation}
    \{ u,v \}_C = \sum_r \left\{ \frac{\partial u}{\partial q_r}\frac{\partial v}{\partial p_r} - \frac{\partial u}{\partial p_r}\frac{\partial v}{\partial q_r} \right\}
\end{equation}
Poisson brackets satisfy the following properties:
\begin{gather}
    \{u,v\}_C = - \{v,u\}_C \\
    \{u,c\}_C = 0 \\
    \begin{cases}
        \{ u_1 + u_2, v\}_C = \{ u_1, v\}_C + \{u_2,v\}_C \notag \\
        \{ u, v_1 + v_2\}_C = \{ u, v_1\}_C + \{u,v_2\}_C 
    \end{cases}
\end{gather}
\begin{align}
    \{u_1u_2,v\}_C 
    &= \sum_r \left\{ \left( \frac{\partial u_1}{\partial q_r} u_2 + u_1\frac{\partial u_2}{\partial q_r}\right)\frac{\partial v}{\partial p_r} \right. 
                \left.- \left( \frac{\partial u_1}{\partial p_r} u_2 + u_1\frac{\partial u_2}{\partial p_r}\right)\frac{\partial v}{\partial q_r} \right\} \notag \\
    &= \{u_1,v\}_C u_2 + u_1 \{ u_2,v \}_C \notag \\
    \{u,v_1v_2\}_C &= \{u,v_1\}_C v_2 + v_1 \{ u,v_2 \}_C
\end{align}
\begin{equation}
    \{u \{v,w\}_C\}_C + \{v \{w,u\}_C\}_C + \{w \{u,v\}_C\}_C = 0
\end{equation}

Now we introduce \textbf{Quantum Poisson bracket}, which we assume to satisfy all the above properties with the same order. 
These requirements sufficiently determine the form of the quantum poisson bracket. 
writting down:
\begin{align}
    \{u_1u_2,v_1v_2\}_Q &= \{u_1,v_1v_2\}_Qu_2 + u_1\{u_2,v_1v_2\}_Q \notag \\
                      &= \{u_1,v_1\}_Qv_2u_2 + v_1\{u_1,v_2\}_Qu_2 + u_1\{u_2,v_1\}_Qv_2 + u_1v_1\{ u_2,v_2 \}_Q \\
    \{u_1u_2,v_1v_2\}_Q &= \{u_1u_2,v_1\}_Qv_2 + v_1\{u_1u_2,v_2\}_Q \notag \\
                      &= \{u_1,v_1\}_Qu_2v_2 + u_1\{u_2,v_1\}_Qv_2 + v_1\{u_1,v_2\}_Qu_2 + v_1u_1\{ u_2,v_2 \}_Q        
\end{align} 
equating the above two results, we have the relationships:
\begin{equation}
    \{u_1,v_1\}_Q(u_2v_2 - v_2u_2) = (u_1v_1 - v_1u_1)\{u_2,v_2\}_Q
\end{equation}
since this equation should hold for $u_1,v_1$ independent of $u_2,v_2$, while 
the commutative property of multiplication is no longer true ($uv-vu\neq 0$) in general. we thus require:
\begin{align}
    u_2v_2 - v_2u_2 \equiv [u_2, v_2] = c\{u_2,v_2\}_Q \\
    u_1v_1 - v_1u_1 \equiv [u_1, v_1] = c\{u_1,v_1\}_Q
\end{align}
with $c$ some universial numerical factor independent of the observable $u$ and $v$. 

we want the poisson bracket of two observables to be real, but $uv - vu$ will be 
imaginary for two operators of real observable. Therefore, numerical factor $c$ must
be purly imaginary.  In order that the theory agree with experiment, we must take 
$c = i\hbar$, where $\hbar$ has the unit of action, giving the final result:
\begin{equation}
    [u,v] = i\hbar\{u,v\}_Q
\end{equation}
which gives the connection between quantum theory and classic theory.
We observe that the commutator $uv - vu$ is of the order of $\hbar$. Approximating $\hbar \to 0$
leads to $uv \approx vu$ and thus the classic limit.

We can check that the quantum poisson bracket of two variables given by their commutation relationship
satisfy all the properties of the classic poisson bracket. Therefore, we further assume that
\emph{the quantum poisson bracket has the same value as the classic poisson bracket} 
$\{u,v\}_Q = \{u,v\}_C$.

In the classical mechanics, the poisson bracket of canonical coordinates and momenta having the value:
\begin{gather}
    \{q_r,q_s\} = \{p_r,p_s\} = 0\\
    \{q_r,p_s\} = \delta_{rs}
\end{gather}
which gives the quantum commutation result:
\begin{gather}
    [q_r,q_s] = [p_r,p_s] = 0\\
    [q_r,p_s] = i\hbar\delta_{rs}
\end{gather}


\end{document}
\documentclass{article}

\usepackage{amssymb, amsmath, amsthm}
\usepackage[margin=1in]{geometry}
\usepackage{verbatim}
\usepackage{graphicx}
\usepackage{hyperref}
\usepackage{docmute}

\begin{document}


\section{Response functions}
\subsection{Linear response}
we wish to find the change of observable $B$ with a perturbation term in addition to the equilibrium Hamiltonian 
that is turned on at 
\begin{equation}
     H = H_0 - Af(t)
\end{equation}
The perturbation is turned on from $t_0$, 
In Schrodinger picture, the time dependence of a state is given by:
\begin{align}
    | \Psi_n(t) \rangle &= U(t,t_0) | \Psi_n(t_0) \rangle 
    %\notag \\ &= e^{-\frac{i}{\hbar}H_0(t-t_0)} U_I(t,t_0) | \Psi_n(t_0) \rangle \notag 
\end{align}
with
\begin{align}
    U(t,t_0) &= e^{-i\frac{H_0}{h}(t-t_0)} U_I(t,t_0) \notag \\
                 &= e^{-i\frac{H_0}{h}(t-t_0)} \exp\left[ -\frac{i}{\hbar} \int_{t_0}^{t} H_I'(t') dt'  \right] \notag \\
                 &\approx e^{-i\frac{H_0}{h}(t-t_0)} \left( 1 + \frac{i}{\hbar} \int_{t_0}^{t} A_I(t')f(t') dt' \right)
\end{align}
and $A_I(t) = \exp(iH_0t/\hbar) A \exp(-iH_0t/\hbar)$. We find the difference:
\begin{align}
    \langle B \rangle(t) - \langle B \rangle(t_0) 
    &= \sum_n \langle \Psi_n(t) | \rho_{0} B | \Psi_n(t) \rangle - \sum_n \langle \Psi_n(t_0) | \rho B | \Psi_n(t_0) \notag \\
    &\approx \sum_n \langle \Psi_n(t) | \rho_{0} B | \Psi_n(t) \rangle - \sum_n \langle \Psi_n(t_0) | \rho_{0} B | \Psi_n(t_0) \rangle
\end{align}
where $\rho_0 = e^{-\beta H_0} / Z$ and we use the adabatic approximation to assume that the probability of 
the states remain the same as in the unperturbed case. 
%The states evolve according to the Schrodinger equation:
%\begin{align}
%    U(t,-\infty) &= e^{-i\frac{H_0}{h}t} U_I(t,-\infty) \notag \\
%                 &= e^{-i\frac{H_0}{h}t} \exp\left[ -\frac{i}{\hbar} \int_{-\infty}^{t} H_I'(t') dt'  \right] \notag \\
%                 &\approx e^{-i\frac{H_0}{h}t} \left( 1 + \frac{i}{\hbar} \int_{-\infty}^{t} A(t')f(t') dt' \right)
%\end{align}
so the expectation value of $B$ at time t is given by:
\begin{align}
    \langle B \rangle(t) &=
    \sum_n \langle \Psi_n(t_0) | \left( 1 - \frac{i}{\hbar} \int_{t_0}^{t} A(t')f(t') dt' \right) e^{i\frac{H_0}{h}(t-t_0)} \rho_{0} B
    e^{-i\frac{H_0}{h}(t-t_0)} \left( 1 + \frac{i}{\hbar} \int_{t_0}^{t} A(t')f(t') dt' \right) | \Psi_n(t_0) \rangle \notag \\
    %&\approx \langle B \rangle_0  + \frac{i}{\hbar} \int_{t_0}^{t} dt' 
    %\left[ \sum_n \langle \Psi_n(t_0) | e^{i\frac{H_0}{h}(t-t_0)} \rho_{0} B e^{-i\frac{H_0}{h}(t-t_0)} A(t')f(t') - A(t')f(t') e^{i\frac{H_0}{h}(t-t_0)} \rho_{eq} B e^{-i\frac{H_0}{h}(t-t_0)} A(t')f(t') | \Psi_n(t_0) \rangle \right] \notag \\
    &\approx \langle B \rangle_0  + \frac{i}{\hbar} \int_{t_0}^{t} dt' 
    \text{Tr} \rho_{0} [B(t-t_0),A(t')] f(t')
\end{align}
So that the difference is given by:
\begin{align}
    \Delta B(t) &= \frac{i}{\hbar} \int_{t_0}^{t} dt' \text{Tr} \rho_{0} [B(t-t_0),A(t')] f(t') 
\end{align}
The response function is defined to be the response after a unit unit impulse at $t = 0$:
\begin{align}
    \phi_{BA} (t) &= \frac{i}{\hbar} \int_{t_0}^{t} dt' \text{Tr} \rho_{0} [B(t-t_0),A(t')] \delta(t'=0) \notag \\
            &= \frac{i}{\hbar} \text{Tr} \rho_{0} [B(t),A(0)] 
\end{align} 
\begin{align}
    \phi_{BA} (t) &=\frac{i}{\hbar} \langle [B(t),A] \rangle_0 = \frac{1}{i\hbar} \text{Tr} \rho_0 [A, B(t)]
                = \frac{1}{i\hbar} \text{Tr} [\rho_0, A] B(t)
\end{align}

Using the identity:
\begin{gather}
    [A, e^{-\beta H_0}] = e^{-\beta H_0} \int_0^{\beta} e^{\lambda H_0} [H_0, A] e^{-\lambda H_0} d\lambda 
    = e^{-\beta H_0} \int_0^{\beta} e^{\lambda H_0} (-i\hbar) \dot{A} e^{-\lambda H_0} d\lambda 
\end{gather}
\begin{align}
    [\rho_{0}, A] &=  i\hbar \rho_{0} \int_0^{\beta} e^{\lambda H_0} \dot{A} e^{-\lambda H_0} d\lambda \notag \\
                &=  i\hbar  \int_0^{\beta} \rho_{0} \dot{A}(-i\hbar\lambda) d\lambda
\end{align}
so that $\exp(-iH_0t/\hbar) \to \exp( -H_0\lambda ) $
and we can arrive at the formula given by Kubo:
\begin{align}
    \phi_{BA} (t) &= \int_0^{\beta} \text{Tr} \rho_0 \dot{A}(-i\hbar\lambda) B(t) d\lambda 
    = \int_0^{\beta} \langle \dot{A}(-i\hbar\lambda) B(t) \rangle_0 d\lambda 
\end{align}

We define the frequency components of the response function as
\footnote{this follows the definition of \emph{Kubo 1957} Eq.2.21, in terms
of the more conventional way, we have:
\begin{equation}
    \chi_{BA}(\omega) = \lim_{\eta\to 0^+} \int_{0}^{\infty} \phi_{BA} (t) e^{i(\omega+i\eta)\tau} dt
\end{equation}
}
:
\begin{equation}
    \chi_{BA}(\omega) = \lim_{\eta\to 0^+} \int_{0}^{\infty} \phi_{BA} (t) e^{-\eta t-i\omega t} dt
\end{equation}
We obtain the frequency response function:
\begin{equation}
    \chi_{BA}(\omega) = 
    \lim_{\eta\to 0^+} \int_0^{\beta} d\lambda \int_{0}^{\infty} dt e^{-\eta t-i\omega t} \langle \dot{A}(-i\hbar\lambda) B(t) \rangle_0 
\end{equation}

Let's also consider the case where a (constant pertrubation) $F$ is applied continuously from $t = -\infty$ to 
$t = 0$ and stops. The system then relax throught internal interaction. The observable will
follow:
\begin{align}
    \Delta B(t) &= \int_{-\infty}^{0} \phi_{BA} (t-t') dt' F \notag \\
                &= \int_{t}^{\infty} \phi_{BA} (t') dt' F \notag \\
                &= \Phi_{BA} (t) F
\end{align}
and
\begin{equation}
    \Phi_{BA} (t) = \lim_{\eta\to 0^+} \int_{t}^{\infty} \phi_{BA} (t') e^{-\eta t'} dt'
\end{equation}
is called the relaxation function.

\subsection{Linear response formula of electrical conductivity}
We consider an uniform external electric field (potential zero is arbitrary) ($V(x) = E(t)x $):
\begin{gather}
    H'(t) = -e \sum_i x_i E(t) = - A E(t) \\
    \dot{A} = e \sum_i \dot{x}_i = J
\end{gather}

where $x_i$ is the position operator of the $i^{th}$ particle and $e$ is the charge associated with that
particle. The current operator is 
defined to be:
\begin{equation}
    J_{\mu} = e \sum_i \dot{x}_i
\end{equation}
The response function is given by:
\begin{align}
    \phi_{\mu\nu} (t) &= \int_0^{\beta} \langle J_{\nu}(-i\hbar\lambda) J_{\mu}(t) \rangle_0 d\lambda \\
    \chi_{\mu\nu}(\omega)&= 
    \lim_{\eta\to 0^+} \int_0^{\beta} d\lambda \int_{0}^{\infty} e^{-\eta t-i\omega t} dt  \langle J_{\nu}(-i\hbar\lambda) J_{\mu}(t) \rangle_0
\end{align}
and the conductivity is given by:
\begin{equation}
    \sigma_{\mu\nu} = \frac{1}{V} \int_0^{\beta} d\lambda \int_{0}^{\infty} dt \langle J_{\nu}(-i\hbar\lambda) J_{\mu}(t) \rangle_0
\end{equation}

\subsection{Linear response formula of thermal conductivity}
Derivation of the expression of thermal conductivity is provided by \emph{Allen and Feldman, 1993}. 
The total current operator is given by:
\begin{equation}
    J_{\alpha} = \sum_{ij\beta\gamma} (R_{i\alpha} - R_{j\alpha}) \Phi_{ij}^{\beta\gamma} u_{i\beta} \dot{u}_{j\gamma}
\end{equation}
We consider the Hamiltonian of the system to be:
\begin{equation}
    H_0 = \int h(x)  d^3x
\end{equation}
where $h(x) = \sum_i h_i \delta(x - x_i)$ consists of the vibration energy of each atom $i$. 
The local current operator $S(x)$ is related to $h(x)$ by the continuity equation:
\begin{gather}
    \frac{\partial h(x)}{\partial t} + \nabla\cdot S(x) = 0 \\
    J = \int S(x) d^3x
\end{gather}
The density matrix can be written as:
\begin{equation}
    \rho = \frac{1}{Z} e^{- \int \beta(x) h(x) d^3x}
\end{equation}
and $\beta(x) \approx \beta[1-\delta T(x)/T]$ with $T$ the average temperature, then
\begin{equation}
    \rho = \frac{1}{Z} e^{- \int \beta[1-\delta T(x)/T] h(x) d^3x} = \frac{1}{Z} e^{- \beta (H_0 + H')}
\end{equation}
with $H'$:
\begin{align}
    H' &= -\frac{1}{T} \int \delta T(x)h(x) d^3x \notag \\
        & = \frac{1}{T} \int d^3x \int _{-\infty}^{0} dt \delta T(x) \nabla\cdot S(x,t) \notag \\
        & = - \left( \frac{1}{T} \int d^3x \int _{-\infty}^{0} dt S(x,t) \right) \nabla T \notag \\
        & = - \left( \frac{1}{T} \int _{-\infty}^{0} dt J(t) \right) \nabla T 
\end{align}

Using the relationship (\emph{Allen 1993}):
\begin{equation}
    e^{-\beta (H_0 + H')} \approx e^{-\beta H_0} \left[ 1 + \int_0^{\beta} d\lambda e^{\lambda H_0} H' e^{-\lambda H_0} \right]
        \approx e^{-\beta H_0} \left[ 1 - \frac{1}{T} \int_0^{\beta} d\lambda e^{\lambda H_0} \int _{-\infty}^{0} dt J(t) e^{-\lambda H_0} \right]
\end{equation} 
The expectation value of current $J_{\mu}$ is then:
\begin{equation}
    \text{Tr}\rho J_{\mu} = \text{Tr}\rho_0 J_{\mu} - \text{Tr}\rho_0\frac{1}{T}\int_0^{\beta} d\lambda \int _{-\infty}^{0} dt 
    e^{\lambda H_0} (J(t) \nabla T) e^{-\lambda H_0} J_{\mu}
\end{equation}
The first term is zero under equilibrium, the second term gives $ - V \kappa_{\mu\nu} \nabla T$ and 
\begin{align}
    \kappa_{\mu\nu} & = \text{Tr}\rho_0 \frac{1}{VT} \int_0^{\beta} d\lambda \int _{-\infty}^{0} dt 
    e^{\lambda H_0} J_{\nu}(t) e^{-\lambda H_0} J_{\mu} \notag \\
    & = \frac{1}{VT}\int_0^{\beta} d\lambda \int _{-\infty}^{0} dt \langle J_{\nu}(t-i\hbar\lambda) J_{\mu}(0) \rangle_0 \notag \\
    & = \frac{1}{VT}\int_0^{\beta} d\lambda \int _{0}^{\infty} dt \langle J_{\nu}(-i\hbar\lambda) J_{\mu}(t) \rangle_0 
\end{align}
which agrees with the Kubo's formula for electrical conductivity. 
\footnote{in the case of electrical conductivity of Kubo's derivation, 
the pertrubation is in the Hamiltonian, but for heat conductivity, pertrubation 
is in the density matrix, thus the process of derivation is different, i.e. The
Hamiltonian is time dependent because of $T(t)$. Specially,
we previously assumed $\rho = \rho_{0}$, which is no longer the approximation. 
If I just directly apply Kubo's form with $H'$, we have an negative sign of conductivity }
%where we take $\nabla T$ to be uniform. Now, we can apply the Kubo's formula:
%The final result take the same form as the Kubo's formula for electrical conductivity.
%\footnote{ I have not yet finish the derivation, since this case seems to be different from the above form,
%where the perturbation is in the Hamiltonian, using the above form of $H' = -Af$ leads to a minus sign, since $J = -\kappa \nabla T$, suggesting
%the form as used in electrical case may not apply here}
%\begin{align}
%    \kappa_{\mu\nu} (t) & = \frac{1}{VT} \int_0^{\beta} \langle J_{\nu}(-i\hbar\lambda) J_{\mu}(t) \rangle d\lambda \\
%    \kappa_{\mu\nu} (\omega) & = \frac{1}{VT} \int_0^{\beta} d\lambda \int_{0}^{\infty} e^{-i\omega t} dt \langle J_{\nu}(-i\hbar\lambda) J_{\mu}(t) \rangle 
%\end{align}

\end{document}
\end{document}
