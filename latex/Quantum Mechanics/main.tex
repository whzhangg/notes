\documentclass{article}

\usepackage{amssymb, amsmath, amsthm}
\usepackage[margin=1in]{geometry}
\usepackage{verbatim}
\usepackage{graphicx}
\usepackage{hyperref}
\usepackage{docmute}

\newcommand{\pfrac}[2]{\frac{\partial #1}{\partial #2}}
\newcommand{\ddt}[1]{\frac{d}{dt}\left( #1 \right)}
\renewcommand{\L}{\mathcal{L}}
\renewcommand{\S}{\mathcal{S}}
\renewcommand{\H}{\mathcal{H}}
\newcommand{\dotx}{\dot{x}}
\newcommand{\dotq}{\dot{q}}
\newcommand{\dotp}{\dot{p}}
\newtheorem{theorem}{Theorem}
\newtheorem{example}{Example}


\begin{document}

\title{Quantum Mechanics}
\author{Wenhao Zhang}
\date{\today}
\maketitle

\tableofcontents
\newpage

\input{M_Lagrange.tex}
\documentclass{article}

\usepackage{amssymb, amsmath, amsthm}
\usepackage[margin=1in]{geometry}
\usepackage{verbatim}
\usepackage{graphicx}
\usepackage{hyperref}
\usepackage{docmute}

\newcommand{\pfrac}[2]{\frac{\partial #1}{\partial #2}}
\newcommand{\ddt}[1]{\frac{d}{dt}\left( #1 \right)}
\renewcommand{\L}{\mathcal{L}}
\renewcommand{\S}{\mathcal{S}}
\renewcommand{\H}{\mathcal{H}}
\newcommand{\dotx}{\dot{x}}
\newcommand{\dotq}{\dot{q}}
\newcommand{\dotp}{\dot{p}}

\newtheorem{theorem}{Theorem}
\newtheorem{example}{Example}

\begin{document}

\section{Hamiltanion method}
\subsection{Hamiltonian}
The quantity $E$ is given:
\begin{equation}
    E(q,\dotq) = \sum_{i=1}^N \pfrac{\L(q,\dotq)}{\dotq_i} \dotq_i - \L(q,\dotq) 
\end{equation}
Define the conjugate momentum and Hamiltonian:
\begin{gather}
    p_i = \pfrac{\L(q,\dotq)}{\dotq_i} \\
    \H = \sum_{i=1}^N p_i \dotq_i - \L(q,\dotq)
\end{gather}
where $\dotq$ are implicit function of $(q,p)$
we can find:
\begin{equation}
    \begin{cases}
        \pfrac{\H}{p_i} = \dotq_i - \pfrac{\L}{p_i} = \dotq_i \notag \\
        \pfrac{\H}{q_i} = 0 - \pfrac{\L}{q_i} = - \dotp_i \notag \\
    \end{cases}
\end{equation}
and thus the Hamiltanion equation:
\begin{align}
        \dotq_i &= \pfrac{\H}{p_i} \\
        \dotp_i &= - \pfrac{\H}{q_i}
\end{align}

The cyclic coordinates are still the ones which $\H$ does not explicitly depend on.

\subsection{Legendre transformation}
Legendre transformation states that if $Z(x) = Y(X) - xX$, then we should have the 
relationship:
\begin{equation}
    \pfrac{Z}{x} = -X
\end{equation}
Using the relationship for Lagrangian and Hamiltanion, we can identify, for one specific coordinate i:
\begin{gather}
    \H(p_i) = p_i \dotq_i - \L(\dotq_i) \notag \\
    Z(x) = -H(p_i) \notag \\
    Y(X) = L(\dotq_i) \notag \\
    \text{leading to}\ \ \pfrac{H(p_i)}{p_i} = \dotq_i
\end{gather}


\end{document}

\documentclass{article}

\usepackage{amssymb, amsmath, amsthm}
\usepackage[margin=1in]{geometry}
\usepackage{verbatim}
\usepackage{graphicx}
\usepackage{hyperref}
\usepackage{docmute}

\begin{document}


\section{Quantum condition}
In quantum mechanics, observables do not obey the commutative law of multiplication so
it is necessary for us to study the value of $\zeta\eta - \eta\zeta$ when $\zeta$ and $\eta$
are two observables. These new relationships are called \textbf{quantum conditions} or
\textbf{commutation relationships} 
\footnote{Principles of Quantum mechanics, Paul Dirac, 1958, Chapter IV}.

Suppose $p$ and $q$ are a set of canonical momentum and coordinates, we can define the classic
\textbf{Poisson bracket} of any two variables $u$ and $v$:
\begin{equation}
    \{ u,v \}_C = \sum_r \left\{ \frac{\partial u}{\partial q_r}\frac{\partial v}{\partial p_r} - \frac{\partial u}{\partial p_r}\frac{\partial v}{\partial q_r} \right\}
\end{equation}
Poisson brackets satisfy the following properties:
\begin{gather}
    \{u,v\}_C = - \{v,u\}_C \\
    \{u,c\}_C = 0 \\
    \begin{cases}
        \{ u_1 + u_2, v\}_C = \{ u_1, v\}_C + \{u_2,v\}_C \notag \\
        \{ u, v_1 + v_2\}_C = \{ u, v_1\}_C + \{u,v_2\}_C 
    \end{cases}
\end{gather}
\begin{align}
    \{u_1u_2,v\}_C 
    &= \sum_r \left\{ \left( \frac{\partial u_1}{\partial q_r} u_2 + u_1\frac{\partial u_2}{\partial q_r}\right)\frac{\partial v}{\partial p_r} \right. 
                \left.- \left( \frac{\partial u_1}{\partial p_r} u_2 + u_1\frac{\partial u_2}{\partial p_r}\right)\frac{\partial v}{\partial q_r} \right\} \notag \\
    &= \{u_1,v\}_C u_2 + u_1 \{ u_2,v \}_C \notag \\
    \{u,v_1v_2\}_C &= \{u,v_1\}_C v_2 + v_1 \{ u,v_2 \}_C
\end{align}
\begin{equation}
    \{u \{v,w\}_C\}_C + \{v \{w,u\}_C\}_C + \{w \{u,v\}_C\}_C = 0
\end{equation}

Now we introduce \textbf{Quantum Poisson bracket}, which we assume to satisfy all the above properties with the same order. 
These requirements sufficiently determine the form of the quantum poisson bracket. 
writting down:
\begin{align}
    \{u_1u_2,v_1v_2\}_Q &= \{u_1,v_1v_2\}_Qu_2 + u_1\{u_2,v_1v_2\}_Q \notag \\
                      &= \{u_1,v_1\}_Qv_2u_2 + v_1\{u_1,v_2\}_Qu_2 + u_1\{u_2,v_1\}_Qv_2 + u_1v_1\{ u_2,v_2 \}_Q \\
    \{u_1u_2,v_1v_2\}_Q &= \{u_1u_2,v_1\}_Qv_2 + v_1\{u_1u_2,v_2\}_Q \notag \\
                      &= \{u_1,v_1\}_Qu_2v_2 + u_1\{u_2,v_1\}_Qv_2 + v_1\{u_1,v_2\}_Qu_2 + v_1u_1\{ u_2,v_2 \}_Q        
\end{align} 
equating the above two results, we have the relationships:
\begin{equation}
    \{u_1,v_1\}_Q(u_2v_2 - v_2u_2) = (u_1v_1 - v_1u_1)\{u_2,v_2\}_Q
\end{equation}
since this equation should hold for $u_1,v_1$ independent of $u_2,v_2$, while 
the commutative property of multiplication is no longer true ($uv-vu\neq 0$) in general. we thus require:
\begin{align}
    u_2v_2 - v_2u_2 \equiv [u_2, v_2] = c\{u_2,v_2\}_Q \\
    u_1v_1 - v_1u_1 \equiv [u_1, v_1] = c\{u_1,v_1\}_Q
\end{align}
with $c$ some universial numerical factor independent of the observable $u$ and $v$. 

we want the poisson bracket of two observables to be real, but $uv - vu$ will be 
imaginary for two operators of real observable. Therefore, numerical factor $c$ must
be purly imaginary.  In order that the theory agree with experiment, we must take 
$c = i\hbar$, where $\hbar$ has the unit of action, giving the final result:
\begin{equation}
    [u,v] = i\hbar\{u,v\}_Q
\end{equation}
which gives the connection between quantum theory and classic theory.
We observe that the commutator $uv - vu$ is of the order of $\hbar$. Approximating $\hbar \to 0$
leads to $uv \approx vu$ and thus the classic limit.

We can check that the quantum poisson bracket of two variables given by their commutation relationship
satisfy all the properties of the classic poisson bracket. Therefore, we further assume that
\emph{the quantum poisson bracket has the same value as the classic poisson bracket} 
$\{u,v\}_Q = \{u,v\}_C$.

In the classical mechanics, the poisson bracket of canonical coordinates and momenta having the value:
\begin{gather}
    \{q_r,q_s\} = \{p_r,p_s\} = 0\\
    \{q_r,p_s\} = \delta_{rs}
\end{gather}
which gives the quantum commutation result:
\begin{gather}
    [q_r,q_s] = [p_r,p_s] = 0\\
    [q_r,p_s] = i\hbar\delta_{rs}
\end{gather}


\end{document}
\documentclass{article}

\usepackage{amssymb, amsmath, amsthm}
\usepackage[margin=1in]{geometry}
\usepackage{verbatim}
\usepackage{graphicx}
\usepackage{hyperref}
\usepackage{docmute}

\begin{document}


\section{Response functions}
\subsection{Linear response}
we wish to find the change of observable $B$ with a perturbation term in addition to the equilibrium Hamiltonian 
that is turned on at 
\begin{equation}
     H = H_0 - Af(t)
\end{equation}
The perturbation is turned on from $t_0$, 
In Schrodinger picture, the time dependence of a state is given by:
\begin{align}
    | \Psi_n(t) \rangle &= U(t,t_0) | \Psi_n(t_0) \rangle 
    %\notag \\ &= e^{-\frac{i}{\hbar}H_0(t-t_0)} U_I(t,t_0) | \Psi_n(t_0) \rangle \notag 
\end{align}
with
\begin{align}
    U(t,t_0) &= e^{-i\frac{H_0}{h}(t-t_0)} U_I(t,t_0) \notag \\
                 &= e^{-i\frac{H_0}{h}(t-t_0)} \exp\left[ -\frac{i}{\hbar} \int_{t_0}^{t} H_I'(t') dt'  \right] \notag \\
                 &\approx e^{-i\frac{H_0}{h}(t-t_0)} \left( 1 + \frac{i}{\hbar} \int_{t_0}^{t} A_I(t')f(t') dt' \right)
\end{align}
and $A_I(t) = \exp(iH_0t/\hbar) A \exp(-iH_0t/\hbar)$. We find the difference:
\begin{align}
    \langle B \rangle(t) - \langle B \rangle(t_0) 
    &= \sum_n \langle \Psi_n(t) | \rho_{0} B | \Psi_n(t) \rangle - \sum_n \langle \Psi_n(t_0) | \rho B | \Psi_n(t_0) \notag \\
    &\approx \sum_n \langle \Psi_n(t) | \rho_{0} B | \Psi_n(t) \rangle - \sum_n \langle \Psi_n(t_0) | \rho_{0} B | \Psi_n(t_0) \rangle
\end{align}
where $\rho_0 = e^{-\beta H_0} / Z$ and we use the adabatic approximation to assume that the probability of 
the states remain the same as in the unperturbed case. 
%The states evolve according to the Schrodinger equation:
%\begin{align}
%    U(t,-\infty) &= e^{-i\frac{H_0}{h}t} U_I(t,-\infty) \notag \\
%                 &= e^{-i\frac{H_0}{h}t} \exp\left[ -\frac{i}{\hbar} \int_{-\infty}^{t} H_I'(t') dt'  \right] \notag \\
%                 &\approx e^{-i\frac{H_0}{h}t} \left( 1 + \frac{i}{\hbar} \int_{-\infty}^{t} A(t')f(t') dt' \right)
%\end{align}
so the expectation value of $B$ at time t is given by:
\begin{align}
    \langle B \rangle(t) &=
    \sum_n \langle \Psi_n(t_0) | \left( 1 - \frac{i}{\hbar} \int_{t_0}^{t} A(t')f(t') dt' \right) e^{i\frac{H_0}{h}(t-t_0)} \rho_{0} B
    e^{-i\frac{H_0}{h}(t-t_0)} \left( 1 + \frac{i}{\hbar} \int_{t_0}^{t} A(t')f(t') dt' \right) | \Psi_n(t_0) \rangle \notag \\
    %&\approx \langle B \rangle_0  + \frac{i}{\hbar} \int_{t_0}^{t} dt' 
    %\left[ \sum_n \langle \Psi_n(t_0) | e^{i\frac{H_0}{h}(t-t_0)} \rho_{0} B e^{-i\frac{H_0}{h}(t-t_0)} A(t')f(t') - A(t')f(t') e^{i\frac{H_0}{h}(t-t_0)} \rho_{eq} B e^{-i\frac{H_0}{h}(t-t_0)} A(t')f(t') | \Psi_n(t_0) \rangle \right] \notag \\
    &\approx \langle B \rangle_0  + \frac{i}{\hbar} \int_{t_0}^{t} dt' 
    \text{Tr} \rho_{0} [B(t-t_0),A(t')] f(t')
\end{align}
So that the difference is given by:
\begin{align}
    \Delta B(t) &= \frac{i}{\hbar} \int_{t_0}^{t} dt' \text{Tr} \rho_{0} [B(t-t_0),A(t')] f(t') 
\end{align}
The response function is defined to be the response after a unit unit impulse at $t = 0$:
\begin{align}
    \phi_{BA} (t) &= \frac{i}{\hbar} \int_{t_0}^{t} dt' \text{Tr} \rho_{0} [B(t-t_0),A(t')] \delta(t'=0) \notag \\
            &= \frac{i}{\hbar} \text{Tr} \rho_{0} [B(t),A(0)] 
\end{align} 
\begin{align}
    \phi_{BA} (t) &=\frac{i}{\hbar} \langle [B(t),A] \rangle_0 = \frac{1}{i\hbar} \text{Tr} \rho_0 [A, B(t)]
                = \frac{1}{i\hbar} \text{Tr} [\rho_0, A] B(t)
\end{align}

Using the identity:
\begin{gather}
    [A, e^{-\beta H_0}] = e^{-\beta H_0} \int_0^{\beta} e^{\lambda H_0} [H_0, A] e^{-\lambda H_0} d\lambda 
    = e^{-\beta H_0} \int_0^{\beta} e^{\lambda H_0} (-i\hbar) \dot{A} e^{-\lambda H_0} d\lambda 
\end{gather}
\begin{align}
    [\rho_{0}, A] &=  i\hbar \rho_{0} \int_0^{\beta} e^{\lambda H_0} \dot{A} e^{-\lambda H_0} d\lambda \notag \\
                &=  i\hbar  \int_0^{\beta} \rho_{0} \dot{A}(-i\hbar\lambda) d\lambda
\end{align}
so that $\exp(-iH_0t/\hbar) \to \exp( -H_0\lambda ) $
and we can arrive at the formula given by Kubo:
\begin{align}
    \phi_{BA} (t) &= \int_0^{\beta} \text{Tr} \rho_0 \dot{A}(-i\hbar\lambda) B(t) d\lambda 
    = \int_0^{\beta} \langle \dot{A}(-i\hbar\lambda) B(t) \rangle_0 d\lambda 
\end{align}

We define the frequency components of the response function as
\footnote{this follows the definition of \emph{Kubo 1957} Eq.2.21, in terms
of the more conventional way, we have:
\begin{equation}
    \chi_{BA}(\omega) = \lim_{\eta\to 0^+} \int_{0}^{\infty} \phi_{BA} (t) e^{i(\omega+i\eta)\tau} dt
\end{equation}
}
:
\begin{equation}
    \chi_{BA}(\omega) = \lim_{\eta\to 0^+} \int_{0}^{\infty} \phi_{BA} (t) e^{-\eta t-i\omega t} dt
\end{equation}
We obtain the frequency response function:
\begin{equation}
    \chi_{BA}(\omega) = 
    \lim_{\eta\to 0^+} \int_0^{\beta} d\lambda \int_{0}^{\infty} dt e^{-\eta t-i\omega t} \langle \dot{A}(-i\hbar\lambda) B(t) \rangle_0 
\end{equation}

Let's also consider the case where a (constant pertrubation) $F$ is applied continuously from $t = -\infty$ to 
$t = 0$ and stops. The system then relax throught internal interaction. The observable will
follow:
\begin{align}
    \Delta B(t) &= \int_{-\infty}^{0} \phi_{BA} (t-t') dt' F \notag \\
                &= \int_{t}^{\infty} \phi_{BA} (t') dt' F \notag \\
                &= \Phi_{BA} (t) F
\end{align}
and
\begin{equation}
    \Phi_{BA} (t) = \lim_{\eta\to 0^+} \int_{t}^{\infty} \phi_{BA} (t') e^{-\eta t'} dt'
\end{equation}
is called the relaxation function.

\subsection{Linear response formula of electrical conductivity}
We consider an uniform external electric field (potential zero is arbitrary) ($V(x) = E(t)x $):
\begin{gather}
    H'(t) = -e \sum_i x_i E(t) = - A E(t) \\
    \dot{A} = e \sum_i \dot{x}_i = J
\end{gather}

where $x_i$ is the position operator of the $i^{th}$ particle and $e$ is the charge associated with that
particle. The current operator is 
defined to be:
\begin{equation}
    J_{\mu} = e \sum_i \dot{x}_i
\end{equation}
The response function is given by:
\begin{align}
    \phi_{\mu\nu} (t) &= \int_0^{\beta} \langle J_{\nu}(-i\hbar\lambda) J_{\mu}(t) \rangle_0 d\lambda \\
    \chi_{\mu\nu}(\omega)&= 
    \lim_{\eta\to 0^+} \int_0^{\beta} d\lambda \int_{0}^{\infty} e^{-\eta t-i\omega t} dt  \langle J_{\nu}(-i\hbar\lambda) J_{\mu}(t) \rangle_0
\end{align}
and the conductivity is given by:
\begin{equation}
    \sigma_{\mu\nu} = \frac{1}{V} \int_0^{\beta} d\lambda \int_{0}^{\infty} dt \langle J_{\nu}(-i\hbar\lambda) J_{\mu}(t) \rangle_0
\end{equation}

\subsection{Linear response formula of thermal conductivity}
Derivation of the expression of thermal conductivity is provided by \emph{Allen and Feldman, 1993}. 
The total current operator is given by:
\begin{equation}
    J_{\alpha} = \sum_{ij\beta\gamma} (R_{i\alpha} - R_{j\alpha}) \Phi_{ij}^{\beta\gamma} u_{i\beta} \dot{u}_{j\gamma}
\end{equation}
We consider the Hamiltonian of the system to be:
\begin{equation}
    H_0 = \int h(x)  d^3x
\end{equation}
where $h(x) = \sum_i h_i \delta(x - x_i)$ consists of the vibration energy of each atom $i$. 
The local current operator $S(x)$ is related to $h(x)$ by the continuity equation:
\begin{gather}
    \frac{\partial h(x)}{\partial t} + \nabla\cdot S(x) = 0 \\
    J = \int S(x) d^3x
\end{gather}
The density matrix can be written as:
\begin{equation}
    \rho = \frac{1}{Z} e^{- \int \beta(x) h(x) d^3x}
\end{equation}
and $\beta(x) \approx \beta[1-\delta T(x)/T]$ with $T$ the average temperature, then
\begin{equation}
    \rho = \frac{1}{Z} e^{- \int \beta[1-\delta T(x)/T] h(x) d^3x} = \frac{1}{Z} e^{- \beta (H_0 + H')}
\end{equation}
with $H'$:
\begin{align}
    H' &= -\frac{1}{T} \int \delta T(x)h(x) d^3x \notag \\
        & = \frac{1}{T} \int d^3x \int _{-\infty}^{0} dt \delta T(x) \nabla\cdot S(x,t) \notag \\
        & = - \left( \frac{1}{T} \int d^3x \int _{-\infty}^{0} dt S(x,t) \right) \nabla T \notag \\
        & = - \left( \frac{1}{T} \int _{-\infty}^{0} dt J(t) \right) \nabla T 
\end{align}

Using the relationship (\emph{Allen 1993}):
\begin{equation}
    e^{-\beta (H_0 + H')} \approx e^{-\beta H_0} \left[ 1 + \int_0^{\beta} d\lambda e^{\lambda H_0} H' e^{-\lambda H_0} \right]
        \approx e^{-\beta H_0} \left[ 1 - \frac{1}{T} \int_0^{\beta} d\lambda e^{\lambda H_0} \int _{-\infty}^{0} dt J(t) e^{-\lambda H_0} \right]
\end{equation} 
The expectation value of current $J_{\mu}$ is then:
\begin{equation}
    \text{Tr}\rho J_{\mu} = \text{Tr}\rho_0 J_{\mu} - \text{Tr}\rho_0\frac{1}{T}\int_0^{\beta} d\lambda \int _{-\infty}^{0} dt 
    e^{\lambda H_0} (J(t) \nabla T) e^{-\lambda H_0} J_{\mu}
\end{equation}
The first term is zero under equilibrium, the second term gives $ - V \kappa_{\mu\nu} \nabla T$ and 
\begin{align}
    \kappa_{\mu\nu} & = \text{Tr}\rho_0 \frac{1}{VT} \int_0^{\beta} d\lambda \int _{-\infty}^{0} dt 
    e^{\lambda H_0} J_{\nu}(t) e^{-\lambda H_0} J_{\mu} \notag \\
    & = \frac{1}{VT}\int_0^{\beta} d\lambda \int _{-\infty}^{0} dt \langle J_{\nu}(t-i\hbar\lambda) J_{\mu}(0) \rangle_0 \notag \\
    & = \frac{1}{VT}\int_0^{\beta} d\lambda \int _{0}^{\infty} dt \langle J_{\nu}(-i\hbar\lambda) J_{\mu}(t) \rangle_0 
\end{align}
which agrees with the Kubo's formula for electrical conductivity. 
\footnote{in the case of electrical conductivity of Kubo's derivation, 
the pertrubation is in the Hamiltonian, but for heat conductivity, pertrubation 
is in the density matrix, thus the process of derivation is different, i.e. The
Hamiltonian is time dependent because of $T(t)$. Specially,
we previously assumed $\rho = \rho_{0}$, which is no longer the approximation. 
If I just directly apply Kubo's form with $H'$, we have an negative sign of conductivity }
%where we take $\nabla T$ to be uniform. Now, we can apply the Kubo's formula:
%The final result take the same form as the Kubo's formula for electrical conductivity.
%\footnote{ I have not yet finish the derivation, since this case seems to be different from the above form,
%where the perturbation is in the Hamiltonian, using the above form of $H' = -Af$ leads to a minus sign, since $J = -\kappa \nabla T$, suggesting
%the form as used in electrical case may not apply here}
%\begin{align}
%    \kappa_{\mu\nu} (t) & = \frac{1}{VT} \int_0^{\beta} \langle J_{\nu}(-i\hbar\lambda) J_{\mu}(t) \rangle d\lambda \\
%    \kappa_{\mu\nu} (\omega) & = \frac{1}{VT} \int_0^{\beta} d\lambda \int_{0}^{\infty} e^{-i\omega t} dt \langle J_{\nu}(-i\hbar\lambda) J_{\mu}(t) \rangle 
%\end{align}

\end{document}
\end{document}
