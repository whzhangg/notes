\documentclass{article}

\usepackage{amssymb, amsmath, amsthm}
\usepackage[margin=1in]{geometry}
\usepackage{verbatim}
\usepackage{graphicx}
\usepackage{hyperref}
\usepackage{docmute}

\newcommand{\pfrac}[2]{\frac{\partial #1}{\partial #2}}
\newcommand{\ddt}[1]{\frac{d}{dt}\left( #1 \right)}
\renewcommand{\L}{\mathcal{L}}
\renewcommand{\S}{\mathcal{S}}
\renewcommand{\H}{\mathcal{H}}
\newcommand{\dotx}{\dot{x}}
\newcommand{\dotq}{\dot{q}}
\newcommand{\dotp}{\dot{p}}

\newtheorem{theorem}{Theorem}
\newtheorem{example}{Example}

\begin{document}

\section{Hamiltanion method}
\subsection{Hamiltonian}
The quantity $E$ is given:
\begin{equation}
    E(q,\dotq) = \sum_{i=1}^N \pfrac{\L(q,\dotq)}{\dotq_i} \dotq_i - \L(q,\dotq) 
\end{equation}
Define the conjugate momentum and Hamiltonian:
\begin{gather}
    p_i = \pfrac{\L(q,\dotq)}{\dotq_i} \\
    \H = \sum_{i=1}^N p_i \dotq_i - \L(q,\dotq)
\end{gather}
where $\dotq$ are implicit function of $(q,p)$
we can find:
\begin{equation}
    \begin{cases}
        \pfrac{\H}{p_i} = \dotq_i - \pfrac{\L}{p_i} = \dotq_i \notag \\
        \pfrac{\H}{q_i} = 0 - \pfrac{\L}{q_i} = - \dotp_i \notag \\
    \end{cases}
\end{equation}
and thus the Hamiltanion equation:
\begin{align}
        \dotq_i &= \pfrac{\H}{p_i} \\
        \dotp_i &= - \pfrac{\H}{q_i}
\end{align}

The cyclic coordinates are still the ones which $\H$ does not explicitly depend on.

\subsection{Legendre transformation}
Legendre transformation states that if $Z(x) = Y(X) - xX$, then we should have the 
relationship:
\begin{equation}
    \pfrac{Z}{x} = -X
\end{equation}
Using the relationship for Lagrangian and Hamiltanion, we can identify, for one specific coordinate i:
\begin{gather}
    \H(p_i) = p_i \dotq_i - \L(\dotq_i) \notag \\
    Z(x) = -H(p_i) \notag \\
    Y(X) = L(\dotq_i) \notag \\
    \text{leading to}\ \ \pfrac{H(p_i)}{p_i} = \dotq_i
\end{gather}


\end{document}
