\documentclass{article}

\usepackage{amssymb, amsmath, amsthm}
\usepackage[margin=1in]{geometry}
\usepackage{verbatim}
\usepackage{graphicx}

\begin{document}

\section{Deriving the diagramatic rules}
Our task is to evaluate the expression for $G(qvv';\tau)$ as:
\begin{align}
    G(qvv';\tau)= \sum_{n=0}^{\infty} \frac{1}{n!} \left( -\frac{1}{\hbar} \right)^{n} 
    \int_0^{\beta\hbar} d\tau_1 \cdots \int_0^{\beta\hbar} d\tau_{n} \left\langle \mathcal{T}[ A_{qv}(\tau) A_{qv'}^{\dagger}(0) H_I(\tau_1) \cdots H_I(\tau_{n}) ] \right\rangle_{0,c} 
\end{align}
we consider the case of first order perturbation from four-phonon interaction, which correspond to Figure.\ref{diagram1}. Now the 
expression is simply:
\begin{align}
    G(q;\tau)&= \left( -\frac{1}{\hbar} \right) 
    \int_0^{\beta\hbar} d\tau_1 \left\langle \mathcal{T}\left[ A_{q}(\tau) A_{q}^{\dagger}(0) \sum_{q_{1\cdots 4}}V(q_1,q_2,q_3,q_4) A_{q_1}(\tau_1)A_{q_2}(\tau_1)A_{q_3}(\tau_1)A_{q_4}(\tau_1)\right] \right\rangle_{0,c} \notag \\
    &= -\frac{12}{\hbar}  
    \sum_{q_1} V(-q,q,-q_1,q_1) \int_0^{\beta\hbar} d\tau_1 \left\langle \mathcal{T}[ A_q(\tau_1)A_{q}^{\dagger}(0) ] \right\rangle
    \left\langle \mathcal{T}[ A_{q}(\tau)A_{q}^{\dagger}(\tau_1) ] \right\rangle 
    \left\langle \mathcal{T}[ A_{q_1}(\tau_1)A_{q_1}^{\dagger}(\tau_1) ] \right\rangle \notag \\
    &= -\frac{12}{\hbar} \sum_{q_1} V(-q,q,-q_1,q_1) \int_0^{\beta\hbar} d\tau_1 G_0(q;\tau_1) G_0(q;\tau-\tau_1) G_0(q_1;0) \notag \\
    &= -\frac{12}{\hbar} \sum_{q_1} V(-q,q,-q_1,q_1) \int_0^{\beta\hbar} d\tau_1 
        \sum_{n_1}G_0(q;i\omega_{n_1}) e^{i\omega_{n_1}\tau_1} 
        \sum_{n_2}G_0(q;i\omega_{n_2}) e^{i\omega_{n_2}(\tau-\tau_1)} 
        \sum_{n_3}G_0(q_1;i\omega_{n_3}) \notag \\
    &= -\frac{12}{\hbar} \sum_{q_1} V(-q,q,-q_1,q_1) \sum_{n_1,n_2} G_0(q;i\omega_{n_1}) G_0(q;i\omega_{n_2}) e^{i\omega_{n_2}\tau} 
        \int_0^{\beta\hbar} d\tau_1 e^{i(\omega_{n_1}-\omega_{n_2})\tau_1} \sum_{n_3}G_0(q_1;i\omega_{n_3}) \notag \\
    &= -12\beta \sum_{q_1} V(-q,q,-q_1,q_1) \sum_{n_1} G_0(q;i\omega_{n_1}) G_0(q;i\omega_{n_1}) e^{i\omega_{n_1}\tau} 
        \sum_{n_3}G_0(q_1;i\omega_{n_3})
\end{align}
where the thermal average is contracted into pairs with Wick's theorem at finite temperature and we used the fact that
\begin{equation}
    \int_0^{\beta\hbar} d\tau_1 e^{i(\omega_{n_1}-\omega_{n_2})\tau_1} = \beta\hbar \delta_{n_1,n_2}
\end{equation}.
To see how it give raise to self-energy, we perform transformation into frequency space:
\begin{align}
    G(q;i\omega_n) &= \frac{1}{\beta\hbar} \int_{0}^{\beta\hbar} G(q;\tau) e^{-i\omega_n\tau} d\tau \notag \\
        &= -\frac{12}{\hbar}\sum_{q_1} V(-q,q,-q_1,q_1) 
        \int_{0}^{\beta\hbar} \sum_{n_1} G_0(q;i\omega_{n_1}) G_0(q;i\omega_{n_1}) e^{i(\omega_{n_1}-\omega_n)\tau}  d\tau
        \sum_{n_3}G_0(q_1;i\omega_{n_3})  \notag \\
        &= -12\beta \sum_{q_1} V(-q,q,-q_1,q_1) G_0(q;i\omega_{n}) G_0(q;i\omega_{n})
        \sum_{n_3}G_0(q_1;i\omega_{n_3}) 
\end{align}
The self-energy is defined to be:
\begin{equation}
    G(q;i\omega_n) = G_0(q;i\omega_n) + \beta\hbar G_0(q;i\omega_n) \Sigma(q;i\omega_n) G(q;i\omega_n)
\end{equation}
or equivalently
\begin{equation}
    G(q;i\omega_n)^{-1} = G_0(q;i\omega_n)^{-1} - \beta\hbar \Sigma(q;i\omega_n) 
\end{equation}
so that we find the self-energy in the above expression:
\begin{equation}
    \Sigma^{(4)}(q;i\omega_n) = -\frac{12}{\hbar} \sum_{q_1} V(-q,q,-q_1,q_1) \sum_{n_1}G_0(q_1;i\omega_{n_1}) 
\end{equation}
\subsection*{General case}
In the general case, we can establish the rules as follow for the expression of phonon Green's function:
\begin{align}
    G(qvv';\tau)= \sum_{n=0}^{\infty} \frac{1}{n!} \left( -\frac{1}{\hbar} \right)^{n} 
    \int_0^{\beta\hbar} d\tau_1 \cdots \int_0^{\beta\hbar} d\tau_{n} \left\langle \mathcal{T}[ A_{qv}(\tau) A_{qv'}^{\dagger}(0) H_I(\tau_1) \cdots H_I(\tau_{n}) ] \right\rangle_{0,c} 
\end{align}
\begin{enumerate}
    \item Factor $1/n!$ is cancelled with the $n!$ permutation of internal time coordinates.
    \item Carry out all integral with respect to internal time give a factor $(\beta\hbar)^n$ .
    \item Finding the number of pairing scheme in the Wick theorem expansion.
\end{enumerate}
Therefore, the remaining factor is only $(-\beta)^n$. There is an additional definition of 
$\frac{1}{\beta\hbar}$ in the self-energy, Therefore, the rules for evaluating self-energy 
is 
\begin{enumerate}
    \item Overall factor of $\frac{1}{\beta\hbar} (-\beta)^n$, where $n$ is the number of the vertexes.
    \item Multiply the number of ways to pair the phonon modes.
    \item Multiply appropriate $V$ for each vertex.
    \item For each internal line, multiply $G_0(qvv';i\omega^i_n)$.
    \item Sum over all internal coordinates $q^i;v^i$ and Matsubara frequency $\omega_n^i$
\end{enumerate}

\end{document}
