\documentclass{article}

\usepackage{amssymb, amsmath, amsthm}
\usepackage[margin=1in]{geometry}
\usepackage{verbatim}
\usepackage{graphicx}

\begin{document}

\section{Unified Theory of Thermal Transport}
\subsection{Transport equation}
We consider that the system of phonons are governed by the equation:
\begin{equation}
    \frac{\partial \rho(t)}{\partial t} + \frac{i}{\hbar} \left[H_0, \rho(t)\right] = \left. \frac{\partial\rho(t) }{\partial t} \right|_{coll} \label{master}
\end{equation}
Define the creation and annhiliation operator $a_{qb}$ and $a^{\dagger}_{qb}$ with 
$b = (b,\alpha)$, which is related 
to phonon creation and annhiliation operator $a_{qv}$ and $a^{\dagger}_{qv}$ by:
\begin{align}
    a_{qb} = \sum_v e^b_{qv} a_{qv} \\
    a^{\dagger}_{qb} = \sum_v e^{*b}_{qv} a^{\dagger}_{qv}
\end{align}
$e^b_{qv}$ gives the transformation between the two set of operators. 
The Harmonic Hamiltonian written using $a_{qb}$ and $a^{\dagger}_{qb}$ is:
\begin{equation}
    H_0 = \sum_{q} \sum_{b,b'} \hbar \sqrt{D_q}_{bb'} \left( a^{\dagger}_{qb}a_{qb'} + \frac{1}{2} \delta_{bb'} \right)
\end{equation}
$\sqrt{D_q}$ is the square root of matrix $D_q$ with matrix 
elements $\Phi_{q,bb'} (m_b m_{b'})^{-\frac{1}{2}}$.
Taking $e^b_{qv}$ to be the $v^{th}$ orthonormal eigenvector of the 
Dynamic matrix $D_q e^b_{qv} = \omega^2_{qv}e^b_{qv}$. $e^b_{qv}$ then also
is the eigenvector of matrix $\sqrt{D_q}$ with eigenvalue $\omega_{qv}$. We can then recovery
the harmonic Hamiltonian in its usual form:
\begin{equation}
    H_0 = \sum_{q,v} \hbar \omega_{qv} \left( a^{\dagger}_{qv}a_{qv} + \frac{1}{2} \right)
\end{equation}

The one body density matrix $\rho_1(q,q',t)$ is defined as:
\begin{equation}
    \rho_1(q,q',t)_{b,b'} = \text{Tr}[\rho(t)a^{\dagger}_{q'b'}a_{qb}]
\end{equation}
% here, the reversed $a^{\dagger}_{q'b'}a_{qb}$ is not very natural but 
% with this we can write the equation in matrix form 
We insert $H_0$ into Eq.\ref{master} and multiply on both side $a^{\dagger}_{q'b'}a_{qb}$ and take the trace:
\begin{align}
    \text{Tr}\left[ \frac{\partial \rho(t) a^{\dagger}_{q'b'}a_{qb} }{\partial t}  \right] &= \frac{\partial \rho_1(q,q',t)_{b,b'} }{\partial t} \\
    \text{Tr}\left[ \left( \frac{\partial\rho(t) a^{\dagger}_{q'b'}a_{qb} }{\partial t} \right) _{coll} \right] 
        &= \left. \frac{\partial \rho_1(q,q',t)_{b,b'} }{\partial t} \right|_{coll}
\end{align} 
For the term $\text{Tr}\left[\frac{i}{\hbar} \left[H_0, \rho(t)\right]a^{\dagger}_{q'b'}a_{qb}\right]$, we can derive:
\begin{align}
    i &\sum_{q_1} \sum_{b_1b_2} \sqrt{D_{q_1}}_{b_1b_2} \text{Tr}
    \left[ \rho (a^{\dagger}_{q'b'}a_{qb}a^{\dagger}_{q_1b_1}a_{q_1b_2} - a^{\dagger}_{q_1b_1}a_{q_1b_2}a^{\dagger}_{q'b'}a_{qb} ) \right] \notag \\
    &= i\sum_{q_1} \sum_{b_1b_2} \sqrt{D_{q_1}}_{b_1b_2} \text{Tr}
    \left[ \rho ( \delta_{q,q_1}\delta_{b,b_1} a^{\dagger}_{q'b'}a_{q_1b_2} - \delta_{q',q_1}\delta_{b',b_2} a^{\dagger}_{q_1b_1}a_{qb} ) \right] \notag \\
    &= i \left( \sum_{q_1} \sum_{b_1b_2} \sqrt{D_{q_1}}_{b_1b_2} \text{Tr} \right.
       [ \rho \delta_{q,q_1}\delta_{b,b_1} a^{\dagger}_{q'b'}a_{q_1b_2} ] -
    \sum_{q_1} \sum_{b_1b_2} \sqrt{D_{q_1}}_{b_1b_2} \text{Tr} 
    \left. [ \rho \delta_{q',q_1}\delta_{b',b_2} a^{\dagger}_{q_1b_1}a_{qb} ] \right) \notag \\
    &= i \left( \sum_{b_2} \sqrt{D_{q}}_{bb_2} \text{Tr} \right.
       [ \rho a^{\dagger}_{q'b'}a_{qb_2} ] - \sum_{b_1} \sqrt{D_{q'}}_{b_1b'} \text{Tr} 
        \left. [ \rho a^{\dagger}_{q'b_1}a_{qb} ] \right) \notag \\
    &= i \left( \sum_{b_2} \sqrt{D_{q}}_{bb_2} \rho_1(q,q',t)_{b_2,b'} - \sum_{b_1} \sqrt{D_{q'}}_{b_1b'} \rho_1(q,q',t)_{b,b_1} \right) \notag \\
    &= i \left[ \sqrt{D_q}\cdot \rho_1(q,q',t) - \rho_1(q,q',t) \cdot \sqrt{D_{q'}} \right] _{bb'} 
\end{align}
So that we obtain the equation:
\begin{equation}
    \frac{\partial \rho_1(q,q',t)_{b,b'} }{\partial t} + i \left[ \sqrt{D_q}\cdot \rho_1(q,q',t) - \rho_1(q,q',t) \cdot \sqrt{D_{q'}} \right] _{bb'}
     = \left. \frac{\partial \rho_1(q,q',t)_{b,b'} }{\partial t} \right|_{coll}  \label{derived1}
\end{equation}
We perform the Weyl transformation to $\rho_1(q,q',t)_{b,b'}$:
\begin{equation}
    N(R,q,t)_{b,b'} = \sum_{q''} \rho_1(q+q'',q-q'',t)_{b,b'} e^{2iq''R}
\end{equation}
to Eq.\ref{derived1}, we will have:
\begin{align}
    \frac{\partial \rho_1(q+q'',q-q'',t)_{b,b'} }{\partial t} &+ i \left[ \sqrt{D_{q+q''}}\cdot \rho_1(q+q'',q-q'',t) - \rho_1(q+q'',q-q'',t) \cdot \sqrt{D_{q-q''}} \right] _{bb'}
    \\ &= \left. \frac{\partial \rho_1(q+q'',q-q'',t)_{b,b'} }{\partial t} \right|_{coll}
\end{align}
Assume the one particle density $\rho_1(q+q'',q-q'',t)_{b,b'}$ is sharply peaked at $q$, $q''$ will be small,
we can then replace frequency $\sqrt{D_{q+q''}}$ and $\sqrt{D_{q-q''}}$ by:
\begin{align}
    \sqrt{D_{q+q''}} &= \sqrt{D_q} + \frac{\partial\sqrt{D_q}}{\partial q''} q''\\
    \sqrt{D_{q-q''}} &= \sqrt{D_q} - \frac{\partial\sqrt{D_q}}{\partial q''} q''
%    \omega_{q+q''v} &= \omega_{qv} + \frac{\partial \omega_{qv}}{\partial q''} q'' \\
%    \omega_{q-q''v'} &= \omega_{qv'} - \frac{\partial \omega_{qv'}}{\partial q''} q'' 
\end{align}
Multiply both side with $e^{2iq''R}$ and integrate, we have:
\begin{align}
    \frac{\partial N(R,q,t)_{bb'}}{\partial t} + &i \left[ \sqrt{D_q}\cdot N(R,q,t) - N(R,q,t) \cdot \sqrt{D_q} \right] _{bb'} \notag \\
    + &\frac{1}{2} \left[ \nabla_q\sqrt{D_q} \cdot \nabla_R N(R,q,t) + \nabla_R N(R,q,t) \cdot \nabla_q\sqrt{D_q}  \right]_{bb'} = \left. \frac{\partial N(R,q,t)_{bb'} }{\partial t} \right|_{coll}
\end{align}
which can be simplified a bit:
\begin{equation}
    \frac{\partial N(R,q,t)_{bb'}}{\partial t} + i \left[ \sqrt{D_q}, N(R,q,t) \right] _{bb'} 
    + \frac{1}{2} \left\{ \nabla_q\sqrt{D_q} , \nabla_R N(R,q,t) \right\}_{bb'} = \left. \frac{\partial N(R,q,t)_{bb'} }{\partial t} \right|_{coll}
\end{equation}
%\begin{align}
%    \frac{\partial N(R,q,t)_{vv'}}{\partial t} + i \left(\omega_{qv'} N(R,q,t)_{vv'} - \omega_{qv} N(R,q,t)_{vv'}\right) + 
%        \frac{1}{2} \left( \nabla_q \omega_{qv'} \nabla_R N(R,q,t)_{vv'} + \nabla_q \omega_{qv} \nabla_R N(R,q,t)_{vv'}  \right) 
%        = \left. \frac{\partial N(R,q,t)_{vv'} }{\partial t} \right|_{coll} \label{derived2}
%\end{align}
%In the form of $ n_v \times n_v$ matrix, we can rewrite Eq.\ref{derived2} as:
%\begin{align}
%    \frac{\partial N(R,q,t)}{\partial t} + i \left[ N(R,q,t), \omega_{q} \right] + 
%        \frac{1}{2} \left\{ \nabla_R N(R,q,t), \nabla_q \omega_{q} \right\} 
%        = \left. \frac{\partial N(R,q,t) }{\partial t} \right|_{coll} \label{derived3}
%\end{align}
% the difference in sequency in communicator come from the different definition of one partial density matrix $\rho_1(q,q',t)$
Finally, we apply the transformation from $(qb)$ to phonon coordinate $(qv)$, obtaining:
\begin{align}
    \frac{\partial N(R,q,t)}{\partial t} + i \left[ \Omega_{q},N(R,q,t) \right] + 
        \frac{1}{2} \left\{ V_{q}, \nabla_R N(R,q,t) \right\} 
        = \left. \frac{\partial N(R,q,t) }{\partial t} \right|_{coll} \label{derived3}
\end{align}
where $\Omega_q$ is a diagonal matrix with diagonal element the frequency of phonon mode $\omega_{qv}$, and $V_{qbb'}$ is 
the velocity matrix containing off-diagonal elements:
\begin{equation}
    V_{q,vv'} = \sum_{bb'} e^{*b}_{qv} ( \nabla_q\sqrt{D_q})_{bb'} e^{b'}_{qv'}
\end{equation}

\subsection{Solving the equation}
The scattering term on the right of Eq.\ref{derived3} is given:
\begin{align}
    \left. \frac{\partial N(R,q,t)_{vv'} }{\partial t} \right|_{coll} 
    = - (1-\delta_{vv'}) \frac{\Gamma_{qv} + \Gamma_{qv'}}{2} N(R,q,t)_{vv'} 
      - \delta_{vv'} \sum_{q''v''} A_{qv,q''v''}  (N(R,q'',t)_{v''v''} - \bar{N}_{q''v''} )
\end{align}
$\Gamma_{qv}$ is the phonon linewidth $\Gamma_{qv} = 1 / \tau_{qv}$, 
$\bar{N}_{qv}$ is the equilibrium Bosen distribution. The scattering matrix A is given by:
\begin{equation}
    A_{q,q'} = \frac{1}{\tau_{q}} \delta_{q,q'} 
            - \sum_{q''} \left( \Lambda_{q,q''}^{q'} -  \Lambda_{q,q'}^{q''} + \Lambda^{q',q''}_{q} \right) \label{A_unified}
\end{equation}
with $\Lambda$ given by:
\begin{align}
    \Lambda_{q,q'}^{q''} &= \frac{\bar{N}_{q} \bar{N}_{q'} (\bar{N}_{q''} + 1)}{\bar{N}_{q} (\bar{N}_{q} + 1)} L_{q,q'}^{q''} \\
    \Lambda^{q',q''}_{q} &= \frac{\bar{N}_{q} (\bar{N}_{q'}+1) (\bar{N}_{q''} + 1)}{\bar{N}_{q} (\bar{N}_{q} + 1)} L^{q',q''}_{q} 
\end{align}
and $L$ is the standard scattering probability of phonon absorption and emission events.
We focus on the case of RTA and ignore the second term in Eq.\ref{A_unified}. Eq.\ref{derived3} is then:
\begin{align}
    &\left[ \frac{\partial N(R,q,t)}{\partial t} + i \left[ \Omega_{q},N(R,q,t) \right] + \right.
      \left.  \frac{1}{2} \left\{ V_{q}, \nabla_R N(R,q,t) \right\} \right]_{vv'} \notag \\
        &=  - (1-\delta_{vv'}) \frac{\Gamma_{qv} + \Gamma_{qv'}}{2} N(R,q,t)_{vv'} 
           - \delta_{vv'} A_{qv,qv} (N(R,q,t)_{vv} - \bar{N}_{qv} ) \label{final}
\end{align}
We aim to solve the Eq.\ref{derived3} under a temperature field $T_l(R)$, $l$ indicate local temperature 
as opposed to the equilibrium temperature $T$. In an steady state, $N(R,q,t)$ will be time independent, we 
linearize $N(R,q)$ as:
\begin{equation}
    N(R,q)_{vv'} = \delta_{vv'} \left[ \bar{N}_{qv} + \frac{\partial \bar{N}_{qv}}{\partial T} (T_l(R)-T) \right] + n^{(1)}_{q,vv'} \cdot \nabla T \label{linear}
\end{equation}
the first term of the right hand side depend only on equilibrium temperature, the second term accounts for the correction 
due to the local temperature, and the third term is the linear response (vector) correspond to a temperature grident.
Putting Eq.\ref{linear} into Eq.\ref{final} and keep only linear term in $\nabla T$, we can write terms on the left side of Eq.\ref{final} as:
\begin{align}
    i \left[ \Omega_{q},N(R,q,t) \right] &= i \left( \omega_{qv} n^{(1)}_{q,vv'} - n^{(1)}_{q,vv'} \omega_{qv'} \right) \nabla T \\
    \frac{1}{2} \left\{ V_{q}, \nabla_R N(R,q,t) \right\} &= 
            \frac{1}{2} \left( V_{qvv'} \frac{\partial \bar{N}_{qv'}}{\partial T} + \frac{\partial \bar{N}_{qv}}{\partial T} V_{qvv'} \right) \nabla T
\end{align}
Heat flux is given by:
\begin{align}
    J(R,t) &= \frac{1}{2NV} \sum_{qv} \hbar \omega_{qv} \left\{ V_q, N(R,q,t)  \right\}_{vv} \notag \\
           &= \frac{1}{2NV} \sum_{qv} \hbar \omega_{qv} \left\{ V_q, n^{(1)}_{q}  \right\}_{vv}\nabla T = - \kappa \nabla T
\end{align}
giving the expression for thermal conductivity:
\begin{equation}
    \kappa = - \frac{1}{2NV} \sum_{qv} \hbar \omega_{qv} \left\{ V_q, n^{(1)}_{q}  \right\}_{vv}
\end{equation}
We can separate matrix $n^{(1)}_{q}$ into a diagonal matrix and an off-diagonal matrix whose diagonal element is zero. 
For the diagonal part, we find:
\begin{gather}
    v_{qv} \frac{\partial \bar{N}_{qv}}{\partial T} \nabla T = - A_{qv,qv} n^{(1)}_{q,vv} \nabla T \\
    n^{(1)}_{q,vv} = - v_{qv} \frac{\partial \bar{N}_{qv}}{\partial T} \tau_{qv}
\end{gather}
The diagonal part gives the contribution to thermal conductivity:
\begin{equation}
    \kappa_{diagonal} = \frac{1}{NV} \sum_{qv} \hbar \omega_{qv} v_{qv} v_{qv} \frac{\partial \bar{N}_{qv}}{\partial T} \tau_{qv}
\end{equation}
which is the usual expression in RTA formulism. The off diagonal part of matrix $n^{(1)}_{q}$ is given by:
\begin{gather}
    i \left( \omega_{qv} n^{(1)}_{q,vv'} - n^{(1)}_{q,vv'} \omega_{qv'} \right) 
    + \frac{1}{2} \left( V_{qvv'} \frac{\partial \bar{N}_{qv'}}{\partial T} + \frac{\partial \bar{N}_{qv}}{\partial T} V_{qvv'} \right)
    = - \frac{\Gamma_{qv}+ \Gamma_{qv'}}{2} n^{(1)}_{q,vv'} \notag 
\end{gather}
\begin{align}
    n^{(1)}_{q,v\neq v'} &= - V_{qvv'} 
        \frac{\partial \bar{N}_{qv} / \partial T + \partial \bar{N}_{qv'} / \partial T }{2i(\omega_{qv}-\omega_{qv'}) + (\Gamma_{qv} + \Gamma_{qv'}) } \notag \\
    &= -\frac{\hbar}{k_b T^2} V_{qvv'}  
    \frac{\omega_{qv}\bar{N}_{qv} (\bar{N}_{qv} + 1) + \omega_{qv'} \bar{N}_{qv} (\bar{N}_{qv'} + 1)}{2i(\omega_{qv} - \omega_{qv'}) + (\Gamma_{qv}+ \Gamma_{qv'})}
\end{align}
The diagonal part of $n^{(1)}_{q}$ contribute to thermal conductivity by:
\begin{align}
    \kappa_{off-diagonal} &= -\frac{1}{2NV} \sum_{qv} \hbar\omega_{qv} \sum_{v'\neq v} \left( V_{qvv'}n^{(1)}_{q,v' v}  + n^{(1)}_{q,vv'} V_{qv'v} \right) \notag \\
        &= \frac{1}{2NV} \sum_{qv,v'\neq v}\hbar\omega_{qv} \left( V_{qvv'} \frac{\partial \bar{N}_{qv} / \partial T + \partial \bar{N}_{qv'} / \partial T }{2i(\omega_{qv}-\omega_{qv'}) + (\Gamma_{qv} + \Gamma_{qv'}) } V_{qv'v} \right.
        \left. + V_{qvv'}V_{qv'v} \frac{\partial \bar{N}_{qv} / \partial T + \partial \bar{N}_{qv'} / \partial T }{2i(\omega_{qv'}-\omega_{qv}) + (\Gamma_{qv} + \Gamma_{qv'}) } \right)  \notag \\
        &= \frac{1}{NV}  \sum_{qv,v'\neq v} \hbar \omega_{qv} V_{qvv'}V_{qv'v} 
        \frac{(\partial \bar{N}_{qv} / \partial T + \partial \bar{N}_{qv'} / \partial T )(\Gamma_{qv}+ \Gamma_{qv'})}{4(\omega_{qv} - \omega_{qv'})^2 + (\Gamma_{qv}+ \Gamma_{qv'})^2} \notag \\
        &= \frac{1}{NV}  \sum_{qv,v'\neq v}\frac{\omega_{qv}+\omega_{qv'}}{2\omega_{qv}\omega_{qv'}} V_{qvv'}V_{qv'v} 
        \frac{ \left( c_{v,qv}\omega_{qv'} + c_{v,qv'}\omega_{qv} \right)(\Gamma_{qv}+ \Gamma_{qv'}) }{4(\omega_{qv} - \omega_{qv'})^2 + (\Gamma_{qv}+ \Gamma_{qv'})^2}
%    \frac{\hbar^2}{k_b T^2} \frac{1}{NV} \sum_{q}\sum_{v\neq v'} &\frac{\omega_{qv} + \omega_{qv'}}{2} V_{qvv'} V_{qv'v} \notag \\
%    & \frac{\omega_{qv}\bar{N}_{qv} (\bar{N}_{qv} + 1) + \omega_{qv'} \bar{N}_{qv} (\bar{N}_{qv'} + 1)}{4(\omega_{qv} - \omega_{qv'})^2 + (\Gamma_{qv}+ \Gamma_{qv'})^2}(\Gamma_{qv}+ \Gamma_{qv'})
\end{align}
where $c_v$ is the specific heat capacity, using $c_v = \hbar\omega_{qv} \bar{N}_{qv}(\bar{N}_{qv}+1) / k_B T^2$ recover the result in the paper. Changing 
$\omega_{qv}$ into $(\omega_{qv} + \omega_{qv'})/2$ in the last equality is for symmeterizing the experssion (exchange $v$ and $v'$ to obtain another expression and then average).
% the expression do not change upon change of index, except for \omega_{qv}
The total thermal conductivity is therefore:
\begin{equation}
    \kappa = \kappa_{diagonal} + \kappa_{off-diagonal}
\end{equation}

\end{document}