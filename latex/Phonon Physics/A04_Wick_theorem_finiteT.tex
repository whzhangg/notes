\documentclass{article}
\usepackage{amsmath}
\usepackage[margin=0.8in]{geometry}
\usepackage{verbatim}
\usepackage{graphicx}

\begin{document}

\section{Wick's theorem at finite temperature}
Evaluating the perturbation expansion involves calculating the thermal average on $N$ operators with  
non-perturbed Hamiltonian $H_0$, where
$N$ must be a even number so that the result is non-zero:
\begin{equation}
    \text{Tr}\left[ \rho_o ABC \cdots XYZ  \right]
\end{equation}
with 
\begin{equation}
    \text{Tr}\left[ \rho_o \mathcal{A} \right] = \sum_i \langle i | e^{\beta(\Omega_0 - H_0 + \mu N)} \mathcal{A} |i \rangle
\end{equation}

\subsection*{Fermion}
we first consider of case of fermions, we use the anticommunication relationship
of fermion creation and annihilation operator: $\{ a_k^{\dagger} , a_{k'} \} = \delta_{kk'}$ to 
exchange the order of operator in the average:
\begin{align}
    &\text{Tr}\left[ \rho_o ABC \cdots XYZ  \right] \notag \\
    = &\text{Tr}\left[ \rho_o \{A,B\} C \cdots XYZ  \right] - \text{Tr}\left[ \rho_o BAC \cdots XYZ  \right] \notag \\
    = &\text{Tr}\left[ \rho_o \{A,B\} C \cdots XYZ  \right] - \left( \text{Tr}\left[ \rho_o B\{A,C\} \cdots XYZ  \right] - \text{Tr}\left[ \rho_o BCA \cdots XYZ  \right] \right) \notag \\
    = &\text{Tr}\left[ \rho_o \{A,B\} C \cdots XYZ  \right] - \text{Tr}\left[ \rho_o B\{A,C\} \cdots XYZ  \right] \notag \\
    & + \cdots + \text{Tr}\left[ \rho_o BC \cdots XY\{A,Z\}  \right] - \text{Tr}\left[ \rho_o BC \cdots XYZA  \right] \label{contraction1}
\end{align}
As an example, 
\begin{align}
    \text{Tr}\left[ \rho_o ABCD  \right] = \text{Tr}\left[ \rho_o\{A,B\} CD \right] - \text{Tr}\left[ \rho_o B\{A,C\} D \right] + \text{Tr}\left[ \rho_o BC\{A,D\} \right] - \text{Tr}\left[ \rho_o BCDA \right]
\end{align}
The sign of each term containing an anticommunicator is given by $(-1)^{n-1}$, the number $n$ of exchange done. For the final term, its sign
is given by the sign of last term that contain an anticommunicator multiplied by $-1$, which will always be negative because we perform maximum
$N-1$ exchange and N is a even number. 

Now, we can verify that:
\begin{align}
    \langle j | a_k^{\dagger} e^{\beta(\Omega_0 - H_0 + \mu N)} | i \rangle 
            &= e^{\beta(\Omega_0 - E_i + \mu N_i)} \langle j |a_k^{\dagger} | i \rangle  \notag \\
    \langle j | e^{\beta(\Omega_0 - H_0 + \mu N)} a_k^{\dagger}  | i \rangle 
            &= e^{\beta(\Omega_0 - E_j + \mu N_j)} \langle j |a_k^{\dagger} | i \rangle  \notag \\
            &= e^{\beta(\Omega_0 - (E_i+\varepsilon_k) + \mu (N_i+1))} \langle j |a_k^{\dagger} | i \rangle  \notag \\
            &= e^{-\beta (\varepsilon_k - \mu)} \langle j | a_k^{\dagger} e^{\beta(\Omega_0 - H_0 + \mu N)} | i \rangle \label{rhoA}
\end{align}
so that 
\begin{equation}
    \rho_0 a_k^{\dagger} = e^{-\beta (\varepsilon_k - \mu)} a_k^{\dagger} \rho_0
\end{equation}
as well as the relation:
\begin{equation}
    \rho_0 a_k = e^{\beta (\varepsilon_k - \mu)} a_k \rho_0
\end{equation}
So that we can conclude
\begin{equation}
    A\rho_0 = e^{\lambda_A \beta (\varepsilon_k - \mu)} \rho_0 A\ \ \
    \begin{cases}
        \lambda_A = 1 , & A = a_k^{\dagger} \\
        \lambda_A = -1 , & A = a_k \\
    \end{cases}
\end{equation}
and 
\begin{align}
    \text{Tr}\left[ \rho_o BC \cdots XYZA  \right] &= \text{Tr}\left[ A \rho_o BC \cdots XYZ  \right] \notag \\
                            &= e^{\lambda_A \beta (\varepsilon_k - \mu)} \text{Tr}\left[ \rho_o ABC \cdots XYZ  \right] \notag \\
\end{align}
Since for fermion operators, their anticommunicator is a C-number, 
Eq.\ref{contraction1} becomes:
\begin{align}
    \text{Tr}\left[ \rho_o ABC \cdots XYZ  \right] &= \frac{\{A,B\}}{e^{\lambda_A \beta (\varepsilon_k - \mu)} + 1} \text{Tr}\left[ \rho_o C \cdots XYZ  \right] \notag \\
                                                & - \frac{\{A,C\}}{e^{\lambda_A \beta (\varepsilon_k - \mu)} + 1} \text{Tr}\left[ \rho_o BD \cdots XYZ  \right] \notag \\
                                                & + \cdots +  (-1)^{n-1} \frac{\{A,Z\}}{e^{\lambda_A \beta (\varepsilon_k - \mu)} + 1} \text{Tr}\left[ \rho_o BC \cdots XY  \right] \label{contraction2}
\end{align}
where $n$ is the number of exchanges of operator needed. 

Next, we calculate the term:
\begin{equation}
    \frac{\{A,B\}}{e^{\lambda_A \beta (\varepsilon_k - \mu)} + 1}
\end{equation}
\textbf{case 1} we let $A = a_k^{\dagger}$, so that $B = a_k$. we have:
\begin{align}
    \frac{\{a_k^{\dagger},a_k\}}{e^{ \beta (\varepsilon_k - \mu)} + 1} = \frac{1}{e^{\beta (\varepsilon_k - \mu)} + 1} = \langle a_k^{\dagger} a_k \rangle = \langle AB \rangle
\end{align}
\textbf{case 2} we let $A = a_k$, so that $B = a_k^{\dagger}$. we have:
\begin{align}
    \frac{\{a_k,a_k^{\dagger}\}}{e^{-\beta (\varepsilon_k - \mu)} + 1} = \frac{1}{e^{ -\beta (\varepsilon_k - \mu)} + 1} = \langle a_k a_k^{\dagger} \rangle = \langle AB \rangle
\end{align}
So that in both case, we have the relationship:
\begin{equation}
    \frac{\{A,B\}}{e^{\lambda_A \beta (\varepsilon_k - \mu)} + 1} = \langle AB \rangle
\end{equation}
and Eq.\ref{contraction2} can now be written:
\begin{align}
    \langle ABC \cdots XYZ  \rangle = & \langle AB  \rangle \langle  C \cdots XYZ \rangle  - \langle AC\rangle  \langle BD \cdots XYZ  \rangle  \notag \\
                                                & + \cdots + (-1)^{n-1} \langle AZ\rangle  \langle BC \cdots XY  \rangle \label{wick1}
\end{align}

Thus we have factorized out all the pairs of operators that contain $A$, the process can continue until the initial thermal average is completely written
as the sum of product of averages of pairs, the sign of which is determined by the number of exchanges to arrive at the specific pairing.

\subsection*{Boson}
Instead of the anticommunicator, we use communicator of boson operators:
\begin{gather}
    [ b_q^{\dagger} , b_{q'} ] = - \delta_{qq'} \notag \\  
    [ b_{q}, b_{q'}^{\dagger} ] = \delta_{qq'}  \notag
\end{gather}
so that Eq.\ref{contraction1} becomes:
\begin{align}
    &\text{Tr}\left[ \rho_o ABC \cdots XYZ  \right] \notag \\
    = &\text{Tr}\left[ \rho_o [A,B] C \cdots XYZ  \right] + \text{Tr}\left[ \rho_o BAC \cdots XYZ  \right] = \cdots \notag \\
    = &\text{Tr}\left[ \rho_o [A,B] C \cdots XYZ  \right] + \text{Tr}\left[ \rho_o B[A,C] \cdots XYZ  \right] \notag \\
    & + \cdots + \text{Tr}\left[ \rho_o BC \cdots XY[A,Z]  \right] + \text{Tr}\left[ \rho_o BC \cdots XYZA  \right] \label{contractionq1}
\end{align}
The relation Eq.\ref{rhoA} still holds, so that:
\begin{align}
    \text{Tr}\left[ \rho_o ABC \cdots XYZ  \right] &= \frac{[A,B]}{1 - e^{\lambda_A \beta (\varepsilon_q - \mu)} } \text{Tr}\left[ \rho_o C \cdots XYZ  \right] \notag \\
                                                & + \frac{[A,C]}{1 - e^{\lambda_A \beta (\varepsilon_q - \mu)} } \text{Tr}\left[ \rho_o BD \cdots XYZ  \right] \notag \\
                                                & + \cdots + \frac{[A,Z]}{1 - e^{\lambda_A \beta (\varepsilon_q - \mu)} } \text{Tr}\left[ \rho_o BC \cdots XY  \right] \label{contractionq2}
\end{align}
For bosons, we can find:
\begin{align}
    \frac{[b_q^{\dagger},b_q]}{1 - e^{ \beta (\varepsilon_q - \mu)} } &= \frac{-1}{1 - e^{\beta (\varepsilon_q - \mu)} } 
        = \langle n_q \rangle = \langle b_q^{\dagger} b_q \rangle  \notag \\
    \frac{[b_q,b_q^{\dagger}]}{1 - e^{ -\beta (\varepsilon_q - \mu)} } &= \frac{1}{1 - e^{-\beta (\varepsilon_q - \mu)} } 
        = \langle n_q + 1\rangle = \langle b_q b_q^{\dagger} \rangle  \notag \\
\end{align}
as a result 
\begin{equation}
    \frac{[A,B]}{1 - e^{ \beta (\varepsilon_q - \mu)} } = \langle AB \rangle 
\end{equation}
and the analogy of Eq.\ref{wick1} for bosons are therefore:
\begin{align}
    \langle ABC \cdots XYZ  \rangle = & \langle AB  \rangle \langle  C \cdots XYZ \rangle  + \langle AC\rangle  \langle BD \cdots XYZ  \rangle  \notag \\
                                                & + \cdots + \langle AZ\rangle  \langle BC \cdots XY  \rangle \label{wick2}
\end{align}


\end{document}
