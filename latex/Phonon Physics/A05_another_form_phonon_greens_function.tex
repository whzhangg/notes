\documentclass{article}
\usepackage{amsmath}
\usepackage[margin=0.8in]{geometry}
\usepackage{verbatim}
\usepackage{graphicx}

\begin{document}

\section{Phonon green's function in terms of $a_{qv}$ and $a_{qv}^{\dagger}$}
We previously defined the green's function as:
\begin{align}
    G(qvv';t) = \langle i | \mathcal{T}[A_{qv}(t)A_{qv'}^{\dagger}(0)]| i \rangle \notag
\end{align}
which is convenient to treat anharmonic interaction. 

In this appendix let's consider 
the green's function in terms of phonon creation and annihilation operator $a_{qv}$ and $a^{\dagger}_{qv}$:
\begin{equation}
    D(qvv';t) = -i \langle  \mathcal{T}[a_{qv}(t)a_{qv'}^{\dagger}(0)] \rangle \notag
\end{equation}
at finite temperature, taking real variable $\tau = it$, we define the temperature green's function as:
\begin{equation}
    D(qvv';\tau) = -\langle  \mathcal{T}[a_{qv}(\tau)a_{qv'}^{\dagger}(0)] \rangle \notag
\end{equation}
with the "imaginary time" Heisenburg operator:
\begin{equation}
    A(\tau) = e^{\tau/\hbar H} A e^{-\tau/\hbar H} \notag 
\end{equation}
we have:
\begin{align}
    D(qvv';\tau) = 
    \begin{cases}
        -\frac{1}{Z} Tr\left\{ e^{-(\beta-\tau/\hbar)H} a_{qv} e^{-\tau H/\hbar } a_{qv'}^{\dagger} \right\}, &\tau > 0 \\
        -\frac{1}{Z} Tr\left\{ e^{-\beta H} a_{qv'}^{\dagger} e^{\tau H/\hbar } a_{qv} e^{- \tau H/\hbar } \right\}, &\tau < 0 \\
    \end{cases}
\end{align}
with $-\beta\hbar < \tau < \beta \hbar$,
$D{qvv';\tau}$ has a period of $\beta\hbar$, we calculate the free phonon green's function:
\textbf{for $\tau > 0$:}
\begin{align}
    D_{qvv';\tau} &= -\frac{1}{Z} \sum_{ij} 
                    \langle i | e^{-(\beta-\tau/\hbar)H} a_{qv} e^{-\tau/\hbar H} | j\rangle \langle j | a^{\dagger}_{qv'} | i \rangle \notag \\
                  &= -\delta_{vv'} e^{-\tau\omega_{qv}} \frac{1}{Z} \sum_i e^{-\beta \varepsilon_i} \langle i | a_{qv} a^{\dagger}_{qv} | i\rangle \notag \\
                  &= -\delta_{vv'} e^{-\tau\omega_{qv}} \langle n_{qv} + 1 \rangle 
\end{align}
\textbf{for $\tau < 0$:}
\begin{align}
    D_{qvv';\tau} &= -\frac{1}{Z} \sum_{ij} 
                    \langle i | e^{-\beta H} a^{\dagger}_{qv} | j \rangle \langle j | e^{\tau/\hbar H}a_{qv'}e^{-\tau/\hbar H} | i \rangle \notag \\
                  &= -\delta_{vv'} e^{-\tau\omega_{qv}} \frac{1}{Z} \sum_i e^{-\beta \varepsilon_i} \langle i | a^{\dagger}_{qv} a_{qv} | i\rangle \notag \\
                  &= -\delta_{vv'} e^{-\tau\omega_{qv}} \langle n_{qv} \rangle 
\end{align}
So that 
\begin{align}
    D(qvv';\tau) = -\delta_{vv'}
    \begin{cases}
        e^{-\tau\omega_{qv}} \langle n_{qv} + 1 \rangle, &\tau > 0 \\
        e^{-\tau\omega_{qv}} \langle n_{qv} \rangle , &\tau < 0 \\
    \end{cases}
\end{align}

We can verify the periodicity as:
\begin{align}
    D(qv;\tau) &= -\frac{e^{-\tau\omega_{qv}}}{e^{\beta\hbar\omega_{qv}}- 1} \ \ (\tau < 0) \notag \\
    D(qv;\tau+\beta\hbar) &= -\frac{e^{-(\tau+\beta\hbar)\omega_{qv}} e^{\beta\hbar\omega_{qv}}}{e^{\beta\hbar\omega_{qv}}- 1} 
                         = -\frac{e^{-\tau\omega_{qv}}}{e^{\beta\hbar\omega_{qv}}- 1} \notag
\end{align}

The fourier transformation is given similar to the previous case:
\begin{gather}
        D(qvv';\tau) = \sum_{n = - \infty}^{\infty} D(qvv';i\omega_n) e^{i\omega_n\tau} \\
        D(qvv';i\omega_n) = \frac{1}{\beta\hbar} \int_{0}^{\beta\hbar} D(qvv';\tau) e^{-i\omega_n\tau} d\tau
\end{gather}
and $\omega_n$ take discrete value $\omega_n = 2n\pi/\beta\hbar$, we have:
\begin{align}
    D(qvv';i\omega_n) &= -\frac{1}{\beta\hbar} \int_{0}^{\beta\hbar} e^{-\tau\omega_{qv}} \langle n_{qv} + 1 \rangle e^{-i\omega_n\tau} d\tau \notag \\
                    &= -\frac{1}{\beta\hbar} \frac{e^{\beta\hbar\omega_{qv}}}{e^{\beta\hbar\omega_{qv}}- 1} \int_0^{\beta\hbar} e^{-(\omega_{qv} + i\omega_n)\tau} d\tau \notag \\
                    &= \frac{1}{\beta\hbar} \frac{e^{\beta\hbar\omega_{qv}}}{e^{\beta\hbar\omega_{qv}}- 1} \frac{e^{-(\omega_{qv} + i\omega_n)\beta\hbar} - 1}{\omega_{qv} + i\omega_n} \notag \\
                    &= -\frac{1}{\beta\hbar} \frac{1}{\omega_{qv} + i\omega_n}\frac{e^{\beta\hbar\omega_{qv}}- e^{-i\omega_n \beta\hbar}}{e^{\beta\hbar\omega_{qv}}- 1}  \notag \\
                    &= -\frac{1}{\beta\hbar} \frac{1}{\omega_{qv} + i\omega_n}
\end{align}
where in the last equality, we use the fact that $\omega_n = 2n\pi/\beta\hbar$ and $e^{-i\omega_n \beta\hbar} = 1$

\subsection*{Retarded green's function}
The general \textbf{retarded green's function} is defined by:
\begin{gather}
    G^R_{mn}(t,t') = 
    \begin{cases}
        -i \theta(t-t') \langle [a_m(t), a_n^{\dagger}(t')] \rangle & \text{Boson} \notag \\
        -i \theta(t-t') \langle \{a_m(t), a_n^{\dagger}(t')\} \rangle & \text{Fermion}
    \end{cases}
\end{gather}
where $\theta(t) = 0$ if $t < 0$.
Taking $t' = 0$ and focus on the phonon case, we have:
\begin{equation}
        G^R_{qvv'}(t) = 
        \begin{cases}
            -i \langle a_{qv}(t)a_{qv'}^{\dagger} - a_{qv'}^{\dagger} a_{qv}(t)\rangle & t > 0 \notag \\
            0 & t < 0
        \end{cases}
\end{equation}
We can follow the above method to calculate the retarded phonon green's function. The result is 
as:
\begin{align}
    \langle a_{qv}(t)a_{qv'}^{\dagger}\rangle &= e^{-it\omega_qv} \langle n_{qv} + 1 \rangle \notag \\
    \langle a_{qv'}^{\dagger}a_{qv}(t)\rangle &= e^{-it\omega_qv} \langle n_{qv} \rangle \notag
\end{align}
so that the retarded green's function is:
\begin{equation}
    G^R_{qvv'}(t) = -i \theta(t) \delta_{vv'} e^{-i\omega_{qv}t}
\end{equation}
To obtain the retarded green's function in the frequency domain:
\begin{equation}
    G^R_{qv}(\omega) = \int_{-\infty}^{\infty} G^R_{qvv'}(t) e^{-i\omega t} dt = \int_{0}^{\infty} G^R_{qvv'}(t) e^{-i\omega t} dt 
\end{equation}
since $G^R = 0$ for $t < 0$. 
\begin{align}
    G^R_{qv}(\omega) &= -i \int_{0}^{\infty} e^{-i(\omega + \omega_{qv})t} dt \notag \\
                     &= -i \lim_{\eta \to 0^+} \int_{0}^{\infty} e^{-[\eta + i(\omega + \omega_{qv})]t} dt \notag \\
                     &= -i \lim_{\eta \to 0^+} \frac{1}{\eta + i(\omega + \omega_{qv})} \\
                     &= -\frac{1}{\omega + \omega_{qv}}
\end{align}
which agree with the temperature green's function by changing $\omega_n$ with $\omega + i\eta$ and multiply $\beta\hbar$, which comes from
the definition of fourier transform.
The inverse transform is given by:
\begin{equation}
    G^R(t) = \frac{1}{2\pi} \int_{-\infty}^{\infty} G^R(\omega) e^{i\omega t} d\omega
\end{equation}

\end{document}
