\documentclass{article}
\usepackage{amsmath}
\usepackage[margin=0.8in]{geometry}
\usepackage{verbatim}
\usepackage{graphicx}

\begin{document}

\section{Wigner function}
%The expectation value of an operator $A$ of a quantum 
%state $|\psi\rangle $ is given by:
%\begin{equation}
%    \langle A \rangle = \int \psi^*(x) A \psi(x) dx
%\end{equation}
%with $\psi(x) = \langle x | \psi \rangle$. 

Define the transformation, called \emph{Weyl transformation} from an operator $A$ to a function $A(x,p)$:
\begin{align}
    \tilde{A}(x,p)  &= \int e^{-ipy/\hbar} \langle x + \frac{y}{2} | A | x - \frac{y}{2} \rangle dy \\
                    &= \int e^{ ixu/\hbar} \langle p + \frac{u}{2} | A | p - \frac{u}{2} \rangle du
\end{align}
where $\langle x | A | x' \rangle$ and $\langle p | A | p' \rangle$ denotes the matrix element of $A$ 
in position or momentum base, and both integral give the same expression $\tilde{A}(x,p)$. 
Suppose the operator $A$ is only a function of $x$, than the Weyl transformationwill give:
\begin{align}
    \tilde{A} &= \int e^{-ipy/\hbar} \langle x + \frac{y}{2} | A | x - \frac{y}{2} \rangle dy \\
              &= \int e^{-ipy/\hbar} \langle x + \frac{y}{2} | A | x - \frac{y}{2} \rangle \delta_{y=0} dy \\
              &= \langle x | A | x \rangle = A(x)
\end{align}
The same will be true if an operator is purely a function of momentum $p$. However, this is not true 
if an operator is a function of $x,p$ at the same time.
It can be shown that:
\begin{equation}
    \text{Tr}[AB] = \frac{1}{\hbar} \int \int \tilde{A}(x,p) \tilde{B}(x,p) dx dp
\end{equation}
define the density operator $\rho$ so that $\text{Tr}[\rho A] = \langle A \rangle$, we thus have:
\begin{equation}
    \langle A \rangle = \frac{1}{\hbar} \int \int \tilde{\rho}(x,p) \tilde{A}(x,p) dx dp
\end{equation}
It is therefore convenient to define a function:
\begin{align}
    W(x,p)  &= \frac{1}{\hbar} \int e^{-ipy/\hbar} \langle x + \frac{y}{2} | \rho | x - \frac{y}{2} \rangle dy \\
            &= \frac{1}{\hbar} \int e^{ ixu/\hbar} \langle p + \frac{u}{2} | \rho | p - \frac{u}{2} \rangle du
\end{align}
This is called \emph{Wigner function}. Now, we can find expectation value of an operator by 
integrating over phase space $x,p$, similar to classical statistic mechanics:
\begin{align}
    \langle A \rangle = \int \int W(x,p) \tilde{A}(x,p) dx dp
\end{align}
Intergrating over one phase space coordinates gives the probability distribution of another:
\begin{equation}
    \langle A \rangle (x) = \int W(x,p) \tilde{A}(x,p) dp
\end{equation}
Wigner function is real and normalized:
\begin{equation}
    \int\int W(x,p) dx dp = 1
\end{equation}
But it is not always positive, therefore, it cannot be interpreted as a classical probability density.

\end{document}