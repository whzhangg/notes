\documentclass{article}
\usepackage{amsmath}
\usepackage[margin=0.8in]{geometry}
\usepackage{verbatim}
\usepackage{graphicx}

\begin{document}

\section{Evaluating the Matsubara Sum}
Evaluating Matsubara sum involve calculating:
\begin{equation}
    \sum_n f(i\omega_n) \notag
\end{equation}
Suppose we have an integral,
\begin{equation}
    \int_c f(i\omega) n(i\omega) d(i\omega) \notag
\end{equation}
where curve $c$ cover all the singularity of the function $f(i\omega)$ and $n(i\omega)$.
If $f(i\omega) \to 0$ at infinity, the integral will vanish:
\begin{equation}
    \int_c f(i\omega) n(i\omega) d(i\omega) = \sum_n Res[n(i\omega_n)] f(i\omega_n) + \sum_p Res[f(i\omega_p)] n(i\omega_p) = 0 \notag
\end{equation}
where the summation run through the poles of function $n(i\omega)$ and $f(i\omega)$. 
If we take $n(i\omega)$ to be the Bose-Einstein distribution function, we have:
\begin{gather}
    n(\omega) = \frac{1}{e^{\beta\hbar\omega} - 1} \notag \\
    n(i\omega) = \frac{1}{e^{i\beta\hbar\omega} - 1} \notag
\end{gather}
$n(i\omega)$ has poles at $\omega_n = \frac{2\pi n}{\beta\hbar}$, the residual of $n(i\omega)$ at those poles 
are: $Res[n(i\omega_n)] = -i/\beta\hbar$. We have the result:
\begin{equation}
    \sum_n f(i\omega_n) = \frac{\beta\hbar}{i} \sum_{p} Res[f(i\omega_p)] n(i\omega_p)
\end{equation}
so that we need to sum over all the residuals in the function $f(i\omega)$.

\end{document}
