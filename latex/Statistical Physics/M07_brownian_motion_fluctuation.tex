\documentclass{article}
\usepackage{amssymb, amsmath, amsthm}
\usepackage[margin=1in]{geometry}
\usepackage{verbatim}
\usepackage{graphicx}
\usepackage{hyperref} % \url \href
\usepackage{docmute}

\newcommand{\pfrac}[2]{\frac{\partial #1}{\partial #2}}
\newcommand{\rms}{\text{rms}}
% \newcommand{\braket}[1]{\langle #1 \rangle}
% \renewcommand{\H}{\mathcal{H}}

\begin{document}

\section{Brownian motion and fluctuations}
\subsection{Brownian motion}
To study the Brownian
\footnote{Blundell, P390, we use the notion $\bar{x}$ for time average here, 
different from $\langle x \rangle$ used in the book}, 
we find the solution to the equation of motion (Langevin equation)
of a particle moving with random force:
\begin{equation}
    m\dot{v} = -\alpha v + F(t) \label{brownian_eom}
\end{equation}
where $\alpha$ is a damping constant due to friction. $F(t)$ is a random force with 
time average $\bar{F} = 0$.
In the absense of the random force, the solution will simply be:
\begin{equation}
    v(t) = v(0)\exp[-t/(m\alpha^{-1})]
\end{equation}
as an exponentially decrease of the velocity due to fraction.

To solve Eq.\ref{brownian_eom}, we multiply both size with $x$ and use the identity:
\begin{equation}
    \frac{d(x\dot{x})}{dt} = x \ddot{x} + \dot{x}^2
\end{equation}
and the equation of motion become:
\begin{equation}
    m \left( \frac{d(x\dot{x})}{dt} - m \dot{x}^2 \right) = - \alpha x \dot{x} + x F(t)
\end{equation}
We calculate the time average with the formula:
$ \overline{x} = \int_{t_1}^{t_2} x(t) dt / (t_2 - t_1)$
and the above equation becomes:
\begin{equation}
    m \frac{d}{dt} \overline{x\dot{x}} = m \overline{ \dot{x}^2 } 
    - \alpha \overline{ x \dot{x} } + \overline{ x F(t) }
\end{equation}
The final term equation to zero because the force is random. Assuming the 
system is in an thermal equilibrium and is ergodic, we can approximate:
\begin{equation}
    m \overline{ \dot{x}^2 } = m \langle \dot{x}^2\rangle = k_B T
\end{equation}
The final equation can be written:
\begin{equation}
    m \frac{d}{dt} \overline{x\dot{x}} = k_B T
    - \alpha \overline{ x \dot{x} } 
\end{equation}
The solution for $\overline{ x \dot{x} }$ is:
\begin{equation}
    \overline{ x \dot{x} } = C e^{-\alpha t/m} + \frac{k_B T}{\alpha}
\end{equation}

Now, we apply a boundary condition by choosing $x = 0$ when $t = 0$ 
(the motion of this molecular does not depend on this, the motion of the molecular
is the same before and after $t=0$, we are only choosing the origin $x = 0$ at this 
moment). Writing the ensemble average instead of time average, We obtain:
\begin{equation}
   \langle x \dot{x} \rangle = \overline{ x \dot{x} } = \frac{k_B T}{\alpha}( 1 - e^{-\alpha t/m})
\end{equation}
Using the identity $dx^2 / dt = 2x \dot{x}$, we have the result for the 
position at later time t:
\begin{equation}
    \langle x^2 \rangle = \overline{x^2} = \frac{2 k_B T}{\alpha}( t - \frac{m}{\alpha} e^{-\alpha t/m})
\end{equation}

For $t \gg m/a$, we have $ \langle x^2 \rangle = \frac{2k_B Tt}{\alpha} $. Using the diffusion 
constant $D$ as $ \langle x^2 \rangle = 2 D t $, we have: $ D = k_B T /\alpha $. This example is 
a example of fluctuation-dissipation theorem. Since $\alpha$ (dissipation) and $\langle x^2 \rangle$ 
(fluctuations) are inversely related.

\subsection{Fluctuations}
We consider how a macroscopic properties $x$ deviate from the average
value. If a system is fixed at a given energy in a microcanonical ensemble, 
We write number of microstates by $\Omega(x,E)$. The entropy of the system
is thus:
\begin{equation}
    S(x,E) = k_B \ln \Omega(x,E)
\end{equation}
The probability of the system with property $x = x_i$ is then:
\begin{equation}
    p(x_i) \propto \Omega(x_i,E) = e^{S(x_i,E)/k_B T}
\end{equation}
The mean value will be given by condition $ (\partial S(x) /\partial x) |_{x_0} = 0$. If we 
Taylor expand $S(x,E)$ around $x_0$:
\begin{equation}
    S(x) = S(x_0) + \frac{1}{2}\left( \frac{\partial^2 S}{\partial x^2} \right)_{x_0} (x - x_0)^2 + \cdots
\end{equation}
The probability function is then a Gaussian:
\begin{equation}
    p(x) \approx \exp\left( - \frac{(\Delta x)^2}{2 \langle (\Delta x)^2 \rangle } \right)
\end{equation}
with the divation:
\begin{equation}
    \langle (\Delta x)^2 \rangle = - k_B / \left( \frac{\partial^2 S}{\partial x^2} \right)_{x_0}
\end{equation}

% \subsection{Availability}
% We can extend the above method to consider other ensembles, consider a ensemble 
% described by temperature, pressure and chemical potential. The change in entropy $dS$
% is related to the change in internal energy, volume and particle number by:
% I don't understand

\subsection{Kramers-Kronig relations}
We define the response function $\chi(t)$ as:
\begin{equation}
    \langle x(t) \rangle_f = \int_{-\infty}^{\infty} \chi(t-t') f(t') dt'
\end{equation}
We require the response function to be causal:
\begin{equation}
    \chi(t) = y(t)\theta(t)
\end{equation}
$y(t)$ is a function that coincide with $\chi(t)$ for $ t > 0 $ and 
we require $y(t) = - \chi(|t|)$ at $t < 0$. This definition ensures that the 
fourier transform of $y(t)$ is purely imaginary:
\begin{align}
    & \int_{-\infty}^{\infty} dt e^{-i\omega t} y(t) \\
    = & \int_{-\infty}^{\infty} dt [\cos(\omega t) - i \sin(\omega t)] y(t) \\
    = & -2i \int_{0}^{\infty} dt \sin(\omega t) y(t)
\end{align}
Here, the fourier transformation is defined to be:
\begin{gather}
    x(\omega) = \int_{-\infty}^{\infty} dt e^{-i\omega t} x(t) \notag \\
    x(t) = \frac{1}{2\pi} \int_{-\infty}^{\infty} d\omega e^{i\omega t} x(\omega) \notag \\
\end{gather}
and we have the relationship:
\begin{equation}
    \langle x(\omega) \rangle _f = \chi(\omega) f(\omega)
\end{equation}
The Fourier transformation of the response function is:
\begin{align}
    \chi(\omega) = & \int_{-\infty}^{\infty}  e^{-i\omega t}\theta(t) y(t) dt \notag \\
    = & \int_{-\infty}^{\infty}  e^{-i\omega t} \theta(t)  \left[ \frac{1}{2\pi}  \int_{-\infty}^{\infty} d\omega' e^{i\omega' t} y(\omega') \right] dt \notag \\
     = & \frac{1}{2\pi} \int_{-\infty}^{\infty} d\omega' \theta(\omega' - \omega) y (\omega') 
     \label{response_transform}
\end{align}

If the $\theta$ function is defined so that 
\begin{equation}
    \theta(t) = \lim_{\epsilon \to 0} 
    \begin{cases}
        e^{-\epsilon t}; & t > 0\\
        0; & t < 0
    \end{cases}
\end{equation}
Its fourier transformation is then given by:
\begin{equation}
    \theta(\omega) = \int_0^{\infty} dt e^{-i\omega t} e^{-\epsilon t}
    = \frac{1}{i\omega + \epsilon} 
    = \frac{\epsilon}{\omega^2 + \epsilon^2} - \frac{i\omega}{\omega^2 + \epsilon^2}
\end{equation}
taking the limit, we have:
\begin{equation}
    \theta(\omega) = \pi \delta(\omega) - \frac{i}{\omega}
\end{equation}
putting it into Eq.\ref{response_transform}, We have, for $\chi(\omega)$:
\begin{equation}
    \chi(\omega) = \frac{1}{2} y(\omega) - \frac{i}{2\pi} \mathcal{P} \int_{-\infty}^{\infty} 
    \frac{y(\omega')d\omega'}{\omega' - \omega}
    = \chi'(\omega) + i\chi''(\omega)
\end{equation}
Since $y(\omega)$ is purely real, we have the relationship:
\begin{gather}
    i\chi''(\omega) = \frac{1}{2} y(\omega) \\
    \chi'(\omega) = - \frac{i}{2\pi} \mathcal{P} \int_{-\infty}^{\infty} \frac{y(\omega')d\omega'}{\omega' - \omega}
\end{gather}
This gives the \textbf{Kramers-Kronig relations}:
\begin{equation}
    \chi'(\omega) = \frac{1}{\pi} 
    \mathcal{P} \int_{-\infty}^{\infty} \frac{ \chi''(\omega) }{\omega' - \omega}d\omega'
\end{equation}

Kramers-Kronig relation can also be derived from the fact that the response
function, now extending $\chi(\omega)$ to $\chi(z)$, is analytic in the upper
plane. The relationship follow directly from this condition. See \emph{Quantum Theory of 
Electron Liquid Page 127, 128}

As an example, we consider the response function of a damped harmonic oscillator 
with equation of motion:
\begin{equation}
    m \ddot{x} + \alpha \dot{x} + kx = f(t)
\end{equation}
writting $\omega_0^2 = k/m$ and $\gamma = \alpha/m$, we have:
\begin{equation}
    \ddot{x} + \gamma \dot{x} + \omega_0^2x = f/m
\end{equation}
Fourier transformation give the result:
\begin{equation}
    \chi(\omega) = \frac{x(\omega)}{f(\omega)} 
    = \frac{1}{m} \left[ \frac{1}{\omega_0^2 - \omega^2 - i\omega \gamma} \right]
\end{equation}
The imaginary part of the response function is given by:
\begin{equation}
    \chi''(\omega)  
    = \frac{1}{m} \left[ \frac{\omega \gamma}{(\omega^2 - \omega_0^2)^2 + (\omega \gamma)^2} \right]
\end{equation}
which vanish at $\omega \to 0$. The statis susceptibility is given by:
\begin{equation}
    \chi'(0) = \frac{1}{m\omega_0^2}
\end{equation}


\subsection{Correlation functions}
We define the autocorrelation function $C_{xx}(t)$ as a time average:
\begin{gather}
    C_{xx}(t) = \langle x(0)x(t) \rangle = \int_{-\infty}^{\infty} x^*(t')x(t'+t) dt' \\
    C_{xx}(\omega) = \int_{-\infty}^{\infty} e^{-i\omega t}\langle x(0)x(t) \rangle dt
\end{gather}
with $t=0$ we have the average of squred:
\begin{equation}
    \langle x^2 \rangle = 
    \frac{1}{2\pi} \int_{-\infty}^{\infty} C_{xx}(\omega) d\omega
\end{equation}

Note that the standard deviation can be written by: 
$\sigma^2 = \langle x^2 \rangle - \langle x \rangle ^2 $, if we take the average value
as $0$, than $\langle x^2 \rangle$ directly measure the fluctuation.

\subsection{Fluctuation-dissipation theorem}
We consider an example of harmonic system, its energy is given by: $E = kx^2 /2 $, where
$k$ is the string constants and $x$ is the amplitude of the vibration. In canonical 
ensemble, the probability of finding the system with amplitude $x$ is then given by:
\begin{equation}
    p(x) = \mathcal{N}' e^{-\beta(kx^2/2)}
\end{equation}
which is an Gaussian form with mean $\langle x \rangle = 0$ and deviation
$\langle x^2 \rangle = \sigma^2 = 1/(\beta k)$. 

Now, if we apply a force $f$ on the oscillator, its energy will be modified:
$E = kx^2/2 - xf$. The probability distribution became:
\begin{equation}
    p(x) = \mathcal{N}' e^{-\beta(kx^2/2 - xf)} = \mathcal{N}'' e^{-\frac{\beta k}{2} (x - \frac{f}{k})^2 }
\end{equation}
where we completed the squre in the exponentially and add the additional 
factor into $\mathcal{N}''$. 

This is an Gaussian distribution with mean value: $\langle x \rangle_f = f/k$ and 
deviation $1/(\beta k) = \langle x^2 \rangle$. (where $\langle x^2 \rangle$ refer to the 
deviation of the undisturbed system). 
We have the relationship:
\begin{equation}
    \frac{\langle x \rangle_f}{\langle x^2 \rangle} = \beta f
\end{equation}
The mean value of $x$ can also be expressed by $\langle x \rangle_f = \chi(0)' f $
(The imaginary part $\chi''(0) = 0$). Then we have the result:
\begin{equation}
    \langle x^2 \rangle = k_B T \chi(0)' = 
    k_BT\int_{-\infty}^{\infty} \frac{d\omega'}{\pi} \frac{\chi''(\omega')}{\omega'} \label{fdt}
\end{equation}
This is the statement of the fluctuations dissipation theorem, connecting the 
fluctuation (autocorrelation function) to the imaginary part of the response function.

\end{document}
