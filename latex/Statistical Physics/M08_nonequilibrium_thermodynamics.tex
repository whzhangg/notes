\documentclass{article}
\usepackage{amssymb, amsmath, amsthm}
\usepackage[margin=1in]{geometry}
\usepackage{verbatim}
\usepackage{graphicx}
\usepackage{hyperref} % \url \href
\usepackage{docmute}

\newcommand{\pfrac}[2]{\frac{\partial #1}{\partial #2}}
\newcommand{\rms}{\text{rms}}
% \newcommand{\braket}[1]{\langle #1 \rangle}
% \renewcommand{\H}{\mathcal{H}}

\begin{document}

\section{Non-equilibrium thermodynamics}
\subsection{Irreversibility of transport process}
We have the following relationship between local entropy density, local energy $u$, particle density and 
charge density:
\begin{gather}
    ds = \frac{1}{T} du - \frac{\mu}{T} dn + \frac{\phi}{T} d\rho_e \\
    ds = \sum_k \phi_k d\rho_k
\end{gather}
where in the second equation, we generalize all possible contribution to a local entropy change 
with a generalized potential $\phi_k = \partial s / partial \rho_k$ and a generalized density $\rho_k$. 
(For example, with $\rho_k = u$ we have $\phi_k = 1/T$). 
Since the generalized density are conserved in the system, we have the continuous equation relating the 
change of $\rho$ with respect to the current:
\begin{equation}
    \pfrac{\rho_k}{t} + \nabla \cdot \mathbf{J}_{k} = 0
\end{equation}

We can calculate the local change in entropy and the flow of entropy:
\begin{gather}
    \pfrac{s}{t} = \sum_k \phi_k \pfrac{\rho_k}{t} \\
    \mathbf{J}_s = \sum_k \phi_k \mathbf{J}_{k}
\end{gather}
We find the following result:
\begin{align}
    \Sigma &= \pfrac{s}{t} +\nabla \cdot  \mathbf{J}_s \\
    & = \sum_k \phi_k \pfrac{\rho_k}{t} + \nabla \cdot \left(\sum_k \phi_k \mathbf{J}_{k} \right)\\
    & = \sum_k \phi_k \pfrac{\rho_k}{t} + \sum_k (\nabla  \phi_k) \mathbf{J}_{k} + \sum_k \phi_k (\nabla \cdot \mathbf{J}_{k}) \\
    & = \sum_k \phi_k \pfrac{\rho_k}{t} + \sum_k (\nabla  \phi_k) \mathbf{J}_{k} - \sum_k \phi_k \pfrac{\rho_k}{t} \\
    & = \sum_k \nabla \phi_k \mathbf{J}_{k}
\end{align}
Which gives the generation rate of entropy per unit volume.

We further consider the current of the generalized density to be a linear response to the 
gradient to the respective potential:
\begin{equation}
    \mathbf{J}_i = \sum_j L_{ij} \nabla \phi_j \label{linearresponse}
\end{equation}
where index $i,j$ denote different components.
% and $L$ is called kinetic coefficient. 

As an example, for thermal conduction $\mathbf{J}_u = - \kappa \nabla T$, we have 
the $\mathbf{J}_u = \kappa T^2 \nabla (1/T)$ for the potential $1/T$. $L_uu = \kappa T^2$.
Using Eq.\ref{linearresponse}, we obtain the entropy generation rate:
\begin{align}
    \Sigma &= \sum_k \nabla \phi_k \sum_j L_{kj} \nabla \phi_j \notag \\
        & = \sum_{kj} \nabla \phi_k L_{kj} \nabla \phi_j
\end{align}

For the local entropy, we require it to increase monotonically in an irreversible process. Therefore,
$\Sigma > 0$ and $L$ is positive definitive.

\subsection{Onsager's reciprocal relation}
We consider a system that is near an equilibrium state, we define the 
variable $\alpha_k = \rho_k - \rho_k^0$ the departure of the $k^{th}$ density
variable. Writing $\alpha = (\alpha_1, \cdots, \alpha_m)$, we can write 
the probability distribution:
\begin{gather}
    P(\alpha) \propto e^{\Delta S/k_B} \\
    \Delta S = -\frac{1}{2} \sum_{ij} g_{ij} \alpha_i \alpha_j \\
    g_{ij} = \left( \frac{\partial^2S}{\partial \alpha_i \partial \alpha_j} \right)_{\alpha=0}
\end{gather}
By definition, we have $g_{ij} = g_{ji}$. We now make the definition:
\begin{equation}
    \Lambda_i = - \pfrac{S}{\alpha_i} = \sum_k g_{ik} \alpha_k \label{conjugate_alpha}
\end{equation}
We call variable $\Lambda_i$ thermodynamic conjugate to $\alpha_i$
\footnote{see Landau, section 110, 111, 118, 119, 120 and last part of 125}. 

We have the relationship:
\begin{equation}
    \pfrac{\ln P}{\alpha_i} = \frac{1}{k_B} \frac{\partial S}{\partial \alpha_i} 
    = - \frac{1}{k_B} \Lambda_i \label{tmp}
\end{equation}

We can derived the following relationship:
\begin{align}
    \left\langle \pfrac{S}{\alpha_i} \alpha_j \right\rangle 
    &= k_B \left\langle \pfrac{\ln P}{\alpha_i} \alpha_j \right\rangle \notag \\
    &= k_B \int \pfrac{\ln P}{\alpha_i} \alpha_j P(\alpha) d\alpha \notag \\
    &= k_B \int \pfrac{P}{\alpha_i} \alpha_j  d\alpha \notag \\
    &= k_B \left( \int [P\alpha_j]_{\alpha_j = -\infty}^{\infty}  d\alpha' - \int \pfrac{\alpha_j}{\alpha_i} P d\alpha \right)\notag \\
\end{align}
where we integrate by parts for variable $d\alpha_i$ and $d\alpha'$ removes $d\alpha_i$ from $d\alpha$. The 
first term is 0 since $P(\alpha_j \to \infty) = 0$. We have the final relationship:
\begin{gather}
    \left\langle \pfrac{S}{\alpha_i} \alpha_j \right\rangle  = - k_B \int \pfrac{\alpha_j}{\alpha_i} P d\alpha = -k_B \delta_{ij} \\
    \left\langle \Lambda_i \alpha_j \right\rangle = k_B \delta_{ij} \label{derive} \\
\end{gather}
where we use $\partial \alpha_j / \partial \alpha_i = \delta_{ij}$ and the fact that $P$ is normalized to 1.

We also then find the relationship:
\begin{gather}
    \left\langle \Lambda_i \Lambda_j \right\rangle = \sum_k g_{jk} \left\langle \Lambda_i \alpha_k \right\rangle 
    = k_B g_{ij} \\
    \left\langle \Lambda_i \alpha_j \right\rangle = \sum_k g_{ik} \left\langle \alpha_i \alpha_j \right\rangle 
    = k_B \delta_{ij} \\
    \left\langle \alpha_i \alpha_j \right\rangle = k_B g^{-1}_{ij}
\end{gather}
with $g^{-1}$ the matrix inverse of $g$. 
\newline
\textbf{Correlation of fluctuation in time} 

If quantity deviate from its equilibrium value, it will tend to reach 
the equilibrium state again. The rate of change $\dot{\alpha}$ will in general 
be a function of the deviation $\alpha$ itself. 
If we now expand up to linear term:
\begin{equation}
    \dot{\alpha} = - \lambda \alpha
\end{equation}
$\lambda$ is a positive constant. In the case of multivariable, we 
have:
\begin{equation}
    \dot{\alpha_i} = - \sum_{j} \lambda_{ij} \alpha_j  \label{recover}
\end{equation}

We now make an important assumption of \textbf{microscopic reversibility}, which 
states\footnote{this is a statement of time reversible symmetry, in some case, such as 
magnetic field, time reversible symmetry is no longer true}:
\begin{align}
    \langle \alpha_i(0)\alpha_j(t) \rangle &= 
    \langle \alpha_i(0)\alpha_j(-t) \rangle \notag \\ 
    &= \langle \alpha_i(t)\alpha_j(0) \rangle
\end{align}
We thus find:
\begin{gather}
    \langle \alpha_i(0)\alpha_j(t) \rangle - \langle \alpha_i(0)\alpha_j(0) \rangle  
    = \langle \alpha_i(t)\alpha_j(0) \rangle - \langle \alpha_i(0)\alpha_j(0) \rangle   \notag \\
    \langle \alpha_i \dot{\alpha}_j \rangle = \langle \dot{\alpha}_i \alpha_j \rangle \label{microreverse}
\end{gather}
where we divided time $t$ and take the limit $t \to 0$.

Using equation Eq.\ref{conjugate_alpha} and Eq.\ref{recover}, we find:
\begin{gather}
    \dot{\alpha}_i = - \sum_{j} \gamma_{ij} \Lambda_j \label{recover2}\\
    \gamma_{ij} = \sum_k \lambda_{ik} g^{-1}_{kj}
\end{gather}
$\gamma_{ij}$ is called \emph{kinetic coefficients}. 
We ubstituting the above equation to Eq.\ref{microreverse}, we have:
\begin{equation}
    \sum_{k} \gamma_{jk} \left\langle \alpha_i \Lambda_k \right\rangle
    = \sum_{k'} \gamma_{ik'} \left\langle \Lambda_k' \alpha_j \right\rangle
\end{equation}
Using Eq.\ref{derive}, we have:
\begin{equation}
    \gamma_{ij} = \gamma_{ji}
\end{equation}
we obtain the \textbf{Onsager reciprocal relations}, which gives the symmetry of the 
kinetic coefficients.

\subsection{Connection between response function and kinetic coefficients}
From fluctuation dissipation theory, we have the relationship between statistic response 
and fluctuation, as given by Eq.\ref{fdt}, generalizing to multivariable:
\begin{equation}
    \langle \alpha_i \alpha_j \rangle = k_BT \chi_{ij}(0) = k_B g^{-1}_{ij}
\end{equation}
where $\chi_{ij}$ is the response function of variable $\alpha_i$ 
with a force coupled to variable $\alpha_j$. The average value of deviation is given by:
\begin{gather}
    \langle \alpha_i \rangle = \sum_{ij} \chi_{ij}(0) f_j = \frac{1}{T} \sum_{ij} g^{-1}_{ij} f_j \label{multilinear}\\
    \langle \Lambda_i \rangle = f_i / T
\end{gather}

The recovery of variable $\alpha$ is then governed by, similar to Eq.\ref{recover2}:
\begin{align}
    \dot{\alpha}_i &= - \sum_j \gamma_{ij} (\Lambda_i - f_i / T) \notag \\
    &= - \sum_j \gamma_{ij} (\sum_k g_{ik} \alpha_k - f_i / T) 
\end{align}
With Eq.\ref{multilinear}, this equation connects $\chi_{ij}$, $\gamma_{ij}$ and $g_{ij}$.
Since $\gamma_{ij} = \gamma_{ji}$ by $g_{ij} = g_{ji}$ (by definition), we thus have
\begin{equation}
    \chi_{ij} = \chi_{ji}
\end{equation} 
when the response function is the transport function $L_{ij}$, we have:
\begin{equation}
    L_{ij} = L_{ji}
\end{equation}

\end{document}
