\documentclass{article}
\usepackage{amssymb, amsmath, amsthm}
\usepackage[margin=1in]{geometry}
\usepackage{verbatim}
\usepackage{graphicx}
\usepackage{hyperref} % \url \href
\usepackage{docmute}

\newcommand{\pfrac}[2]{\frac{\partial #1}{\partial #2}}
\newcommand{\rms}{\text{rms}}
% \newcommand{\braket}[1]{\langle #1 \rangle}
% \renewcommand{\H}{\mathcal{H}}

\begin{document}

\section{Transport properties in Ideal gas}

\subsection{Viscosity}
Viscosity $\eta$ measures th resistance of a liquid (gas) to deform by shear stress. Suppose we sandwich fluid between 
two infinitly large plate. We fix the bottom plate and apply some force $F$ to the top plate. The fraction between the fluid and 
the top plate will accelerate the fluid near it, the the momentum is passed down inside the liquid through internal interaction (collision)

Now, we wait for long enough time until the whole system is in equilibrium. The top plate will reach some final speed $u$ and 
the fluid near the top plate will have the same macroscopic speed (condition of equilibrium). The bottom plate and the fluid near the bottom 
will be stationary (An external force is necessary to keep the bottom plate stationary). 
Therefore, a velocity gradient will be set up in the fluid between the two plates (in $z$ direction). We define 
the viscosity by equation:
\begin{equation}
    \tau_{shear} = \frac{F}{A} = \eta \frac{d \langle u_x \rangle}{dz}
\end{equation}

With force $F$ constantly applied on the top plate, we are inputing momentum $Fdt$ per unit area in time $dt$. In equilibrium, 
These momentum will be transported completely to the bottom plate. Therefore, we have momentum flux through $z$ direction 
(note that the momentum itself is along the plate, $x$ dirction, but the flux is along $z$):
\begin{equation}
    \Pi_z = - F / A = - \eta \frac{d \langle u_x \rangle}{dz}
\end{equation}
We have a negative sign because the momentum flow from high velocity area to low velocity area, which is opposite the velocity gradient.

We can calculate the momentum flux by considering the microscopic motion of gas molecules. We consider that the motion of free ideal gas superimposed 
with the collective drifting motion with velocity $\langle u_z \rangle$. A molecules travelling along $z$ direction will
change their momentum by collision with other molecules in the final positon $z_2$ ($z_1 \to z_2$). The number of particle with velocity $v$ travelling
at an angle $\theta$ with $z$ direction is $ v\cos \theta n \cdot \frac{1}{2} f(v) dv \sin\theta d\theta $ per unit area $A$ and time $dt$. How far they 
travel will be given by the mean free path $\lambda$. The momentum difference of a molecule just after collision at $z_1$ and just after collision at $z_2$ is 
given by:
\begin{equation}
    - m \left( \frac{\partial \langle u_x \rangle}{\partial z} \right) \lambda \cos\theta
\end{equation}
which is the momentum flux created by this single molecule. Summing over all molecules with different speed and angle, we have:
\begin{align}
    \Pi_z &= \int_0^{\infty} dv \int_0^{\pi/2} d\theta 
     v \cos \theta n \frac{1}{2} f(v) \sin\theta \cdot m \left( \frac{\partial \langle u_x \rangle}{\partial z} \right) \lambda \cos\theta \\
     &= -\frac{1}{3} n m \lambda \langle v \rangle \frac{\partial \langle u_x \rangle}{\partial z} \label{momentum_transfer}
\end{align}
We obtain the viscosity:
\begin{equation}
    \eta = \frac{1}{3} n m \lambda \langle v \rangle
\end{equation}
Using the previous result $\lambda = (\sqrt{2}n\sigma)^{-1}$ and $\langle v \rangle = ()\frac{8k_BT}{\pi m})^{1/2}$, we can also write:
\begin{equation}
    \eta = \frac{2}{3\sigma}\left( \frac{mk_BT}{\pi} \right)^{1/2}
\end{equation}
We have the following observations:
\begin{enumerate}
    \item $\eta$ is independent of pressure.
    \item $\eta \propto T^{1/2}$.
    \item For the above approximation of momentum transfer to be correct, we require $ L \ll \lambda \ll d $, where $d$ is the size of the molecule and $L$ is the size scale of the container.
\end{enumerate}

\subsection{Thermal conductivity}
We define the heat flux in the $z$ direction
\begin{gather}
    J_z = -\kappa \left( \frac{\partial T}{\partial z} \right) \\
    \mathbf{J} = -\kappa \nabla T
\end{gather}
The gas molecules carrier heat through their kinetic energy, which depend on temperature $T$ via $\langle E_k \rangle = 3k_BT / 2$. Defining heat capacity of a 
molecule as $C$, we can calculate the total heat flux:
\begin{align}
    J_z &= \int_0^{\infty} dv \int_0^{\pi/2} d\theta 
    v \cos \theta n \frac{1}{2} f(v) \sin\theta \cdot - C \left( \frac{\partial T}{\partial z} \right) \lambda \cos\theta \\
    &= -\frac{1}{3} n C \lambda \langle v \rangle \frac{\partial T}{\partial z} \label{heat_transfer}
\end{align}
Using $C_V = n C$, the thermal conductivity of gas is therefore
\begin{equation}
    \kappa = \frac{1}{3} C_V \lambda \langle v \rangle
\end{equation}

We observe that Eq.\ref{heat_transfer} is very similar to Eq.\ref{momentum_transfer}, and we have:
\begin{equation}
    \frac{\kappa}{\eta} = \frac{C}{m}
\end{equation} 

\subsection{Particle Diffusion}
We consider a gas of molecules in which some of them is labelled, if those labelled molecules are initially confined in 
certain area and the confinement is removed, they will start to diffuse (Self-diffuse). Suppose that the diffusion is 
along $z$ direction and we use $n^*(z)$ to denote the density of those labelled particles, we can define the 
diffusion coefficient: 
\begin{equation}
    \Phi_z = -D \left( \frac{\partial n^*}{\partial z} \right)
\end{equation}
Following the above microscopic picture, we have:
\begin{align}
    \Phi_z &= \int_0^{\infty} dv \int_0^{\pi/2} d\theta 
    v \cos \theta \frac{1}{2} f(v) \sin\theta \cdot - \left( \frac{\partial n^*}{\partial z} \right) \lambda \cos\theta \\
    &= -\frac{1}{3} \lambda \langle v \rangle \frac{\partial T}{\partial z} 
\end{align}
giving the self-diffusion coefficient
\begin{equation}
    D = \frac{1}{3} \langle v \rangle
\end{equation}

We have the following relationship:
\begin{enumerate}
    \item $ D \propto T^{3/2}$
    \item $D \rho = \eta $, where $rho$ is the density $\rho = nm$
\end{enumerate}

\subsection{Heat diffusion equation}
Given the heat flux $J = - \kappa \nabla T$, the total heat flow out of a closed surface $S$ is $\int_S J \cdot dS$.
This value should equal to the loss of total thermal energy $\int_V CT dV$, where $C$ here is the volume heat capacity. 
We have the Result:
\begin{equation}
    \int_S J \cdot dS = \int_V \nabla \cdot J dV = -\frac{\partial}{\partial t} \int_V CT dV
\end{equation}
where we obtain the first equality through divergence theorem. We have
\begin{align}
    \nabla \cdot J = -C \frac{\partial T}{\partial t} \\
    \frac{\partial T}{\partial t} = D \nabla^2 T \label{thermaldiffusion}
\end{align}
with $D = \kappa/C$ is the thermal diffusivity.
Eq.\ref{thermaldiffusion} is called \textbf{Thermal diffusion equation}.

In a steady state, we have $\partial T / \partial t = 0 $ so that the 
diffusion equation is reduced to 
\begin{equation}
    \nabla^2 T = 0
\end{equation}

Suppose we have a gas between two hot plates separate with a distance $L$, 
one maintained at temperature $T_1$ and the other at $T_2 < T_1$. By integrating
the equation $\partial^2 T / \partial x^2 = 0$ twice 
(note that this equation imply a linear temperature distribution)
and using the boundary condition,
we have:
\begin{equation}
    T = \frac{(T_2-T_1)x}{L} + T_1 
\end{equation}
The heat flux is given by:
\begin{equation}
    J = - \kappa \left( \frac{\partial T}{\partial x} \right) = \frac{\kappa}{L} (T_1 - T_2)
\end{equation}
The value $\kappa / L$ is called \textbf{thermal conductance}.

If heat is generated at a rate $H$ per unit volume, the divergence of $J$ will 
be:
\begin{equation}
    \nabla \cdot J = -C \frac{\partial T}{\partial t} + H
\end{equation}
and the thermal diffusion equation will be modified to be:
\begin{equation}
    \frac{\partial T}{\partial t} = D \nabla^2 T + \frac{H}{C}
\end{equation}

\subsection*{Thermal diffusion equation in 1D}
We want to solve the equation:
\begin{equation}
    \frac{\partial T}{\partial t} = D \frac{\partial^2T}{\partial x^2}
\end{equation}
to obtain the temperature profile.

Since this equation is a second order linear partial equation, we can look for 
wave-like solutions
\begin{equation}
    T(x,t) \propto \exp(i(kx - \omega t))
\end{equation}
with $k = 2\pi/\lambda $ and $\omega = 2\pi/f$ the wave vector and angular frquency. Solution is 
given by:
\begin{gather}
    - i \omega = -D k^2 \\
    k = \pm (1 + i) \sqrt{\frac{\omega}{2D}}
\end{gather}
Noting that with $ k = -(1 + i) \sqrt{\frac{\omega}{2D}}$, $T \to \infty$ as $x \to \infty$. 
Therefore, the solution for the temperature profile can be written as a summation of frequency:
\begin{equation}
    T(x,t) = \sum_{\omega} A(\omega)\exp(-i\omega t) \exp((i-1)\sqrt{\frac{\omega}{2D}}x) \label{thermaldiffusionsolution}
\end{equation}

As an application, consider solving the 1D problem of heat diffusion into the earth ground. The 
boundary profile is given by
\begin{equation}
    T(0,t) = T_0 + \Delta T \cos(\Omega t) = T_0 + \frac{\Delta T}{2} e^{i\Omega t} + \frac{\Delta T}{2} e^{-i\Omega t} \label{thermaldiffusionboundary}
\end{equation}
where the period is given by the alternation of day and night. 

Requiring Eq.\ref{thermaldiffusionsolution} to give the boundary condition Eq.\ref{thermaldiffusionboundary}, we obtain
the solution:
\begin{equation}
    T(x,t) = T_0 + \Delta T e^{-x/\delta} \cos(\Omega t - \frac{x}{\delta})
\end{equation}
where $\delta = \sqrt{2D/\Omega}$ is called the skin depth, and temperature fall off exponentially as $e^{-x/\delta}$.

\subsection*{Newton's law of cooling}
Newton's low of cooling states that the heat loss of a surface is proportional to the area of the surface
multiplied by the temperature difference. The heat flux is given by:
\begin{equation}
    J = h \Delta T
\end{equation}
with $h$ the heat transfer coefficient of the surface. As an example, suppose a cup of tea at temperature $T_{hot}$ 
is placed in a room at temperature $T_{air}$ and the heat loss is through the surface area $A$. Suppose the air temperature
near the surface is maintained at $T_{air}$ with convection, we have:
\begin{equation}
    -C \frac{\partial T}{\partial t} = J A  = h A (T - T_{air})
\end{equation}
$T(t)$ is the temperature of the tea. We have the solution
\begin{equation}
    T(t) = T_{air} + (T_{hot} - T_{air})e^{-\lambda t}
\end{equation}
with $\lambda = A h / C $

\subsection*{Prandtl number}
Apart from thermal diffusion, convection also play a part in the heat transfer in solid and gas. Convection will
dominate if momentum diffusion dominates. We can thus compare the magnititude of the two mechanism by
\begin{equation}
    \sigma_p = \frac{v}{D} = \frac{\eta c_p}{\kappa}
\end{equation}
where $v = \eta / \rho$ is the kinematic viscosity and $D$ is the thermal diffusivity $D = \kappa/(\rho c_p)$ ($c_p$ is the specific heat).
This value of called \textbf{Prandtl number}. For $\sigma_p \gg 1$, the convection is the dominant mode of heat transport. 
For gas, $\sigma_p$ can be found to be $2/3$ with the previous results. For liquid, $\sigma_p \gg 1$.

\end{document}
