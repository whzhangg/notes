\documentclass{article}
\usepackage{amssymb, amsmath, amsthm}
\usepackage[margin=1in]{geometry}
\usepackage{verbatim}
\usepackage{graphicx}
\usepackage{hyperref} % \url \href
\usepackage{docmute}

\newcommand{\pfrac}[2]{\frac{\partial #1}{\partial #2}}
\newcommand{\dbar}{\mathbf{\tilde{d}}}
\newcommand{\dnor}{\text{d}}
% \renewcommand{\H}{\mathcal{H}}

\begin{document}


\section{Thermodynamic potentials}
We call \textbf{Thermodynamic potentials} functions that are constructed from 
the functions of state. 
We define the following Thermodynamic potentials, The first being the internal 
energy, the following are the Enthalpy, Helmholtz function and the Gibbs 
function.

\begin{table*}[h]
    \centering
    \begin{tabular}{rrrr}
     $U(S,V)$ &   $U$          & $\dnor U =   T \dnor S - p \dnor V$ & $T = \left(\pfrac{U}{S}\right)_V$, $p = -\left(\pfrac{U}{V}\right)_S$ \\
     $H(S,p)$ &   $H = U + PV$ & $\dnor H =   T \dnor S + p \dnor V$ & $T = \left(\pfrac{H}{S}\right)_V$, $V = \left(\pfrac{H}{p}\right)_S$ \\
     $F(T,V)$ &   $F = U - TS$ & $\dnor F = - S \dnor T - p \dnor V$ & $S = -\left(\pfrac{F}{T}\right)_V$, $p = -\left(\pfrac{F}{V}\right)_T$ \\
     $G(T,p)$ &   $G = H - TS$ & $\dnor G = - S \dnor T + V \dnor p$ & $S = -\left(\pfrac{G}{T}\right)_p$, $V = \left(\pfrac{G}{p}\right)_T$ \\
     \end{tabular}
\end{table*}

Regarding the thermodynamic potentials, we have the following comments:
\begin{itemize}
    \item Different thermodynamic potentials are expressed in different \textbf{natural variables} because 
            their differential form. For example, because $\dnor F = - S \dnor T - p \dnor V$, the natural variables for $F$ are $U, V$. 
            The differential form implies that the change in $F$ is related to the change in $T$ or $V$. if $\dnor V = 0$, then the change
            in $F$ is independent of pressure $p$. 
    \item Each form of the thermodynamic potential gives different formula of thermodynamic quantities, which can be easier to access. For 
            example, $S = -\left(\partial{F}/\partial{T}\right)_V$ gives the value of entropy with easily accessible quantities $F, T$.
    \item Each thermodynamic potentials describe the system subject to different constrains: For a system with fixed pressure and fixed temperature
            in a process, its Gibbs function will remain the same through the process. While for a system with fixed volume and temperature, 
            it's free energy $F$ will be fixed over the process.
\end{itemize}

\subsection{Maxwell's relationship}
Maxwell's relationship can be derived from the above  definition of the thermodynamic
potential function, and they are stated as below:
\begin{gather}
    \left(\pfrac{T}{V}\right)_S = - \left(\pfrac{p}{S}\right)_V \\
    \left(\pfrac{T}{p}\right)_S = \left(\pfrac{V}{S}\right)_p \\
    \left(\pfrac{S}{V}\right)_T = - \left(\pfrac{p}{T}\right)_V \\
    \left(\pfrac{S}{p}\right)_T = - \left(\pfrac{V}{T}\right)_V 
\end{gather}

\subsection{Legendre transformation}
In general, Legendre transformation is a transformation on the real valued convex functions
of one of the variable and is usually used to convert functions of one quantity (position, pressure) into function of
the conjugate quantity (momentum, volume)
Consider function $F(x)$ with $dF/dx = s(x)$, we can write a function
$G(s)$ with property $dG/ds = x(s)$. They are related by:
\begin{equation}
    d(F+G) = sdx + xds = d(xs)
\end{equation}
so that $F(x) + G(s) = xs$. One property of Legendre transformation that the result of the transformation 
is also convex function. For details of the geometry meaning and convex requirement, see Morin's chapter.

For thermodynamic potentials, we use a \emph{non-standard definition} $dG/ds = -x(s)$, 
so that the transformation agree with the thermal dynamic variables.
leading to 
\begin{equation}
    d(F-G) = sdx + xds = d(xs)
\end{equation}
and therefore $F-G = xs$
\footnote{
    Both is obviously correct. An example of the standard usage is Legendre transformation 
    for Lagrangian and Hamiltonian is given by $F+G = xs$. see "Understanding the transformation in terms of derivatives"
    section in \url{https://en.wikipedia.org/wiki/Legendre_transformation}.
}.

We have seem that intensity and extensity properties would come in pairs, such as $(T,S)$, $(\mu,N)$
and $(P,V)$ due to their definition. We have also see that by defining different ensemble, we can
write down different potential function for each ensemble:
\begin{align}
    \text{NVT ensemble} &\Rightarrow dA(N,V,T) = -SdT - pdV + \mu dN\notag \\
    \text{$\mu$VT ensemble} &\Rightarrow d\Omega(\mu,V,T) = -SdT - pdV - Nd\mu \notag \\
    (Gibbs)\ \text{NPT ensemble} &\Rightarrow dG(N,p,T) = -SdT + Vdp + \mu dN \notag \\
    &\cdots \notag
\end{align}
The relationship between different potential function can be expressed with \textbf{Legendre transformation}.

Writting $x$ for an intensive property and $S$ as the conjugating extensive property. For a potential $F(x)$
with the thermodynamic relationship $dF = Sdx$, if another potential $G$ is related to $F(x)$ by $F(x) - G(S) = xS$,
then, we can find $dG(S) = -xdS$.

As an example, to find the Legendre transformation of the free energy in canonical ensemble $A(N,V,T)$ in terms of 
variable $V$, we have:
\begin{equation}
    dA = -pdV \notag
\end{equation} 
The transformed potential is thus:
\begin{equation}
    dG(N,p,T) = Vdp \notag
\end{equation} 
corresponding to the 
Gibbs potential. 

\subsection{Isothermal magnetization and adiabatic demagnetization}
To generalize the above treatment in general system, we can write:
\begin{equation}
    \dbar W = X dx
\end{equation}
where $X$ is some intensive generalized force and $x$ is some 
extensive generalized displacement.

Consider a system of magnetic moments arranged in a lattice. We 
assume paramagnetic system with no interaction between the magnetic moments. 
The first law of Thermodynamic for such system is 
\begin{equation}
    \dnor U = T \dnor S - m \dnor B
\end{equation}
We define the magnetic susceptibility as:
\begin{equation}
    \chi = \lim_{H\to0} \frac{M}{H} \approx \frac{\mu_0 M}{B}
\end{equation}
since $B = \mu_0(H+M)$ and $M \ll H$.
Curie's law states $\chi \propto 1/T$ and therefore
\begin{equation}
    \left(\pfrac{\chi}{T}\right)_B < 0
\end{equation}

Consider the Helmholtz function:
\begin{equation}
    \dnor F = - S \dnor T - m \dnor B
\end{equation}
which gives the Maxwell relation:
\begin{equation}
    \left(\pfrac{S}{B}\right)_T = \left(\pfrac{m}{T}\right)_B 
    \approx \frac{VB}{\mu_0} \left(\pfrac{\chi}{T}\right)_B
\end{equation}
The heat absorbed in an isothermal change of $B$ is:
\begin{equation}
    \Delta Q = T \left(\pfrac{S}{B}\right)_T \Delta B
    = \frac{TVB}{\mu_0} \left(\pfrac{\chi}{T}\right)_B \Delta B < 0
\end{equation}
So that an isothermal increase of $B$ will 
release heat to the environment.

The change in temperature in an adiabatic change of $B$ is 
\begin{equation}
    \left( \pfrac{T}{B} \right)_S = - \left( \pfrac{T}{S} \right)_B \left( \pfrac{S}{B} \right)_T
\end{equation}
using 
\begin{equation}
    C_B = T \left( \pfrac{S}{T} \right)_B    
\end{equation}
we have:
\begin{equation}
    \left( \pfrac{T}{B} \right)_S = - \frac{TVB}{\mu_0 C_B} \left(\pfrac{\chi}{T}\right)_B > 0
\end{equation}
So that in an adiabatic process, the temperature will decrease with the decrease of magnetization,
which is called \textbf{adiabatic demagnetization}

\section{Equipartition of energy}
To derive the equipartition theorem, we first consider a system
whose energy is given by a quadratic form:
\begin{equation}
    E = \alpha x^2
\end{equation}
where $\alpha$ is some positive constant and $x$ is a variable
that descript the microscopic configuration of the system. The 
probability of the system taking configuration $x$ is then 
given by the canonical distribution:
\begin{equation}
    P(x) = \frac{e^{-\beta \alpha x^2}}{\int_{-\infty}^{\infty} e^{-\beta \alpha x^2 dx}}
\end{equation}
The mean energy of the system is then:
\begin{align}
    \langle E \rangle &= \int_{-\infty}^{\infty} E P(x) dx \\
        & = \frac{\int_{-\infty}^{\infty} \alpha x^2 e^{-\beta \alpha x^2} dx}{\int_{-\infty}^{\infty} e^{-\beta \alpha x^2 dx}} \\
        & = \frac{1}{2} k_B T
\end{align}

For a system with multiple variables, we similarly have:
\begin{equation}
    E = \sum_{i=1}^n \alpha_i x_i^2
\end{equation}
The total energy is then:
\begin{align}
    \langle E \rangle 
        & = \frac{\int_{-\infty}^{\infty}\cdots\int_{-\infty}^{\infty} \sum_{i=1}^n \alpha_i x_i^2 e^{-\beta \sum_{j=1}^n \alpha_j x_j^2} dx_1 \cdots dx_n}{\int_{-\infty}^{\infty}\cdots\int_{-\infty}^{\infty} e^{-\beta \sum_{i=1} \alpha_i x_i^2} dx_1 \cdots dx_n} \\
        & = \sum_{i=1}^n \frac{\int_{-\infty}^{\infty}\cdots\int_{-\infty}^{\infty} \alpha_i x_i^2 e^{-\beta \sum_{j=1}^n \alpha_j x_j^2} dx_1 \cdots dx_n}{\int_{-\infty}^{\infty}\cdots\int_{-\infty}^{\infty} e^{-\beta \sum_{i=1} \alpha_i x_i^2} dx_1 \cdots dx_n} \\
        & = \sum_{i=1}^n \frac{\int_{-\infty}^{\infty} \alpha_i x_i^2 e^{-\beta \alpha_i x_i^2}  dx_i }{\int_{-\infty}^{\infty} e^{-\beta \alpha_i x_i^2}  dx_i } \\
        & = \sum_{i=1}^n  \frac{1}{2} k_B T = \frac{n}{2} k_B T
\end{align}
where $n$ the number of variable of the system ( microscope variable that determine the energy of the system), which we can call degree of freedom.

The formal equipartition theorem states: \emph{ If the energy of a classic system is the sum
of the $n$ quadratic variables and the system is in contact with a thermal reservior at temperature $T$, 
then its average energy is $n\times \frac{1}{2}k_B T$}

\subsection{Application of equipartition theorem to heat capacitiy of crystals}
We consider each atom in the crystal are kept at their respective equilibrium positions
by a spring. Each atom is therefore described by 6 variables: their kinetic energy
is given by their velocity $\mathbf{v}$ and their potential energy by their 
position $\mathbf{r}$. Therefore, there are $6N$ degree of freedom in a solid 
with $N$ numbers of atoms. The total energy of the crystal at temperature $T$
is then:
\begin{equation}
    \langle E \rangle = 3N k_B T
\end{equation}
and the Molar heat capacitiy of a solid is then $3N_A k_B = 3R$, known as the Dulong-Petit rule.

\subsection{Assumption and limitation of equipartition theorem}
The critical assumption we made in deriving the equipartition theorem is 
as follows:
\begin{enumerate}
    \item We have assumed that the system variable $x$ is continuous through the intergration $\int_{-\infty}^{\infty} E(x) P(x) dx$.
            However, for quantum system, the system are found to take distinct states with quantized energy levels. For example, at low temperature,
            thermal energy may not be enough to excite the system to the next energy level and the system will be in the ground state with $P = 1$.
            A classical system in this case assumes that there are always levels for system to distribute to even at low temperature and the above 
            integral is valid. 
    \item We assume that that the energy of the system can be written in a quadratic form. In most case this is valid, since the system will 
            minimize its energy in equilibrium and the energy can therefore be written as an expansion to second order: $ E = E_0 + \alpha (x - x_0)^2$ 
            around $x =x_0$. However, when temperature is high and $x$ start to deviate largely from $x_0$, 
            higher order terms become important and quadratic form is no longer valid.
\end{enumerate}
As a conclusion, we learned that the equipartition theorem is valid in a temperature range high enough so that states are almost continuous ($\Delta E \ll k_B T$ with
$\Delta E$ the energy difference between states) but 
low enough that the quadratic form of the energy is valid.

\section{The partition function}
We define the partition function as:
\begin{equation}
    Z = \sum_{\alpha} e^{-\beta E_{\alpha}}
\end{equation}
which allow us to derive all the important thermodynamic quantities. 

The internal energy is given by:
\begin{equation}
    U = \langle E \rangle = \frac{\sum_{i} E_i e^{-\beta E_i}}{\sum_{i} e^{-\beta E_i}} = -\frac{\dnor \ln Z}{\dnor\beta} = k_BT^2\frac{\dnor \ln Z}{\dnor T}
\end{equation}

The entropy is found by:
\begin{align}
    S = -k_B \sum_i P_i \ln P_i = k_B \sum_i P_i (\beta E_i + \ln Z) = k_B (\beta U + \ln Z)
\end{align}
or we can write:
\begin{equation}
    S = U/T + k_B \ln Z
\end{equation}

Helmholtz function can be written by $F = U - TS$, using the above result, we have:
\begin{gather}
    F = -k_B T \ln Z \\
    Z = e^{-\beta F}
\end{gather}

Other derived quantities can also be found:
\begin{gather}
    C_V = \left(\pfrac{U}{T}\right)_V = k_B T \left[ 2\left(\pfrac{\ln Z}{T}\right)_V + T \left( \frac{\partial^2\ln Z}{\partial T^2} \right)_V \right] \\
    p = - \left(\pfrac{F}{V}\right)_T = k_B T \left( \pfrac{\ln Z}{V} \right)_T \\
    H = U + pV =  k_B T \left[ \left( T\pfrac{\ln Z}{T}\right)_V + V \left( \pfrac{\ln Z}{V} \right)_T \right] \\
    G = F + pV =  k_B T \left[ -\ln Z + V \left( \pfrac{\ln Z}{V} \right)_T \right]
\end{gather}

It should be noted that the value of the partition function itself
is not uniquely defined, because the zero of the energy is arbitrary,
so that partitiona function can be defined up to an arbitrary multiplicative constant.
However, In terms of the thermodynamic quantities calculated in the above equations,
The result not on $Z$ but on $\ln Z$ and its derivative, so the 
those quantities are up to a additive constants or show no dependence on the 
arbitrarness of the partition function.

\section{Chemical potential and Grand potential}
We consider adding particles to a system, then the internal energy of the system will
increase by an amount, which we denote \textbf{chemical potential}. For a large system
with $\mu$ not changing significantly by adding or removing a particle, we have
\begin{equation}
    \dnor U = T \dnor S - p \dnor V + \mu \dnor N \label{du_dmu}
\end{equation}
with $N$ the particle number and $\mu$ can be written:
\begin{equation}
    \mu = \left( \pfrac{U}{N} \right)_{S,V}
\end{equation}
In terms of other constrains, we have:
\begin{gather}
    \dnor F = -p \dnor V - S \dnor T + \mu \dnor N \\
    \dnor G = V \dnor p - S \dnor T + \mu \dnor N
\end{gather}
and we have the expression for chemical potential as:
\begin{equation}
    \mu = \left( \pfrac{F}{N} \right)_{T,V} = \left( \pfrac{G}{N} \right)_{T,p}
\end{equation}

For an isolated system, the entrope will increase as system go to equilibrium. 
We write:
\begin{align}
    \dnor S &= \left( \pfrac{S}{U} \right)_{N,V} \dnor U + 
              \left( \pfrac{S}{V} \right)_{N,U} \dnor V + 
              \left( \pfrac{S}{N} \right)_{U,V} \dnor N \\
            &= \frac{\dnor U}{T} + \frac{p \dnor V}{T} - \frac{\mu \dnor N}{T}
\end{align}
where the second equality follow from Eq.\ref{du_dmu}. 
Therefore, we identify:
\begin{equation}
    \left( \pfrac{S}{U} \right)_{N,V} = \frac{1}{T}; \ \ \
    \left( \pfrac{S}{V} \right)_{N,U} = \frac{p}{T}; \ \ \
    \left( \pfrac{S}{N} \right)_{U,V} =-\frac{\mu}{T}
\end{equation}

We consider two system connected to each other and isolated from 
from the environment, If heat is allowed to follow, we have:
\begin{align}
    \dnor S &= \left( \pfrac{S_1}{U_1} \right)_{N,V} \dnor U_1 + \left( \pfrac{S_2}{U_2} \right)_{N,V} \dnor U_2 \\
        &= \left( \frac{1}{T_1} - \frac{1}{T_2} \right) \dnor U_1 \ge 0
\end{align}
Therefore, the equilibrium can be found at $T_1 = T_2$.

Similarly, if the system are allowed to exchange particle, we have:
\begin{align}
    \dnor S = \left( \frac{\mu_1}{T_1} - \frac{\mu_2}{T_2} \right) \dnor N_1 \ge 0
\end{align}
the equilibrium can be found at $\mu_1 = \mu_2$, if the two subsystem have 
already the same temperature.

\subsection{Grand partition function}
For a grand canonical ensemble, The probability that a system
will be in a state with energy $E_i$ and particle number $N_i$ are 
given by
\begin{equation}
    P_i = \frac{e^{\beta(\mu N_i - E_i)}}{\mathcal{Z}}
\end{equation}
and the grand partition function $\mathcal{Z}$ is given by:
\begin{equation}
    \mathcal{Z} = \sum_i e^{\beta(\mu N_i - E_i)}
\end{equation}

We can find the thermodynamic quantities using $\mathcal{Z}$ in 
a similar way as with $Z$, with the result:
\begin{gather}
   \langle N \rangle = \sum_i N_i P_i = k_B T \left( \pfrac{\ln \mathcal{Z}}{\mu} \right)_T \\
    U = - \left( \pfrac{\ln \mathcal{Z}}{\beta} \right)_{\mu} + \mu N \\
    S = -k_B \sum_i P_i \ln P_i = \frac{U - \mu N + k_B T\ln \mathcal{Z}}{T} \label{grand_entropy}
\end{gather}

We further define the grand potential:
\begin{equation}
    \Phi_G = -k_B T \ln \mathcal{Z}
\end{equation}
which is a function of state. We have, using Eq.\ref{grand_entropy}:
\begin{gather}
    \Phi_G = -k_B T \ln \mathcal{Z} = U - TS - \mu N = F - \mu N \\
    \dnor \Phi_G = \dnor F - \mu \dnor N - N \dnor \mu
\end{gather}

\subsection{Different types of particles}
If there are different types of particle, we can write:
\begin{gather}
    \dnor U = T \dnor S - p \dnor V + \sum_i \mu_i \dnor N_i \\
    \dnor F = -p \dnor V - S \dnor T + \sum_i \mu_i \dnor N_i \\
    \dnor G = V \dnor p - S \dnor T + \sum_i \mu_i \dnor N_i 
\end{gather}

\section{Quantum Statistical Mechanics}
For a quantum \textbf{ensemble} $|\Psi\rangle$ expressed in a complete set of basis $|\phi_i\rangle$ (state of system)as 
% both \Phi and \phi_i are the state of the whole system
\begin{equation}
    |\Psi\rangle = \sum_i c_i |\phi_i\rangle
\end{equation}
we define the density matrix 
\begin{equation}
    \rho = |\Psi\rangle \langle \Psi | = \sum_{ij} |\phi_i\rangle \langle \phi_i| \Psi\rangle \langle \Psi |\phi_j\rangle \langle \phi_j|
    = \sum_{ij} c_i c_j^* |\phi_i\rangle \langle \phi_j| \label{density1}
\end{equation}
and therefore
\begin{equation}
    \langle \phi_i | \rho | \phi_j \rangle = c_i c_j^*
\end{equation}
So that the expectation value of observables $A$ is then:
\begin{equation}
    \langle A \rangle = \langle \Psi | A | \Psi \rangle
    = \sum_{ij} \langle \Psi |\phi_i\rangle \langle \phi_i| A | \phi_j\rangle \langle \phi_j|\Psi \rangle
    = \sum_{ij} c_i^* c_j \langle \phi_i| A | \phi_j\rangle
\end{equation}
while we also have:
\begin{equation}
    \text{Tr}[\rho A] = \sum_i \langle \phi_i | \rho A | \phi_i \rangle
        = \sum_{ij} \langle \phi_i | \rho | \phi_j \rangle \langle \phi_j | A | \phi_i \rangle
        = \sum_{ij} c_i c_j^* \langle \phi_j | A | \phi_i \rangle = \langle A \rangle
\end{equation}
So that we find the relationship
\begin{equation}
    \langle A \rangle = \text{Tr}[\rho A]
\end{equation}

Now, we consider the time dependence of the trace operator, using the time evolution operator:
\begin{align}
    -ih\frac{\partial}{\partial t} \rho &= -ih \frac{\partial}{\partial t}  \sum_{ij} c_i c_j^* |\phi_i\rangle \langle \phi_j| \notag \\
        &= \sum_{ij} c_i c_j^* \left( -ih\frac{\partial}{\partial t} |\phi_i\rangle \langle \phi_j| - |\phi_i\rangle ih\frac{\partial}{\partial t} \langle \phi_j|  \right) \notag \\
        &= \sum_{ij} c_i c_j^* \left(  H|\phi_i\rangle \langle \phi_j| - |\phi_i\rangle \langle \phi_j| H \right) \notag \\
        &= [H\rho  - \rho H] = [H,\rho]
\end{align}
In an equilibrium, the density operator will be time dependent: $\partial \rho / \partial t = 0$, suggesting that 
$H$ and $\rho$ commute:
$[H,\rho] = 0$. The eigenstates of the Hamiltonian thus are also the eigenstates of the density operator.
Therefore, the density operator in Eq.\ref{density1} can be written in this form:
\begin{equation}
    \rho = \sum_n w_n | n \rangle \langle n | \label{density2}
\end{equation}
where $|n\rangle$ are the eigenstates of the Hamiltonian, and $w_n = c_n^* c_n = |c_n|^2$ is the probability for the 
system $|\Psi\rangle$ to be in the energy eigenstate $|n\rangle$
For Microcanonical ensemble with $\Omega$ accessible states, we simply have:
\begin{equation}
    w_n = \frac{1}{\Omega}
\end{equation}
For Canonical ensemble, we have:
\begin{equation}
    w_n = \frac{e^{-\beta E_n}}{\sum_{n'} e^{-\beta E_{n'}}}
\end{equation}
and the partition function $Q = \text{Tr}e^{-\beta H}$ (an observables).
For Grand Canonical ensemble, taking account of the particle number in 
an system state $|n\rangle$, we have:
\begin{equation}
    w_n = \frac{e^{-\beta (E_n-\mu N_n)}}{\sum_{n'} e^{-\beta (E_{n'}-\mu N_{n'})}}
\end{equation}
and partition function $Q =  \text{Tr}e^{-\beta (H-\mu N}$, where $N$ is the operator for 
particle density.

\subsection{Distribution function}
Suppose the Hamiltonian can be written as a function of particle number
$H = \sum_p \varepsilon_p n_p$ with $\varepsilon_p$, $n_p$ denoting 
the energy and occupation number of a single particle state $p$.
We start by calculating the partition function of a quantum ensemble, since we wish 
to study the probability of system containing different particle number, we 
use the grand canonical potential:
\begin{align}
    Q(\mu,V,T) = \sum_{N=0}^{\infty} \sum_{\{n_p\}_N} e^{-\beta\sum_p (\varepsilon_p - \mu) n_p}
               = \sum_{\{n_p\}} e^{-\beta\sum_p (\varepsilon_p - \mu) n_p}
\end{align}
where in the first expression, $\{n_p\}_N$ denotes a many body state with occupation $\{n_{p_1},n_{p_2}, \cdots, n_{p_i}\}$
in each of the single particle with the constraint that the total number of particles sum up to $N$. In the second 
expression, the two summation in the first expression are combined.
We now take the summation $p$ in the exponential out:
\begin{equation}
    Q(\mu,V,T) = \sum_{\{n_p\}} e^{-\beta\sum_p (\varepsilon_p - \mu) n_p} = \prod_p \sum_{n_p} e^{-\beta(\varepsilon_p - \mu) n_p} = \prod_p Q_p
\end{equation}
We can now study the sum of $e^{-\beta(\varepsilon_p - \mu) n_p}$ over all possible occupation number:
\textbf{Fermion}
we can only take $n_p = 0$ or $1$, so that 
\begin{gather}
    Q_p = 1 + e^{-\beta(\varepsilon_p - \mu)} \\
    \Omega_p = -k_BT\ln Q_p
\end{gather}
and 
\begin{equation}
    N_p = -\left(\frac{\partial \Omega}{\partial \mu}\right)_{T,V} = \frac{1}{e^{\beta(\varepsilon_p - \mu)}+1}
\end{equation}
\textbf{Boson}
the sum of $n_p$ is from $0$ to $\infty$, thus:
\begin{equation}
    \sum_{n_p=0}^{\infty} e^{-\beta(\varepsilon_p - \mu) n_p} 
    =\frac{1}{1-e^{-\beta(\varepsilon_p - \mu)}} 
\end{equation}
giving
\begin{equation}
    N_p =  \frac{1}{e^{\beta(\varepsilon_p - \mu)}-1}
\end{equation}
The partition function and total particle number can be summarized by given by:
\begin{gather}
    \Omega = -ak_BT\sum_p\ln(1 + ae^{-\beta(\varepsilon_p - \mu)}) \\
    N = \sum_p \frac{1}{e^{\beta(\varepsilon_p - \mu)} + a}
\end{gather}
with $a = 1$ for fermion and $-1 $ for boson.

\subsection{Boson gas}
For an ideal gas of spinless bosons \footnote{for spin $S$, each momentum eigenstate $k$ will have $(2S+1)$ fold degeneracy}, 
the single particle state will be the momentum eigenstate given by 
$k$ and kinetic energy $\varepsilon_k = \frac{\hbar^2k^2}{2m}$. The chemical potential $\mu$ will be negative otherwise 
the state with energy smaller than $\mu$ will have negative occupation.
We calculate the grand potential:
\begin{align}
    \Omega &= k_BT \sum_k \ln(1-e^{-\beta(\varepsilon_p - \mu)}) \notag \\
            &= k_BT \int_0^{\infty} \ln(1-e^{-\beta(\varepsilon_p - \mu)}) g(\varepsilon) d\varepsilon \notag \\
            &= k_BT \int_0^{\infty} \ln(1-e^{-\beta(\varepsilon_p - \mu)}) \frac{V\varepsilon^{1/2}}{(2\pi)^2}\left(\frac{2m}{\hbar^2}\right)^{3/2} d\varepsilon \notag \\
            &= -\frac{2}{3} \frac{V}{(2\pi)^2}\left(\frac{2m}{\hbar^2}\right)^{3/2} \int_0^{\infty} \frac{\varepsilon^{3/2}d\varepsilon}{e^{\beta(\varepsilon_p - \mu)}-1} \label{todos}
\end{align}
define $z = e^{\beta\mu}$, the particle number and internal energy can be written as:
\begin{align}
    N = \frac{V}{(2\pi)^2}\left(\frac{2m}{\hbar^2}\right)^{3/2} \int_0^{\infty} \frac{\varepsilon^{1/2}d\varepsilon}{e^{\beta\varepsilon_p}/z-1} \\
    U = \frac{V}{(2\pi)^2}\left(\frac{2m}{\hbar^2}\right)^{3/2} \int_0^{\infty} \frac{\varepsilon^{3/2}d\varepsilon}{e^{\beta\varepsilon_p}/z-1}
\end{align}
The integral we can write in terms of \textbf{polylogarithm} function:
\begin{align}
    \int_0^{\infty} \frac{\varepsilon^{n-1}d\varepsilon}{e^{\beta\varepsilon_p}/z-1} = (k_BT)^n (n-1)! L_n(z)
\end{align}
with $n/2! = n/2 \times (n-2)/2 \times \cdots \times 1/2$ with n an odd number.
So that 
\begin{align}
    N &= \frac{V}{(2\pi)^2}\left(\frac{2m}{\hbar^2}\right)^{3/2} (k_BT)^{3/2} (1/2)! L_{3/2}(z) = \frac{V}{\lambda^3} L_{3/2}(z) \label{eqbosonN} \\
    U &= \frac{V}{(2\pi)^2}\left(\frac{2m}{\hbar^2}\right)^{3/2} (k_BT)^{5/2} (3/2)! L_{5/2}(z) = \frac{3}{2}\frac{Vk_BT}{\lambda^3} L_{5/2}(z)
\end{align}
$\lambda$ is defined to absorbed all the numerical factor in Eq.\ref{eqbosonN} and $\lambda \propto T^{-1/2}$. 
Let's now consider Eq.\ref{eqbosonN}
\begin{equation}
    n \lambda^3 = L_{3/2}(z) \label{conflict}
\end{equation}
since $\mu$ need to be negative, the value of $z$ is bound to $(0,1)$, the function $L_{3/2}$ is monoclinically increaing with $z$ and bound
between $(0,L_{3/2}(1)) \approx (0,2.612)$, however, as we decrease temperature and $z \to 1$, 
$\lambda$ will increase without bound, which conflict with Eq.\ref{conflict}. 

This inconsistency come from the fact that when we convert the summation of $k$ into integral of energy $\varepsilon$, we essentially omitted 
a single state $(k=0,\varepsilon=0)$ since $g(\varepsilon=0)=0$. Its density of state $g(\varepsilon)$ is ignorable compared to other states but
at low temperature, it's occupation maybe very large. Including the occupation of this ground state explicitly in Eq.\ref{eqbosonN}:
\begin{equation}
    N = N_0 + N_1 = \frac{1}{1/z-1} + \frac{V}{\lambda^3} L_{3/2}(z)
\end{equation}
if we define 
\begin{equation}
    n \lambda(T_c)^3 = L_{3/2}(z=1)
\end{equation}
then
\begin{equation}
    \frac{N_0}{N} = \frac{N-N_1}{N} = 1-\left(\frac{T}{T_c}\right)^{3/2}
\end{equation}
if the temperature is low enough so that $N_0/N \approx 1$, then almost all particles are condensed in the ground state. $T_c$ for an ideal boson gas can be calculated:
\begin{equation}
    k_B T_c \approx 0.061 \frac{\hbar^2}{m} n^{2/3}
\end{equation}
For example, liquid Helium$^4$ with a density of $1.5e22 cm^{-3}$ gives a $T_c \approx 3K$. Expreiment observed a phase transition to a 
new phase with superfluid properties around this temperature, associated with the condensation.

\section{Thermodynamic Properties of magnetic system}
We write the internal energy of an magnetic system as:
\begin{equation}
    \dnor U = \dbar Q + \mu_0 H \dnor M
\end{equation}
where $H$ is the external field. For reversible process, we have:
\begin{equation}
    \dnor U = T \dnor S + \mu_0 H \dnor M
\end{equation}
and we can express the temperature $T$ and field $H$ by:
\begin{align}
    T = \left(\pfrac{U}{S}\right)_M \ \text{ and }\ \ H = \frac{1}{\mu_0} \left(\pfrac{U}{M}\right)_S
\end{align}
and we obtain the Maxwell relationship with internal energy:
\begin{equation}
    \left( \pfrac{T}{M} \right)_S = \mu_0 \left( \pfrac{H}{S} \right)_M
\end{equation}
Heat absorbed can be written as:
\begin{align}
    \dbar Q &= \left(\pfrac{U}{T}\right)_M \dnor T + \left[ \left(\pfrac{U}{M}\right)_T -\mu_0 H \right] \dnor M \\
            &= C_M \dnor T + l \dnor M \label{q_cm}
\end{align}
where we identify $C_M$ to be the specific heat at constant magnetization: when $\dnor M = 0$, we have $\dbar Q = C_M \dnor T$, and
\begin{equation}
    C_M = \left(\pfrac{U}{T}\right)_M = \left(\frac{\dbar Q}{\dnor T}\right)_M = T \left(\pfrac{S}{T}\right)_M
\end{equation}
where we have used $\dbar Q = T \dnor S$. For $l$, we can find, using Maxwell relation:
\begin{equation}
    l = \left( \frac{\dbar Q}{\dnor M} \right)_T = T \left(\pfrac{S}{M}\right)_T = - \mu_0 T \left(\pfrac{H}{T}\right)_M
\end{equation}

We introduce thermodynamic quantities called magnetic enthalpy:
\begin{gather}
    \mathcal{H} = U - \mu_0 H M \\
    \dnor \mathcal{H} = T \dnor S - \mu_0 M \dnor H
\end{gather}
which gives the Maxwell relationship:
\begin{equation}
    \left( \pfrac{T}{H} \right)_S = - \mu_0 \left( \pfrac{M}{S} \right)_H
\end{equation}
Similar to above, we have the heat absorbed:
\begin{align}
    \dbar Q &= \left(\pfrac{\mathcal{H}}{T}\right)_H \dnor T + \left[ \left(\pfrac{\mathcal{H}}{H}\right)_T + \mu_0 H \right] \dnor M \\
            &= C_H \dnor T + h \dnor M \label{q_ch}
\end{align}
where
\begin{equation}
    C_H = \left(\pfrac{\mathcal{H}}{T}\right)_H = T \left(\pfrac{S}{T}\right)_H
\end{equation}
and similar to the calculation of $l$, we have:
\begin{eqnarray}
    h = T \left(\pfrac{S}{H}\right)_T = \mu_0 T \left(\pfrac{M}{T}\right)_H
\end{eqnarray}

We can obtain the relationship between $C_M$ and $C_H$ that is similar between $C_V$ and $C_p$. 
Substructing Eq.\ref{q_cm} and Eq.\ref{q_ch}, we obtain:
\begin{equation}
    (C_H - C_M) \dnor T = -T\mu_0 \left[ \left(\pfrac{H}{T}\right)_M \dnor M + \left(\pfrac{M}{T}\right)_H \dnor H \right]
\end{equation}
since $M$, $T$ and $H$ are all function of state, we can express the relation:
\begin{equation}
    \dnor T = \left(\pfrac{M}{T}\right)_H \dnor M + \left(\pfrac{T}{H}\right)_M \dnor H
\end{equation}
So that we have the result:
\begin{equation}
    C_H - C_M = - T\mu_0 \left(\pfrac{H}{T}\right)_M \left(\pfrac{M}{T}\right)_H 
    = T\mu_0 \left(\pfrac{M}{T}\right)_H^2 \left(\pfrac{H}{M}\right)_T 
\end{equation}
where we used the relation:
\begin{equation}
    \left(\pfrac{H}{T}\right)_M = - \left(\pfrac{M}{T}\right)_H \left(\pfrac{H}{M}\right)_T 
\end{equation}

For system with fixed energy, we maximize the entropy to find the equilibrium
\footnote{That is to say, for microcanonical system, the equilibrium is found 
at states that correspond to the most number of microstates}.
However, for system in contact with the reservior, we need to minimize the free 
energy, defined by:
\begin{gather}
    F = U - TS = -\frac{1}{\beta} \ln Z \\
    \dnor F = \mu_0 H \dnor M - S \dnor T
\end{gather}
which gives the Maxwell relationship:
\begin{equation}
    \mu_0 \left( \pfrac{H}{T} \right)_M = - \left( \pfrac{S}{M} \right)_T
\end{equation} 

When the system is fixed at temperature $T$ and external field $H$, The 
Gibbs potential should be minimized:
\begin{equation}
    \dnor G = - \mu_0 M \dnor H - S \dnor T
\end{equation}
with the corresponding Maxwell relationship:
\begin{eqnarray}
    \mu_0 \left( \pfrac{M}{T} \right)_H = - \left( \pfrac{S}{H} \right)_T
\end{eqnarray}

Finally, if the number of particle in the system is not fixed:
we define extra thermodynamic potentials:
\begin{table*}[h]
    \centering
    \begin{tabular}{rr}
     $\Omega_F = F - \mu N$ & $ \dnor \Omega_F = \mu_0 H \dnor M - S\dnor T - N \dnor \mu $ \\
     $\Omega_G = G - \mu N$ & $ \dnor \Omega_G = \mu_0 M \dnor H - S\dnor T - N \dnor \mu $ \\
     \end{tabular}
\end{table*}
The particle number $N$ is given by:
\begin{equation}
    N = - \left( \pfrac{\Omega_F}{\mu} \right)_{T,M} = - \left( \pfrac{\Omega_G}{\mu} \right)_{T,M} 
\end{equation}


\end{document}


