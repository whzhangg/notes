\documentclass{article}
\usepackage{amssymb, amsmath, amsthm}
\usepackage[margin=1in]{geometry}
\usepackage{verbatim}
\usepackage{graphicx}
\usepackage{hyperref} % \url \href
\usepackage{docmute}

\newcommand{\pfrac}[2]{\frac{\partial #1}{\partial #2}}
\newcommand{\dbar}{\mathbf{\tilde{d}}}
\newcommand{\dnor}{\text{d}}
% \renewcommand{\H}{\mathcal{H}}

\begin{document}


\section{The fundamental postulation and Liouville's theorem}

\subsection{The Fundamental postulate}
An isolated system in equilibrium is equally likely to be found in any
of the microstates accessible to it.

\begin{itemize}
    \item \textbf{System} an part of universe that is only weakly coupled to the rest of the universe 
    so that its dynamic is dominated by internal interactions.
    \item \textbf{Equilibrium} the measurement of quantities are time independent.
    \item \textbf{Microstate} a complete microscope specification of coordinates of every particles (position and velocity).
    \item \textbf{Ensemble} a collection of the system that are macroscopically the same but microscopically different. 
\end{itemize}

\subsection{Motivation for the fundamental postulation}
From the above fundamental postulation and use the ergodic hypothesis, we can relate the macroscopic properties of a system that we can measure 
to the statistic (probabilistic) description of microstate. 

\textbf{Ergodic hypothesis} assumes that 1) the system's internal dynamics are such that the microstates of the system are 
constantly changing and 2) the system will visit all possible microstate and spend an equal time in each of them. 
As a result, as we carry out measurement, the system will likely to be found in a configuration (macroscopic properties) that 
is represented by the most microstates\footnote{Blundell, p35-37}. 

The following example illustrate this notation: consider a box of 100 identical coins, shaked hard and we measure the number 
of the coins facing up as we open the box (macroscopic result). We do not really care the actual configuration of the outcome 
(coin 1 face up, $\cdots$ coin N face down)
which is a microscopic property since we know each of these configurations is equally possible. 
It's easy to see that the most probable result is 50 up and 50 down, but let's see the possibility of 
outcome that deviate from this average value:
\begin{align}
    \text{50 up and 50 down}\ &= \frac{100!}{(50!)^2} \approx 4 \times 10^{27} \notag \\ 
    \text{53 up and 47 down}\ &= \frac{100!}{53! 47!} \approx 3 \times 10^{27} \notag \\
    \text{90 up and 10 down}\ &= \frac{100!}{90! 10!} \approx 10^{13} \notag \\
    \text{100 up and 0 down}\ &= 1
\end{align}
where each result is determined by counting their configurations. We see that the probability that result deviate far from the 
average decay exponentially. For actual physic system, the number of particles are $\propto 10^{23}$, which essentially 
mean that the possibility of deviation is ignorable and when we measure the macroscopic properties of physical system, we almostly
certainly obtain the value given from probabilistic calculation.

The macroscopic propertie of the system can thus be calculated as following:
For $N$ particles we have in total $6N$ coordinates $(q_1, q_2, \cdots , q_{3N}, p_1, p_2, \cdots , p_{3N})$ which completely
define a microscope state. We define the "phase density" as:
\begin{align}
    &\rho(q_1, q_2, \cdots , q_{3N}, p_1, p_2, \cdots , p_{3N}, t) \\
     &\ \ \   \to \text{Probability of finding a system near}\ (q_1, q_2, \cdots , q_{3N}, p_1, p_2, \cdots , p_{3N}) \text{at time}\ t
\end{align}
If property of this system is given by a function $f(q,p)$, then the ensemble average of $f$ at time $t$ will be 
given as:
\begin{align}
    \langle f(t) \rangle = \frac{\int\int\cdots\int f(q,p)\rho(q,p,t)dq^{3N}dp^{3N}}{\int\int\cdots\int \rho(q,p,t)dq^{3N}dp^{3N}}
\end{align}
With the above definition, $\rho(q,p,t)dq^{3N}dp^{3N}$ gives the number of states (points) that are included in the phase space 
volume $dq^{3N}dp^{3N}$ near $(q,p)$.

\subsection{Liouville's theorem}
Liouville's theorem states that the evolution of $\rho$ is given by:
\begin{equation}
    \frac{d\rho}{dt} = 0
\end{equation}
which is to say that if we follow the trajectory of a state $(q,p)$ as it 
evolve over time, its phase space density will not change (total derivative): $\rho(q(0),p(0),t=0) = \rho(q(t'),p(t'), t = t')$. 

\textbf{Proof 1} In this proof, we consider the phase space points inclosed by a volume at $t = 0$ at $(q_1,p_1)$, at a later time $\delta t$, 
we locate those phase space points agian and we show that the volume of phase space that enclose these points are the same. This will
thus mean the phase (point) density do not change following the trajectory.

let's consider an area in a two dimensional phase space that is a rectangle specified by its 2 diagonal points $(q_1,p_1),(q_2,p_2)$ at some 
initial time $t$, then at time $t+\delta t$, the points changed to $(q_1 + \dot{q_1}\delta t, p_1 + \dot{p_1}\delta t)$ and 
$(q_2 + \dot{q_2}\delta t, p_2 + \dot{p_2}\delta t)$. The volume difference, to first order in $\delta t$ is:
\begin{align}
    \Delta V &= (q_2 + \dot{q_2}\delta t - q_1 - \dot{q_1}\delta t)(p_2 + \dot{p_2}\delta t - p_1 - \dot{p_1}\delta t) - (q_2 - q_1)(p_2 - p_1) \notag \\
             &= (\dot{q_2} - \dot{q_1}) (p_2 - p_1) + (\dot{p_2} - \dot{p_1}) (q_2 - q_1) \notag \\
             &= \frac{1}{V} \left( \frac{\dot{q_2} - \dot{q_1}}{q_2-q_1} + \frac{\dot{p_2} - \dot{p_1}}{p_2-p_1} \right) \notag \\
             &= \frac{1}{V} \left( \frac{\partial \dot{q}}{\partial q} + \frac{\partial \dot{p}}{\partial p} \right) \delta t
\end{align} 
If a system envolve under Hamiltonian dynamics:
\begin{align}
    \dot{q_i} &= \frac{\partial H}{\partial p_i} ; \ \ \dot{p_i} = -\frac{\partial H}{\partial q_i}
\end{align}
then 
\begin{equation}
    \frac{\partial \dot{q}}{\partial q} + \frac{\partial \dot{p}}{\partial p} 
    = \frac{\partial^2 H}{\partial p \partial q} - \frac{\partial^2 H}{\partial q\partial p} = 0
\end{equation}
which shows that the enclosing volume of those phase space points do not change as the system evolve, and therefore, the phase space density
in this volume do not change over time, giving the result:
\begin{equation}
    \frac{d\rho}{dt} = \frac{\partial \rho}{\partial t} + 
        \sum_i \left( \frac{\partial \rho}{\partial q_i}\frac{\partial q_i}{\partial t} + \frac{\partial \rho}{\partial p_i}\frac{\partial p_i}{\partial t} \right)
        = 0
\end{equation}
Where the first equality is given merely by the definition of total derivative.

\textbf{Proof 2} In this proof, we compute the partial derivatives first and show that they result in the result of Liouville's theorem
\footnote{Taken from \url{https://hepweb.ucsd.edu/ph110b/110b_notes/node93.html}}.

We first compute $\partial \rho / \partial t$. Consider the flow the phase space points in and out of a cubic volume element 
in the phase space arount $(q,p)$, The net flow of phase space points is given by:
\begin{align}
    \frac{\partial N}{\partial t} 
    &= -\sum_i \left( \frac{\partial (\rho \dot{q_i})}{\partial q_i} + \frac{\partial (\rho \dot{p_i})}{\partial p_i} \right) dq \cdots dp \notag \\
    \frac{\partial \rho}{\partial t} 
    &= -\sum_i \left( \frac{\partial (\rho \dot{q_i})}{\partial q_i} + \frac{\partial (\rho \dot{p_i})}{\partial p_i} \right)
\end{align}
the total derivatve is then:
\begin{align}
    \frac{d\rho}{dt} &= \frac{\partial \rho}{\partial t}
    + \sum_i \left( \frac{\partial \rho}{\partial q_i}\frac{\partial q_i}{\partial t} + \frac{\partial \rho}{\partial p_i}\frac{\partial p_i}{\partial t} \right) \notag \\
    &= -\sum_i \left( \frac{\partial (\rho \dot{q_i})}{\partial q_i} + \frac{\partial (\rho \dot{p_i})}{\partial p_i} \right) 
    + \sum_i \left( \frac{\partial \rho}{\partial q_i}\frac{\partial q_i}{\partial t} + \frac{\partial \rho}{\partial p_i}\frac{\partial p_i}{\partial t} \right) \notag \\
    &= - \sum_i \left( \rho\frac{\partial \dot{q_i}}{\partial q_i} + \rho\frac{\partial \dot{p_i}}{\partial p_i}  \right) = 0
\end{align}
thus proving the Liouville's theorem.


\section{Ensembles}
We can define an ensemble as a collection of possible configurations that satisfy a given macroscopic 
property\footnote{Blundell, p38}.
We have already seen an ensemble containing possible configurations of \textbf{an isolated} system where energy
is known and all configurations are of equal probability. 
This ensemble is known as the \textbf{Microcanonical ensemble}. For a microcanonical ensemble, particle 
number, volume and energy are specified at the same time, thus it is also called \textbf{NVE ensemble}.

Now consider a system that can exchange energy through a contact with a temperature bath and eventually 
come to an equilibrium. This ensemble is called \textbf{Canonical ensemble}. For canonical ensemble, 
it's dynamic is still governed by its internal interaction but now its energy may vary. Since now 
the energy of this system can change, it is more appropriate to describe it in terms of 
temperature $T$, from which we can find its energy. 

Now we want to find the probability distribution of its microstate (probability of finding the system 
to be in a specific microstate), The system and reservior together is described by a microcanonical 
ensemble, in which each state is described by $(q_1,\cdots, q_n, q_{n+1}, \cdots q_{N}, p_1,\cdots, p_n, p_{n+1}, \cdots p_{N})$
where the first $n$ coordinates describe the system, and the rest coordinates describe the 
microstate of the reservior. Every microstates of the combined system are equal likely, therefore,
the probability to find the microstate of the canonical ensemble $(q_1',\cdots, q_n',p_1',\cdots, p_n')$ depend on the number 
of possible configurations of the coordinates in the reservior, i.e. the number of the 
microstate of the combined system in which $(q_1,\cdots, q_n,p_1,\cdots, p_n) = (q_1',\cdots, q_n',p_1',\cdots, p_n')$.

We have, with $E^r_i$ denote the energy of the reservior for 
the microstate $i$ of system, and $\Delta E = E_j - E_i$, the relative probability of two 
distinct microstates of the system(which actually depend on the number of corresponding states in the reservior):
\begin{equation}
    \frac{P_j}{P_i} = \frac{\Omega_r(E^r_i - \Delta E)}{\Omega_r(E^r_i)} = \exp\left( \ln \Omega_r(E^r_i - \ln\Delta E)-\Omega_r(E^r_i) \right)
    \approx \exp\left( -\frac{E_j-E_i}{k_BT} \right)
\end{equation}
Where we introduced the statistic definition of temperature
\begin{equation}
    \frac{\dnor \ln\Omega}{\dnor E} = \frac{1}{k_B T}
\end{equation}
This give the result, with $\beta = 1/k_BT$
\begin{equation}
    P_i \propto e^{-\beta E_i}
\end{equation}
Define the canonical partition function $Q_N$
\begin{equation}
    Q_N(T,V,N) = \sum_j e^{-\beta E_j}
\end{equation}
The free energy is defined by:
\begin{gather}
    A(T,V,N) = -k_BT\ln Q_N
\end{gather}
The probability of finding a given microstate of the system is then
\begin{equation}
    P_i = \frac{1}{Q_N} e^{-\beta E_i} = e^{\beta(A - E_i)}
\end{equation}

To specific a canconical potential, we need to specify $V, N, T$, therefore, canonical potential 
is also known as the \textbf{NVT ensemble}

%Finally, let's consider a system that can exchange both energy and particle with 
%a reservior. The equilibrium will be given by equal $T$ and $\mu$ between the 
%two part. This ensemble is named \textbf{Grand Canonical Ensemble}.
%We can derive the probability of the microstate of the system similar to the 
%case of the canconical ensemble, but now we need to consider the microstates with 
%different number of particles. 
%\begin{equation}
%    \frac{P_j}{P_i} = \frac{\Omega_r(E^r_i - \Delta E, N^r_i - \Delta N)}{\Omega_r(E^r_i,N^r_i)} = \exp\left( \frac{S_r(E^r_i - \Delta E, N^r_i - \Delta N)-S_r(E^r_i,N^r_i)}{k_B} \right)
%\end{equation}
%with $\Delta E = E_j - E_i, \Delta N = N_j - N_i$, to linear in $\Delta N, \Delta E$, we have:
%\begin{equation}
%    \frac{P_j}{P_i} = \exp\left( -\frac{1}{k_B} \frac{\partial S}{\partial E} \Delta E - \frac{1}{k_B} \frac{\partial S}{\partial N} \Delta N \right)
%    = \exp\left( -\frac{1}{k_BT} (E_j - E_i) + \frac{\mu}{k_BT} (N_j - N_i) \right)
%\end{equation}
%giving
%\begin{equation}
%    P_i \propto e^{-\beta (E_i - \mu N)}
%\end{equation}
%Define the grand canonical partition function 
%\begin{equation}
%    Q(\mu,T,V) = \sum_N\sum_j e^{-\beta (E_j - \mu N)}
%\end{equation}
%and the grand potential 
%\begin{eqnarray}
%    \Omega(\mu,T,V) = -k_BT\ln Q
%\end{eqnarray}
%we have the probability to find a microstate:
%\begin{equation}
%    P_{i,N} = \frac{1}{Q} e^{-\beta (E_i - \mu N)} = e^{\beta(\Omega - E_i + \mu N)}
%\end{equation}
%The grand canonical ensemble is known as the \textbf{$\mu$VT ensemble}


\end{document}


