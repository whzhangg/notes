\documentclass{article}
\usepackage{amssymb, amsmath, amsthm}
\usepackage[margin=1in]{geometry}
\usepackage{verbatim}
\usepackage{graphicx}
\usepackage{hyperref} % \url \href
\usepackage{docmute}

\newcommand{\pfrac}[2]{\frac{\partial #1}{\partial #2}}
\newcommand{\dbar}{\mathbf{\tilde{d}}}
\newcommand{\dnor}{\text{d}}
% \renewcommand{\H}{\mathcal{H}}

\begin{document}

\section{Intensive and extensive properties (previous version)}

\subsection{Heat}
We provide a definition of heat as \emph{The thermal energy in transit}, denoted as $Q$\footnote{Blundell, p14}. We define the heat capacity
$C$ of an object as the amount of heat that is needed to increase its temperature:
\begin{equation}
    C = \frac{dQ}{dT}
\end{equation}
thus $C$ has the unit $J/K$. The specific heat is defined to be the heat capacity per unit mass, having the unit $J/(kg\cdot K)$

\subsection{Entropy}
We define entropy $S$ for an isolated macroscopic system of $N$ particles in volume $V$ and energy $E$ to be:
\begin{equation}
    S(E,N,V,x) = k_B \ln\Omega(E,N,V,x)
\end{equation}
where $\Omega(E,N,V,x)$ is the number of accessible states at a given value of $E, N, V$, and $x$ is some constraints which 
influence the number of accessible states.
As a non-equilibrium isolated system allow to relax to equilibrium, entropy will increase monotonically and eventually maximize 
at equilibrium.
%imagine that the system will attempt to visit as many configurations as possible. 

\textbf{Temperature} 
consider an isolated system with two subsystem in weak contact but heat is allowed to follw between the two subsystem.
The total number of accessible states (configurations) are given by the product of the number of configurations of the 
two subsystem. Entropy will be additive. Consider the energy of one of the subsystem $E_1$
as the constrain for the configurations.
\begin{align}
    \Omega(E,E_1) = \Omega_1(E_1) \Omega_2(E_2) \\
    S(E,E_1) = S_1(E_1) S_2(E_2)   
\end{align}
Relaxation process will increase entropy by changing $E_1$, towards a macroscopic that correspond to 
more configurations. At equilibrium (heat no longer exchange), we have:
\begin{gather}
    \frac{\partial S}{\partial E_1} = 0 \ \Rightarrow \ 
    \left. \frac{\partial S_1(E_1)}{\partial E_1} \right|_{N_1,V_1} = \left. \frac{\partial S_2(E_2)}{\partial E_2}\right|_{N_2,V_2} = \frac{1}{T} \label{defineT}
\end{gather}
where the final equality gives the definition of temperature, thus if two subsystem reaches equilibrium in terms of energy flow, their 
temperature will be equal. This process is irreversible and thus define a arrow of time.

\textbf{Chemical potential}
now, we fix only volume $V$ of each subsystem and allow both energy and particles to exchange, then:
\begin{equation}
    \left. \frac{\partial S_1(N_1)}{\partial N_1} \right|_{E_1,V_1} = \left. \frac{\partial S_2(N_2)}{\partial N_2}\right|_{E_2,V_2} = -\frac{\mu}{T}
\end{equation}
the last equality defines the chemical potential $\mu$. 
% minus sign the divide by temperature is because of the tradition

\textbf{Pressure}
finally, we allow the volume of the system to exchange, and we can similar define pressure:
\begin{equation}
    \left. \frac{\partial S_1(V_1)}{\partial V_1} \right|_{E_1,N_1} = \left. \frac{\partial S_2(V_2)}{\partial V_2}\right|_{E_2,N_2} = \frac{P}{T}
\end{equation}

Thus, we can see that if we set an initial system not in equilibrium, the two subsystem will
start to exchange energy, particles and volume until $T$, $P$ and $\mu$ become the same 
for the two subsystem. 

We can separate the macroscopic properties of a system into two different catagory:
\begin{itemize}
    \item \textbf{Extensive properties} that will increase proportional to the system size, such as $N,E,V$
    \item \textbf{Intensive properties} that will be same for any of the subsystem, such as  $P,\mu,T$
\end{itemize}


\end{document}


