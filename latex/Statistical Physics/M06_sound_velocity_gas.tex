\documentclass{article}
\usepackage{amssymb, amsmath, amsthm}
\usepackage[margin=1in]{geometry}
\usepackage{verbatim}
\usepackage{graphicx}
\usepackage{hyperref} % \url \href
\usepackage{docmute}

\newcommand{\pfrac}[2]{\frac{\partial #1}{\partial #2}}
\newcommand{\rms}{\text{rms}}
% \newcommand{\braket}[1]{\langle #1 \rangle}
% \renewcommand{\H}{\mathcal{H}}

\begin{document}

\section{Sound velocity in gas}
In fluids , only longnitudinal wave (compression) can be transitted. Transverse wave cannot be transitted since gas cannot transmit shear.
Speed of sound for a fluid can be derived from the \textbf{continuity equation} and \textbf{Euler equation}. 
The continuity equation for a fluid is:
\begin{gather}
    \int_S \rho \textbf{u} dS = \int_V \nabla\cdot(\rho \textbf{u}) dV = -\frac{\partial}{\partial t} \int_V \rho dV \label{continuity}\\
    \nabla\cdot(\rho \textbf{u}) = - \frac{\partial \rho}{\partial t}
\end{gather} 
where $\rho$ is the local fluid density, $S$ is the surface area of a volume element $V$ and $\textbf{u}$ is the 
local fluid velocity. The first term in Eq.\ref{continuity} is the flux through the surface, the second term is obtained 
by divergence theorem. In 1D, the equation is reduced to:
\begin{equation}
    \frac{\partial (\rho u)}{\partial x} = - \frac{\partial \rho}{\partial t}
\end{equation}

The Euler equation, on the other hand, determines the dynamics of a fluid, as:
\begin{gather}
    -\frac{1}{\rho} \nabla p = \frac{\partial \textbf{u}}{\partial t} + (\textbf{u} \cdot \nabla) \textbf{u} \\
    -\frac{1}{\rho} \frac{\partial p}{\partial x} = \frac{\partial u}{\partial t} + u \frac{\partial u}{\partial x}
\end{gather}

To obtain the velocity of sound in fluid in 1D, we expand the continuity equation and divide 
both side by $\rho$:
\begin{gather}
    \frac{u}{\rho} \pfrac{\rho}{x} + \pfrac{u}{x} = - \frac{1}{\rho} \pfrac{\rho}{t} \\ 
    \pfrac{u}{x} = - \frac{1}{\rho} \pfrac{\rho}{t}
\end{gather}
where we ignored the first term since it is second order ($u d\rho$). 
We also discard the second order term in the Euler equation and obtain:
\begin{gather}
    -\frac{1}{\rho} \frac{\partial p}{\partial x} = \frac{\partial u}{\partial t} \\
    - \frac{1}{\rho^2} \left( \rho \pfrac{P}{\rho} \right) \pfrac{\rho}{x} = \frac{\partial u}{\partial t} \\
    - \frac{B}{\rho^2} \pfrac{\rho}{x} = \frac{\partial u}{\partial t}
\end{gather}
where we used the definition of the Bulk modulus:
\begin{gather}
    B = -V \pfrac{p}{V} = \rho \pfrac{p}{\rho} \\
    \text{Using  } \ d\rho = M d\left( \frac{1}{V} \right) = - M \frac{dV}{V^2} = - \rho \frac{dV}{V}
\end{gather}
Now, using the result:
\begin{align}
    \pfrac{u}{x} &= - \frac{1}{\rho} \pfrac{\rho}{t} \\
    \frac{\partial u}{\partial t} &= - \frac{B}{\rho^2} \pfrac{\rho}{x}
\end{align}
and partial differentiate with respect to $t$ and $x$, we can remove $u$:
\begin{equation}
    - \frac{\partial^2 \rho}{\partial t^2} = - \frac{B}{\rho} \frac{\partial^2 \rho}{\partial x^2}
\end{equation}
which is a wave equation and the solution is given by:
\begin{equation}
    \rho \approx e^{i(kx-\omega t)}
\end{equation}
with propogating velocity:
\begin{equation}
    v_s = \frac{\omega}{k} = \sqrt{\frac{B}{\rho}}
\end{equation}

\subsection{Sound velocity under isothermal and adiabatic conditions}
The difference between isothermal and adiabatic condition is whether temperature
or entropy (no relaxation process) is fixed during the transmission of the 
sound wave. In isothermal condition, we have:
\begin{equation}
    B_T = -V \left( \pfrac{p}{V} \right)_T = p
\end{equation}
so that the sound velocity is:
\begin{equation}
    v_s = \sqrt{\frac{B_T}{\rho}} = \sqrt{\frac{nm\langle v^2\rangle}{3\rho}} = \sqrt{\frac{\langle v^2\rangle}{3}}
\end{equation}
which coincide with the mean molecular speed in a given direction.

In the adiabatic condition, the gas obey:
\begin{equation}
    \frac{dp}{p} = -\gamma \frac{dV}{V}
\end{equation}
The adiabatic bulk modulus is then:
\begin{equation}
    B_S = -V \left( \pfrac{p}{V} \right)_S = \gamma p
\end{equation}
and the speed of sound is given by:
\begin{equation}
    v_s = \sqrt{\frac{\gamma\langle v^2 \rangle }{3}}    
\end{equation}

Since sound wave is passed by compressing and decompressing the fluid and as the fluid is compressed, their 
temperature will increase. Therefore,
The isothermal process is realized if there are enough time for thermal equilibration to take place.  
In the real case, the compression process is generally fast and the sound waves are almost always adiabatic.

\end{document}
