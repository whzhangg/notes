\documentclass{article}

\usepackage{amssymb, amsmath, amsthm}
\usepackage[margin=1in]{geometry}
\usepackage{verbatim}
\usepackage{graphicx}
\usepackage{hyperref} % \url \href
\usepackage{docmute}
\usepackage[title]{appendix}

\newtheorem{definition}{Definition}
\newtheorem{theorem}{Theorem}
\newcommand{\heff}{\mathbb{H}^{\text{eff}}}
\newcommand{\pfrac}[2]{\frac{\partial #1}{\partial #2}}
\newcommand{\huptb}{\text{H}_0}
\newcommand{\order}[2]{#1^{(#2)}}
\newcommand{\statebra}[1]{\langle #1 |}
\newcommand{\stateket}[1]{| #1 \rangle}
\newcommand{\MO}{\textbf{MO}}
\newcommand{\AO}{\textbf{AO}}

\begin{document}

\section{Symmetry Consideration}
The detailed treatment of symmetry consideration can be found in other books. Here I only
collect some notation and points important for our purpose. 
\subsection{Notation for Irreducible representation}
The notation for irreducible representation are usually used as follows:
\begin{enumerate}
    \item $A$ singly degenerate, symmetrical with respect to rotation about the principle axis (characters of these rotation are 1)
    \item $B$ singly degenerate, but antisymmetrical with repect to rotation about the principle axis (characters of these rotation are -1)
    \item $E$ doubly and $T$ triply degenerate representations.
    \item If there are more than one representation with the same label, we can use subscript: $A_1$, $A_2$ and so on.
    \item The representation with $\chi(i) = 1$ (symmetry under inversion) are given subscript $g$ (German \emph{gerade} = even)
    \item The representation that is antisymmetrical under inversion are given subscript $u$ (\emph{ungerade})
    \item The representation that is symmetrical with respect to $\sigma_h$ ($\chi(\sigma_h) = 1$) are given $'$, the representation 
            that is antisymmetrical with respect to $\sigma_h$ are given $''$ superscript.
\end{enumerate}
For a full description of the symbol, refer to \emph{Point Group Theory Table, Altmann, 1994, Page 63}. 
Usually, uppercase characters for the representation are used to denote electronic states. For 
atomic and molecular orbitals, as well as molecular vibrations, lowercase notation is used, e.g. 
$a_1, e, t_g$.

\subsection{Symmetry Consideration in Integrals} 
Symmetry require all integrals of the overlap $\statebra{\psi_a} \psi_b \rangle$ or 
interaction $\statebra{\psi_a} H \stateket{\psi_b}$ to vanish unless the wavefunction
$\stateket{\psi_a}$ and $\stateket{\psi_b}$ transform as the same irreducible 
representation of the molecular point group. Therefore, only the orbitals of the 
same symmetry may interact with each other. 

\end{document}
