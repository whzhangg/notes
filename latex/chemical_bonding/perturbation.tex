\documentclass{article}
\usepackage{amssymb, amsmath, amsthm}
\usepackage[margin=1in]{geometry}
\usepackage{verbatim}
\usepackage{graphicx}
\usepackage{hyperref} % \url \href
\usepackage{docmute}

\newtheorem{definition}{Definition}
\newtheorem{theorem}{Theorem}
\newcommand{\heff}{\mathbb{H}^{\text{eff}}}
\newcommand{\pfrac}[2]{\frac{\partial #1}{\partial #2}}

\newcommand{\huptb}{\text{H}_0}
\newcommand{\order}[2]{#1^{(#2)}}
\newcommand{\statebra}[1]{\langle #1 |}
\newcommand{\stateket}[1]{| #1 \rangle}

\begin{document}

\section{Perturbation Theory}
Here we review the time independent perturbation theory following the treatment 
given in Sakurai's book.
\subsection{Nondegenerate case}
Suppose in the un-perturbed case, we have:
\begin{gather}
    \huptb \stateket{\order{n}{0}} = E_0 \stateket{\order{n}{0}} \quad \text{and} \quad
    \sum_n \stateket{\order{n}{0}}\statebra{\order{n}{0}} = 1
\end{gather}
and introduce perturbation $\text{H} = \huptb + \lambda V$ and solve the 
perturbed eigenequation:
\begin{equation}
    (\huptb + \lambda V) \stateket{n} = E \stateket{n}
\end{equation}
The parameter $\lambda$ switches on the perturbation and setting $\lambda = 1$ 
recovers the actual perturbed Hamiltonian. 
Introduce $\Delta_n = E_n - \order{E_n}{0}$, we 
can write:
\begin{align}
    (\huptb + \lambda V) \stateket{\order{n}{\lambda}} = (\Delta_n + \order{E_n}{0}) \stateket{\order{n}{\lambda}} \\
    (\huptb - \order{E_n}{0}) \stateket{\order{n}{\lambda}} = (\Delta_n - \lambda V) \stateket{\order{n}{\lambda}}
\end{align}
since $(\huptb - \order{E_n}{0})\stateket{\order{n}{0}} = 0$, we have:
\begin{equation}
    \label{E:result1}
    \statebra{\order{n}{0}} \Delta_n - \lambda V \stateket{n}
    = \statebra{\order{n}{0}} \huptb - \order{E_n}{0} \stateket{n} 
    = 0
\end{equation} 

Define a projection operator which exclude unperturbed states $|\order{n}{0}\langle$:
$\Phi_n = 1 - \stateket{\order{n}{0}}\statebra{\order{n}{0}} = \sum_{k\neq n} \stateket{\order{k}{0}}\statebra{\order{k}{0}}$. 
The projection operator commutes with the unperturbed Hamiltonian $\huptb$:
\begin{equation}
    ( 1 - \stateket{\order{n}{0}}\statebra{\order{n}{0}} ) \huptb
    = \huptb - \order{E_n}{0} \stateket{\order{n}{0}}\statebra{\order{n}{0}} 
    = \huptb ( 1 - \stateket{\order{n}{0}}\statebra{\order{n}{0}} )
\end{equation}
using the projection operator, we have:
\begin{equation}
    (\Delta_n - \lambda V) \stateket{n}
    = (\stateket{\order{n}{0}}\statebra{\order{n}{0}} + \Phi_n) (\Delta_n - \lambda V) \stateket{n} 
    = \Phi_n (\Delta_n - \lambda V) \stateket{n}
\end{equation}
On the other hand, we are free to add:
\begin{equation}
    (\huptb - \order{E_n}{0}) \stateket{n} = 
    (\huptb - \order{E_n}{0}) (-c\stateket{\order{n}{0}} + \stateket{n} )
\end{equation}
where $c$ is some arbitrary constants, gathering the results, we have equation:
\begin{gather}
    (\huptb - \order{E_n}{0}) (-c\stateket{\order{n}{0}} + \stateket{n} ) 
    = \Phi_n (\Delta_n - \lambda V) \stateket{n} \\
    \label{E:main2}
    \stateket{n} = \stateket{\order{n}{0}} 
    + \underbrace{\frac{1}{\huptb - \order{E_n}{0}} \Phi_n (\Delta_n - \lambda V) \stateket{n}}_{\text{orthogonal to } \stateket{\order{n}{0}}}
\end{gather}
where $c$ is set to $1$ so that 
when $\lambda = 0$, the second term will be zero since both $\Delta_n$ and $\lambda V$ are zero. 
At this moment, the perturbed states are not normalized. \eqref{E:main2}
We also note that the entire second term on the right hand side is orthogonal to state $\stateket{\order{n}{0}}$
because of the projection operator can also be written explicitly to be:
\begin{align}
    \frac{1}{\huptb - \order{E_n}{0}} \Phi_n (\Delta_n - \lambda V) \stateket{n}
    = \sum_{k\neq n} \ \frac{1}{\order{E_k}{0} - \order{E_n}{0}} \stateket{\order{k}{0}} \statebra{\order{k}{0}} (\Delta_n - \lambda V) \stateket{n}
\end{align}
Suppose we write $\stateket{n} = \stateket{\order{n}{0}} + \stateket{\delta  n}$, we find that equation
\eqref{E:result1} can be reduced to 
\begin{equation}
    \label{E:main1}
    \Delta_n = \statebra{\order{n}{0}} \lambda V \stateket{n}
\end{equation}
since $\statebra{\order{n}{0}} \Delta_n \stateket{\delta n} = 0$

Now, we write out the approximate expansion in orders of $\lambda$:
\begin{align}
    \Delta_n &= \lambda \order{\Delta_n}{1} + \lambda^2 \order{\Delta_n}{2} + \lambda^3 \order{\Delta_n}{3} + \cdots \\
    \stateket{n} &= \stateket{\order{n}{0}} + \lambda \stateket{\order{n}{1}} + 
                    \lambda^2 \stateket{\order{n}{2}} + \lambda^3 \stateket{\order{n}{3}} + \cdots
\end{align}
Substituting $\stateket{n}$ and $\Delta_n$ into equation \eqref{E:main1} and equating the terms in the same order,
we have the relationship:
\begin{gather}
    \order{\Delta_n}{1} = \statebra{\order{n}{0}} V \stateket{\order{n}{0}} \\
    \order{\Delta_n}{2} = \statebra{\order{n}{0}} V \stateket{\order{n}{1}} \\
    \cdots \\
    \order{\Delta_n}{i+1} = \statebra{\order{n}{0}} V \stateket{\order{n}{i}} \\
\end{gather}
Where we find that the $(i+1)$th perturbation energy depend on the $i$th perturbed states, 
now we can simply work up the order using equation \eqref{E:main2}, for example, we can expand to 
the first few orders:
\begin{equation}
    \lambda \stateket{\order{n}{1}} + \lambda^2 \stateket{\order{n}{2}} + \cdots 
    = \frac{1}{\huptb - \order{E_n}{0}} \Phi_n (\lambda \order{\Delta_n}{1} + \lambda^2 \order{\Delta_n}{2} - \lambda V) 
    \left( 
        \stateket{\order{n}{0}} + \lambda \stateket{\order{n}{1}} + \cdots
    \right)
\end{equation}
We have:
\begin{align}
    \lambda \stateket{\order{n}{1}} &= \frac{\Phi_n}{\huptb - \order{E_n}{0}} \lambda (\order{\Delta_n}{1} - V) \stateket{\order{n}{0}} \\ 
    \lambda^2 \stateket{\order{n}{2}} &= 
    \frac{\Phi_n}{\huptb - \order{E_n}{0}} \lambda^2 
        \left[ \order{\Delta_n}{2} \stateket{\order{n}{0}} +  (\order{\Delta_n}{1} - V) \stateket{\order{n}{1}}  \right]
\end{align}
clean up the results and setting $\lambda = 1$, we have:
\begin{equation}
    O(\lambda^2): \Delta_n = V_{nn} - \sum_{k\neq n}\frac{V_{nk}V_{kn}}{\order{E_k}{0} - \order{E_n}{0}}
\end{equation}
where we find $\Phi_n \order{\Delta_n}{1} \stateket{\order{n}{0}} = \order{\Delta_n}{1} \Phi_n \stateket{\order{n}{0}} = 0$. 
For perturbed states, we have:
\begin{align}
    O(\lambda^2): \stateket{n} = \stateket{\order{n}{0}} &- \sum_{k\neq n} \frac{V_{kn}}{\order{E_k}{0} - \order{E_n}{0}} \stateket{\order{k}{0}} \\ 
    & - \left( \sum_{m\neq n}\sum_{k\neq n} \frac{V_{mk}}{\order{E_m}{0} - \order{E_n}{0}} \frac{V_{kn}}{\order{E_k}{0} - \order{E_n}{0}} \stateket{\order{m}{0}} \right.
    \left. + \sum_{k\neq n} \frac{V_{nn}V_{kn}}{(\order{E_k}{0} - \order{E_n}{0})^2} \stateket{\order{k}{0}} \right)
\end{align}


\subsection{Degenerate case}
We now consider the degenerate case where a set of eigenstates are degenerate in 
energy. We denote those states to be $\stateket{m}\in D$ with energy $\order{E_D}{0}$. 
However, instead of using $\stateket{\order{m}{0}}$ as starting points for the
degenerate set, we use $\stateket{\order{l}{0}}$ which are a linear combination of $\stateket{\order{m}{0}}$
with:
\begin{equation}
    \stateket{\order{l}{0}} = \sum_m c_m^l \stateket{\order{m}{0}}
\end{equation} 
We define the projector:
\begin{align}
    P_0 = \sum_{m\in D} \stateket{\order{m}{0}} \statebra{ \order{m}{0} } \\
    P_1 = \sum_{k\notin D} \stateket{\order{k}{0}} \statebra{ \order{k}{0} }
\end{align}
The perturbation equation can be written as:
\begin{align}
    (E_l - \huptb - \lambda V) \stateket{l} = 
    (E_l - \order{E_D}{0} - \lambda V) P_0 \stateket{l} + (E_l - \huptb - \lambda V) P_1 \stateket{l} = 0
\end{align}
again, we write the expansion:
\begin{align}    
    \Delta_l &= \lambda \order{\Delta_l}{1} + \lambda^2 \order{\Delta_l}{2} + \lambda^3 \order{\Delta_l}{3} + \cdots \\
    \stateket{l} &= \stateket{\order{l}{0}} + \lambda \stateket{\order{l}{1}} + 
                    \lambda^2 \stateket{\order{l}{2}} + \lambda^3 \stateket{\order{l}{3}} + \cdots
\end{align}
Multiplying $\statebra{\order{l}{0}}$ to the left, we find:
\begin{gather}
    \statebra{\order{l}{0}} (E_l - \huptb - \lambda V) \stateket{l} 
    = \delta_l \statebra{\order{l}{0}} l \rangle - \lambda \statebra{\order{l}{0}} V \stateket{l} = 0 \\ 
    \Delta_l = \lambda \statebra{\order{l}{0}} V \stateket{l}
\end{gather}
which is similar to the non-degenerate case with:
\begin{equation}
    \label{E:Delta_l}
    \order{\Delta_l}{i+1} = \statebra{\order{l}{0}} V \stateket{\order{l}{i}} 
\end{equation}
But now, the coefficients of $\order{l}{0}$ is not known at this moment.

We multiply projector $P_0$ and $P_1$ to the above equation using:
\begin{align}
    (E_l - \order{E_D}{0} - \lambda P_0 V) P_0 \stateket{l} - \lambda P_0 V P_1 \stateket{l} = 0 \\
    - \lambda P_1 V P_0 \stateket{l} + \underbrace{(E_l - \huptb - \lambda P_1 V)}_{\text{can be inverted}} P_1 \stateket{l} = 0
\end{align}
where we used $P_0 \huptb P_1 = 0$ to arrive at the first equation.
In the second equation, since we have projector $P_1$ in the second term, we can safely invert 
to arrive at:
\begin{equation}
    \label{E:inverted}
    P_1 \stateket{l} = \frac{\lambda }{E_l - \huptb - \lambda P_1 V} P_1 V P_0 \stateket{l}
\end{equation}
Using the relationship $\frac{1}{1-x} \approx 1 + x + x^2 + \cdots$, we can expand the fractional 
terms to be:
\begin{equation}
    \frac{1}{\order{E_l}{0} - \huptb + \lambda \order{\Delta_l}{1} - \lambda P_1 V} \approx 
    \frac{1}{\order{E_l}{0} - \huptb} + \frac{\lambda P_1 V - \lambda \order{\Delta_l}{1}}{(\order{E_l}{0} - \huptb)^2}
    + \frac{(\lambda P_1 V - \lambda \order{\Delta_l}{1})^2}{(\order{E_l}{0} - \huptb)^3} + \cdots
\end{equation}
Therefore, to first order, with $P_1 \stateket{\order{l}{0}} = 0$ 
and $P_0 \stateket{\order{l}{0}} = \stateket{\order{l}{0}}$ we have:
\begin{align}
    P_1 \stateket{\order{l}{1}} &= \frac{1}{\order{E_l}{0} - \huptb} P_1 V P_0 \stateket{\order{l}{0}} \\ 
    &= \sum_{m\in D}\sum_{ k\notin D} \frac{1}{\order{E_l}{0} - \huptb} 
    \stateket{\order{k}{0}} \statebra{ \order{k}{0} } V \stateket{\order{l}{0}} \\ 
    &= \sum_{k\notin D} \frac{V_{kl}}{\order{E_l}{0} - \order{E_k}{0}} \stateket{\order{k}{0}}
\end{align}
Putting equation \eqref{E:inverted} into the first equation, we have:
\begin{equation}
    \label{E:degenerate_main}
    \left[ (E_l - \order{E_D}{0} - \lambda P_0 V) - \lambda^2 P_0 V \frac{1}{E_l - \huptb - \lambda P_1 V} P_1 V \right] P_0 \stateket{l} = 0
\end{equation}
Keeping up to first order, we have:
\begin{equation}
    \left( \order{\Delta_l}{1} - P_0 V P_0 \right) \stateket{\order{l}{0}} = 0
\end{equation}
The part $P_0 V P_0$ is simplily a matrix:
\begin{equation}
    P_0 V P_0 = \sum_m \sum_n \statebra{m} V_{mn} \stateket{n}
\end{equation}
so that we have an eigenequation whose solution gives the coefficients $c_m^l$ and 
therefore the desired starting basis for $\stateket{\order{l}{0}}$, as well as the 
eigenvalues of $\order{\Delta_l}{1}$, which is just $\statebra{\order{l}{0}} V \stateket{\order{l}{0}}$
once $\stateket{\order{l}{0}}$ are eigenvectors.

Now we try to solve $P_0 \stateket{\order{l}{0}}$. We go back to equation \eqref{E:degenerate_main}.
However, since we know the coefficients for $\stateket{\order{l}{0}}$, we redefine the projection 
operator to be:
\begin{equation}
    P_0 = \sum_{i\in D} \stateket{\order{l_i}{0}} \statebra{\order{l_i}{0}}; \quad P_1 = 1 - P_0
\end{equation}
Furthermore, define:
\begin{align}    
    \Delta_{l_i} &= \lambda \order{\Delta_{l_i}}{1} + \lambda^2 \order{\Delta_{l_i}}{2} + \cdots \\
    \stateket{l_i} &= \stateket{\order{l_i}{0}} + \lambda \stateket{\order{l_i}{1}} + 
                    \lambda^2 \stateket{\order{l_i}{2}} + \cdots
\end{align}
Write out major terms from \eqref{E:degenerate_main}, we have:
\begin{align}
    \left[ (\lambda \order{\Delta_{l_i}}{1} - \lambda P_0 V ) - \lambda^2 P_0 V \frac{1}{\order{E_D}{0} - \huptb} P_1 V \right] \\ 
    \times P_0 \left[ \stateket{\order{l_i}{0}} + \lambda \stateket{\order{l_i}{1}} \right] = 0
\end{align}
We have the relationship:
\begin{equation}
    \lambda^2 P_0 V \frac{1}{\order{E_D}{0} - \huptb} P_1 V P_0 \stateket{\order{l_i}{0}} 
    = \lambda^2 (\order{\Delta_{l_i}}{1} - P_0 V ) P_0 \stateket{\order{l_i}{1}} 
\end{equation}
For the left hand side, with $P_0 \stateket{\order{l_i}{0}} = \stateket{\order{l_i}{0}}$ we have:
\begin{align}
    & P_0 V \frac{1}{\order{E_D}{0} - \huptb} P_1 V P_0 \stateket{\order{l_i}{0}} \\
    &= \sum_{j\in D} \stateket{ \order{l_j}{0} } \statebra{\order{l_j}{0}} V \frac{1}{\order{E_D}{0} - \huptb} 
    \sum_{k\notin D} \stateket{ \order{k}{0} } \statebra{\order{k}{0}} V \stateket{\order{l_i}{0}} \\ 
    &= \sum_{j\in D} 
        \sum_{k\notin D} \frac{V_{jk}V_{ki}}{\order{E_D}{0} - \order{E_k}{0}} \stateket{ \order{l_j}{0} }
\end{align}
For the term on the right, we have:
\begin{align}
    (\order{\Delta_{l_i}}{1} - P_0 V ) P_0 \stateket{\order{l_i}{1}}  
    &= \sum_{j\in D} \order{\Delta_{l_i}}{1} \stateket{ \order{l_j}{0} } \statebra{\order{l_j}{0}} \order{l_i}{1} \rangle
    - \sum_{j\in D} \order{\Delta_{l_j}}{1} \stateket{ \order{l_j}{0} } \statebra{\order{l_j}{0}} \order{l_i}{1} \rangle \\ 
    &= \sum_{j\in D} (\order{\Delta_{l_i}}{1} - \order{\Delta_{l_j}}{1}) \stateket{ \order{l_j}{0} } \statebra{\order{l_j}{0}} \order{l_i}{1} \rangle
\end{align}
so that we have:
\begin{equation}
    \sum_{j\in D} (\order{\Delta_{l_i}}{1} - \order{\Delta_{l_j}}{1}) \stateket{ \order{l_j}{0} } \statebra{\order{l_j}{0}} \order{l_i}{1} \rangle
    = \sum_{j\in D} \sum_{k\notin D} \frac{V_{jk}V_{ki}}{\order{E_D}{0} - \order{E_k}{0}} \stateket{ \order{l_j}{0} }
\end{equation}
if $j = i$, we find the term on the left become zero. So that we can ignore $i$ in the summation and 
invert, obtaining\footnote{This result has a different sign compared to the one given in Equation 5.2.14, but I cannot see the problem}:
\begin{equation}
    P_0 \stateket{ \order{l_i}{1}} = \sum_{j\neq i} \sum_{k\notin D} \frac{V_{jk}V_{ki}}{(\order{\Delta_{l_i}}{1} - \order{\Delta_{l_j}}{1})(\order{E_D}{0} - \order{E_k}{0})} \stateket{ \order{l_j}{0} }
\end{equation}
Finally, we can obtain the first order correction to the eigenstates:
\begin{equation}
    \stateket{\order{l_i}{1}} = P_0 \stateket{\order{l_i}{1}} + P_1 \stateket{\order{l_i}{1}}
\end{equation}
and subsequently the second order energy correction using 
$\order{\Delta_l}{2} = \statebra{\order{l}{0}} V \stateket{\order{l}{1}}$. Higher order corrections 
can be worked up similarly as in the non-degenerate case. 

\end{document}
