\documentclass{article}
\usepackage{amssymb, amsmath, amsthm}
\usepackage[margin=1in]{geometry}
\usepackage{verbatim}
\usepackage{graphicx}
\usepackage{hyperref} % \url \href
\usepackage{docmute}

\newtheorem{definition}{Definition}
\newtheorem{theorem}{Theorem}
\newcommand{\heff}{\mathbb{H}^{\text{eff}}}
\newcommand{\pfrac}[2]{\frac{\partial #1}{\partial #2}}

\newcommand{\huptb}{\text{H}_0}
\newcommand{\order}[2]{#1^{(#2)}}
\newcommand{\statebra}[1]{\langle #1 |}
\newcommand{\stateket}[1]{| #1 \rangle}

\begin{document}

\section{Perturbation Theory}
Suppose in the un-perturbed case, we have:
\begin{gather}
    \huptb \stateket{\order{n}{0}} = E_0 \stateket{\order{n}{0}} \quad \text{and} \quad
    \sum_n \stateket{\order{n}{0}}\statebra{\order{n}{0}} = 1
\end{gather}
and introduce perturbation $\text{H} = \huptb + \lambda V$ and solve the 
perturbed eigenequation:
\begin{equation}
    (\huptb + \lambda V) \stateket{n} = E \stateket{n}
\end{equation}
The parameter $\lambda$ switches on the perturbation and setting $\lambda = 1$ 
recovers the actual perturbed Hamiltonian. 
Introduce $\Delta_n = E_n - \order{E_n}{0}$, we 
can write:
\begin{align}
    (\huptb + \lambda V) \stateket{\order{n}{\lambda}} = (\Delta_n + \order{E_n}{0}) \stateket{\order{n}{\lambda}} \\
    (\huptb - \order{E_n}{0}) \stateket{\order{n}{\lambda}} = (\Delta_n - \lambda V) \stateket{\order{n}{\lambda}}
\end{align}
since $(\huptb - \order{E_n}{0})\stateket{\order{n}{0}} = 0$, we have:
\begin{equation}
    \label{E:result1}
    \statebra{\order{n}{0}} \Delta_n - \lambda V \stateket{n}
    = \statebra{\order{n}{0}} \huptb - \order{E_n}{0} \stateket{n} 
    = 0
\end{equation} 

Define a projection operator which exclude unperturbed states $|\order{n}{0}\langle$:
$\Phi_n = 1 - \stateket{\order{n}{0}}\statebra{\order{n}{0}} = \sum_{k\neq n} \stateket{\order{k}{0}}\statebra{\order{k}{0}}$. 
The projection operator commutes with the unperturbed Hamiltonian $\huptb$:
\begin{equation}
    ( 1 - \stateket{\order{n}{0}}\statebra{\order{n}{0}} ) \huptb
    = \huptb - \order{E_n}{0} \stateket{\order{n}{0}}\statebra{\order{n}{0}} 
    = \huptb ( 1 - \stateket{\order{n}{0}}\statebra{\order{n}{0}} )
\end{equation}
using the projection operator, we have:
\begin{equation}
    (\Delta_n - \lambda V) \stateket{n}
    = (\stateket{\order{n}{0}}\statebra{\order{n}{0}} + \Phi_n) (\Delta_n - \lambda V) \stateket{n} 
    = \Phi_n (\Delta_n - \lambda V) \stateket{n}
\end{equation}
On the other hand, we are free to add:
\begin{equation}
    (\huptb - \order{E_n}{0}) \stateket{n} = 
    (\huptb - \order{E_n}{0}) (-c\stateket{\order{n}{0}} + \stateket{n} )
\end{equation}
where $c$ is some arbitrary constants, gathering the results, we have equation:
\begin{gather}
    (\huptb - \order{E_n}{0}) (-c\stateket{\order{n}{0}} + \stateket{n} ) 
    = \Phi_n (\Delta_n - \lambda V) \stateket{n} \\
    \label{E:main2}
    \stateket{n} = \stateket{\order{n}{0}} 
    + \underbrace{\frac{1}{\huptb - \order{E_n}{0}} \Phi_n (\Delta_n - \lambda V) \stateket{n}}_{\text{orthogonal to } \stateket{\order{n}{0}}}
\end{gather}
where $c$ is set to $1$ so that 
when $\lambda = 0$, the second term will be zero since both $\Delta_n$ and $\lambda V$ are zero. 
At this moment, the perturbed states are not normalized. \eqref{E:main2}
We also note that the entire second term on the right hand side is orthogonal to state $\stateket{\order{n}{0}}$
because of the projection operator can also be written explicitly to be:
\begin{align}
    \frac{1}{\huptb - \order{E_n}{0}} \Phi_n (\Delta_n - \lambda V) \stateket{n}
    = \sum_{k\neq n} \ \frac{1}{\order{E_k}{0} - \order{E_n}{0}} \stateket{\order{k}{0}} \statebra{\order{k}{0}} (\Delta_n - \lambda V) \stateket{n}
\end{align}
Suppose we write $\stateket{n} = \stateket{\order{n}{0}} + \stateket{\delta  n}$, we find that equation
\eqref{E:result1} can be reduced to 
\begin{equation}
    \label{E:main1}
    \Delta_n = \statebra{\order{n}{0}} \lambda V \stateket{n}
\end{equation}
since $\statebra{\order{n}{0}} \Delta_n \stateket{\delta n} = 0$

Now, we write out the approximate expansion in orders of $\lambda$:
\begin{align}
    \Delta_n &= \lambda \order{\Delta_n}{1} + \lambda^2 \order{\Delta_n}{2} + \lambda^3 \order{\Delta_n}{3} + \cdots \\
    \stateket{n} &= \stateket{\order{n}{0}} + \lambda \stateket{\order{n}{1}} + 
                    \lambda^2 \stateket{\order{n}{2}} + \lambda^3 \stateket{\order{n}{3}} + \cdots
\end{align}
Substituting $\stateket{n}$ and $\Delta_n$ into equation \eqref{E:main1} and equating the terms in the same order,
we have the relationship:
\begin{gather}
    \order{\Delta_n}{1} = \statebra{\order{n}{0}} V \stateket{\order{n}{0}} \\
    \order{\Delta_n}{2} = \statebra{\order{n}{0}} V \stateket{\order{n}{1}} \\
    \cdots \\
    \order{\Delta_n}{i+1} = \statebra{\order{n}{0}} V \stateket{\order{n}{i}} \\
\end{gather}
Where we find that the $(i+1)$th perturbation energy depend on the $i$th perturbed states, 
now we can simply work up the order using equation \eqref{E:main2}:


\end{document}
