\documentclass{article}
\usepackage{amssymb, amsmath, amsthm}
\usepackage[margin=1in]{geometry}
\usepackage{verbatim}
\usepackage{graphicx}
\usepackage{hyperref} % \url \href
\usepackage{docmute}

\newtheorem{definition}{Definition}
\newtheorem{theorem}{Theorem}
\newcommand{\heff}{\mathbb{H}^{\text{eff}}}
\newcommand{\pfrac}[2]{\frac{\partial #1}{\partial #2}}

\begin{document}

\section{Two-orbital solution}
We illustrate the application of the direct solution method using the two (atomic) orbital problem. Suppose the 
value of the parameters $H_{11} = \varepsilon_1^0$, $H_{22} = \varepsilon_2^0$, $H_{12}$ are known, as well as 
for the overlap: $S_{11} = S_{22} = 1$ and $S_{12}$. 
\begin{align}
    (\varepsilon_1^0 - \varepsilon_1) c_{11} + (H_{12} - \varepsilon_1 S_{12}) c_{21} &= 0 \\
    (H_{12} - \varepsilon_2 S_{12}) c_{12} + (\varepsilon_2^0 - \varepsilon_2) c_{22} &= 0 \\ 
    S_{ii} = c_{1i}^2 + c_{2i}^2 + 2 c_{1i} c_{2i} S_{12} &= 1
\end{align}
the energies can be solved by the determinant:
\begin{align}
    \left| \begin{matrix}
        \varepsilon_1^0 - \varepsilon_i & H_{12} - \varepsilon_i S_{12} \\
        H_{12} - \varepsilon_i S_{12} & \varepsilon_2^0 - \varepsilon_i
    \end{matrix} \right| &= 0 
\end{align}
therefore given by a second order equation:
\begin{equation}
    (\varepsilon_1^0 - \varepsilon_i)(\varepsilon_2^0 - \varepsilon_i) - (H_{12} - \varepsilon_i S_{12})^2 = 0
\end{equation}

\subsection{Degenerate case}
When $\varepsilon_1^0 = \varepsilon_2^0 = \varepsilon^0$, the solution for the energy can be found to be:
\begin{align}
    \varepsilon_1 = \frac{\varepsilon^0 + H_{12}}{1 + S_{12}} \approx \varepsilon^0 + (H_{12} - \varepsilon^0 S_{12}) - S_{12} (H_{12} - \varepsilon_i^0 S_{12}) \\
    \varepsilon_2 = \frac{\varepsilon^0 - H_{12}}{1 - S_{12}} \approx \varepsilon^0 - (H_{12} - \varepsilon^0 S_{12}) - S_{12} (H_{12} - \varepsilon_i^0 S_{12}) 
\end{align}
$S_{12} > 0$ and $H_{12} < 0$, we have thus $H_{12} - \varepsilon^0 S_{12} < 0$ so that $\varepsilon_1 < \varepsilon_2$, which are further shifted down by a small amount 
given by the second term.
The normalization coefficient can be solved using the secular relationship. We have:
\begin{equation}
    \frac{c_{21}}{c_{11}} = 1; \quad \frac{c_{22}}{c_{12}} = -1
\end{equation}
and the wave functios are:
\begin{align}
    \psi_1 = \frac{1}{\sqrt{2 + 2 S_{12}}} (\chi_1 + \chi_2) \\
    \psi_2 = \frac{1}{\sqrt{2 - 2 S_{12}}} (\chi_1 - \chi_2)
\end{align}

\subsection{Nondegenerate case}
Approximate results can be given for non-degenerate cases, with the solution for energies:
\begin{align}
    \varepsilon_1 \approx \varepsilon_1^0 + \frac{(H_{12} - \varepsilon_1^0 S_{12})^2}{\varepsilon_1^0 - \varepsilon_2^0} < \varepsilon_1^0 \\
    \varepsilon_2 \approx \varepsilon_2^0 + \frac{(H_{12} - \varepsilon_1^0 S_{12})^2}{\varepsilon_2^0 - \varepsilon_1^0} > \varepsilon_2^0
\end{align}
since we take $\varepsilon_2^0 > \varepsilon_1^0$. Furthermore, we have:
\begin{equation}
    \varepsilon_2^0 - \varepsilon_2 < \varepsilon_1 - \varepsilon_1^0
\end{equation}
The solution for the coefficients are given for $\psi_1$:
\begin{gather}
    \psi_1 \approx \left(1 - tS_{12} - \frac{1}{2}t^2\right) \chi_1 + t \chi_2 \\
    t = \frac{H_{12}-\varepsilon_1^0S_{12}}{\varepsilon_1^0 - \varepsilon_2^0} > 0
\end{gather}
so that $\psi_1$ is obtained by combining $\chi_1$ and $\chi_2$ in phase. 
The solution for $\psi_2$ is:
\begin{gather}
    \psi_2 \approx t' \chi_1 + \left( 1 - t' S_{12} - \frac{1}{2}t'^2 \right) \chi_2 \\
    t' = \frac{H_{12}-\varepsilon_2^0S_{12}}{\varepsilon_2^0 - \varepsilon_1^0} < 0
\end{gather}
$\psi_2$ is obtained by mixing $\chi_1$ out of phase with $\chi_2$.


\end{document}
