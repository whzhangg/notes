\documentclass{article}
\usepackage{amssymb, amsmath, amsthm}
\usepackage[margin=1in]{geometry}
\usepackage{verbatim}
\usepackage{graphicx}
\usepackage{hyperref} % \url \href
\usepackage{docmute}

\newcommand{\pfrac}[2]{\frac{\partial #1}{\partial #2}}
\newcommand{\dbar}{\mathbf{\tilde{d}}}
\newcommand{\dnor}{\text{d}}
% \renewcommand{\H}{\mathcal{H}}

\begin{document}

\section{Thermodynamic potentials(previous version)}

It follows from the definition of  $T$, $P$ and $\mu$ that 
\begin{gather}
    \frac{1}{T} = \left(\frac{\partial S}{\partial E}\right)_{N,V};\  
    \frac{\mu}{T} = \left(\frac{\partial S}{\partial N}\right)_{E,V};\ 
    \frac{P}{T} = \left(\frac{\partial S}{\partial V}\right)_{E,N}
\end{gather}
we can organize:
\begin{align}
    dS &= \frac{1}{T} dE - \frac{\mu}{T}dN + \frac{P}{T}dV \\
    dE &= TdS + \mu dN - P dV \label{partial_extensive}
\end{align}
leading to:
\begin{align}
    T = \left(\frac{\partial E}{\partial S}\right)_{N,V};\  
    \mu = \left(\frac{\partial E}{\partial N}\right)_{S,V};\ 
    P = -\left(\frac{\partial E}{\partial V}\right)_{S,N}
\end{align}
For the Eq.\ref{partial_extensive}, we can integrate the subsystems parts by parts into 
the whole system ($S$, $N$ and $V$ are extensive parts), 
since $T, \mu$ and $P$ will be the same for all subsystems, the integration
gives:
\begin{equation}
    E = TS + \mu N - PV \label{energy}
\end{equation}

Now, let's consider the canonical ensemble. The energy of the system
is no longer fixed, as opposed to microcanonical ensemble in which all the 
microstates in the ensemble have fixed energy. 
However, for each energy, it is associated with a 
probability and thus we only consider the average energy, which we define
as \textbf{internal energy}:
\begin{equation}
    U \equiv \langle E \rangle = \sum_j P_j E_j = \frac{\sum_j E_j e^{-\beta E_j}}{\sum_j e^{-\beta E_j}} = -\frac{1}{Q_N} \frac{\partial Q_N}{\partial \beta}
\end{equation}
The internal energy is now the macroscopic observable instead of energy, as
in the microcanonical case.
We wish to establish an relationship between the internal energy and 
the free energy $Q_N = \exp(-\beta A)$

Let's write
\begin{equation}
    Q_N = \sum_j e^{-\beta E_j} = \sum_{E_j} \Omega(E_j) e^{-\beta E_j}
\end{equation}
with $\Omega(E_j)$ gives the number of microstates that have energy $E_j$. With the definition of entropy $S(E) = k_B \ln \Omega(E)$, we have:
\begin{eqnarray}
    Q_N = \sum_{E_j} e^{-\beta (E_j - S(E_j)T)} \simeq e^{-\beta (E' - S(E')T)} \label{approx}
\end{eqnarray}
with $E'$ be the value that minimize the function $E - S(E)T$. If the fluctuation is small, we can consider only
the term $e^{-\beta (E' - S(E')T)}$ in the summation is not ignorable, which gives the approximation.
The requirement that $E'$ minimize function $E - S(E)T$ is:
\begin{equation}
    \left. \frac{\partial (E-S(E)T)}{\partial E}\right|_{E'} = \left. 1 - T\frac{\partial S(E)}{\partial E}\right|_{E'} = 0
\end{equation}
therefore, $E'$ is the energy in which:
\begin{equation}
    \left. \frac{\partial S(E)}{\partial E}\right|_{E'} = \frac{1}{T}
\end{equation}
that is to say, $E'$ is the energy where the 
system's temperature defined by Eq.\ref{defineT} is equal to
the temperature of the heat bath, which is the equilibrium condition. 
This motivate us to equal $E' = \langle E \rangle = U$
leading to the result that 
\begin{equation}
    A = U - TS(U) \label{eqA}
\end{equation}
$U$ minimize $U - TS(U)$ at equilibrium, therefore, the equilibrium condition of 
a canonical potential is the minimization of free energy $A$.

With Eq.\ref{eqA}, Eq.\ref{partial_extensive} and $U = \langle E \rangle$,  we have
\begin{equation}
    dA(N,V,T) = dU - TdS - SdT = -SdT - pdV + \mu dN
\end{equation}
and 
\begin{equation}
    S = -\left(\frac{\partial A}{\partial T}\right)_{N,V};\  
    \mu = \left(\frac{\partial A}{\partial N}\right)_{T,V};\ 
    P = -\left(\frac{\partial A}{\partial V}\right)_{T,N}
\end{equation}

For Grand canonical ensemble, we can apply similar method as in the canonical ensemble, but 
this time adding the particle number as a summation variable in the calculation of 
partition function, we can obtain:
\begin{equation}
    \Omega(\mu,T,V) = U - TS - \mu \bar{N}
\end{equation}
with $\bar{N}$ being the average particle number. The equilibrium 
condition of a grand canonical ensemble is then the minimization of $\Omega$. 
We also have the following relationship:
\begin{equation}
    d\Omega(\mu,T,V) = -SdT - Nd\mu - PdV
\end{equation}
and
\begin{equation}
    S = -\left(\frac{\partial \Omega}{\partial T}\right)_{\mu,V};\  
    N = -\left(\frac{\partial \Omega}{\partial \mu}\right)_{T,V};\ 
    P = -\left(\frac{\partial \Omega}{\partial V}\right)_{T,\mu}
\end{equation}

\subsection{Gibbs expression of entropy}
For the canonical ensemble, we have:
\begin{gather}
    P_i = \frac{1}{Q} e^{-\beta E_i} \notag \\
    A = -k_B T\ln Q = U - TS \notag
\end{gather}
This lead to the relationship:
\begin{equation}
    S = -k_B \sum_i P_i \ln P_i
\end{equation}
This is the Gibbs' form of entropy in terms of probability.


\end{document}


