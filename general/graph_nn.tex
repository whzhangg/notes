\documentclass{article}
\usepackage{amssymb, amsmath, amsthm}
\usepackage[margin=1in]{geometry}
\usepackage{verbatim}
\usepackage{graphicx}
\usepackage{hyperref} % \url \href

\newcommand{\bbz}{\mathbb{Z}}
\newcommand{\bbr}{\mathbb{R}}
\newcommand{\calf}{\mathcal{F}}


\begin{document}

\title{Summary for graphy neural network in chemistry}
\author{Wenhao Zhang\\National Institute of Materials Science}
\date{\today}
\maketitle

\section{Introduction}
A graph is represented as $G = (V,E)$ where $V$ is a set of nodes and $E$ is the set of edges\cite{wu_2021}, we can
write a node as $v_i$ with node index $i$ and edge $e_{ij}$ between node $i$ and $j$. 
We can put features vectors (attributes) $x_i$ and $x^e_{ij}$ on nodes and edges. 
In general, the edges can have directions. the undirected graph correspond to the special case when edge of both
direction exist between node $i$ and $j$.

For chemistry application, we in general need to present a crystal structure. 
It seems very natural that we represent a crystal structure using a graph, with atoms as nodes and bonds between atoms 
as edges. 
Since a transformation that perserve distances and angles (isometry) do not change the representation of a graph 
(if a nodes' absolute location is not included in the features), isometric transformation on the crystal structure
leave the its graph representation invariant. This is a desirable property for strcture representation. 
However, such invariance comes at a cost that some informations about the crystal structure at lost: for example, if two 
components of a molecular is connected by a single edge, then rotating only one part of the molecular while keeping the 
other part still will not change the graph representation, while it may change the molecular properties. 
Furthermore, in most cases, angular informations are not included, until \cite{directional_klicpera}. 

We describe some of the most cited graph network used in material chemistry. Sometime they are also called 
'message passing network', where informations of nodes are passed throught edges. But in general, graph network 
have more flavors, such as graph convolutions that envolve pooling, or recurrent ones. 

\section{Review of related works}
The difference between different works are mainly:
\begin{enumerate}
    \item How the graph is constructed.
    \item How the nodes and edges' feature are created
    \item How the hidden layers are updated
    \item How the result is output
\end{enumerate}
\subsection{Gilmer's message passing network}



%Cohen and Welling 2016, Weiler et al. 2018a and 2018b
\begin{thebibliography}{99}
    \bibitem{wu_2021} 
    Z. Wu, S. Pan, F. Chen, G. Long, C. Zhang, and P. S. Yu, “A Comprehensive Survey on Graph Neural Networks,” IEEE Trans. Neural Netw. Learning Syst., vol. 32, no. 1, pp. 4–24, Jan. 2021, doi: 10.1109/TNNLS.2020.2978386.
    \bibitem{directional_klicpera}
    J. Klicpera, J. Groß, and S. Günnemann, “Directional Message Passing for Molecular Graphs,” arXiv:2003.03123 [physics, stat], Mar. 2020, Accessed: Jan. 25, 2022. [Online]. Available: http://arxiv.org/abs/2003.03123
\end{thebibliography}

\end{document}