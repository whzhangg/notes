\documentclass{article}
\usepackage{amsmath}
\usepackage[margin=0.8in]{geometry}

\begin{document}
    
\title{Atomic Units and conversion}
\date{\today}

\section{Atomic units}
Hartree units are named after the physicist Douglas Hartree\footnote{the reference of this note is Wikipedia}, 
in this unit
\emph{the numerical values of the following four fundamental physical constants 
are all unity by definition}. They are:
\begin{itemize}
    \item \textbf{Reduced Planck Constant} $\hbar=1$ (unit of action)
    \item \textbf{Elementary charge} $e=1$ (unit of charge)
    \item \textbf{Bohr radius} $a_0=1$ (unit of length)
    \item \textbf{Electron mass} $m_e=1$ (unit of mass)
\end{itemize}

Each unit in this system can be expressed as a product of powers of four physical
constants \emph{without a multiplying constant}. Therefore, they are 
consistent units, meaning that given any mathematical expression, if all the values 
are in atomic unit, the result will come out as atomic unit without the need for 
conversion.
The derived units in atomic unit system
are converted to SI unit by replacing the value of those constants (the constants
are used as units).
\begin{table}[h]
    \centering
    \caption{Defining constants}
    \begin{tabular}{|l|l|l|}
        \hline 
        \textbf{Symbol} & \textbf{Definition} & \textbf{Value in SI units} \\ \hline
        $\hbar$         & $\hbar$             &$1.054571\times 10^{-34} J\cdot s$ \\ \hline
        $e$             & $e$                 & $1.602176\times 10^{-19} C$  \\ \hline
        $a_0$           & $4\pi\epsilon_0\hbar^2/(m_ee^2)$ & $5.291772\times 10^{-11} m$  \\ \hline
        $m_e$           & $m_e$               & $9.109383\times 10^{-31} kg$ \\ \hline
        $E_h$           & $\hbar^2/(m_ea_0^2) $ & $\hbar^2/(m_e a_0^2)$ \\ \hline
    \end{tabular}
\end{table}

\section{Unit conversion}
As an example, the speed of light in atomic unit of velocity is approximately 137.036,
The atomic unit of velocity is expressed by the constants $a_0 E_h /\hbar$, which is converted 
to SI unit by $a_0 E_h / \hbar = 2.187\times10^6 m/s$, and therefore 
$137.036\times 2.187\times10^6 = 2.99 \times 10^8 m/s$

\begin{table}[h]
    \centering
    \caption{Derived atomic units}
    \begin{tabular}{|l|l|l|}
        \hline
        \textbf{Atomic unit of} & \textbf{Expression} & \textbf{Value in SI} \\ \hline
        Action                  & $\hbar$           & $1.054\times 10^{-34} J\cdot s$ \\ \hline
        Charge                  & $e$               & $1.602\times 10^{-19} C$ \\ \hline
        Charge Density          & $e/a_0^3$         & $1.081\times 10^12 C\cdot m^{-3}$ \\ \hline
        Current                 & $eE_h/\hbar$      & $6.623\times 10^{-3} A$ \\ \hline
        Electric Field          & $E_h/(ea_0)$      & $5.142\times 10^11 V/m $ \\ \hline
        Electric Potential      & $E_h/e$           & $27.211V $ \\ \hline
        Electric Dipole moment  & $ea_0$            & $8.478\times 10^{-30} C\cdot m $ \\ \hline
        Energy                  & $E_h$             & $4.359\times 10^{-18} J $ \\ \hline
        Force                   & $E_h/a_0$         & $8.238\times 10^{-8} N$ \\ \hline
        Length                  & $a_0$             & $5.291\times 10^{-11} m $ \\ \hline
        Magnetic Dipole Moment  & $e\hbar/m_e$      & $1.854\times 10^{-23} J/T$ \\ \hline
        Mass                    & $m_e$             & $9.109\times 10^{-31} kg $ \\ \hline
        Momentum                & $\hbar/a_0$       & $1.992\times 10^{-24} kg\cdot m/s $ \\ \hline
        Permitivity             & $e^2/(a_0E_h)$    & $1.112\times 10^{-10} F/m$ \\ \hline
        Time                    & $\hbar/E_h$       & $2.418\times 10^{-17} s $ \\ \hline   
        Pressure                & $E_h/a_0^3$       & $2.942\times 10^{13} Pa $ \\ \hline
        Velocity                & $a_0E_h/\hbar$    & $2.187\times 10^6 m/s $ \\ \hline
    \end{tabular}
\end{table}

\end{document}