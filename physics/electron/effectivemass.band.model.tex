\documentclass{article}
\usepackage{amsmath}
\usepackage[margin=0.8in]{geometry}

\begin{document}

\title{Derivation of band model effective mass}
\date{\today}
\author{W.H.}
\maketitle

\section{Derivation of single parabolic band}
In single parabolic band, the energy dispersion relation is:
\begin{equation}
    \varepsilon(k) = \hbar^2 k^2 / 2m^*
\end{equation}
and velocity is given by:
\begin{equation}
    v =\sqrt{2\varepsilon/m^* }
\end{equation}

The energy dependent density of state $g(\varepsilon)$ is calculated, for a single spin state,
according to
\begin{gather}
    N_k(\varepsilon) = \frac{V}{8\pi^3} \frac{4}{3} \pi k^3 \notag \\
    \frac{dN_k}{d\varepsilon} = \frac{dN_k}{dk} \frac{dk}{d\varepsilon} 
            = \frac{V}{8\pi^3} 4\pi k^2 \frac{\sqrt{2m^*}}{\hbar} \frac{1}{2} \varepsilon^{-1/2} \\
    \frac{dN_k}{d\varepsilon} = \frac{V}{2\pi^2} \frac{\sqrt{2}{m^*}^{3/2}}{\hbar^3} \sqrt{\varepsilon } \\
    g(\varepsilon) = \frac{V}{4\pi^2} \left( \frac{2m^*}{\hbar^2} \right)^{3/2} \sqrt{\varepsilon }
\end{gather}

The relationship between $k$ points summation and energy integration is given by:
\begin{equation}
    \lim_{V \to \infty } \sum_k F(\varepsilon(k)) = \int_{BZ} \frac{Vdk}{8\pi^3}  F(\varepsilon(k)) 
        =V \int_{-\infty}^{\infty} d\varepsilon g(\varepsilon)F(\varepsilon)
\end{equation}

The total number of carrier is therefore:
\begin{align}
    n &= \int_{-\infty}^{\infty} g(\varepsilon)f(\varepsilon) d\varepsilon \notag \\
      &= \frac{1}{4\pi^2} \left( \frac{2m^*}{\hbar^2} \right)^{3/2} \int_{-\infty}^{\infty} \frac{\sqrt{\varepsilon}}{e^{(\varepsilon - \mu)/k_b T}+1} d\varepsilon
\end{align}

The Seebeck is given by:
\begin{equation}
    S = \zeta / \sigma
\end{equation}
with $\zeta$ and $\sigma$ given by, taking account of spin factor 2:
\begin{align}
    \sigma &= \frac{2e^2}{V} \int g(\varepsilon)\tau(\varepsilon)v^2(\varepsilon)\left(-\frac{\partial f}{\partial \varepsilon}\right) d\varepsilon \\
    \zeta  &= \frac{2e}{VT} \int (\varepsilon - \mu) g(\varepsilon)\tau(\varepsilon)v^2(\varepsilon)\left(-\frac{\partial f}{\partial \varepsilon}\right) d\varepsilon \\
\end{align}
Assuming the form of $\tau$ to be:
\begin{equation}
    \tau = A\varepsilon^\eta
\end{equation}
In the case of simplest acoustic phonon scattering, the coefficient 
is given by:
\begin{equation}
    \tau = \frac{\hbar C_1 N_v}{\pi k_b T \Xi^2} g(\varepsilon)^{-1} f(\varepsilon)
\end{equation}

We find the energy independent term in the integral can be moved out and get cancelled in the division. 
So we are left with:
\begin{equation}
    S = \frac{1}{eT} \frac{\int \varepsilon^{3/2+\eta} (\varepsilon -\mu) \left(-\frac{\partial f}{\partial \varepsilon}\right) d\varepsilon} 
                {\int \varepsilon^{3/2+\eta} \left(-\frac{\partial f}{\partial \varepsilon}\right) d\varepsilon}
\end{equation}

\section{Anisotropy of Effective mass}
In the case of ellipsoid valleys, we can define anisotropic effective mass $m^*_{\perp}$ and $m^*_{\parallel}$. The
DOS effective mass of such a valley can be averaged as:
\begin{equation}
    m^*_{DOS} = ({m^*_{\perp}}^2 m^*_{\parallel})^{1/3}
\end{equation}

But for the mobility, the inertial effective mass can be given 
as 
\begin{equation}
    m^*_{I} = 3/(\frac{2}{m^*_{\perp}}+\frac{1}{m^*_{\parallel}})
\end{equation}

\section{In the case of multivalley}
For multivalley, we take the degeneracy to be $N_v$ and the 
total DOS effective mass can be written using the single 
band value:
\begin{equation}
    m^*_{total} = N_v^{2/3}m^*
\end{equation}.

\section{Some reference}
Below is a table of seebeck coefficients of simple metal, for collaboration purpose:
\begin{table}[h]
    \caption{Parameter in simple metals}
    \centering
    \begin{tabular}{lrrr}
        \hline
        Elements & $Z$ & $n(10^{22} cm^{-3})$ & Seebeck ($\mu V/K$) \\ \hline
        Na       &  1  & 2.65                 & -7   \\
        Ag       &  1  & 5.86                 & 1.5  \\
        K        &  1  & 1.40                 & -14  \\
        Al       &  3  & 18.1                 & -1.5 \\ \hline
    \end{tabular}
\end{table}

\end{document}