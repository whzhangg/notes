\documentclass{article}
\usepackage{amsmath}
\usepackage[margin=0.8in]{geometry}
\newcommand{\pfrac}[2]{\frac{\partial #1}{\partial #2}}

\begin{document}

\title{Adaptive Smearing Method}
\author{WH}
\date{\today}
\maketitle

\section{The case of three phonon interation}
we follow the derivation of the \emph{ShengBTE} paper. 
\footnote{ShengBTE: a solver for boltzmann transport equation for phonon}
To approximate a delta function $\delta(q_1 - q_2 -q_3)\delta(\omega_1 - \omega_2 - \omega_3)$, we 
first write:
\begin{equation}
    W(q_2) = \omega_2 + \omega_3
\end{equation}
to linear approximation, we have:
\begin{equation}
    W(q_2') \approx W(q_2^0) + \sum_\mu \left( \pfrac{W}{q_2^{\mu}} \right) \left( {q'}_2^{\mu} - {q^0}_2^{\mu}\right) \label{3ph}
\end{equation}
where $q^0$ indicate a $q$ point that we sampled in the BZ. The mean square deviation of $W(q_2')$
in the volume $\Omega_{BZ} / N_q$ around $q^0$ will be given by:
\begin{equation}
    \sigma^2_{W,q_2^0} = \sum_\mu \left( \pfrac{W}{q_2^{\mu}} \right)^2 E\left\{ \left( q_2'^{\mu} - q_2^{0\mu}\right)^2 \right\}
\end{equation}
using:
\begin{equation}
    E\left\{ \left( q_2'^{\mu} - q_2^{0\mu}\right)^2 \right\} 
        = \left( \frac{ \Delta q_2^{\mu} }{\sqrt{12}} \right)^2
\end{equation}
so that we have, for the mean square deviation:
\begin{align}
    \sigma_{W,q_2^0} &= \frac{1}{\sqrt{12}} \sqrt{ \sum_\mu \left( \pfrac{W}{q_2^{\mu}} \right)^2
                            ( \Delta q_2^{\mu} )^2  } \\
        &= \frac{1}{\sqrt{12}} \sqrt{ \sum_\mu \left[ \sum_{\alpha}\left( \pfrac{W}{q_2^{\alpha}} \right)
                 \frac{Q_{\alpha}^{\mu}}{ N_{\mu}}  \right]^2  }
\end{align}
where we use $\mu$ to indicate direction along a reciprocial lattice vector 
and $\alpha$ for the cartesian direction.
The term $ \partial W / \partial q_2^{\alpha}$ is given by:
\begin{equation}
   \frac{ \partial(\omega_2 + \omega_3) } {\partial q_2^{\alpha} } 
    = v_{q_2}^{\alpha} + \sum_{\beta} v_{q_3}^{\beta} \frac{\partial q_3^{\beta}}{\partial q_2^{\alpha}}
    = v_{q_1}^{\alpha} - v_{q_2}^{\alpha}
\end{equation}
where the second equal comes from the requirement that $q_1 + q_2 = q$. The final smearing width is given 
by the group velocity of phonons:
\begin{equation}
    \sigma_{W,q_2^0} = \frac{1}{\sqrt{12}} \sqrt{ \sum_\mu \left[ \sum_{\alpha}\left( v_{q_2}^{\alpha} - v_{q_3}^{\alpha} \right)
                 \frac{Q_{\alpha}^{\mu}}{ N_{\mu}}  \right]^2  } \label{final1}
\end{equation}
The second delta function $\delta(q_1 - q_2 -q_3)\delta(\omega_1 + \omega_2 - \omega_3)$ gives:
\begin{equation}
    \frac{ \partial( - \omega_2 + \omega_3) } {\partial q_2^{\alpha} } 
    = - v_{q_2}^{\alpha} + \sum_{\beta} v_{q_3}^{\beta} \frac{\partial q_3^{\beta}}{\partial q_2^{\alpha}}
    = - ( v_{q_1}^{\alpha} + v_{q_2}^{\alpha} )
\end{equation}
so that 
\begin{equation}
    \sigma_{W,q_2^0} = \frac{1}{\sqrt{12}} \sqrt{ \sum_\mu \left[ \sum_{\alpha}\left( v_{q_2}^{\alpha} + v_{q_3}^{\alpha} \right)
                 \frac{Q_{\alpha}^{\mu}}{ N_{\mu}}  \right]^2  } \label{final2}
\end{equation}

\section{The case of four phonon}
The delta function in the case of four phonon interaction is:
\begin{gather}
    \delta(q_1 - q_2 - q_3 - q_4)\delta(\omega_1 - \omega_2 - \omega_3 - \omega_4) \notag \\
    \delta(q_1 - q_2 - q_3 - q_4)\delta(\omega_1 + \omega_2 - \omega_3 - \omega_4) \notag \\
    \delta(q_1 - q_2 - q_3 - q_4)\delta(\omega_1 - \omega_2 + \omega_3 + \omega_4) \notag \\
\end{gather}
with now two free paramter, we have, for the first case:
\begin{equation}
    W(q_2, q_3) = \omega_2 + \omega_3 + \omega_4
\end{equation}
To linear approximation, we can write:
\begin{align}
    W(q_2',q_3') &\approx W(q_2^0,q_3^0) 
                + \sum_\mu \left( \pfrac{W}{q_2^{\mu}} \right) \left( q_2'^{\mu} - q_2^{0\mu}\right)
                + \sum_\nu \left( \pfrac{W}{q_3^{\nu}} \right) \left( q_3'^{\nu} - q_3^{0\nu}\right)
\end{align}
The mean square deviation is given by:
\begin{equation}
    \sigma^2_{W,q_2^0,q_3^0} = \sum_\mu \left( \pfrac{W}{q_2^{\mu}} \right)^2 E\left\{ \left( q_2'^{\mu} - q_2^{0\mu}\right)^2 \right\}
                             + \sum_\nu \left( \pfrac{W}{q_3^{\nu}} \right)^2 E\left\{ \left( q_3'^{\nu} - q_3^{0\nu}\right)^2 \right\}
\end{equation}
so that sigma is given by:
\begin{equation}
    \sigma_{W,q_2^0,q_3^0} = \frac{1}{\sqrt{12}} \sqrt{ \sum_\mu \left( \pfrac{W}{q_2^{\mu}} \right)^2 ( \Delta q_2^{\mu} )^2 
                                                      + \sum_\nu \left( \pfrac{W}{q_3^{\nu}} \right)^2 ( \Delta q_3^{\nu} )^2   } 
\end{equation}
changing the coordinate from reciprocial axis to cartesian axis, we have:
\begin{equation}
    \sigma_{W,q_2^0,q_3^0} = \frac{1}{\sqrt{12}} 
        \sqrt{ \sum_\mu \left[ \sum_{\alpha}\left( \pfrac{W}{q_2^{\alpha}} \right) \frac{Q_{\alpha}^{\mu}}{ N_{\mu}}  \right]^2 
             + \sum_\nu \left[ \sum_{\beta}\left( \pfrac{W}{q_3^{\beta}} \right) \frac{Q_{\beta}^{\nu}}{ N_{\nu}}  \right]^2  } 
\end{equation}
the linear coefficients are given by:
\begin{align}
    \pfrac{W}{q_2^{\alpha}} &= \frac{ \partial \left[ \omega_2 + \omega_3 + \omega_4 \right] }{\partial q_2^{\alpha}} 
                          = v_{q_2}^{\alpha} - v_{q_4}^{\alpha} \\
    \pfrac{W}{q_3^{\alpha}} &= \frac{ \partial \left[ \omega_2 + \omega_3 + \omega_4 \right] }{\partial q_3^{\alpha}} 
                          = v_{q_3}^{\alpha} - v_{q_4}^{\alpha}
\end{align}
since $\frac{\partial \omega_2(q_2)}{\partial q_1^{\alpha}}\bigg|_{q_2^0} = 0$ to first order.
The final result is thus:
\begin{equation}
    \sigma_{W,q_2^0,q_3^0} = \frac{1}{\sqrt{12}} 
        \sqrt{ \sum_\mu \left[ \sum_{\alpha}\left( v_{q_2}^{\alpha} - v_{q_4}^{\alpha} \right) \frac{Q_{\alpha}^{\mu}}{ N_{\mu}}  \right]^2 
             + \sum_\nu \left[ \sum_{\beta}\left( v_{q_3}^{\beta} - v_{q_4}^{\beta} \right) \frac{Q_{\beta}^{\nu}}{ N_{\nu}}  \right]^2  } 
\end{equation}

For the other two delta function, the mean square deviation will be the same, we go through the above step to find:
\begin{align}
    \frac{ \partial \left[ \omega_2 - \omega_3 - \omega_4 \right] }{\partial q_2^{\alpha}} 
                          = v_{q_2}^{\alpha} + v_{q_4}^{\alpha} \\
    \frac{ \partial \left[ \omega_2 - \omega_3 - \omega_4 \right] }{\partial q_3^{\alpha}} 
                          = v_{q_3}^{\alpha} - v_{q_4}^{\alpha}
\end{align}
and we will have:
\begin{equation}
    \sigma_{W,q_2^0,q_3^0} = \frac{1}{\sqrt{12}} 
        \sqrt{ \sum_\mu \left[ \sum_{\alpha}\left( v_{q_2}^{\alpha} + v_{q_4}^{\alpha} \right) \frac{Q_{\alpha}^{\mu}}{ N_{\mu}}  \right]^2 
             + \sum_\nu \left[ \sum_{\beta}\left( v_{q_3}^{\beta} - v_{q_4}^{\beta} \right) \frac{Q_{\beta}^{\nu}}{ N_{\nu}}  \right]^2  } 
\end{equation}

\end{document}