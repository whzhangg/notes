\documentclass{article}
\usepackage{amssymb, amsmath, amsthm}
\usepackage[margin=1in]{geometry}
\usepackage{verbatim}
\usepackage{graphicx}
\usepackage{hyperref} % \url \href

\newcommand{\pfrac}[2]{\frac{\partial #1}{\partial #2}}
\newcommand{\dbar}{\mathbf{\tilde{d}}}
\newcommand{\dnor}{\text{d}}
% \renewcommand{\H}{\mathcal{H}}

\begin{document}

\title{Law of Thermodynamics}
\author{Wenhao}
\date{\today}
\maketitle

\tableofcontents

\section{The fundamental postulation and Liouville's theorem}

\subsection{The Fundamental postulate}
An isolated system in equilibrium is equally likely to be found in any
of the microstates accessible to it.

\begin{itemize}
    \item \textbf{System} an part of universe that is only weakly coupled to the rest of the universe 
    so that its dynamic is dominated by internal interactions.
    \item \textbf{Equilibrium} the measurement of quantities are time independent.
    \item \textbf{Microstate} a complete microscope specification of coordinates of every particles (position and velocity).
    \item \textbf{Ensemble} a collection of the system that are macroscopically the same but microscopically different. 
\end{itemize}

\subsection{Motivation for the fundamental postulation}
From the above fundamental postulation and use the ergodic hypothesis, we can relate the macroscopic properties of a system that we can measure 
to the statistic (probabilistic) description of microstate. 

\textbf{Ergodic hypothesis} assumes that 1) the system's internal dynamics are such that the microstates of the system are 
constantly changing and 2) the system will visit all possible microstate and spend an equal time in each of them. 
As a result, as we carry out measurement, the system will likely to be found in a configuration (macroscopic properties) that 
is represented by the most microstates\footnote{Blundell, p35-37}. 

The following example illustrate this notation: consider a box of 100 identical coins, shaked hard and we measure the number 
of the coins facing up as we open the box (macroscopic result). We do not really care the actual configuration of the outcome 
(coin 1 face up, $\cdots$ coin N face down)
which is a microscopic property since we know each of these configurations is equally possible. 
It's easy to see that the most probable result is 50 up and 50 down, but let's see the possibility of 
outcome that deviate from this average value:
\begin{align}
    \text{50 up and 50 down}\ &= \frac{100!}{(50!)^2} \approx 4 \times 10^{27} \notag \\ 
    \text{53 up and 47 down}\ &= \frac{100!}{53! 47!} \approx 3 \times 10^{27} \notag \\
    \text{90 up and 10 down}\ &= \frac{100!}{90! 10!} \approx 10^{13} \notag \\
    \text{100 up and 0 down}\ &= 1
\end{align}
where each result is determined by counting their configurations. We see that the probability that result deviate far from the 
average decay exponentially. For actual physic system, the number of particles are $\propto 10^{23}$, which essentially 
mean that the possibility of deviation is ignorable and when we measure the macroscopic properties of physical system, we almostly
certainly obtain the value given from probabilistic calculation.

The macroscopic propertie of the system can thus be calculated as following:
For $N$ particles we have in total $6N$ coordinates $(q_1, q_2, \cdots , q_{3N}, p_1, p_2, \cdots , p_{3N})$ which completely
define a microscope state. We define the "phase density" as:
\begin{align}
    &\rho(q_1, q_2, \cdots , q_{3N}, p_1, p_2, \cdots , p_{3N}, t) \\
     &\ \ \   \to \text{Probability of finding a system near}\ (q_1, q_2, \cdots , q_{3N}, p_1, p_2, \cdots , p_{3N}) \text{at time}\ t
\end{align}
If property of this system is given by a function $f(q,p)$, then the ensemble average of $f$ at time $t$ will be 
given as:
\begin{align}
    \langle f(t) \rangle = \frac{\int\int\cdots\int f(q,p)\rho(q,p,t)dq^{3N}dp^{3N}}{\int\int\cdots\int \rho(q,p,t)dq^{3N}dp^{3N}}
\end{align}
With the above definition, $\rho(q,p,t)dq^{3N}dp^{3N}$ gives the number of states (points) that are included in the phase space 
volume $dq^{3N}dp^{3N}$ near $(q,p)$.

\subsection{Liouville's theorem}
Liouville's theorem states that the evolution of $\rho$ is given by:
\begin{equation}
    \frac{d\rho}{dt} = 0
\end{equation}
which is to say that if we follow the trajectory of a state $(q,p)$ as it 
evolve over time, its phase space density will not change (total derivative): $\rho(q(0),p(0),t=0) = \rho(q(t'),p(t'), t = t')$. 

\textbf{Proof 1} In this proof, we consider the phase space points inclosed by a volume at $t = 0$ at $(q_1,p_1)$, at a later time $\delta t$, 
we locate those phase space points agian and we show that the volume of phase space that enclose these points are the same. This will
thus mean the phase (point) density do not change following the trajectory.

let's consider an area in a two dimensional phase space that is a rectangle specified by its 2 diagonal points $(q_1,p_1),(q_2,p_2)$ at some 
initial time $t$, then at time $t+\delta t$, the points changed to $(q_1 + \dot{q_1}\delta t, p_1 + \dot{p_1}\delta t)$ and 
$(q_2 + \dot{q_2}\delta t, p_2 + \dot{p_2}\delta t)$. The volume difference, to first order in $\delta t$ is:
\begin{align}
    \Delta V &= (q_2 + \dot{q_2}\delta t - q_1 - \dot{q_1}\delta t)(p_2 + \dot{p_2}\delta t - p_1 - \dot{p_1}\delta t) - (q_2 - q_1)(p_2 - p_1) \notag \\
             &= (\dot{q_2} - \dot{q_1}) (p_2 - p_1) + (\dot{p_2} - \dot{p_1}) (q_2 - q_1) \notag \\
             &= \frac{1}{V} \left( \frac{\dot{q_2} - \dot{q_1}}{q_2-q_1} + \frac{\dot{p_2} - \dot{p_1}}{p_2-p_1} \right) \notag \\
             &= \frac{1}{V} \left( \frac{\partial \dot{q}}{\partial q} + \frac{\partial \dot{p}}{\partial p} \right) \delta t
\end{align} 
If a system envolve under Hamiltonian dynamics:
\begin{align}
    \dot{q_i} &= \frac{\partial H}{\partial p_i} ; \ \ \dot{p_i} = -\frac{\partial H}{\partial q_i}
\end{align}
then 
\begin{equation}
    \frac{\partial \dot{q}}{\partial q} + \frac{\partial \dot{p}}{\partial p} 
    = \frac{\partial^2 H}{\partial p \partial q} - \frac{\partial^2 H}{\partial q\partial p} = 0
\end{equation}
which shows that the enclosing volume of those phase space points do not change as the system evolve, and therefore, the phase space density
in this volume do not change over time, giving the result:
\begin{equation}
    \frac{d\rho}{dt} = \frac{\partial \rho}{\partial t} + 
        \sum_i \left( \frac{\partial \rho}{\partial q_i}\frac{\partial q_i}{\partial t} + \frac{\partial \rho}{\partial p_i}\frac{\partial p_i}{\partial t} \right)
        = 0
\end{equation}
Where the first equality is given merely by the definition of total derivative.

\textbf{Proof 2} In this proof, we compute the partial derivatives first and show that they result in the result of Liouville's theorem
\footnote{Taken from \url{https://hepweb.ucsd.edu/ph110b/110b_notes/node93.html}}.

We first compute $\partial \rho / \partial t$. Consider the flow the phase space points in and out of a cubic volume element 
in the phase space arount $(q,p)$, The net flow of phase space points is given by:
\begin{align}
    \frac{\partial N}{\partial t} 
    &= -\sum_i \left( \frac{\partial (\rho \dot{q_i})}{\partial q_i} + \frac{\partial (\rho \dot{p_i})}{\partial p_i} \right) dq \cdots dp \notag \\
    \frac{\partial \rho}{\partial t} 
    &= -\sum_i \left( \frac{\partial (\rho \dot{q_i})}{\partial q_i} + \frac{\partial (\rho \dot{p_i})}{\partial p_i} \right)
\end{align}
the total derivatve is then:
\begin{align}
    \frac{d\rho}{dt} &= \frac{\partial \rho}{\partial t}
    + \sum_i \left( \frac{\partial \rho}{\partial q_i}\frac{\partial q_i}{\partial t} + \frac{\partial \rho}{\partial p_i}\frac{\partial p_i}{\partial t} \right) \notag \\
    &= -\sum_i \left( \frac{\partial (\rho \dot{q_i})}{\partial q_i} + \frac{\partial (\rho \dot{p_i})}{\partial p_i} \right) 
    + \sum_i \left( \frac{\partial \rho}{\partial q_i}\frac{\partial q_i}{\partial t} + \frac{\partial \rho}{\partial p_i}\frac{\partial p_i}{\partial t} \right) \notag \\
    &= - \sum_i \left( \rho\frac{\partial \dot{q_i}}{\partial q_i} + \rho\frac{\partial \dot{p_i}}{\partial p_i}  \right) = 0
\end{align}
thus proving the Liouville's theorem.


\section{Ensembles}
We can define an ensemble as a collection of possible configurations that satisfy a given macroscopic 
property\footnote{Blundell, p38}.
We have already seen an ensemble containing possible configurations of \textbf{an isolated} system where energy
is known and all configurations are of equal probability. 
This ensemble is known as the \textbf{Microcanonical ensemble}. For a microcanonical ensemble, particle 
number, volume and energy are specified at the same time, thus it is also called \textbf{NVE ensemble}.

Now consider a system that can exchange energy through a contact with a temperature bath and eventually 
come to an equilibrium. This ensemble is called \textbf{Canonical ensemble}. For canonical ensemble, 
it's dynamic is still governed by its internal interaction but now its energy may vary. Since now 
the energy of this system can change, it is more appropriate to describe it in terms of 
temperature $T$, from which we can find its energy. 

Now we want to find the probability distribution of its microstate (probability of finding the system 
to be in a specific microstate), The system and reservior together is described by a microcanonical 
ensemble, in which each state is described by $(q_1,\cdots, q_n, q_{n+1}, \cdots q_{N}, p_1,\cdots, p_n, p_{n+1}, \cdots p_{N})$
where the first $n$ coordinates describe the system, and the rest coordinates describe the 
microstate of the reservior. Every microstates of the combined system are equal likely, therefore,
the probability to find the microstate of the canonical ensemble $(q_1',\cdots, q_n',p_1',\cdots, p_n')$ depend on the number 
of possible configurations of the coordinates in the reservior, i.e. the number of the 
microstate of the combined system in which $(q_1,\cdots, q_n,p_1,\cdots, p_n) = (q_1',\cdots, q_n',p_1',\cdots, p_n')$.

We have, with $E^r_i$ denote the energy of the reservior for 
the microstate $i$ of system, and $\Delta E = E_j - E_i$, the relative probability of two 
distinct microstates of the system(which actually depend on the number of corresponding states in the reservior):
\begin{equation}
    \frac{P_j}{P_i} = \frac{\Omega_r(E^r_i - \Delta E)}{\Omega_r(E^r_i)} = \exp\left( \ln \Omega_r(E^r_i - \ln\Delta E)-\Omega_r(E^r_i) \right)
    \approx \exp\left( -\frac{E_j-E_i}{k_BT} \right)
\end{equation}
Where we introduced the statistic definition of temperature
\begin{equation}
    \frac{\dnor \ln\Omega}{\dnor E} = \frac{1}{k_B T}
\end{equation}
This give the result, with $\beta = 1/k_BT$
\begin{equation}
    P_i \propto e^{-\beta E_i}
\end{equation}
Define the canonical partition function $Q_N$
\begin{equation}
    Q_N(T,V,N) = \sum_j e^{-\beta E_j}
\end{equation}
The free energy is defined by:
\begin{gather}
    A(T,V,N) = -k_BT\ln Q_N
\end{gather}
The probability of finding a given microstate of the system is then
\begin{equation}
    P_i = \frac{1}{Q_N} e^{-\beta E_i} = e^{\beta(A - E_i)}
\end{equation}

To specific a canconical potential, we need to specify $V, N, T$, therefore, canonical potential 
is also known as the \textbf{NVT ensemble}

%Finally, let's consider a system that can exchange both energy and particle with 
%a reservior. The equilibrium will be given by equal $T$ and $\mu$ between the 
%two part. This ensemble is named \textbf{Grand Canonical Ensemble}.
%We can derive the probability of the microstate of the system similar to the 
%case of the canconical ensemble, but now we need to consider the microstates with 
%different number of particles. 
%\begin{equation}
%    \frac{P_j}{P_i} = \frac{\Omega_r(E^r_i - \Delta E, N^r_i - \Delta N)}{\Omega_r(E^r_i,N^r_i)} = \exp\left( \frac{S_r(E^r_i - \Delta E, N^r_i - \Delta N)-S_r(E^r_i,N^r_i)}{k_B} \right)
%\end{equation}
%with $\Delta E = E_j - E_i, \Delta N = N_j - N_i$, to linear in $\Delta N, \Delta E$, we have:
%\begin{equation}
%    \frac{P_j}{P_i} = \exp\left( -\frac{1}{k_B} \frac{\partial S}{\partial E} \Delta E - \frac{1}{k_B} \frac{\partial S}{\partial N} \Delta N \right)
%    = \exp\left( -\frac{1}{k_BT} (E_j - E_i) + \frac{\mu}{k_BT} (N_j - N_i) \right)
%\end{equation}
%giving
%\begin{equation}
%    P_i \propto e^{-\beta (E_i - \mu N)}
%\end{equation}
%Define the grand canonical partition function 
%\begin{equation}
%    Q(\mu,T,V) = \sum_N\sum_j e^{-\beta (E_j - \mu N)}
%\end{equation}
%and the grand potential 
%\begin{eqnarray}
%    \Omega(\mu,T,V) = -k_BT\ln Q
%\end{eqnarray}
%we have the probability to find a microstate:
%\begin{equation}
%    P_{i,N} = \frac{1}{Q} e^{-\beta (E_i - \mu N)} = e^{\beta(\Omega - E_i + \mu N)}
%\end{equation}
%The grand canonical ensemble is known as the \textbf{$\mu$VT ensemble}


\section{First Law}
Suppose we have the function $f(x)$, where $x$ are parameters of the state. 
If the integral:
\begin{equation}
    \Delta f = \int_{x_i}^{x_f} df = f(x_f) - f(x_i) 
\end{equation}
i.e., if $\Delta f$ is independent of the path chosen, then we call $f$ a 
\textbf{function of state}, which only depend on the system parameter at a 
certain instance.

\textbf{Equation of states} are equation that contain only function of states:
i.e., for ideal gas, we have $pV = RT$ which depend on $p, V, T$, all of them are 
function of states.

\textbf{First law}: energy is conserved in the form of heat and work.

Writting $U$ as the internal energy, The equation of states can be written as
\begin{gather}
    \Delta U = \Delta Q + \Delta W \\
    \dnor U = \dbar Q + \dbar W = \dbar Q - p \dnor V
\end{gather}
where $\Delta$ gives the total change and $\dnor$ represent differential change. $\dbar$
represent changes that are path dependent. It is clear that 
the heat absorbed and the work done to the system depend on the specific process, and 
thus they cannot be exact differentiated with respect to the system parameters, while 
volume $V$ is a function of states.

\subsection{Heat capacitiy of gas at constant volume or pressure}
For ideal gas, we have:
\begin{align}
    \dbar Q &= \dnor U + p \dnor V \\
    \dbar Q & = \left(\pfrac{U}{T}\right)_V \dnor T + \left(\pfrac{U}{V}\right)_T \dnor V + p \dnor V \\
    \frac{\dbar Q}{\dnor T} &= \left(\pfrac{U}{T}\right)_V + \left[ \left(\pfrac{U}{V}\right)_T  + p \right] \frac{\dnor V}{\dnor T}
\end{align}
Therefore, at fixed volume, $\dnor V = 0$ and we have:
\begin{equation}
    C_V = \left(\frac{\dbar Q}{\dnor T}\right)_V = \left(\pfrac{U}{T}\right)_V
\end{equation}
at fixed pressure:
\begin{align}
    C_p & = \left(\frac{\dbar Q}{\dnor T}\right)_p \\
        & = C_V + \left[ \left(\pfrac{U}{V}\right)_T  + p \right] \left(\frac{\dnor V}{\dnor T}\right)_p
\end{align}
For ideal gas, we have $\left(\pfrac{U}{V}\right)_T = 0$. and $p \left(\frac{\dnor V}{\dnor T}\right)_p = R$,
so that we have: $C_p = C_V + R$. For an equilibrium gas at constant pressure, the volume of the gas also expand when
the temperature of the gas is increased (gas do work to the environment). 
So we in generally need to input more heat to the increase the temperature of the gas, compared to the case of constant 
volume. 

In general gas, we define the ratio $\gamma$ called \textbf{adiabatic index}:
\begin{equation}
    \gamma = \frac{C_p}{C_V}
\end{equation}


\section{Second Law}

\subsection{Reversible process}
we define a process to be reversible if \textbf{at every moment during the process, the system
is in equilibrium}. As an example, for an ideal gas, if the equation of state $pV = RT$ is hold true
at every moment during the process, then it implies that the gas is always in an equilibrium state
and the process is reversible: we could not notice if this process happens in reversed. 
We consider two types of reversible process in terms of gas: First is called \textbf{Isothermal process}
and the second is called \textbf{Adiabatic process}.

In the isothermal process, the temperature of the system is fixed, in contact with a heat reservior:
\begin{equation}
    \Delta T = 0
\end{equation}
For ideal gas, this implies that $\Delta U = 0$, for reversible process of ideal gas, The 
equation of states $pV = RT$ is always true. This gives:
\begin{gather}
    \Delta U = \dbar Q - \frac{RT}{V} \dnor V = 0
\end{gather}
\[
\boxed{\Delta Q = \int_{V_1}^{V_2} \dbar Q = \int_{V_1}^{V_2} \frac{RT}{V} \dnor V = RT \ln \frac{V_2}{V_1}}
\]

In the adiabatic process, the system is not allowed to exchange heat with the 
environment, thus we have $\dbar Q = 0$. We find:
\begin{gather}
    \dnor U = \dbar W \\
    C_V dT = - p \dnor V = - \frac{RT}{V} \dnor V \\
    \ln \frac{T_2}{T_1} = - \frac{R}{C_V} \ln \frac{V_2}{V_1}
\end{gather}
For ideal gas, $C_p = C_V + R$, $\gamma = 1 + R / C_V$ and we find:
\begin{gather}
    \ln \frac{T_2}{T_1} = (1-\gamma) \ln \frac{V_2}{V_1} \\
    \frac{T_2}{T_1} = \left( \frac{V_2}{V_1} \right) ^ {1-\gamma} = \left( \frac{V_1}{V_2} \right) ^ {\gamma-1} \\
    T V^{\gamma - 1} = \text{const}
\end{gather}
\[ \boxed{p V^{\gamma} = \text{const}} \]

where we used relationship $pV = RT$ to obtain the last relationship. Equation $p V^{\gamma}$ 
enable us to find the pressure of the system if we know the volume of the 
system during a adiabatic expansion or compression. Temperature can be derived using the ideal 
gas law then, if we know the pressure and volume.

\subsection{Second law}
The second law can be stated in the following two equivalent form:

\textbf{Clausius statement} No process whose sole effect is to transfer heat from cold to hot body

\textbf{Kelvin statement} No process is possible whose sole effect is to convert heat into work

It is important to note the word "sole effect". For example, in an isothermal process of ideal gas, 
$\dbar Q + \dbar W = 0$ so all the heat is converted into work done by the system. However, the 
accompanying effect is that the volume of the gas expanded. 
Therefore, To study the ability to convert heat into work, we should consider process which has no 
other effect other than the conversion, such as Carnot engine which work in cycles

\subsection{Carnot engine}
We define a carnot engine which consist of two reversible adiabatic and two reversible 
isothermal process, working between temperature $T_h$ and $T_l$. The carnot engine do work
in the following process:

\begin{table*}[h]
    \centering
    \begin{tabular}{ccc}
        $ A \to B $ & isothermal expansion & $Q_h = RT_h \ln \frac{V_B}{V_A}$ \\
        $ B \to C $ & adiabatic & $ \frac{T_h}{T_l} = \left( \frac{V_C}{V_B} \right) ^ {\gamma-1} $ \\
        $ C \to D $ & isothermal compression & $Q_l = - RT_h \ln \frac{V_D}{V_C}$ \\
        $ D \to A $ & adiabatic & $ \frac{T_h}{T_l} = \left( \frac{V_D}{V_A} \right) ^ {\gamma-1}$ \\
    \end{tabular}
\end{table*}

The adiabatic process lead to $\frac{V_C}{V_B} = \frac{V_D}{V_A} $, with $V_B > V_A$. Finally, we 
find the important result for a carnot energy:
\[\boxed{\frac{Q_h}{T_h} = \frac{Q_l}{T_l} }\]
\begin{equation}
    Q_h - Q_l = W 
\end{equation}

where $W$ is the work done in one carnot cycle. We can find the efficiency of the engine
\footnote{This carnot engine seems to convert all the heat absorbed into work, $Q_h - Q_l \to W$. But
it has the additonal effect to transport heat from one source to another. Therefore, this does not 
conflict with Kelvin's statement}:
\begin{equation}
    \eta = \frac{W}{Q_h} = 1 - \frac{T_l}{T_h}
\end{equation}

We have the following statement related to the (reversible) carnot engines:
\begin{itemize}
    \item Carnot engine is the most efficient engine
    \item all reversible engine operating at the same temperature environment have the same efficiency
    \item Clausius statement of the second law is the same as Kelvin's statement \footnote{See book}
\end{itemize}

\subsection{Refrigerator}
We consider the above carnot energy run backwards: it absorb heat $Q_l$ from the 
low temperature side with an isothermal expansion. Input work $W$ is provided 
during the adiabatic process, and dump heat $Q_h$ into the the hot side 
with an isothermal compression process. The result for such a refrigerator is:
\begin{gather}
    Q_l + W = Q_h \\
    \frac{Q_h}{T_h} = \frac{Q_l}{T_l} \\
    \eta = \frac{Q_l}{W} = \frac{T_l}{T_h - T_l}
\end{gather} 

\subsection{Clausius theorem}
Suppose we have a seriers of heat exchange in a working cycle of an engine. At 
each step, the engine exchange heat $\dbar Q_i $ from a \textbf{contact} with 
temperature $T_i$. The maximum work this engine can produce is then:
\begin{equation}
    \Delta W = \sum_i \dbar Q_i
\end{equation}

Furthermore, suppose each of these contact are connect to \textbf{a single environment} 
with temperature $T$ through a carnot cycle. Each of these carnot cycle can be considered
as heat pumps or refrigerators, which operates to provide (absorb) heat $\dbar Q_i $
at each contact.
We have the relationship for these carnot cycles:
\begin{gather}
    \frac{\dbar Q}{T} = \frac{\dbar Q_i}{T_i} \\
    \dbar Q = \dbar Q_i + \dbar W_i 
\end{gather}

Since this system work in cycles, it cannot absorb heat from a single heat resevior (environment at $T$) 
and output work (second law). So the total work output by this system is necessary zero (do nothing) or negative 
(external work need to be provided to get this system going). So we have the following requirement:
\begin{equation}
    \sum_i \dbar W_i + \Delta W \le 0
\end{equation}
We have:
\begin{gather}
    \dbar W_i = \dbar Q - \dbar Q_i = \dbar Q_i (\frac{T}{T_i} - 1) \\
    \Delta W = \sum_i \dbar Q_i 
\end{gather}
So we can find:
\begin{gather}
    \sum_i \dbar Q_i \frac{T}{T_i} \le 0 
\end{gather}
\[
\boxed{\sum_i \frac{\dbar Q_i}{T_i} \le 0 \ \ \text{or}\ \  \oint \frac{\dbar Q}{T} \le 0}
\]
The second equation is the result of the Clausius theorem. The equal sign is only achieved for a reversible energy,
which we write:
\[
   \boxed{ \oint \frac{\dbar Q_{rev}}{T} = 0 }
\]

\subsection{Entropy}
For an reversible process, we have the relationship:
\begin{equation}
    \oint \frac{\dbar Q_{rev}}{T} = 0
\end{equation}
suggesting that the integral:
\begin{equation}
    \int_{A}^{B} \frac{\dbar Q_{rev}}{T} \ \ \text{is path independent}
\end{equation}
We therefore define a value $\dnor S = \dbar Q_{rev} / T$ and the value 
of $S$ is an function of state. $S$ is called the (Clausius) \textbf{entropy}. 
\textbf{For adiabatic process, $\dbar Q = 0$ and $\dnor S = 0$}.

In general, we consider a cycle consists of both reversible and 
irreversible process. According to Clausius theorem, we have:
\begin{equation}
    \oint \frac{\dbar Q}{T} = \int_A^B \frac{\dbar Q}{T} + \int_B^A \frac{\dbar Q_{rev}}{T} \le 0
\end{equation}
so that we have:
\begin{equation}
    \int_A^B \frac{\dbar Q}{T} \le \int_A^B \frac{\dbar Q_{rev}}{T} = \int_A^B \dnor S  \label{entropy_increase}
\end{equation}
taking the process from $A$ to $B$ infinitsemial, we thus find:
\begin{equation}
    \frac{\dbar Q}{T} \le \dnor S
\end{equation}
and the equality only happens for reversible process.

In thermally isolated system, $\dbar Q = 0$, and we reach an
important conclusion that for such isolated system, the entropy
of the system can only increase: $ \dnor S \ge 0$.

We take note that in this example, we entropy increase from process A to process B, 
as indicated by the integral in Eq.\ref{entropy_increase}. If we choose the direction
to be B $\to$ A, then we will have  $ \dnor S \le 0$. However, for a irreversible process,
the direction is fixed and only the direction that lead to an increase of entropy
will be naturally occuring (A $\to$ B). 

For a reversible process, $\dbar Q = T \dnor S$ and the first law gives:
\begin{align}
    \dnor U & = T \dnor S + \dbar W \\
            & = T \dnor S - p \dnor V \label{internal energy}
\end{align}
Since value $U, T, S, p, V$ are all function of state. This 
result is independent of the process and also holds for 
irreversible process: 
for an irreversible process, $\dnor U = \dbar Q + \dbar W$
but:
\begin{gather}
    \dbar Q \neq  T \dnor S \\
    \dbar W \neq  - p \dnor V 
\end{gather}
and Eq.\ref{internal energy} still holds
\begin{equation}
    \dbar Q + \dbar W = \dbar Q_{rev} - p \dnor V = T \dnor S - p \dnor V
\end{equation}

Equation.\ref{internal energy} also give the result:
\begin{gather}
    p = - \left(\frac{\dnor U}{\dnor V}\right)_S \\
    T = \left(\frac{\dnor U}{\dnor S}\right)_V
\end{gather}

\subsection{Joule Expansion}
We consider a process of Joule expansion as an example of irreversible process:
suppose a container is separate into half, each with volume $V_0$. The gas is 
initially confined one side of the container with pressure $p_i$. The other side
of the container is vacuum. Now we remove the separation and let the 
gas take up the volume of the whole container.

We have the equation of states:
\begin{gather}
    p_i V_0 = RT_i \\
    p_f 2V_0 = RT_f
\end{gather}
since the internal energy of the gas does not change, $\Delta U = 0$ and therefore 
for ideal gas, its temperature remain the same: $T_i = T_f$. We thus have:
$p_f = \frac{1}{2} p_i$. and
\begin{gather}
    \dnor U = T \dnor S - p \dnor V = 0 \\
    \dnor S = \frac{p}{T} \dnor V \\ 
    \Delta S = \int_i^f \dnor S = \int_i^f \frac{p}{T} \dnor V = \int_i^f \frac{R}{V} \dnor V = R \ln2
\end{gather}
where we consider the process as a reversible isothermal expansion, but is since $S$ is function
of state, the result is true for any process
\footnote{
Joule expansion is associated with Maxwell's Demon, which is an interesting read at page 149 of the book
}.

\subsection{Entropy of macrostates and total entropy}
Since we have the statistical definition of temperature
\begin{equation}
    \frac{1}{k_BT} = \frac{\dnor \ln\Omega}{\dnor E}
\end{equation}
where $\Omega$ is the number of microstates with energy $E$. 
Assume a microcanonical ensemble (each microstates have the same energy
and are equally probably), then the internal energy coinside 
with energy $E$. Using the relation:
\begin{equation}
    T = \left( \pfrac{U}{S} \right)_V
\end{equation}
We find the result:
\begin{equation}
    S = k_B \ln \Omega
\end{equation}
which we call the \textbf{Boltzmann's definition of entropy}.
\footnote{\url{https://physics.stackexchange.com/questions/141321/what-is-the-conceptual-difference-between-gibbs-and-boltzmann-entropies}}.
The entropy increase of the joule expansion can be obtained from the Boltzmann's entropy formula
by considering the number of phase space before and after the expansion: The phase space of 
velocity remain the same while the phase space of possible particle position is doubled. Giving the 
result: $\Delta S = R \ln2$.

The above case correspond to an isolated system with fixed energy (microcanonical ensemble). We now
consider the case where the system of interest is connected to a reservior. 
The number of states microstate of the whole system is $N$, We have:
\begin{equation}
    S_{tot} = S_{sys} + S_{res} = k_B \ln N
\end{equation}
suppose that the $i^{th}$ microstate of our system correspond to $n_i$ microstate of the reservior,
then the probability of our system to be in the $i^{th}$ microstate is then $P_i = n_i / N$
with corresponding entropy of the reservior $S_{res,i} = k_B \ln n_i$. The average entropy 
of the reservior can then be calculated:
\begin{equation}
    S_{res} = \langle S_{res} \rangle = \sum_i P_i S_{res,i} = \sum_i P_i k_B\ln n_i
\end{equation}
We have than the entropy of our system:
\begin{align}
    S & = k_B \left(\sum_i p_i\right) \ln N - \sum_i P_i k_B\ln n_i \\
    & = - k_B \sum_i p_i \ln p_i
\end{align}
where we used the result $\sum_i p_i = 1$. This is known as the \textbf{Gibb expression of entropy}.
The reasoning applied here is the same as finding the probability of an canonical system. 
The probability here can be given, for example, by canonical distribution function $p_i = e^{-\beta E_i} / Z$

Gibbs entropy formula is useful because most of the system we consider are connected to some 
heat bath at some tmperature $T$. Similar to why canonical ensemble is more useful than microcanonical 
ensemble. 
Also, Gibb's formula only contain probability, which can be measured, compared to number of microstates
in Boltzmann formula, which cannot be measured.
\footnote{
This is original expression given by the book, which seems to express a slightly different concept
related to the macroscopic properties. So I kept it here in the footnote. But I think the reasoning 
is the same.

Suppose the system 
have in total N possible microstates of equal probability, 
which can be divided into $i$ macrostates that we can distinguish by experimental measurement.
Each of those macrostates contain $n_i$ microstates. We define the Gibb's entropy:
\begin{equation}
    S_{tot} = S + S_{micro}
\end{equation}
where $ S_{tot} = k_B \ln N$. The microscopic entropy is given by:
\begin{equation}
    S_{micro} = \langle S \rangle = \sum_i P_i S_i = \sum_i P_i k_B\ln n_i
\end{equation}
where the probability $p_i = n_i / N$ and $\sum_i p_i = 1$. We have:
\begin{align}
    S & = k_B \left(\sum_i p_i\right) \ln N - \sum_i P_i k_B\ln n_i \\
    & = - k_B \sum_i p_i \ln p_i
\end{align}
}

\subsection{Entropy of information}
In information theory, we define the information content $Q$ of 
a statement (or a prediction) by:
\begin{equation}
    Q = - k \log P
\end{equation}
where $P$ is the probability that the statement turn out to be true, and
$k$ is a positive scaling constants. If we take $k = 1$, and take $\log_2$ for logarithm, then
the information content is measured in bits. 

As an example, consider the 
following statement: \emph{a single coin toss give a result of "face up"}. The 
information content is:
\begin{equation}
    Q = - \log_2 P = 1
\end{equation}
Therefore, this information is 1 bit long: to store the above statement, we only require 1 bit
to store whether the result is face up with ${0, 1}$.

Now, to describe(transmit) a probability distribution, we can use(transmit) a series of statements. For example, 
to describe the probability distribution of weather (sunny, cloudy and rainy) in a day, we 
have different events with different probabilities and information contents:
\begin{itemize}
    \item the probability of a rainy day is $20\%$
    \item the probability of a sunny day is $40\%$
    \item the probability of a cloudy day is $40\%$
\end{itemize}
We associate each statement with its information content and define the \textbf{average information content}
as:
\begin{equation}
    S = \langle Q \rangle = -k \sum_i P_i \log P_i
\end{equation}
information content $S$ is called \textbf{Shannon entropy}. Shannon entropy therefore give the average information
content of succesive measurements.

As an example, consider a Bernoulli trial with outcome probability $p$. Using $k=1$, the 
Shannon entropy is calculated by:
\begin{equation}
    S = - \sum_i P_i \log_2 P_i = -p \log_2 p - (1-p) \log_2 (1-p)
\end{equation}
$S$ is maximum when $p = 0.5$, showing that the information content of a Bernoulli distribution is greatest
when the uncertainty of output is greatest.
When $p$ deviates from $0.5$, the outcome become biased and it is very likely that most measurement
give the same result. The information content with this probability distribution decrease. In the extreme case, if the event is 
deterministic with single outcome, than we will gain no information after measurement of the result.

\subsection{Entropy and encoding}
The Shannon entropy measures the average information content, which can be interpreted as the \textbf{average bits needed} to
transit the measurement, when measured in bits. Consider the following two case:

\textbf{Case 1.} Suppose an event have 8 possible outcomes with equal probability, 
the Shannon entropy is then:
\begin{equation}
    S = - 8 \times \frac{1}{8} \log_2\frac{1}{8} = 3\ bits
\end{equation}
the distribution is random, so we need to use 3 bits to store each outcome and there is no way to compress the informations.

\textbf{Case 2.} Suppose now that we have different probability of the outcome given by the following distribution:
\begin{equation*}
    {1/2, 1/4, 1/8, 1/16, 1/64, 1/64, 1/64, 1/64}
\end{equation*}
Now, the Shannon entropy is:
\begin{equation}
    S = - \sum_i P_i \log_2 P_i = 2\ bits
\end{equation}
It is now possible to compress the result by the following code: $0, 10, 110, 1110, 111100, 111101, 111110, 111111$ and the average length
of the code we need to transmit is:
\begin{equation}
    L = 1/2 \times 1 + 1/4 \times 2 + 1/8 \times 3 + 1/16 \times 4 + 4 \times 1/64 \times 6= 3\ bits
\end{equation}
which agree with the value of the Shannon entropy where the length required to store one result is given by $Q = - \log_2 P$.

\section{Third Law}
We can measure the change of entropy of a system by measuring its heat capacitiy:
\begin{gather}
    C_p = T\left(\pfrac{S}{T}\right)_p \label{thirdlawcp}\\
    S = \int \frac{C_p}{T} \dnor T
\end{gather}
so that we have:
\begin{equation}
    S(T) = S(T_0) + \int_{T_0}^{T} \frac{C_p}{T} \dnor T
\end{equation}

The third law states that the entropy of a system at absolute zero will be zero:

\textbf{Planck's statement of the third law} The entropy of all systems in internal
equilibrium is the same at absolute zero and may be taken to be zero.

It is noted that the system should be in a relaxed internal equilibrium for the third law
to hold, it is possible,
for example, to freeze a metastable glass phase instead of a crystalline phase and go down
o $T \to 0$ with non zero entropy.

As a consequence of the third law, Heat capacitiy, given by Eq.\ref{thirdlawcp} will
necessarily go to zero as $T\to0$.

As another implication, Using the statistical definition of the entropy $S = k_B \ln \Omega$
implies that the ground states of a system at absolute zero, is necessary non-degenerate with
$\Omega = 1$ so that the entropy will be zero. Such non-degeneracy would be enforced by the 
third law for real systems, as well as the requirement for $C_pto 0$. 
As an example, the curie's law gives a susceptibility $\chi \to \infty$ as $T \to 0$. But with
the third law, we require $\partial \chi / \partial T \to 0$. This problem is caused because
for Curie law, we assumed a mean field interaction where the interaction of magnetic
moment is ignored\footnote{See discussion in page 204-205}. 

\section{Thermodynamic potentials}
We call \textbf{Thermodynamic potentials} functions that are constructed from 
the functions of state. 
We define the following Thermodynamic potentials, The first being the internal 
energy, the following are the Enthalpy, Helmholtz function and the Gibbs 
function.

\begin{table*}[h]
    \centering
    \begin{tabular}{rrrr}
     $U(S,V)$ &   $U$          & $\dnor U =   T \dnor S - p \dnor V$ & $T = \left(\pfrac{U}{S}\right)_V$, $p = -\left(\pfrac{U}{V}\right)_S$ \\
     $H(S,p)$ &   $H = U + PV$ & $\dnor H =   T \dnor S + p \dnor V$ & $T = \left(\pfrac{H}{S}\right)_V$, $V = \left(\pfrac{H}{p}\right)_S$ \\
     $F(T,V)$ &   $F = U - TS$ & $\dnor F = - S \dnor T - p \dnor V$ & $S = -\left(\pfrac{F}{T}\right)_V$, $p = -\left(\pfrac{F}{V}\right)_T$ \\
     $G(T,p)$ &   $G = H - TS$ & $\dnor G = - S \dnor T + V \dnor p$ & $S = -\left(\pfrac{G}{T}\right)_p$, $V = \left(\pfrac{G}{p}\right)_T$ \\
     \end{tabular}
\end{table*}

Regarding the thermodynamic potentials, we have the following comments:
\begin{itemize}
    \item Different thermodynamic potentials are expressed in different \textbf{natural variables} because 
            their differential form. For example, because $\dnor F = - S \dnor T - p \dnor V$, the natural variables for $F$ are $U, V$. 
            The differential form implies that the change in $F$ is related to the change in $T$ or $V$. if $\dnor V = 0$, then the change
            in $F$ is independent of pressure $p$. 
    \item Each form of the thermodynamic potential gives different formula of thermodynamic quantities, which can be easier to access. For 
            example, $S = -\left(\partial{F}/\partial{T}\right)_V$ gives the value of entropy with easily accessible quantities $F, T$.
    \item Each thermodynamic potentials describe the system subject to different constrains: For a system with fixed pressure and fixed temperature
            in a process, its Gibbs function will remain the same through the process. While for a system with fixed volume and temperature, 
            it's free energy $F$ will be fixed over the process.
\end{itemize}

\subsection{Maxwell's relationship}
Maxwell's relationship can be derived from the above  definition of the thermodynamic
potential function, and they are stated as below:
\begin{gather}
    \left(\pfrac{T}{V}\right)_S = - \left(\pfrac{p}{S}\right)_V \\
    \left(\pfrac{T}{p}\right)_S = \left(\pfrac{V}{S}\right)_p \\
    \left(\pfrac{S}{V}\right)_T = - \left(\pfrac{p}{T}\right)_V \\
    \left(\pfrac{S}{p}\right)_T = - \left(\pfrac{V}{T}\right)_V 
\end{gather}

\subsection{Legendre transformation}
In general, Legendre transformation is a transformation on the real valued convex functions
of one of the variable and is usually used to convert functions of one quantity (position, pressure) into function of
the conjugate quantity (momentum, volume)
Consider function $F(x)$ with $dF/dx = s(x)$, we can write a function
$G(s)$ with property $dG/ds = x(s)$. They are related by:
\begin{equation}
    d(F+G) = sdx + xds = d(xs)
\end{equation}
so that $F(x) + G(s) = xs$. One property of Legendre transformation that the result of the transformation 
is also convex function. For details of the geometry meaning and convex requirement, see Morin's chapter.

For thermodynamic potentials, we use a \emph{non-standard definition} $dG/ds = -x(s)$, 
so that the transformation agree with the thermal dynamic variables.
leading to 
\begin{equation}
    d(F-G) = sdx + xds = d(xs)
\end{equation}
and therefore $F-G = xs$
\footnote{
    Both is obviously correct. An example of the standard usage is Legendre transformation 
    for Lagrangian and Hamiltonian is given by $F+G = xs$. see "Understanding the transformation in terms of derivatives"
    section in \url{https://en.wikipedia.org/wiki/Legendre_transformation}.
}.

We have seem that intensity and extensity properties would come in pairs, such as $(T,S)$, $(\mu,N)$
and $(P,V)$ due to their definition. We have also see that by defining different ensemble, we can
write down different potential function for each ensemble:
\begin{align}
    \text{NVT ensemble} &\Rightarrow dA(N,V,T) = -SdT - pdV + \mu dN\notag \\
    \text{$\mu$VT ensemble} &\Rightarrow d\Omega(\mu,V,T) = -SdT - pdV - Nd\mu \notag \\
    (Gibbs)\ \text{NPT ensemble} &\Rightarrow dG(N,p,T) = -SdT + Vdp + \mu dN \notag \\
    &\cdots \notag
\end{align}
The relationship between different potential function can be expressed with \textbf{Legendre transformation}.

Writting $x$ for an intensive property and $S$ as the conjugating extensive property. For a potential $F(x)$
with the thermodynamic relationship $dF = Sdx$, if another potential $G$ is related to $F(x)$ by $F(x) - G(S) = xS$,
then, we can find $dG(S) = -xdS$.

As an example, to find the Legendre transformation of the free energy in canonical ensemble $A(N,V,T)$ in terms of 
variable $V$, we have:
\begin{equation}
    dA = -pdV \notag
\end{equation} 
The transformed potential is thus:
\begin{equation}
    dG(N,p,T) = Vdp \notag
\end{equation} 
corresponding to the 
Gibbs potential. 

\subsection{Isothermal magnetization and adiabatic demagnetization}
To generalize the above treatment in general system, we can write:
\begin{equation}
    \dbar W = X dx
\end{equation}
where $X$ is some intensive generalized force and $x$ is some 
extensive generalized displacement.

Consider a system of magnetic moments arranged in a lattice. We 
assume paramagnetic system with no interaction between the magnetic moments. 
The first law of Thermodynamic for such system is 
\begin{equation}
    \dnor U = T \dnor S - m \dnor B
\end{equation}
We define the magnetic susceptibility as:
\begin{equation}
    \chi = \lim_{H\to0} \frac{M}{H} \approx \frac{\mu_0 M}{B}
\end{equation}
since $B = \mu_0(H+M)$ and $M \ll H$.
Curie's law states $\chi \propto 1/T$ and therefore
\begin{equation}
    \left(\pfrac{\chi}{T}\right)_B < 0
\end{equation}

Consider the Helmholtz function:
\begin{equation}
    \dnor F = - S \dnor T - m \dnor B
\end{equation}
which gives the Maxwell relation:
\begin{equation}
    \left(\pfrac{S}{B}\right)_T = \left(\pfrac{m}{T}\right)_B 
    \approx \frac{VB}{\mu_0} \left(\pfrac{\chi}{T}\right)_B
\end{equation}
The heat absorbed in an isothermal change of $B$ is:
\begin{equation}
    \Delta Q = T \left(\pfrac{S}{B}\right)_T \Delta B
    = \frac{TVB}{\mu_0} \left(\pfrac{\chi}{T}\right)_B \Delta B < 0
\end{equation}
So that an isothermal increase of $B$ will 
release heat to the environment.

The change in temperature in an adiabatic change of $B$ is 
\begin{equation}
    \left( \pfrac{T}{B} \right)_S = - \left( \pfrac{T}{S} \right)_B \left( \pfrac{S}{B} \right)_T
\end{equation}
using 
\begin{equation}
    C_B = T \left( \pfrac{S}{T} \right)_B    
\end{equation}
we have:
\begin{equation}
    \left( \pfrac{T}{B} \right)_S = - \frac{TVB}{\mu_0 C_B} \left(\pfrac{\chi}{T}\right)_B > 0
\end{equation}
So that in an adiabatic process, the temperature will decrease with the decrease of magnetization,
which is called \textbf{adiabatic demagnetization}

\section{Equipartition of energy}
To derive the equipartition theorem, we first consider a system
whose energy is given by a quadratic form:
\begin{equation}
    E = \alpha x^2
\end{equation}
where $\alpha$ is some positive constant and $x$ is a variable
that descript the microscopic configuration of the system. The 
probability of the system taking configuration $x$ is then 
given by the canonical distribution:
\begin{equation}
    P(x) = \frac{e^{-\beta \alpha x^2}}{\int_{-\infty}^{\infty} e^{-\beta \alpha x^2 dx}}
\end{equation}
The mean energy of the system is then:
\begin{align}
    \langle E \rangle &= \int_{-\infty}^{\infty} E P(x) dx \\
        & = \frac{\int_{-\infty}^{\infty} \alpha x^2 e^{-\beta \alpha x^2} dx}{\int_{-\infty}^{\infty} e^{-\beta \alpha x^2 dx}} \\
        & = \frac{1}{2} k_B T
\end{align}

For a system with multiple variables, we similarly have:
\begin{equation}
    E = \sum_{i=1}^n \alpha_i x_i^2
\end{equation}
The total energy is then:
\begin{align}
    \langle E \rangle 
        & = \frac{\int_{-\infty}^{\infty}\cdots\int_{-\infty}^{\infty} \sum_{i=1}^n \alpha_i x_i^2 e^{-\beta \sum_{j=1}^n \alpha_j x_j^2} dx_1 \cdots dx_n}{\int_{-\infty}^{\infty}\cdots\int_{-\infty}^{\infty} e^{-\beta \sum_{i=1} \alpha_i x_i^2} dx_1 \cdots dx_n} \\
        & = \sum_{i=1}^n \frac{\int_{-\infty}^{\infty}\cdots\int_{-\infty}^{\infty} \alpha_i x_i^2 e^{-\beta \sum_{j=1}^n \alpha_j x_j^2} dx_1 \cdots dx_n}{\int_{-\infty}^{\infty}\cdots\int_{-\infty}^{\infty} e^{-\beta \sum_{i=1} \alpha_i x_i^2} dx_1 \cdots dx_n} \\
        & = \sum_{i=1}^n \frac{\int_{-\infty}^{\infty} \alpha_i x_i^2 e^{-\beta \alpha_i x_i^2}  dx_i }{\int_{-\infty}^{\infty} e^{-\beta \alpha_i x_i^2}  dx_i } \\
        & = \sum_{i=1}^n  \frac{1}{2} k_B T = \frac{n}{2} k_B T
\end{align}
where $n$ the number of variable of the system ( microscope variable that determine the energy of the system), which we can call degree of freedom.

The formal equipartition theorem states: \emph{ If the energy of a classic system is the sum
of the $n$ quadratic variables and the system is in contact with a thermal reservior at temperature $T$, 
then its average energy is $n\times \frac{1}{2}k_B T$}

\subsection{Application of equipartition theorem to heat capacitiy of crystals}
We consider each atom in the crystal are kept at their respective equilibrium positions
by a spring. Each atom is therefore described by 6 variables: their kinetic energy
is given by their velocity $\mathbf{v}$ and their potential energy by their 
position $\mathbf{r}$. Therefore, there are $6N$ degree of freedom in a solid 
with $N$ numbers of atoms. The total energy of the crystal at temperature $T$
is then:
\begin{equation}
    \langle E \rangle = 3N k_B T
\end{equation}
and the Molar heat capacitiy of a solid is then $3N_A k_B = 3R$, known as the Dulong-Petit rule.

\subsection{Assumption and limitation of equipartition theorem}
The critical assumption we made in deriving the equipartition theorem is 
as follows:
\begin{enumerate}
    \item We have assumed that the system variable $x$ is continuous through the intergration $\int_{-\infty}^{\infty} E(x) P(x) dx$.
            However, for quantum system, the system are found to take distinct states with quantized energy levels. For example, at low temperature,
            thermal energy may not be enough to excite the system to the next energy level and the system will be in the ground state with $P = 1$.
            A classical system in this case assumes that there are always levels for system to distribute to even at low temperature and the above 
            integral is valid. 
    \item We assume that that the energy of the system can be written in a quadratic form. In most case this is valid, since the system will 
            minimize its energy in equilibrium and the energy can therefore be written as an expansion to second order: $ E = E_0 + \alpha (x - x_0)^2$ 
            around $x =x_0$. However, when temperature is high and $x$ start to deviate largely from $x_0$, 
            higher order terms become important and quadratic form is no longer valid.
\end{enumerate}
As a conclusion, we learned that the equipartition theorem is valid in a temperature range high enough so that states are almost continuous ($\Delta E \ll k_B T$ with
$\Delta E$ the energy difference between states) but 
low enough that the quadratic form of the energy is valid.

\section{The partition function}
We define the partition function as:
\begin{equation}
    Z = \sum_{\alpha} e^{-\beta E_{\alpha}}
\end{equation}
which allow us to derive all the important thermodynamic quantities. 

The internal energy is given by:
\begin{equation}
    U = \langle E \rangle = \frac{\sum_{i} E_i e^{-\beta E_i}}{\sum_{i} e^{-\beta E_i}} = -\frac{\dnor \ln Z}{\dnor\beta} = k_BT^2\frac{\dnor \ln Z}{\dnor T}
\end{equation}

The entropy is found by:
\begin{align}
    S = -k_B \sum_i P_i \ln P_i = k_B \sum_i P_i (\beta E_i + \ln Z) = k_B (\beta U + \ln Z)
\end{align}
or we can write:
\begin{equation}
    S = U/T + k_B \ln Z
\end{equation}

Helmholtz function can be written by $F = U - TS$, using the above result, we have:
\begin{gather}
    F = -k_B T \ln Z \\
    Z = e^{-\beta F}
\end{gather}

Other derived quantities can also be found:
\begin{gather}
    C_V = \left(\pfrac{U}{T}\right)_V = k_B T \left[ 2\left(\pfrac{\ln Z}{T}\right)_V + T \left( \frac{\partial^2\ln Z}{\partial T^2} \right)_V \right] \\
    p = - \left(\pfrac{F}{V}\right)_T = k_B T \left( \pfrac{\ln Z}{V} \right)_T \\
    H = U + pV =  k_B T \left[ \left( T\pfrac{\ln Z}{T}\right)_V + V \left( \pfrac{\ln Z}{V} \right)_T \right] \\
    G = F + pV =  k_B T \left[ -\ln Z + V \left( \pfrac{\ln Z}{V} \right)_T \right]
\end{gather}

It should be noted that the value of the partition function itself
is not uniquely defined, because the zero of the energy is arbitrary,
so that partitiona function can be defined up to an arbitrary multiplicative constant.
However, In terms of the thermodynamic quantities calculated in the above equations,
The result not on $Z$ but on $\ln Z$ and its derivative, so the 
those quantities are up to a additive constants or show no dependence on the 
arbitrarness of the partition function.

\section{Chemical potential and Grand potential}
We consider adding particles to a system, then the internal energy of the system will
increase by an amount, which we denote \textbf{chemical potential}. For a large system
with $\mu$ not changing significantly by adding or removing a particle, we have
\begin{equation}
    \dnor U = T \dnor S - p \dnor V + \mu \dnor N \label{du_dmu}
\end{equation}
with $N$ the particle number and $\mu$ can be written:
\begin{equation}
    \mu = \left( \pfrac{U}{N} \right)_{S,V}
\end{equation}
In terms of other constrains, we have:
\begin{gather}
    \dnor F = -p \dnor V - S \dnor T + \mu \dnor N \\
    \dnor G = V \dnor p - S \dnor T + \mu \dnor N
\end{gather}
and we have the expression for chemical potential as:
\begin{equation}
    \mu = \left( \pfrac{F}{N} \right)_{T,V} = \left( \pfrac{G}{N} \right)_{T,p}
\end{equation}

For an isolated system, the entrope will increase as system go to equilibrium. 
We write:
\begin{align}
    \dnor S &= \left( \pfrac{S}{U} \right)_{N,V} \dnor U + 
              \left( \pfrac{S}{V} \right)_{N,U} \dnor V + 
              \left( \pfrac{S}{N} \right)_{U,V} \dnor N \\
            &= \frac{\dnor U}{T} + \frac{p \dnor V}{T} - \frac{\mu \dnor N}{T}
\end{align}
where the second equality follow from Eq.\ref{du_dmu}. 
Therefore, we identify:
\begin{equation}
    \left( \pfrac{S}{U} \right)_{N,V} = \frac{1}{T}; \ \ \
    \left( \pfrac{S}{V} \right)_{N,U} = \frac{p}{T}; \ \ \
    \left( \pfrac{S}{N} \right)_{U,V} =-\frac{\mu}{T}
\end{equation}

We consider two system connected to each other and isolated from 
from the environment, If heat is allowed to follow, we have:
\begin{align}
    \dnor S &= \left( \pfrac{S_1}{U_1} \right)_{N,V} \dnor U_1 + \left( \pfrac{S_2}{U_2} \right)_{N,V} \dnor U_2 \\
        &= \left( \frac{1}{T_1} - \frac{1}{T_2} \right) \dnor U_1 \ge 0
\end{align}
Therefore, the equilibrium can be found at $T_1 = T_2$.

Similarly, if the system are allowed to exchange particle, we have:
\begin{align}
    \dnor S = \left( \frac{\mu_1}{T_1} - \frac{\mu_2}{T_2} \right) \dnor N_1 \ge 0
\end{align}
the equilibrium can be found at $\mu_1 = \mu_2$, if the two subsystem have 
already the same temperature.

\subsection{Grand partition function}
For a grand canonical ensemble, The probability that a system
will be in a state with energy $E_i$ and particle number $N_i$ are 
given by
\begin{equation}
    P_i = \frac{e^{\beta(\mu N_i - E_i)}}{\mathcal{Z}}
\end{equation}
and the grand partition function $\mathcal{Z}$ is given by:
\begin{equation}
    \mathcal{Z} = \sum_i e^{\beta(\mu N_i - E_i)}
\end{equation}

We can find the thermodynamic quantities using $\mathcal{Z}$ in 
a similar way as with $Z$, with the result:
\begin{gather}
   \langle N \rangle = \sum_i N_i P_i = k_B T \left( \pfrac{\ln \mathcal{Z}}{\mu} \right)_T \\
    U = - \left( \pfrac{\ln \mathcal{Z}}{\beta} \right)_{\mu} + \mu N \\
    S = -k_B \sum_i P_i \ln P_i = \frac{U - \mu N + k_B T\ln \mathcal{Z}}{T} \label{grand_entropy}
\end{gather}

We further define the grand potential:
\begin{equation}
    \Phi_G = -k_B T \ln \mathcal{Z}
\end{equation}
which is a function of state. We have, using Eq.\ref{grand_entropy}:
\begin{gather}
    \Phi_G = -k_B T \ln \mathcal{Z} = U - TS - \mu N = F - \mu N \\
    \dnor \Phi_G = \dnor F - \mu \dnor N - N \dnor \mu
\end{gather}

\subsection{Different types of particles}
If there are different types of particle, we can write:
\begin{gather}
    \dnor U = T \dnor S - p \dnor V + \sum_i \mu_i \dnor N_i \\
    \dnor F = -p \dnor V - S \dnor T + \sum_i \mu_i \dnor N_i \\
    \dnor G = V \dnor p - S \dnor T + \sum_i \mu_i \dnor N_i 
\end{gather}

\pagebreak
\section{Quantum Statistical Mechanics}
For a quantum \textbf{ensemble} $|\Psi\rangle$ expressed in a complete set of basis $|\phi_i\rangle$ (state of system)as 
% both \Phi and \phi_i are the state of the whole system
\begin{equation}
    |\Psi\rangle = \sum_i c_i |\phi_i\rangle
\end{equation}
we define the density matrix 
\begin{equation}
    \rho = |\Psi\rangle \langle \Psi | = \sum_{ij} |\phi_i\rangle \langle \phi_i| \Psi\rangle \langle \Psi |\phi_j\rangle \langle \phi_j|
    = \sum_{ij} c_i c_j^* |\phi_i\rangle \langle \phi_j| \label{density1}
\end{equation}
and therefore
\begin{equation}
    \langle \phi_i | \rho | \phi_j \rangle = c_i c_j^*
\end{equation}
So that the expectation value of observables $A$ is then:
\begin{equation}
    \langle A \rangle = \langle \Psi | A | \Psi \rangle
    = \sum_{ij} \langle \Psi |\phi_i\rangle \langle \phi_i| A | \phi_j\rangle \langle \phi_j|\Psi \rangle
    = \sum_{ij} c_i^* c_j \langle \phi_i| A | \phi_j\rangle
\end{equation}
while we also have:
\begin{equation}
    \text{Tr}[\rho A] = \sum_i \langle \phi_i | \rho A | \phi_i \rangle
        = \sum_{ij} \langle \phi_i | \rho | \phi_j \rangle \langle \phi_j | A | \phi_i \rangle
        = \sum_{ij} c_i c_j^* \langle \phi_j | A | \phi_i \rangle = \langle A \rangle
\end{equation}
So that we find the relationship
\begin{equation}
    \langle A \rangle = \text{Tr}[\rho A]
\end{equation}

Now, we consider the time dependence of the trace operator, using the time evolution operator:
\begin{align}
    -ih\frac{\partial}{\partial t} \rho &= -ih \frac{\partial}{\partial t}  \sum_{ij} c_i c_j^* |\phi_i\rangle \langle \phi_j| \notag \\
        &= \sum_{ij} c_i c_j^* \left( -ih\frac{\partial}{\partial t} |\phi_i\rangle \langle \phi_j| - |\phi_i\rangle ih\frac{\partial}{\partial t} \langle \phi_j|  \right) \notag \\
        &= \sum_{ij} c_i c_j^* \left(  H|\phi_i\rangle \langle \phi_j| - |\phi_i\rangle \langle \phi_j| H \right) \notag \\
        &= [H\rho  - \rho H] = [H,\rho]
\end{align}
In an equilibrium, the density operator will be time dependent: $\partial \rho / \partial t = 0$, suggesting that 
$H$ and $\rho$ commute:
$[H,\rho] = 0$. The eigenstates of the Hamiltonian thus are also the eigenstates of the density operator.
Therefore, the density operator in Eq.\ref{density1} can be written in this form:
\begin{equation}
    \rho = \sum_n w_n | n \rangle \langle n | \label{density2}
\end{equation}
where $|n\rangle$ are the eigenstates of the Hamiltonian, and $w_n = c_n^* c_n = |c_n|^2$ is the probability for the 
system $|\Psi\rangle$ to be in the energy eigenstate $|n\rangle$
For Microcanonical ensemble with $\Omega$ accessible states, we simply have:
\begin{equation}
    w_n = \frac{1}{\Omega}
\end{equation}
For Canonical ensemble, we have:
\begin{equation}
    w_n = \frac{e^{-\beta E_n}}{\sum_{n'} e^{-\beta E_{n'}}}
\end{equation}
and the partition function $Q = \text{Tr}e^{-\beta H}$ (an observables).
For Grand Canonical ensemble, taking account of the particle number in 
an system state $|n\rangle$, we have:
\begin{equation}
    w_n = \frac{e^{-\beta (E_n-\mu N_n)}}{\sum_{n'} e^{-\beta (E_{n'}-\mu N_{n'})}}
\end{equation}
and partition function $Q =  \text{Tr}e^{-\beta (H-\mu N}$, where $N$ is the operator for 
particle density.

\subsection{Distribution function}
Suppose the Hamiltonian can be written as a function of particle number
$H = \sum_p \varepsilon_p n_p$ with $\varepsilon_p$, $n_p$ denoting 
the energy and occupation number of a single particle state $p$.
We start by calculating the partition function of a quantum ensemble, since we wish 
to study the probability of system containing different particle number, we 
use the grand canonical potential:
\begin{align}
    Q(\mu,V,T) = \sum_{N=0}^{\infty} \sum_{\{n_p\}_N} e^{-\beta\sum_p (\varepsilon_p - \mu) n_p}
               = \sum_{\{n_p\}} e^{-\beta\sum_p (\varepsilon_p - \mu) n_p}
\end{align}
where in the first expression, $\{n_p\}_N$ denotes a many body state with occupation $\{n_{p_1},n_{p_2}, \cdots, n_{p_i}\}$
in each of the single particle with the constraint that the total number of particles sum up to $N$. In the second 
expression, the two summation in the first expression are combined.
We now take the summation $p$ in the exponential out:
\begin{equation}
    Q(\mu,V,T) = \sum_{\{n_p\}} e^{-\beta\sum_p (\varepsilon_p - \mu) n_p} = \prod_p \sum_{n_p} e^{-\beta(\varepsilon_p - \mu) n_p} = \prod_p Q_p
\end{equation}
We can now study the sum of $e^{-\beta(\varepsilon_p - \mu) n_p}$ over all possible occupation number:
\textbf{Fermion}
we can only take $n_p = 0$ or $1$, so that 
\begin{gather}
    Q_p = 1 + e^{-\beta(\varepsilon_p - \mu)} \\
    \Omega_p = -k_BT\ln Q_p
\end{gather}
and 
\begin{equation}
    N_p = -\left(\frac{\partial \Omega}{\partial \mu}\right)_{T,V} = \frac{1}{e^{\beta(\varepsilon_p - \mu)}+1}
\end{equation}
\textbf{Boson}
the sum of $n_p$ is from $0$ to $\infty$, thus:
\begin{equation}
    \sum_{n_p=0}^{\infty} e^{-\beta(\varepsilon_p - \mu) n_p} 
    =\frac{1}{1-e^{-\beta(\varepsilon_p - \mu)}} 
\end{equation}
giving
\begin{equation}
    N_p =  \frac{1}{e^{\beta(\varepsilon_p - \mu)}-1}
\end{equation}
The partition function and total particle number can be summarized by given by:
\begin{gather}
    \Omega = -ak_BT\sum_p\ln(1 + ae^{-\beta(\varepsilon_p - \mu)}) \\
    N = \sum_p \frac{1}{e^{\beta(\varepsilon_p - \mu)} + a}
\end{gather}
with $a = 1$ for fermion and $-1 $ for boson.

\subsection{Boson gas}
For an ideal gas of spinless bosons \footnote{for spin $S$, each momentum eigenstate $k$ will have $(2S+1)$ fold degeneracy}, 
the single particle state will be the momentum eigenstate given by 
$k$ and kinetic energy $\varepsilon_k = \frac{\hbar^2k^2}{2m}$. The chemical potential $\mu$ will be negative otherwise 
the state with energy smaller than $\mu$ will have negative occupation.
We calculate the grand potential:
\begin{align}
    \Omega &= k_BT \sum_k \ln(1-e^{-\beta(\varepsilon_p - \mu)}) \notag \\
            &= k_BT \int_0^{\infty} \ln(1-e^{-\beta(\varepsilon_p - \mu)}) g(\varepsilon) d\varepsilon \notag \\
            &= k_BT \int_0^{\infty} \ln(1-e^{-\beta(\varepsilon_p - \mu)}) \frac{V\varepsilon^{1/2}}{(2\pi)^2}\left(\frac{2m}{\hbar^2}\right)^{3/2} d\varepsilon \notag \\
            &= -\frac{2}{3} \frac{V}{(2\pi)^2}\left(\frac{2m}{\hbar^2}\right)^{3/2} \int_0^{\infty} \frac{\varepsilon^{3/2}d\varepsilon}{e^{\beta(\varepsilon_p - \mu)}-1} \label{todos}
\end{align}
define $z = e^{\beta\mu}$, the particle number and internal energy can be written as:
\begin{align}
    N = \frac{V}{(2\pi)^2}\left(\frac{2m}{\hbar^2}\right)^{3/2} \int_0^{\infty} \frac{\varepsilon^{1/2}d\varepsilon}{e^{\beta\varepsilon_p}/z-1} \\
    U = \frac{V}{(2\pi)^2}\left(\frac{2m}{\hbar^2}\right)^{3/2} \int_0^{\infty} \frac{\varepsilon^{3/2}d\varepsilon}{e^{\beta\varepsilon_p}/z-1}
\end{align}
The integral we can write in terms of \textbf{polylogarithm} function:
\begin{align}
    \int_0^{\infty} \frac{\varepsilon^{n-1}d\varepsilon}{e^{\beta\varepsilon_p}/z-1} = (k_BT)^n (n-1)! L_n(z)
\end{align}
with $n/2! = n/2 \times (n-2)/2 \times \cdots \times 1/2$ with n an odd number.
So that 
\begin{align}
    N &= \frac{V}{(2\pi)^2}\left(\frac{2m}{\hbar^2}\right)^{3/2} (k_BT)^{3/2} (1/2)! L_{3/2}(z) = \frac{V}{\lambda^3} L_{3/2}(z) \label{eqbosonN} \\
    U &= \frac{V}{(2\pi)^2}\left(\frac{2m}{\hbar^2}\right)^{3/2} (k_BT)^{5/2} (3/2)! L_{5/2}(z) = \frac{3}{2}\frac{Vk_BT}{\lambda^3} L_{5/2}(z)
\end{align}
$\lambda$ is defined to absorbed all the numerical factor in Eq.\ref{eqbosonN} and $\lambda \propto T^{-1/2}$. 
Let's now consider Eq.\ref{eqbosonN}
\begin{equation}
    n \lambda^3 = L_{3/2}(z) \label{conflict}
\end{equation}
since $\mu$ need to be negative, the value of $z$ is bound to $(0,1)$, the function $L_{3/2}$ is monoclinically increaing with $z$ and bound
between $(0,L_{3/2}(1)) \approx (0,2.612)$, however, as we decrease temperature and $z \to 1$, 
$\lambda$ will increase without bound, which conflict with Eq.\ref{conflict}. 

This inconsistency come from the fact that when we convert the summation of $k$ into integral of energy $\varepsilon$, we essentially omitted 
a single state $(k=0,\varepsilon=0)$ since $g(\varepsilon=0)=0$. Its density of state $g(\varepsilon)$ is ignorable compared to other states but
at low temperature, it's occupation maybe very large. Including the occupation of this ground state explicitly in Eq.\ref{eqbosonN}:
\begin{equation}
    N = N_0 + N_1 = \frac{1}{1/z-1} + \frac{V}{\lambda^3} L_{3/2}(z)
\end{equation}
if we define 
\begin{equation}
    n \lambda(T_c)^3 = L_{3/2}(z=1)
\end{equation}
then
\begin{equation}
    \frac{N_0}{N} = \frac{N-N_1}{N} = 1-\left(\frac{T}{T_c}\right)^{3/2}
\end{equation}
if the temperature is low enough so that $N_0/N \approx 1$, then almost all particles are condensed in the ground state. $T_c$ for an ideal boson gas can be calculated:
\begin{equation}
    k_B T_c \approx 0.061 \frac{\hbar^2}{m} n^{2/3}
\end{equation}
For example, liquid Helium$^4$ with a density of $1.5e22 cm^{-3}$ gives a $T_c \approx 3K$. Expreiment observed a phase transition to a 
new phase with superfluid properties around this temperature, associated with the condensation.

\section{Thermodynamic Properties of magnetic system}
We write the internal energy of an magnetic system as:
\begin{equation}
    \dnor U = \dbar Q + \mu_0 H \dnor M
\end{equation}
where $H$ is the external field. For reversible process, we have:
\begin{equation}
    \dnor U = T \dnor S + \mu_0 H \dnor M
\end{equation}
and we can express the temperature $T$ and field $H$ by:
\begin{align}
    T = \left(\pfrac{U}{S}\right)_M \ \text{ and }\ \ H = \frac{1}{\mu_0} \left(\pfrac{U}{M}\right)_S
\end{align}
and we obtain the Maxwell relationship with internal energy:
\begin{equation}
    \left( \pfrac{T}{M} \right)_S = \mu_0 \left( \pfrac{H}{S} \right)_M
\end{equation}
Heat absorbed can be written as:
\begin{align}
    \dbar Q &= \left(\pfrac{U}{T}\right)_M \dnor T + \left[ \left(\pfrac{U}{M}\right)_T -\mu_0 H \right] \dnor M \\
            &= C_M \dnor T + l \dnor M \label{q_cm}
\end{align}
where we identify $C_M$ to be the specific heat at constant magnetization: when $\dnor M = 0$, we have $\dbar Q = C_M \dnor T$, and
\begin{equation}
    C_M = \left(\pfrac{U}{T}\right)_M = \left(\frac{\dbar Q}{\dnor T}\right)_M = T \left(\pfrac{S}{T}\right)_M
\end{equation}
where we have used $\dbar Q = T \dnor S$. For $l$, we can find, using Maxwell relation:
\begin{equation}
    l = \left( \frac{\dbar Q}{\dnor M} \right)_T = T \left(\pfrac{S}{M}\right)_T = - \mu_0 T \left(\pfrac{H}{T}\right)_M
\end{equation}

We introduce thermodynamic quantities called magnetic enthalpy:
\begin{gather}
    \mathcal{H} = U - \mu_0 H M \\
    \dnor \mathcal{H} = T \dnor S - \mu_0 M \dnor H
\end{gather}
which gives the Maxwell relationship:
\begin{equation}
    \left( \pfrac{T}{H} \right)_S = - \mu_0 \left( \pfrac{M}{S} \right)_H
\end{equation}
Similar to above, we have the heat absorbed:
\begin{align}
    \dbar Q &= \left(\pfrac{\mathcal{H}}{T}\right)_H \dnor T + \left[ \left(\pfrac{\mathcal{H}}{H}\right)_T + \mu_0 H \right] \dnor M \\
            &= C_H \dnor T + h \dnor M \label{q_ch}
\end{align}
where
\begin{equation}
    C_H = \left(\pfrac{\mathcal{H}}{T}\right)_H = T \left(\pfrac{S}{T}\right)_H
\end{equation}
and similar to the calculation of $l$, we have:
\begin{eqnarray}
    h = T \left(\pfrac{S}{H}\right)_T = \mu_0 T \left(\pfrac{M}{T}\right)_H
\end{eqnarray}

We can obtain the relationship between $C_M$ and $C_H$ that is similar between $C_V$ and $C_p$. 
Substructing Eq.\ref{q_cm} and Eq.\ref{q_ch}, we obtain:
\begin{equation}
    (C_H - C_M) \dnor T = -T\mu_0 \left[ \left(\pfrac{H}{T}\right)_M \dnor M + \left(\pfrac{M}{T}\right)_H \dnor H \right]
\end{equation}
since $M$, $T$ and $H$ are all function of state, we can express the relation:
\begin{equation}
    \dnor T = \left(\pfrac{M}{T}\right)_H \dnor M + \left(\pfrac{T}{H}\right)_M \dnor H
\end{equation}
So that we have the result:
\begin{equation}
    C_H - C_M = - T\mu_0 \left(\pfrac{H}{T}\right)_M \left(\pfrac{M}{T}\right)_H 
    = T\mu_0 \left(\pfrac{M}{T}\right)_H^2 \left(\pfrac{H}{M}\right)_T 
\end{equation}
where we used the relation:
\begin{equation}
    \left(\pfrac{H}{T}\right)_M = - \left(\pfrac{M}{T}\right)_H \left(\pfrac{H}{M}\right)_T 
\end{equation}

For system with fixed energy, we maximize the entropy to find the equilibrium
\footnote{That is to say, for microcanonical system, the equilibrium is found 
at states that correspond to the most number of microstates}.
However, for system in contact with the reservior, we need to minimize the free 
energy, defined by:
\begin{gather}
    F = U - TS = -\frac{1}{\beta} \ln Z \\
    \dnor F = \mu_0 H \dnor M - S \dnor T
\end{gather}
which gives the Maxwell relationship:
\begin{equation}
    \mu_0 \left( \pfrac{H}{T} \right)_M = - \left( \pfrac{S}{M} \right)_T
\end{equation} 

When the system is fixed at temperature $T$ and external field $H$, The 
Gibbs potential should be minimized:
\begin{equation}
    \dnor G = - \mu_0 M \dnor H - S \dnor T
\end{equation}
with the corresponding Maxwell relationship:
\begin{eqnarray}
    \mu_0 \left( \pfrac{M}{T} \right)_H = - \left( \pfrac{S}{H} \right)_T
\end{eqnarray}

Finally, if the number of particle in the system is not fixed:
we define extra thermodynamic potentials:
\begin{table*}[h]
    \centering
    \begin{tabular}{rr}
     $\Omega_F = F - \mu N$ & $ \dnor \Omega_F = \mu_0 H \dnor M - S\dnor T - N \dnor \mu $ \\
     $\Omega_G = G - \mu N$ & $ \dnor \Omega_G = \mu_0 M \dnor H - S\dnor T - N \dnor \mu $ \\
     \end{tabular}
\end{table*}
The particle number $N$ is given by:
\begin{equation}
    N = - \left( \pfrac{\Omega_F}{\mu} \right)_{T,M} = - \left( \pfrac{\Omega_G}{\mu} \right)_{T,M} 
\end{equation}

\newpage

\section*{Appendix A. Intensive and extensive properties (previous version)}

\subsection{Heat}
We provide a definition of heat as \emph{The thermal energy in transit}, denoted as $Q$\footnote{Blundell, p14}. We define the heat capacity
$C$ of an object as the amount of heat that is needed to increase its temperature:
\begin{equation}
    C = \frac{dQ}{dT}
\end{equation}
thus $C$ has the unit $J/K$. The specific heat is defined to be the heat capacity per unit mass, having the unit $J/(kg\cdot K)$

\subsection{Entropy}
We define entropy $S$ for an isolated macroscopic system of $N$ particles in volume $V$ and energy $E$ to be:
\begin{equation}
    S(E,N,V,x) = k_B \ln\Omega(E,N,V,x)
\end{equation}
where $\Omega(E,N,V,x)$ is the number of accessible states at a given value of $E, N, V$, and $x$ is some constraints which 
influence the number of accessible states.
As a non-equilibrium isolated system allow to relax to equilibrium, entropy will increase monotonically and eventually maximize 
at equilibrium.
%imagine that the system will attempt to visit as many configurations as possible. 

\textbf{Temperature} 
consider an isolated system with two subsystem in weak contact but heat is allowed to follw between the two subsystem.
The total number of accessible states (configurations) are given by the product of the number of configurations of the 
two subsystem. Entropy will be additive. Consider the energy of one of the subsystem $E_1$
as the constrain for the configurations.
\begin{align}
    \Omega(E,E_1) = \Omega_1(E_1) \Omega_2(E_2) \\
    S(E,E_1) = S_1(E_1) S_2(E_2)   
\end{align}
Relaxation process will increase entropy by changing $E_1$, towards a macroscopic that correspond to 
more configurations. At equilibrium (heat no longer exchange), we have:
\begin{gather}
    \frac{\partial S}{\partial E_1} = 0 \ \Rightarrow \ 
    \left. \frac{\partial S_1(E_1)}{\partial E_1} \right|_{N_1,V_1} = \left. \frac{\partial S_2(E_2)}{\partial E_2}\right|_{N_2,V_2} = \frac{1}{T} \label{defineT}
\end{gather}
where the final equality gives the definition of temperature, thus if two subsystem reaches equilibrium in terms of energy flow, their 
temperature will be equal. This process is irreversible and thus define a arrow of time.

\textbf{Chemical potential}
now, we fix only volume $V$ of each subsystem and allow both energy and particles to exchange, then:
\begin{equation}
    \left. \frac{\partial S_1(N_1)}{\partial N_1} \right|_{E_1,V_1} = \left. \frac{\partial S_2(N_2)}{\partial N_2}\right|_{E_2,V_2} = -\frac{\mu}{T}
\end{equation}
the last equality defines the chemical potential $\mu$. 
% minus sign the divide by temperature is because of the tradition

\textbf{Pressure}
finally, we allow the volume of the system to exchange, and we can similar define pressure:
\begin{equation}
    \left. \frac{\partial S_1(V_1)}{\partial V_1} \right|_{E_1,N_1} = \left. \frac{\partial S_2(V_2)}{\partial V_2}\right|_{E_2,N_2} = \frac{P}{T}
\end{equation}

Thus, we can see that if we set an initial system not in equilibrium, the two subsystem will
start to exchange energy, particles and volume until $T$, $P$ and $\mu$ become the same 
for the two subsystem. 

We can separate the macroscopic properties of a system into two different catagory:
\begin{itemize}
    \item \textbf{Extensive properties} that will increase proportional to the system size, such as $N,E,V$
    \item \textbf{Intensive properties} that will be same for any of the subsystem, such as  $P,\mu,T$
\end{itemize}

\newpage
\section*{Appendix B. Thermodynamic potentials(previous version)}

It follows from the definition of  $T$, $P$ and $\mu$ that 
\begin{gather}
    \frac{1}{T} = \left(\frac{\partial S}{\partial E}\right)_{N,V};\  
    \frac{\mu}{T} = \left(\frac{\partial S}{\partial N}\right)_{E,V};\ 
    \frac{P}{T} = \left(\frac{\partial S}{\partial V}\right)_{E,N}
\end{gather}
we can organize:
\begin{align}
    dS &= \frac{1}{T} dE - \frac{\mu}{T}dN + \frac{P}{T}dV \\
    dE &= TdS + \mu dN - P dV \label{partial_extensive}
\end{align}
leading to:
\begin{align}
    T = \left(\frac{\partial E}{\partial S}\right)_{N,V};\  
    \mu = \left(\frac{\partial E}{\partial N}\right)_{S,V};\ 
    P = -\left(\frac{\partial E}{\partial V}\right)_{S,N}
\end{align}
For the Eq.\ref{partial_extensive}, we can integrate the subsystems parts by parts into 
the whole system ($S$, $N$ and $V$ are extensive parts), 
since $T, \mu$ and $P$ will be the same for all subsystems, the integration
gives:
\begin{equation}
    E = TS + \mu N - PV \label{energy}
\end{equation}

Now, let's consider the canonical ensemble. The energy of the system
is no longer fixed, as opposed to microcanonical ensemble in which all the 
microstates in the ensemble have fixed energy. 
However, for each energy, it is associated with a 
probability and thus we only consider the average energy, which we define
as \textbf{internal energy}:
\begin{equation}
    U \equiv \langle E \rangle = \sum_j P_j E_j = \frac{\sum_j E_j e^{-\beta E_j}}{\sum_j e^{-\beta E_j}} = -\frac{1}{Q_N} \frac{\partial Q_N}{\partial \beta}
\end{equation}
The internal energy is now the macroscopic observable instead of energy, as
in the microcanonical case.
We wish to establish an relationship between the internal energy and 
the free energy $Q_N = \exp(-\beta A)$

Let's write
\begin{equation}
    Q_N = \sum_j e^{-\beta E_j} = \sum_{E_j} \Omega(E_j) e^{-\beta E_j}
\end{equation}
with $\Omega(E_j)$ gives the number of microstates that have energy $E_j$. With the definition of entropy $S(E) = k_B \ln \Omega(E)$, we have:
\begin{eqnarray}
    Q_N = \sum_{E_j} e^{-\beta (E_j - S(E_j)T)} \simeq e^{-\beta (E' - S(E')T)} \label{approx}
\end{eqnarray}
with $E'$ be the value that minimize the function $E - S(E)T$. If the fluctuation is small, we can consider only
the term $e^{-\beta (E' - S(E')T)}$ in the summation is not ignorable, which gives the approximation.
The requirement that $E'$ minimize function $E - S(E)T$ is:
\begin{equation}
    \left. \frac{\partial (E-S(E)T)}{\partial E}\right|_{E'} = \left. 1 - T\frac{\partial S(E)}{\partial E}\right|_{E'} = 0
\end{equation}
therefore, $E'$ is the energy in which:
\begin{equation}
    \left. \frac{\partial S(E)}{\partial E}\right|_{E'} = \frac{1}{T}
\end{equation}
that is to say, $E'$ is the energy where the 
system's temperature defined by Eq.\ref{defineT} is equal to
the temperature of the heat bath, which is the equilibrium condition. 
This motivate us to equal $E' = \langle E \rangle = U$
leading to the result that 
\begin{equation}
    A = U - TS(U) \label{eqA}
\end{equation}
$U$ minimize $U - TS(U)$ at equilibrium, therefore, the equilibrium condition of 
a canonical potential is the minimization of free energy $A$.

With Eq.\ref{eqA}, Eq.\ref{partial_extensive} and $U = \langle E \rangle$,  we have
\begin{equation}
    dA(N,V,T) = dU - TdS - SdT = -SdT - pdV + \mu dN
\end{equation}
and 
\begin{equation}
    S = -\left(\frac{\partial A}{\partial T}\right)_{N,V};\  
    \mu = \left(\frac{\partial A}{\partial N}\right)_{T,V};\ 
    P = -\left(\frac{\partial A}{\partial V}\right)_{T,N}
\end{equation}

For Grand canonical ensemble, we can apply similar method as in the canonical ensemble, but 
this time adding the particle number as a summation variable in the calculation of 
partition function, we can obtain:
\begin{equation}
    \Omega(\mu,T,V) = U - TS - \mu \bar{N}
\end{equation}
with $\bar{N}$ being the average particle number. The equilibrium 
condition of a grand canonical ensemble is then the minimization of $\Omega$. 
We also have the following relationship:
\begin{equation}
    d\Omega(\mu,T,V) = -SdT - Nd\mu - PdV
\end{equation}
and
\begin{equation}
    S = -\left(\frac{\partial \Omega}{\partial T}\right)_{\mu,V};\  
    N = -\left(\frac{\partial \Omega}{\partial \mu}\right)_{T,V};\ 
    P = -\left(\frac{\partial \Omega}{\partial V}\right)_{T,\mu}
\end{equation}

\subsection{Gibbs expression of entropy}
For the canonical ensemble, we have:
\begin{gather}
    P_i = \frac{1}{Q} e^{-\beta E_i} \notag \\
    A = -k_B T\ln Q = U - TS \notag
\end{gather}
This lead to the relationship:
\begin{equation}
    S = -k_B \sum_i P_i \ln P_i
\end{equation}
This is the Gibbs' form of entropy in terms of probability.


\end{document}


