\documentclass{amsart}
%\usepackage{amsmath}
\usepackage[margin=1in]{geometry}
\usepackage{verbatim}
\usepackage{graphicx}
\usepackage{hyperref} % \url \href

\newcommand{\pfrac}[2]{\frac{\partial #1}{\partial #2}}
\newtheorem{definition}{Definition}
\newtheorem{theorem}{Theorem}

% this note here aim to outline the main principles of the 
% group theory and representation theory
% it is based on two separate materials:
% group theory for mathematic, physic, chemistry and materials science students
% this note has very formal mathematic notation but are slightly difficult to follow, since many proof is 
% very short

% Group theory, application to physics of condensed matter
% which is more straight forward to understand but did not include 
% many mathematic concepts

\begin{document}

\title{Group Theory and Representation}
\author{Wenhao}
\date{\today}
\maketitle

\section{Group Theory}
\begin{definition}[Group]
    A group is a set plus an operation, that map an ordered pair of group element $(g,h)$ of $G$ into another element $g\cdot h \in G$, satisfying
    the following properties:
    \begin{enumerate}
        \item operation is associative: $g\cdot (h \cdot k) = (g\cdot h) \cdot k$ for $g,h,k \in G$;
        \item $G$ contain an identity element $e$, that satisfies $g\cdot e = e\cdot g = g$ for all $g \in G$ and 
        \item Each element of $G$ has an inverse, denoted by $g^{-1}$.
    \end{enumerate}  
\end{definition}

\begin{definition}[Order of the group]
    Order of the group $G$, or the cardinality of the group, is the number of elements in the set $G$, denoted by $|G|$
\end{definition}

\begin{definition}[Abelian group]
    A group is called abelian if for all $g, h \in G$, $g\cdot h = h \cdot g$ (commutative)    
\end{definition}

\vspace{10pt}

\subsubsection*{Permutation group}
We denote a set by $X$. All the bijections of $X$ to itself form a group, which we denote $\text{Sym}(X)$. 
If $|X| = n$, then $|\text{Sym}(X)| = n!$.
If $|X| = |Y|$, then $\text{Sym}(X) = \text{Sym}(Y)$ and we denote it as $\text{Sym}(n)$ or $S_n$
For example, $S_3 = \{ e, (1,2), (2,3), (1,3), (1,2,3), (3,2,1) \}$. We have $|S_3| = 6$. 
The permutation $(3,2,1)$ means $3\to 2, 2\to 1, 1\to 3$ with 1,2,3 indicate the position in the set.

\subsubsection*{Linear transformation group}
Denoting $V$ as a vector space, we write $\text{GL}(V)$ as the group of all linear transformation in $V$

\vspace{10pt}

\begin{definition}[Subgroup]
    Definition: $H$ is a non empty subset of $G$ and $H$ is a group, then $H$ is a subgroup of $G$
\end{definition}

\begin{theorem}
    The intersections of subgroups of $G$ is also a subgroup of $G$.
\end{theorem}
\begin{proof}
    Suppose $H$ and $L$ are subgroups of $G$. $M$ is the intersections between $H$ and $L$, then:
    \begin{enumerate}
        \item identity $e \in M$;
        \item if $h_1, h_2 \in H$ and $h_1 \cdot h_2 = e$, If $h_1\in L$, then inevitably $h_2 \in L$, therefore the intersections of $H$ and $L$ is closed under inverse;
        \item similarly, if $h_1, h_2 \in H$ and $h_1, h_2 \in L$, then $h_1\cdot h_2$ belong to both $H$ and $L$ are therefore in the intersections. $M$ is closed under group operation.
    \end{enumerate}
\end{proof}

\begin{definition}[Generator]
    For a set $S$, the intersections of all subgroups contain $S$ is a subgroup. This intersections is denoted by $\langle S\rangle$ and we say that it is generated by $S$. 
\end{definition}
For a group element $g$, we write that group that is generated by $g$ as $\langle g \rangle$, the order of $\langle g \rangle$ is also called the order of $g$. 

\begin{theorem}
    if a group $G$ is finite, we must have $g^n = e$ for $g \in G$. Since any $g^a$ is a number in $G$
\end{theorem}

\begin{definition}[Cyclic group]
    If a group is generated by a single element, i.e. $G = \langle g \rangle$ for $g \in G$
\end{definition}

\begin{definition}
    A group is called normal (self-conjugate) if 
    \[
        gBg^{-1} = B\ \text{for}\ g \in G \qquad \text{(group automorphism)}    
    \]
\end{definition}

\begin{theorem}[Rearrangement theorem]
    for group $G$ and a group element $g'\in G$, the set 
    \[\{g'g \mid g \in G\}\]
    contain each group element once and only once.
\end{theorem}
\begin{proof}
    It is equivalent to say that if $g_1 \neq g_2$, then $g'g_1 \neq g'g_2$. all group element in $G$ are mapped 
    to another distinct elements in $G$ (rearrangement).

    If $g'g_1 = g'g_2$ but $g_1 \neq g_2$, then 
    \[ g'^{-1}g'g_1 = g'^{-1}g'g_2 \] which apperant conflict with the assumption
\end{proof}

\vspace{10pt}

\subsubsection*{Multiplication} 
For $S$ and $T$, both subset of group $G$, we define their produce:
\begin{equation}
    ST = \{st\mid s\in S, t \in T\}
\end{equation}
and $sT \equiv \{s\}T $ and $Ts \equiv T\{s\} $ for $s \in S$.

\vspace{10pt}

\begin{definition}[Left cosets]
    For $H$ a subgroup of $G$ and $g \in G$, $gH$ is called a left coset. $Hg$ is called a right coset. 
    The set $\{gH \mid g \in G, H\ \text{is subgroup of}\ G\}$ is written as $G\setminus H$
\end{definition}

For example, for $S_3 = \{e, (12), (23), (13), (123), (132)\}$ 
and $H = \{e,(123),(132)\}$, We can work out the following relationship:
\begin{align*}
    (123)(123) &= (132) &  (132)(123) &= e \\
    (132)(132) &= (123) &  (123)(132) &= e 
\end{align*}
i.e.\,, $H$ is a subset of $S_4$. Applying element $g \in \{(12), (23), (13)\}$ on $H$
give the set $\{(12), (23), (13)\}$. Therefore, the left cosets of $H$ is:
\[
    \left\{\, \{e,(123),(132)\}, \{(12), (23), (13)\}\, \right\}    
\]

\begin{theorem}
    The left cosets of the subgroup $H$ of $G$ partition $G$
\end{theorem}
\begin{proof}
    This is equivalent to say that $gH$ is either $H$ itself, or share no comment elements with $H$.
    if $g\in H$, then $gH = H$. On the other hand, 
    if $g\notin H$ but $gh \in H$ for an element $h\in H$, then, by the requirement of group $h^{-1}\in H$. $ghh^{-1} = g \in H$ which conflict with the assumption.
    Therefore, we $gH$ cannot share element with $H$: $|gH| = |H|$, so that left cosets of a subgroup partition the group.
\end{proof}
As a result, the whole group can be written as:
\[
    G = H + g_1 H + g_2 H + \dots + g_n H    
\]
\begin{theorem}[Lagrange's theorem]
    For a finite group $G$ and $H$ is a subgroup of $G$, $|H|$ can divide $G$.
\end{theorem}
\begin{definition}[Index of $H$ in $G$]
    The number of left cosets of a subgroup $H$ is called the index of $H$ in $G$, denoted as $[G:H]$.
\end{definition}

If $G$ is a finite group and $g\in G$. Then the order of $\langle g\rangle$ divide $|G|$. This is because 
$\langle g\rangle$ is a subgroup of G.

\newpage
\section{Representation Theory}

\newpage
\section{Crystal Structure}


\end{document}
