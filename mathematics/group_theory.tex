\documentclass{amsart}
%\usepackage{amssymb, amsmath, amsthm}
%\usepackage[margin=1in]{geometry}
\usepackage{verbatim}
\usepackage{graphicx}
\usepackage{hyperref} % \url \href

\newcommand{\pfrac}[2]{\frac{\partial #1}{\partial #2}}
\newcommand{\setM}{\mathcal{M}}
\newtheorem{definition}{Definition}
\newtheorem{theorem}{Theorem}
\newtheorem{lemma}{Lemma}
\DeclareMathOperator{\Aut}{Aut}
\DeclareMathOperator{\Image}{Im}
\DeclareMathOperator{\AO}{AO}
\DeclareMathOperator{\E}{E}
\DeclareMathOperator{\Sym}{Sym}

% the highest level is part
% in each part, different section give different topics that are lossly connected
% subsection* should be used for giving subsequent definitions that are less important
\begin{document}

\title{Group Theory and Representation}
\author{Wenhao}
\date{\today}
\maketitle

\part{Group Theory}

\section*{Mapping}
In this note, mapping and function are used interchangably. 
% https://en.wikipedia.org/wiki/Bijection,_injection_and_surjection
\subsection*{Injective}
A mapping is called injective (one-to-one) if each element of the codomain is mapped by 
at most one element of the domain (arguments, input). 
For all $x,x' in X$, $f(x)=f(x')$ only if $x=x'$.
\vspace{-10pt} % reduce the space in successive definition
\subsection*{Surjective}
A function is surjective (onto), if each element of the codomain is mapped to by at least one element of 
the domain. That is, the image of of the domain equals the codomain. 
For all $y\in Y$, there exist an $x\in X$ so that $y=f(x)$.
\vspace{-10pt} % reduce the space in successive definition
\subsection*{Bijection}
If the function is both injective and bijective, than it is called bijective. Bijective is also called invertible.


\section*{Definition of Group}
\begin{definition}[Group]
    A group is a set plus an operation, that map an ordered pair of group element $(g,h)$ of $G$ into another element $g\cdot h \in G$, satisfying
    the following properties:
    \begin{enumerate}
        \item operation is associative: $g\cdot (h \cdot k) = (g\cdot h) \cdot k$ for $g,h,k \in G$;
        \item $G$ contain an identity element $e$, that satisfies $g\cdot e = e\cdot g = g$ for all $g \in G$ and 
        \item Each element of $G$ has an inverse, denoted by $g^{-1}$.
    \end{enumerate}  
\end{definition}

\subsection*{Order of the group}
    Order of the group $G$, or the cardinality of the group, is the number of elements in the set $G$, denoted by $|G|$
\vspace{-10pt} % reduce the space in successive definition
\subsection*{Abelian group}
    A group is called abelian if for all $g, h \in G$, $g\cdot h = h \cdot g$ (commutative)


\begin{definition}[Subgroup]
    Definition: $H$ is a non empty subset of $G$ and $H$ is a group, then $H$ is a subgroup of $G$
\end{definition}

\begin{theorem}
    The intersections of subgroups of $G$ is also a subgroup of $G$.
\end{theorem}
\begin{proof}
    Suppose $H$ and $L$ are subgroups of $G$. $M$ is the intersections between $H$ and $L$, then:
    \begin{enumerate}
        \item identity $e \in M$;
        \item if $h_1, h_2 \in H$ and $h_1 \cdot h_2 = e$, If $h_1\in L$, then inevitably $h_2 \in L$, therefore the intersections of $H$ and $L$ is closed under inverse;
        \item similarly, if $h_1, h_2 \in H$ and $h_1, h_2 \in L$, then $h_1\cdot h_2$ belong to both $H$ and $L$ are therefore in the intersections. $M$ is closed under group operation.
    \end{enumerate}
\end{proof}

\vspace{10pt}

\begin{definition}[Generator]
    For a set $S$, the intersections of all subgroups contain $S$ is a subgroup. This intersections is denoted by $\langle S\rangle$ and we say that it is generated by $S$. 
\end{definition}
For a group element $g$, we write that group that is generated by $g$ as $\langle g \rangle$, the order of $\langle g \rangle$ is also called the order of $g$. 

\subsection*{Cyclic group}
    If a group is generated by a single element, i.e. $G = \langle g \rangle$ for $g \in G$

\begin{theorem}
    if a group $G$ is finite, we must have $g^n = e$ for $g \in G$. Since any $g^a$ is a number in $G$
\end{theorem}


\section*{Two important group}

\subsection*{Linear transformation group}
Denoting $V$ as a vector space, we write $\text{GL}(V)$ as the group of all linear transformation in $V$

\subsection*{Permutation group}
Denoting a set by $X$. All the bijections of $X$ to itself form a group, which we denote $\Sym(X)$. 

If $|X| = n$, then $|\Sym(X)| = n!$.
If $|X| = |Y|$, then $\Sym(X) = \Sym(Y)$ and we denote it as $\Sym(n)$ or $S_n$.

We denote a permutation by $\pi\colon \{1,2,\dots,n\} \to \{ \cdots \}$. For example:
\[
    \pi = \left(  
    \begin{matrix} 
    1&2&3&4&5\\
    5&4&1&2&3
    \end{matrix}
    \right) \in S_5
\]
gives a permutation $\{1,2,3,4,5\} \to \{ 5,4,1,2,3 \}$. Such notation is called 'two-line notation'
We should note that the numbers are to be interpreted as indices of the objects in the set.
In matrix form, we have:
\begin{equation}
    \left( \begin{matrix} 5 \\ 4 \\ 1 \\ 2\\ 3 \end{matrix} \right)
    = \left( \begin{matrix} 
        0 & 0 & 0 & 0 & 1 \\
        0 & 0 & 0 & 1 & 0 \\
        1 & 0 & 0 & 0 & 0 \\
        0 & 1 & 0 & 0 & 0 \\
        0 & 0 & 1 & 0 & 0 
    \end{matrix} \right) 
    \left( \begin{matrix} 1 \\ 2 \\ 3 \\ 4 \\ 5 \end{matrix} \right)
\end{equation}
We call the matrix of permutation $A_{\pi}$. 
With the matrix notation, we can define the sign of a permutation:
\[\text{sign}\pi = \det A_{\pi}\]

\textbf{Cyclic notation}
For the example permutation $\{1,2,3,4,5\} \to \{ 5,4,1,2,3 \}$, we can simply write
$\pi = (153)(24)$
The interpretion of cyclic notation is as follows:
\begin{itemize}
    \item a cyclic is a permutation of a subset without affecting other elements,
    \item fixed point can be omitted,
    \item $(153)$ is read as $1\to 5, 5\to 3, 3\to 1$. $1\to5$ reads 'object in position 1 become object in position 5'.
\end{itemize}
We can generate the two-line notation from the cyclic notation as follows:
\begin{equation*}
    \begin{matrix}
                & 1 & 2 & 3 & 4 & 5 \\
        1 \to 5 \quad & 5 &   &   &   &   \\
        5 \to 3 \quad &   &   &   &   & 3 \\
        3 \to 1 \quad &   &   & 1 &   &   \\
        2 \to 4 \quad &   & 4 &   &   &   \\
        4 \to 2 \quad &   &   &   & 2 &   \\
                & 5 & 4 & 1 & 2 & 3
    \end{matrix}
\end{equation*}

The following arrow form of permutation is also convenient to keep track of the operation, especially for successive 
application of permutations. Note that the arrow $5\to1$ reads 'the object in position 5 is moved to position 1'.
\begin{figure*}[!h]
    \centering
    \includegraphics[width=2in]{figures/permutation_arrow.png}
\end{figure*}

Using cyclic notation, we can write $S_3 = \{ e, (1,2), (2,3), (1,3), (1,2,3), (3,2,1) \}$. We have $|S_3| = 6$. The group table of $S_3$ can be calculated 
and tabulated to be:
\begin{table}[h]
    \centering
    \caption{Multiplication table of $S_3$}
    \begin{tabular}{|c|ccc|ccc|}
        \hline
                & e     & (123) & (321) & (12)  & (13)  & (23)  \\ \hline
           e    & e     & (123) & (321) & (12)  & (13)  & (23)  \\ 
          (123) & (123) & (321) & e     & (13)  & (23)  & (12)  \\
          (321) & (321) & e     & (321) & (23)  & (12)  & (13)  \\ \hline
          (12)  & (12)  & (23)  & (13)  &     e & (321) & (123) \\
          (13)  & (13)  & (12)  & (23)  & (123) & e     & (321) \\
          (23)  & (23)  & (13)  & (12)  & (321) & (123) & e     \\ \hline
    \end{tabular}
    \label{T:s3}
\end{table}

\section{Multiplication of group} 

\subsection*{Group multiplication}
For $S$ and $T$, both subset of group $G$, we define their product:
\begin{equation}
    ST = \{st\mid s\in S, t \in T\}
\end{equation}
and $sT \equiv \{s\}T $ and $Ts \equiv T\{s\} $ for $s \in S$.

\vspace{10pt}

\begin{definition}[Left cosets]
    For $H$ a subgroup of $G$ and $g \in G$, $gH$ is called a left coset. $Hg$ is called a right coset. 
    The set $\{gH \mid g \in G, H\ \text{is subgroup of}\ G\}$ is written as $G\setminus H$
\end{definition}

For example, for $S_3$ and its subgroup $H = \{e,(123),(132)\}$, 
Using the multiplication table \ref{T:s3}, we can find that
applying element $g \in \{(12), (23), (13)\}$ on $H$
give the set $\{(12), (23), (13)\}$. Therefore, the left cosets of $H$ is:
\[
    \left\{\, \{e,(123),(132)\}, \{(12), (23), (13)\}\, \right\}    
\]

\vspace{10pt}

\begin{theorem}
    The left cosets of the subgroup $H$ of $G$ partition $G$
\end{theorem}
\begin{proof}
    This is equivalent to say that $gH$ is either $H$ itself, or share no comment elements with $H$.
    if $g\in H$, then $gH = H$. On the other hand, 
    if $g\notin H$ but $gh \in H$ for an element $h\in H$, then, by the requirement of group $h^{-1}\in H$. $ghh^{-1} = g \in H$ which conflict with the assumption.
    Therefore, we $gH$ cannot share element with $H$: $|gH| = |H|$, so that left cosets of a subgroup partition the group.
\end{proof}
As a result, the whole group can be written as:
\[
    G = H + g_1 H + g_2 H + \dots + g_n H    
\]
\begin{theorem}[Lagrange's theorem]
    For a finite group $G$ and $H$ is a subgroup of $G$, $|H|$ can divide $G$.
\end{theorem}
\subsection*{Index of $H$ in $G$}
    The number of left cosets of a subgroup $H$ is called the index of $H$ in $G$, denoted as $[G:H]$.

If $G$ is a finite group and $g\in G$. Then the order of $\langle g\rangle$ divide $|G|$. This is because 
$\langle g\rangle$ is a subgroup of G.

\section{Mapping}

\begin{definition}[Homomorphism]
    We define a mapping $\Phi\colon G \to H$ from group $G$ to $H$. If
    \[
        \Phi(g_1 g_2) = \Phi(g_1) \Phi(g_2)    
    \] 
    is satisfied for all $g_1, g_2 \in G$, then we call $\Phi$ a homomorphism
\end{definition}

\subsection*{Isomorphism}
    If the mapping $\Phi$ is a bijection, then we call $\Phi$ an isomorphism. If $G$ and $H$ are related by 
    an isomorphism, we write $G\cong H$

\subsection*{Automorphism}
    We call the mapping $\Phi\colon G \to G$ (from $G$ onto itself) an automorphism

Automorphism are bijections. For example, $\Phi\colon G \to gG$ is an automorphism, 
this is shown in the rearrangement theorem. We write the set of all automorphism by $\Aut(G)$ 

\vspace{10pt}

\begin{theorem}[Rearrangement theorem]
    for group $G$ and a group element $g'\in G$, the set 
    \[\{g'g \mid g \in G\}\]
    contain each group element once and only once.
\end{theorem}
\begin{proof}
    It is equivalent to say that if $g_1 \neq g_2$, then $g'g_1 \neq g'g_2$. all group element in $G$ are mapped 
    to another distinct elements in $G$ (rearrangement).

    If $g'g_1 = g'g_2$ but $g_1 \neq g_2$, then 
    \[ g'^{-1}g'g_1 = g'^{-1}g'g_2 \] which apperant conflict with the assumption
\end{proof}

\vspace{10pt}

\begin{definition}
    [Kernel] For a homomorphism $\Phi\colon G \to H$, we define the kernel $\ker\Phi = \{g\in G\mid \Phi(g) = e_h \}$, i.\,e.\,,
    kernel of $\Phi$ is the elements in $G$ that are mapped to identity of group $H$. $\ker\Phi \subset G$
\end{definition}

\begin{definition}
    [Image] For homomorphism $\Phi\colon G \to H$, we define the image $\Image\Phi = \{ \Phi(g) \mid g \in G \}$, i.\,e.\,,
    the elements in $H$ that are obtained from the mapping. $\Image\Phi \subset H$
\end{definition}

\begin{lemma}
    $\ker\Phi$ is a subgroup of $G$
\end{lemma}
\begin{lemma}
    $\Image\Phi$ is a subgroup of $H$
\end{lemma}

\vspace{10pt}

\begin{theorem}
    Homomorphism $\Phi\colon G\to H$ is injective if and only if $\ker \Phi = e$
\end{theorem}
\begin{proof}
    If $\Phi$ is injective, then we require $\ker \Phi = e_g$; 

    If $\ker \Phi = e_g$. Suppose $\Phi(a) = h$ and $\Phi(b) = h$. If in group $G$, we have $ax = b$, with $x\in G$, 
    Then we have:
    \[  
        \Phi(b) = \Phi(a)\Phi(x)    
    \]
    This implies that $\Phi(x)=e_h$ and $x = e_g$. Therefore, $a = b$: if $\ker \Phi = e_g$ then we cannot have
    two different elements in $G$ that are mapped to the same element in $H$.
\end{proof}

\vspace{10pt}

\begin{definition}
    [Conjugation]
    We call the mapping $\Phi_g\colon G\to G$ with $\Phi_g(h) = ghg^{-1}$ a conjugation with $g$, or an \emph{inner conjugation}
\end{definition}
According to definition, $\Phi_g \in \Aut(G)$

\begin{definition}
    A subgroup is called normal, or self-conjugating, if every left cosets of $H$ in $G$ is also a right coset of $H$ in $G$
\end{definition}
If $H$ is a normal subgroup, then for the set $G/H$, we can show that $(g_1 H)\cdot(g_2 H)$ is also in 
set $G/H$:
\begin{proof}
    Since $H$ is a normal subgroup, by definition: 
    \[gH = Hg = gHg^{-1}g\] so that we fine:
    \[gHg^{-1} = H\]
    Then:
    \begin{align*}
        (g_1 H)\cdot(g_2 H) &= g_1 H g_1^{-1}g_1 g_2 H = H g_1 g_2 H \\
            &= (g_1g_2)(g_1g_2)^{-1} H (g_1g_2) H \\
            &= g_1g_2 H H = g_1 g_2 H 
    \end{align*}
    since $g_1 g_2 H$ is also a left coset and belong to $G/H$, we have shown that the set $G/H$ is indeed a group with 
    operation $(g_1 H)\cdot(g_2 H)$
\end{proof}

we call \textbf{quotinet group}, or factor group.
For example, for $H = \{e,(123),(132)\}$, we can verify that $H$ is a normal group, and the 
set ${H, (12)H}$ is a group with the defined operation.

\begin{lemma}
    The kernel $\ker\Phi$ of $\Phi\colon G\to H$ is a normal subgroup.
\end{lemma}

\begin{lemma}
    For a group homomorphism $\Phi\colon G\to H$, the quotinet group is isomorphic to $\Image\Phi$ (bijection).
\end{lemma}


\begin{definition}
    A group is called normal (self-conjugate) if 
    \[
        gBg^{-1} = B\ \text{for}\ g \in G \qquad \text{(group automorphism)}    
    \]
\end{definition}

\section{Action of a group}

\begin{definition}
    [Action of a group]
    For a group $G$ and a set $\setM$, the action of $G$ on $\setM$ is a mapping 
    $\alpha\colon G\times \setM \to \setM$ satisfying:
    \begin{enumerate}
        \item $\alpha(e,m) = m$ for $m \in \setM$;
        \item $\alpha(g, \alpha(h,m)) = \alpha(gh, m)$ for $m\in \setM$ and $h,g\in G$
    \end{enumerate}
\end{definition}
where $\alpha(g,m)$ means action of a group element $g$ on a set element in $\setM$

For an action $\alpha$, we have a homomorphism that map group $G$ to permutation group on $\setM$, 
given by:
\[g\to (m\to \alpha(g,m))\]
$m\to \alpha(g,m)$ being a permutation. 

\subsection*{Group action on itself}
For a group $G$, $G$ can act on itself by conjugation: $\alpha\colon G\times G \to G$ given by
$\alpha(h,g) = hgh^{-1}$

\vspace{10pt}

\begin{definition}
    [Orbit under $G$]
    For $G$ acting on $\setM$, the sets $Gm = \{gm\mid g\in G\}$ partition the set $\setM$. 
    \[\setM = Gm_1 + Gm_2 + \dots + Gm_n\]
    Each set $Gm$ is called the orbit of $m$ under $G$.
\end{definition}

\subsection*{Transitive}
An action is called transitive if it has only one orbit

\begin{definition}
    [Symmetry]
    For a group $G$ acting on $\setM$, $T\subset M$ is called invariant under $S \subset G$ if $ST\subset T$, 
    with $ST = {\alpha(g,m)\mid g \in G, m\in T}$. 
    The elements in $S$ are called 'symmetry'.
\end{definition}

\subsection*{Stabilizer}
For group $G$ acting on $\setM$ and $m \in \setM$, the set 
\[G_m = \{ g\in G\mid gm = m \}\]
are called the stabilizer of $m$. $G_m \subset G$

\begin{lemma}
    $G_m$ is a subgroup of $G$, $|G_m|$ divide $G$. Further more, $G_m$ is a normal subgroup. 
    The mapping $f\colon Gm \to G/G_m$ is a bijection and $|G|=|G_m||Gm|$
\end{lemma}

\section{Isometry}

\begin{definition}
    [Isometry]
    The distance in euclidean space is given by 
    \[
        d(x,y) = \sqrt{\sum_i(x_i-y_i)^2}    
    \]
    A mapping $g\colon \mathbb{R}^n \to \mathbb{R}^n$ satisfying 
    \[
        d(g(x),g(y)) = d(x,y)    
    \] is called an isometry, the transformation that keep the distance. We denote them as $\AO(n)$ or $\E(n)$
\end{definition}
The group of all isometries is a subgroup of permutation between points in n-dimensional space $\Sym(\mathbb{R}^n)$

\subsection*{Translation}
with $x, a \in \mathbb{R}^n$, transformation $t_a\colon x \to x+a$ is called translation.

\subsection*{Orthogonal linear map}
An orthogonal linear map is a linear transformation that perserve the inner product, the group of all
orthogonal linear map is denoted as O($3$)

\begin{theorem}
    For any group element $g\in \AO(n)$ (isometry), there exist a unique pair $(a,r)$ such that 
    \[g = t_a \cdot r\]
    with $r\in \AO(n)$ that fix the point at origin (rotation). $r$ is a anorthogonal linear map
\end{theorem}
As a result, any isometry is an orthogonal linear map followed by translation.

\vspace{10pt} %%%%%%%%%%%%%%%%%%%%%%%%%%%%%%%%%%%%

For a subgroup $G\subset \AO(n)$, the map $R\colon G \to R(G)$ with $R(G)\in \text{O}(n)$ given by
$g = t_a R(g)$ is a group homomorphism. It's kernel is $T(G)$:
\begin{equation}
    T(G) = \{ a \in \mathbb{R}^n \mid t_a \in G\}
\end{equation}
the set of all translation. 

\begin{lemma}
    $T(G)$ is invariant under $R(G)$
\end{lemma}
i.\,e.\,, the application of $r\in R(G)$ on $t\in T(G)$ gives another translation in $T(G)$

\vspace{10pt} %%%%%%%%%%%%%%%%%%%%%%%%%%%%%%%%%%%%

\begin{definition}
    [Lattice]
    a discrete subgroup of $\mathbb{R}^n$ is called Lattice (with $+$ as group operation). The dimension of its span is called rank.
\end{definition}
Let $a_1,\dots,a_k \in \mathbb{R}^n$ be linearly independent, then the set 
$L = \sum_i \mathbb{Z} a_i$ is a lattice.

A subgroup $G\subset \AO(n)$ act discretely if all orbits in $\mathbb{R}^n$ is discrete.

\begin{lemma}
    If $G$ is a subgroup of $\AO(n)$ leaving a lattice $T$ invariant, then $G$ is finite
\end{lemma}

\begin{definition}
    [Crystallographic group]
    If $G$ is a subgroup of $\AO(n)$ for which $T(G)$ is the lattice translation, then $G$ is called 
    a crystallographic group.
\end{definition}
\subsection*{Crystallographic point group}
If $G\subset \text{O}(n)$ leave invariant a lattice of rank $n$, then $G$ is called a crystallographic point group

\begin{lemma}
    If $G$ is a crystallographic group, then $R(G)$ is a crystallographic point group
\end{lemma}


\newpage
\section{Representation Theory}

\newpage
\section{Crystal Structure}


\end{document}
