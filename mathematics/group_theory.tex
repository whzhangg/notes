\documentclass{amsart}
%\usepackage{amssymb, amsmath, amsthm}
%\usepackage[margin=1in]{geometry}
\usepackage{verbatim}
\usepackage{graphicx}
\usepackage{hyperref} % \url \href

\newcommand{\pfrac}[2]{\frac{\partial #1}{\partial #2}}
\newtheorem{definition}{Definition}
\newtheorem{theorem}{Theorem}
\newtheorem{lemma}{Lemma}
\DeclareMathOperator{\Aut}{Aut}
\DeclareMathOperator{\Image}{Im}

% the highest level is part
% in each part, different section give different topics that are lossly connected
% subsection* should be used for giving subsequent definitions that are less important
\begin{document}

\title{Group Theory and Representation}
\author{Wenhao}
\date{\today}
\maketitle

\part{Group Theory}

\section{Definition of Group}
\begin{definition}[Group]
    A group is a set plus an operation, that map an ordered pair of group element $(g,h)$ of $G$ into another element $g\cdot h \in G$, satisfying
    the following properties:
    \begin{enumerate}
        \item operation is associative: $g\cdot (h \cdot k) = (g\cdot h) \cdot k$ for $g,h,k \in G$;
        \item $G$ contain an identity element $e$, that satisfies $g\cdot e = e\cdot g = g$ for all $g \in G$ and 
        \item Each element of $G$ has an inverse, denoted by $g^{-1}$.
    \end{enumerate}  
\end{definition}

\subsection*{Order of the group}
    Order of the group $G$, or the cardinality of the group, is the number of elements in the set $G$, denoted by $|G|$

\subsection*{Abelian group}
    A group is called abelian if for all $g, h \in G$, $g\cdot h = h \cdot g$ (commutative)

\vspace{10pt}

\subsection*{Permutation group}
We denote a set by $X$. All the bijections of $X$ to itself form a group, which we denote $\text{Sym}(X)$. 
If $|X| = n$, then $|\text{Sym}(X)| = n!$.
If $|X| = |Y|$, then $\text{Sym}(X) = \text{Sym}(Y)$ and we denote it as $\text{Sym}(n)$ or $S_n$
For example, $S_3 = \{ e, (1,2), (2,3), (1,3), (1,2,3), (3,2,1) \}$. We have $|S_3| = 6$. 
The permutation $(3,2,1)$ means $3\to 2, 2\to 1, 1\to 3$ with 1,2,3 indicate the position in the set.

\subsection*{Linear transformation group}
Denoting $V$ as a vector space, we write $\text{GL}(V)$ as the group of all linear transformation in $V$

\vspace{10pt}

\begin{definition}[Subgroup]
    Definition: $H$ is a non empty subset of $G$ and $H$ is a group, then $H$ is a subgroup of $G$
\end{definition}

\begin{theorem}
    The intersections of subgroups of $G$ is also a subgroup of $G$.
\end{theorem}
\begin{proof}
    Suppose $H$ and $L$ are subgroups of $G$. $M$ is the intersections between $H$ and $L$, then:
    \begin{enumerate}
        \item identity $e \in M$;
        \item if $h_1, h_2 \in H$ and $h_1 \cdot h_2 = e$, If $h_1\in L$, then inevitably $h_2 \in L$, therefore the intersections of $H$ and $L$ is closed under inverse;
        \item similarly, if $h_1, h_2 \in H$ and $h_1, h_2 \in L$, then $h_1\cdot h_2$ belong to both $H$ and $L$ are therefore in the intersections. $M$ is closed under group operation.
    \end{enumerate}
\end{proof}

\vspace{10pt}

\begin{definition}[Generator]
    For a set $S$, the intersections of all subgroups contain $S$ is a subgroup. This intersections is denoted by $\langle S\rangle$ and we say that it is generated by $S$. 
\end{definition}
For a group element $g$, we write that group that is generated by $g$ as $\langle g \rangle$, the order of $\langle g \rangle$ is also called the order of $g$. 

\subsection*{Cyclic group}
    If a group is generated by a single element, i.e. $G = \langle g \rangle$ for $g \in G$

\begin{theorem}
    if a group $G$ is finite, we must have $g^n = e$ for $g \in G$. Since any $g^a$ is a number in $G$
\end{theorem}

\vspace{10pt}

\section{Multiplication of group} 

\subsection*{Group multiplication}
For $S$ and $T$, both subset of group $G$, we define their produce:
\begin{equation}
    ST = \{st\mid s\in S, t \in T\}
\end{equation}
and $sT \equiv \{s\}T $ and $Ts \equiv T\{s\} $ for $s \in S$.

\vspace{10pt}

\begin{definition}[Left cosets]
    For $H$ a subgroup of $G$ and $g \in G$, $gH$ is called a left coset. $Hg$ is called a right coset. 
    The set $\{gH \mid g \in G, H\ \text{is subgroup of}\ G\}$ is written as $G\setminus H$
\end{definition}

For example, for $S_3 = \{e, (12), (23), (13), (123), (132)\}$ 
and $H = \{e,(123),(132)\}$, We can work out the following relationship:
\begin{align*}
    (123)(123) &= (132) &  (132)(123) &= e \\
    (132)(132) &= (123) &  (123)(132) &= e 
\end{align*}
i.e.\,, $H$ is a subset of $S_4$. Applying element $g \in \{(12), (23), (13)\}$ on $H$
give the set $\{(12), (23), (13)\}$. Therefore, the left cosets of $H$ is:
\[
    \left\{\, \{e,(123),(132)\}, \{(12), (23), (13)\}\, \right\}    
\]

\vspace{10pt}

\begin{theorem}
    The left cosets of the subgroup $H$ of $G$ partition $G$
\end{theorem}
\begin{proof}
    This is equivalent to say that $gH$ is either $H$ itself, or share no comment elements with $H$.
    if $g\in H$, then $gH = H$. On the other hand, 
    if $g\notin H$ but $gh \in H$ for an element $h\in H$, then, by the requirement of group $h^{-1}\in H$. $ghh^{-1} = g \in H$ which conflict with the assumption.
    Therefore, we $gH$ cannot share element with $H$: $|gH| = |H|$, so that left cosets of a subgroup partition the group.
\end{proof}
As a result, the whole group can be written as:
\[
    G = H + g_1 H + g_2 H + \dots + g_n H    
\]
\begin{theorem}[Lagrange's theorem]
    For a finite group $G$ and $H$ is a subgroup of $G$, $|H|$ can divide $G$.
\end{theorem}
\subsection*{Index of $H$ in $G$}
    The number of left cosets of a subgroup $H$ is called the index of $H$ in $G$, denoted as $[G:H]$.

If $G$ is a finite group and $g\in G$. Then the order of $\langle g\rangle$ divide $|G|$. This is because 
$\langle g\rangle$ is a subgroup of G.

\section{Mapping}

\begin{definition}[Homomorphism]
    We define a mapping $\Phi\colon G \to H$ from group $G$ to $H$. If
    \[
        \Phi(g_1 g_2) = \Phi(g_1) \Phi(g_2)    
    \] 
    is satisfied for all $g_1, g_2 \in G$, then we call $\Phi$ a homomorphism
\end{definition}

\subsection*{Isomorphism}
    If the mapping $\Phi$ is a bijection, then we call $\Phi$ an isomorphism. If $G$ and $H$ are related by 
    an isomorphism, we write $G\cong H$

\subsection*{Automorphism}
    We call the mapping $\Phi\colon G \to G$ (from $G$ onto itself) an automorphism

Automorphism are bijections. For example, $\Phi\colon G \to gG$ is an automorphism, 
this is shown in the rearrangement theorem. We write the set of all automorphism by $\Aut(G)$ 

\vspace{10pt}

\begin{theorem}[Rearrangement theorem]
    for group $G$ and a group element $g'\in G$, the set 
    \[\{g'g \mid g \in G\}\]
    contain each group element once and only once.
\end{theorem}
\begin{proof}
    It is equivalent to say that if $g_1 \neq g_2$, then $g'g_1 \neq g'g_2$. all group element in $G$ are mapped 
    to another distinct elements in $G$ (rearrangement).

    If $g'g_1 = g'g_2$ but $g_1 \neq g_2$, then 
    \[ g'^{-1}g'g_1 = g'^{-1}g'g_2 \] which apperant conflict with the assumption
\end{proof}

\vspace{10pt}

\begin{definition}
    [Kernel] For a homomorphism $\Phi\colon G \to H$, we define the kernel $\ker\Phi = \{g\in G\mid \Phi(g) = e_h \}$, i.\,e.\,,
    kernel of $\Phi$ is the elements in $G$ that are mapped to identity of group $H$. $\ker\Phi \subset G$
\end{definition}

\begin{definition}
    [Image] For homomorphism $\Phi\colon G \to H$, we define the image $\Image\Phi = \{ \Phi(g) \mid g \in G \}$, i.\,e.\,,
    the elements in $H$ that are obtained from the mapping. $\Image\Phi \subset H$
\end{definition}

\begin{lemma}
    $\ker\Phi$ is a subgroup of $G$
\end{lemma}
\begin{lemma}
    $\Image\Phi$ is a subgroup of $H$
\end{lemma}

\vspace{10pt}

\begin{theorem}
    Homomorphism $\Phi\colon G\to H$ is injective if and only if $\ker \Phi = e$
\end{theorem}
\begin{proof}
    If $\Phi$ is injective, then we require $\ker \Phi = e_g$; 

    If $\ker \Phi = e_g$. Suppose $\Phi(a) = h$ and $\Phi(b) = h$. If in group $G$, we have $ax = b$, with $x\in G$, 
    Then we have:
    \[  
        \Phi(b) = \Phi(a)\Phi(x)    
    \]
    This implies that $\Phi(x)=e_h$ and $x = e_g$. Therefore, $a = b$: if $\ker \Phi = e_g$ then we cannot have
    two different elements in $G$ that are mapped to the same element in $H$.
\end{proof}

\vspace{10pt}

\begin{definition}
    [Conjugation]
    We call the mapping $\Phi_g\colon G\to G$ with $\Phi_g(h) = ghg^{-1}$ a conjugation with $g$, or an \emph{inner conjugation}
\end{definition}
According to definition, $\Phi_g \in \Aut(G)$


\begin{definition}
    A group is called normal (self-conjugate) if 
    \[
        gBg^{-1} = B\ \text{for}\ g \in G \qquad \text{(group automorphism)}    
    \]
\end{definition}


\vspace{10pt}

\newpage
\section{Representation Theory}

\newpage
\section{Crystal Structure}


\end{document}
