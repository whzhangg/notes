\documentclass{amsart}
\usepackage{bm}
\usepackage[margin=1.5in]{geometry}
\usepackage{graphicx}

\begin{document}
\title{Tight Binding Model by Maximally Localized Wannier Functions}
\author{Wenhao}
\date{\today}
\maketitle

\section{Bloch Theorm}
It is apparent that in the lattice, the Hamiltonian is periodic and commute with any 
translation of lattice\footnote{ J\"{u}rgen K\"{u}bler, \emph{Theory of Itinerant Electronic Magnetism} page 79}:
\begin{equation}
    [T_j,H]=0
\end{equation}
Where operator $T_j$ translate the state by $\bm{R_j}$. Therefore, any eigenstate of the Hamiltonian should be also the eigenstate of translation:
\begin{equation}
    T_j \psi_i(\bm{r})=\psi_i(\bm{r}+\bm{R_j})=\lambda_j \psi_i(\bm{r})
\end{equation}
This is satisfied if the eigenvalue $\lambda_j$ is:
\begin{equation}
    \lambda_j=e^{i\bm{k} \cdot \bm{R_j}}
\end{equation}
And the state is indicated by the number k.
Finally, Bloch Theorm states that eigenstate of the Hamiltonian in a periodic system need to satisfy such condition:
\begin{equation}
    \psi_{n\bm{k}}(\bm{r} + \bm{R_j})=e^{i\bm{k} \cdot \bm{R_j}} \psi_{n\bm{k}}(\bm{r})
\end{equation}
Function that satisfy such condition can be written as:
\begin{equation}
    \psi_{n\bm{k}}(r)=e^{i\bm{k} \cdot \bm{r}} u_{n\bm{k}}(\bm{r})
\end{equation}
where $u_{n\bm{k}}(\bm{r})$ has the periodicity of the lattice.
now $\bm{k}$ is restricted only in the unit cell of the reciprocal lattice 
if we define reciprocal lattice vector $\bm{K_s}$ by $\bm{K_s} \cdot \bm{R_j} = 2\pi n $: 
\begin{equation}
    \bm{k'}=\bm{k}+\bm{K_s}
\end{equation}
gives
\begin{equation}
    e^{i\bm{k} \cdot \bm{R_j}} = e^{i\bm{k'} \cdot \bm{R_j}} 
\end{equation}
Consequently, both wavefunctions $\psi_{n\bm{k}}(r)$ and $\psi_{n\bm{k'}}(r)$ prossess the same translation eigenvalue. 
The wavevector $\bm{k}$ and $\bm{k'}$ are said to be equivalent and we have:
\begin{align}
    \psi_{n\bm{k}}(\bm{r}) = \psi_{n\bm{k}+\bm{K_s}}(r)
    \varepsilon_{n\bm{k}}  = \varepsilon_{n\bm{k}+\bm{K_s}}
\end{align}
where eigenvalue $\varepsilon_{n\bm{k}}$ is for the Schr\"{o}dinger equation:
\begin{equation}
    H \psi_{n\bm{k}}(\bm{r}) = \varepsilon_{n\bm{k}} \psi_{n\bm{k}}(\bm{r})
\end{equation}

\section{The Tight-binding Method}
In the crystal, the electron wavefunction near the cores looks like atomic orbitals, this suggests that 
we can construct an electron wavefunction by combining the atomic orbitals, each localized on a particular atom,
to represent a state running throughout the crystal.\footnote{J.M. Ziman \emph{Theory of Solids}, page 91}
We suppose that $\phi_a(\bm{r}-\bm{l})$ is an atomic orbital for an atom centered at $l$. Then we construct an function that satisfies
the Bloch Theorm:
\begin{equation}
    \phi_{a\bm{k}}(\bm{r}) = \sum_{\bm{l}} e^{i\bm{k}\cdot\bm{l}} \phi_a(\bm{r}-\bm{l})
\end{equation}
where $a$ can be interpreted as the atom index. In bra-ket notation we express:
\begin{equation}
    \left| a\bm{k} \right\rangle = \sum_{\bm{l}} e^{i\bm{k}\cdot\bm{l}} \left| al \right\rangle
\end{equation}
where $\left| al \right\rangle$ is the state of atom $a$ centered at $l$ and $\left| a\bm{k} \right\rangle$ is the state indexed by $a$ and $\bm{k}$.
We can construct the tight-binding Hamiltonian for a given $\bm{k}$:
\begin{align} 
    \label{tb}
    \left\langle a\bm{k} \middle| H \middle| b\bm{k} \right\rangle &= \sum_{\bm{l_1}}\sum_{\bm{l_2}} e^{-i\bm{k}(\bm{l_2}-\bm{l_1})} \left\langle a\bm{l_1} \middle| H \middle| b\bm{l_2} \right\rangle \\
                        &= N\sum_R e^{-i\bm{k}\bm{R}} \left\langle a0 \middle| H \middle| b\bm{R} \right\rangle \notag
\end{align}
$\left\langle a\bm{k} \middle| H \middle| b\bm{k} \right\rangle$ is just the matrix element $H_{ab}^{\bm{k}}$ 
while $\left\langle a\bm{l_1} \middle| H \middle| b\bm{l_2} \right\rangle$ is the matrix element $H_{ab}^{0\bm{R}}$:
\begin{equation}
    H_{ab}^{0\bm{R}}=\int \phi^*_a(\bm{r}-0) \left\{ -\frac{\hbar^2}{2m}\nabla^{2} + v^{eff}(\bm{r}) \right\} \phi_b(\bm{r}-\bm{R}) d\bm{r}
\end{equation}

\section{Maximally Localized Wannier Function}
Wannier function is constructed according to 
\begin{equation}
    \left| n\bm{R} \right\rangle = \frac{V}{(2\pi)^2}\int_{BZ} d\bm{k} e^{-i\bm{k}\cdot\bm{R}} \left| n\bm{k} \right\rangle 
\end{equation}
where $V/(2\pi)^2$ is the $\bm{k}$ point density in the $BZ$ and $V/(2\pi)^2\int d\bm{k}$ replace the summation of $\bm{k}$.
It can be shown that under a lattice translation $(\bm{R'}-\bm{R})$ state  $\left| n\bm{R} \right\rangle$ transform into $\left| n\bm{R'} \right\rangle$.
Therefore we interpret  $\left| n\bm{R} \right\rangle$ as the wannier function of $n$ orbital centered on $\bm{R}$.

In a multiband case we can further contruct Wannier function by:
\begin{equation}
    \left| n\bm{R} \right\rangle = \frac{V}{(2\pi)^2}\sum_{BZ} d\bm{k} e^{-i\bm{k}\cdot\bm{R}} \sum_{m=1}^J U_{mn}^{\bm{k}}\left| m\bm{k} \right\rangle 
\end{equation}
where the unitary matrix $U_{mn}^{\bm{k}}$ is defined to maximize the localization of the wannier state $\left| n\bm{R} \right\rangle$

\section{Application to CrSi\textsubscript{2}}
The wannierization is done with the \emph{Wannier90} code, which let us extract the Hamiltonian matrix element between the wannier function $H_{ab}^{0\bm{R}}$.
The output is written to the \emph{seedname\_hr.dat} file. For the description of the find see \emph{Wannier90 user guide}. 
It contains the matrix element $H_{ab}^{0\bm{R}}$ in each line with $R$ indicated by leading three integer with two orbital index. The rest two float point number
is the real and imaginary part of the Hamiltonian, respectively. The matrix element should be real.

The following is some result for the CrSi\textsubscript{2} where there are 42 wannier functions in a unit cell: Cr $4s,3d$ and Si $3s,3p$. 
The localization of the Cr $d$ orbital are quite well but its $4s$ orbitals have large spread and are delocalized. 
For the Si atom, its $s$ orbital localized well but the $p$ orbitals have slightly larger spread. The Hamiltonian is generated according to equation \ref{tb}. 
The tight-binding band structure is given and compared with the \emph{PWscf} bandstructure:
\begin{figure}[h]
    \includegraphics{pwscf_band/Tight_binding_PWscf_compare.pdf}
    \centering
    \caption{Comparsion between the band extracted from the tight-binding Hamiltionian and the bandstructure from PWscf}
\end{figure}
It seems that the tight-binding do not reproduce the Bandstructure perfectly, but the generally shape. 
However, we should especially not that in the Tight-binding picture, the highly dispersive s band crosses the fermi surface along the direction $\Gamma-M$: 
the hybirdization does not open a gap. Probably the tight-binding mode here is not so complete.
\begin{figure}[h]
    \centering
    \includegraphics{pwscf_band/bandstructure_PWscf.pdf}
    \caption{The complete bandstructure}
\end{figure}

\end{document}
