\documentclass{article}
\usepackage{amsmath}
\usepackage[margin=0.8in]{geometry}
\usepackage{verbatim}
\usepackage{graphicx}
\usepackage{hyperref} % \url \href

\newcommand{\pfrac}[2]{\frac{\partial #1}{\partial #2}}
% \renewcommand{\H}{\mathcal{H}}

\begin{document}

\title{First principle defect calculation}
\author{Wenhao}
\date{\today}
\maketitle

\section{Defect formation energy}

Defect formation energy of an isolated defect $H^f_q$ is expressed by the reaction equation:
\begin{equation}
    \sum_i N_i \mu_i + E_f dn_e + H^f_q \to E^{def}_{iso,q}
\end{equation}
where the summation is over all the components $i$. $N$ and $\mu$ is the counts and 
chemical potential of each components. $dn_e$ is the number of electrons that we additionally 
added or removed to the system, $E_f$ is their energy. The chemical potential is calculated by 
considering the equilibrium between coexisting phase, which is obtained from the phase 
diagram\cite{Kuwabara2004}. $E^{def}_{iso,q}$ is the energy of a supercell that contain the isolated defect with total charge
charge $q$, $q > 0$ for removing electrons and $q < 0$ for adding electrons ($dn_e = - q$ from definition).

In actual calculation, we have:
\begin{equation}
    E^{def}_{iso,q} = E^{def}_{periodic,q} - E^{corr}_q
\end{equation}
$E^{corr}_q$ is the correction necessary to remove the contribution of calculated total 
energy in the defected supercell. Rearranging, we have the expression for the 
defect formation energy:
\begin{equation}
    H^{f}_q = E^{def}_{periodic,q} - \sum_i N_i \mu_i + qE_f - E^{corr}_q
\end{equation}

\section{Correction of the image charge}
In this section, we provide a summary of the popular methods for the 
finite size correction in charged defect calculation. We review the method
by Freysoldt\cite{Freydoldt2009} and Kumagai\cite{Kumagai2014}

\subsection*{Freysoldt's method}
In Freysoldt's article, the creation of a charged point defect is split into three steps, 
In the first step, we consider a charge neutral cell which contain the defect. Take the 
As defect in Silicon as an example, compare to the perfect crystal, the additional ionic charge
on the As atoms and an additional electron are introduced. This extra electron will form an orbital
$\phi_d(r)$. Then we attend to remove this additional electron from this orbital 
(to the conduction band bottom in some far away unitcell), leading to ionization. In the defect 
supercell, we have a charge:
\begin{equation}
    q_d(r) = q |\phi_d(r)|2
\end{equation}

Next, we relax the system, allowing the electrons (and ions) to screen the introduced charge. The overall
resulting electron distribution give raise to a change in electrostatic potential, compared to the 
case of neutral defect:
\begin{equation}
    V_{q/0}(r) = V_{defect,q}(r) - V_{defect,0}(r) \label{charge_potential}
\end{equation}
$V_{q/0}(r)$ is the potential introduced from ionization. 
Finally, we introduce the artificial periodicity and add a compensating homogeneous background charge density
$n = -q/\Omega$, where $\Omega$ is the volume of the supercell. The compensating background removes the 
divergence of the electrostatic potential:
\begin{align}
    V_{q/0}(G) = \int d^3r V_{q/0}(r) e^{-iGr} \\
    \tilde{V}_{q/0}(r) = \frac{1}{\Omega} \sum_{G\neq 0} V_{q/0}(G) e^{iGr} \label{fourier}
\end{align}
where $q\neq 0$ removes the divergence with compensating background charge and $\tilde{V}_{q/0}(r)$ is now the 
additional periodic potential (compared to charge neutral case) from charged periodic defects. 

We keep track of the artificial elements we introduced in our calculation: the periodic image charge and their 
potentials as well as the neutral compensating background. Thus the energy correction consist of two parts:
\begin{align}
    E^{inter} = \frac{1}{2} \int_{\Omega}  [q_d(r) + n ]  [ \tilde{V}_{q/0}(r) - V_{q/0}(r)  ] d^3 r \\
    E^{intra} = \int_{\Omega} n V_{q/0}(r) d^3r = -\frac{q}{\Omega} \left( \int_{\Omega} V_{q/0}(r) d^3 r \right)
\end{align}

The first part is the interaction between the defect charge and compensating background in the home supercell with
the potential from imaginary charge and background from image cells. The second part is the interaction of the 
defect potential in the supercell introduced by the actual defect and the compensating background. The integral is 
confined to the supercell, whose total energy is used.

We now separate the potential $V_{q/0}(r)$ into two parts: 1) a long ranged part $V^{lr}_q(r)$, that we can approximate using 
simple form. 2) a short ranged part $V^{sr}_q(r)$ that is zero outside of the supercell:
\begin{align}
    V_{q/0}(r) &= V^{lr}_{q}(r) + V^{sr}_{q/0}(r) \\
    V^{lr}_{q}(r) &= \frac{1}{\epsilon} \int \frac{q_d(r')}{|r-r'|} d^3 r' \label{vlr}
\end{align} 

For their periodic form, we have:
\begin{align}
    \tilde{V}^{sr}_{q/0}(r) &= \sum_R V^{sr}_{q/0}(r+R) + C \\ \notag 
                &\approx  V^{sr}_q(r) + C \\
    \tilde{V}_{q/0}(r) = \tilde{V}^{lr}_{q}(r) &+ \tilde{V}^{sr}_{q/0}(r) = \tilde{V}^{lr}_{q}(r) + V^{sr}_q(r) + C \label{align}
\end{align}
the periodic part,$\tilde{V}^{lr}_q(r)$ is obtained from Eq.\ref{vlr} and Eq.\ref{fourier}. The constant C to reproduce the 
absolute value of $\tilde{V}_{q/0}(r)$ (the value of $\tilde{V}^{lr}_{q/0}(r)$ is fixed once we choose the form the $q_d(r)$).

The correction energy can be simplified to be:
\begin{equation}
    E^{corr} = E^{inter} + E^{intra} = E^{lat}_q + q\Delta_{q/0}
\end{equation}
with each part
\begin{gather}
    E^{lat}_q = \int_{\Omega} \left( \frac{1}{2} [ q_d(r) + n ][\tilde{V}^{lr}_{q}(r) - V^{lr}_{q}(r) ] +  n V^{lr}_{q}(r) \right) d^3r \\
    \Delta_{q/0} = \frac{1}{\Omega} \int_{\Omega} V^{sr}_{q/0}(r) d^3r  \\
    V^{sr}_{q/0}(r) = \tilde{V}_{q/0}(r) - \tilde{V}^{lr}_{q}(r) - C
\end{gather}
The correction energy thus depends only on values of $\epsilon$, $q_d(r)$, $\tilde{V}_{q/0}$ and $C$.
the dielectric constant $\epsilon$ can be calculated by first principle methods such as DFPT, $\tilde{V}_{q/0}$ 
are calculated by Eq.\ref{charge_potential} using the potential calculated for the charged supercell and 
neutral defect (or perfect bulk potential). The constant $C$ is calculated by Eq.\ref{align}.
For the charge density $q_d(r)$, reasonable approximation suffices (point charge $q_d(r) = \delta(r)q$ or 
a Gaussian charge).

Finally, the form of defect formation energy is:
\begin{equation}
    H^{f}_q = E^{def} - \sum_i N_i \mu_i - E^{lat}_q + q(E_f + \Delta_{q/0})
\end{equation}
$q$ is positive if we remove the electron from the supercell (The charge of the whole supercell is positive).

\subsection*{Modified Freysoldt's method (Kumagai)}
In Kumagai's paper, some method as Freysoldt is used. However, some modification is introduced 
to consider the anisotropy and the difficult in potential alignment when the ions are polarizable.
Kumagai's paper also reviews some simpler formula, summarized as follows:

\textbf{Point Charge Correction} In the simplest consideration, we treat the defect charge as a point charge
and the potential of the imagine charge and the compensating background can then be written as Madelung 
potental in isotropic system:
\begin{equation}
    V_{PC,q}^{isot} = \frac{\alpha q}{\epsilon L}
\end{equation}
where \emph{isot} means isotropic, and the PC correction energy is then given as:
\begin{equation}
    E_{PC,q}^{isot} = \frac{1}{2}\int_{\Omega} V_{PC,q}^{isot} q \delta(r) d^3r = \frac{\alpha q^2}{2 \epsilon L}
\end{equation}

\textbf{Makov-Payne Correction} Makov and Payne derived the energy correction with order $L^{-3}$, in addition to the 
correction term $L^{-1}$ given above when the defect charge density is no longer a simple point charge:
\begin{align}
    V_{MP,q}^{isot}(r) &=  V_{PC,q}^{isot} - \frac{2\pi q}{3 \epsilon L^3} r^2 + \frac{4\pi}{3 \epsilon L^3} p \cdot r - \frac{2\pi Q}{3 \epsilon L^3} + O(r^4) \\
    E_{MP,q}^{isot} &= \frac{1}{2}\int_{\Omega} V_{MP,q}^{isot}(r) q_d(r) d^3r \\
        &= E_{PC,q}^{isot} - \frac{2\pi q Q}{3 \epsilon L^3} + \frac{2\pi p^2}{3\epsilon L^3} + O(L^{-5})
\end{align} 
where $p = \int r q_d(r) d^3r$ and $Q = \int r^2 q_d(r) d^3 r$ is the dipole and second radial moment of the defect charge density. Usually, 
using the point charge correction or Makov-Payne correction involves fittig the formation energy with respect to $L$ (supercell lattice constant)
using $L^{-1}$ or $L^{-1} + L^{-3}$ and extend $L \to \infty$.

Based on Freysoldt's method of energy correction, Kumagai et. al. purposed \textbf{Improved Freysoldt's method} to be used in the case of 
an anisotropic ionic crystal. Firstly, the term $\tilde{V}^{lr}_{q}(r)$ is replaced using anisotropy dielectric constant $\mathbf{\epsilon}$ and point charge
approximation of the defect charge density
\footnote{
    The notataion used in Kumagai and Freysoldt are different: In Kumagai, $V_{bulk}$ is the periodic potential in a perfect bulk system, $V_{defect,q}$
    is the (calculated) potential in a defect supercell with periodic array of defect charge and neutralization background charge, $V_{isolated,q}$ is the 
    potential of an system with an isolated defect. They correspond to:
    \begin{align}
        V_{isolated,q} - V_{bulk} &\to V_{q/0} \ (V_{q/b}) \notag \\
        V_{defect,q} - V_{bulk} &\to \tilde{V}_{q/0} \ (V_{q/b}) \notag \\
        V_{PC,d} \text{ (Eq.(5))} &\to \tilde{V}^{lr}_{q}(r=0) \notag \\
        \Delta V_{PC,q/b} &\to  V^{sr} + C \ (\Delta V_{PC,q/b} |_{far} = C) \notag
    \end{align}
    where $V_{q/0}$ play the same role as $V_{q/b}$ depending on the reference potential (Bulk or neutral defect). 
    Furthermore, In Kumagai's paper, the term $E_{PC}$ also differs from the term $E_{lattice}$ in Freysoldt, with the relationship:
    \begin{equation}
        E_{PC} = E_{lattice} - \frac{q}{\Omega} \int_{\Omega} d^3 r [ \tilde{V}_{q/0}(r) - \tilde{V}^{lr}_{q}(r) ]
    \end{equation} 
    and the fact $\int_{\Omega} \tilde{V}_{q/0} d^3 r = 0 $ by their convention:
    \begin{equation}
        \int_{\Omega} V_{defect,q} d^3 r = \int_{\Omega} V_{bulk} d^3 r = 0
    \end{equation}
}:
\begin{align}
    V_{PC,q}^{aniso} (r\neq 0) 
    = \sum_{R_i} \frac{q}{\sqrt{|\bar{\epsilon}|}} \frac{\text{erfc}(\gamma \sqrt{(R_i-r) \bar{\epsilon}^{-1} (R_i-r) })}{\sqrt{(R_i-r) \bar{\epsilon}^{-1} (R_i-r) }} \\
        - \frac{\pi q}{\Omega \gamma^2} + \sum_{G_i}^{i\neq 0} \frac{4\pi q}{\Omega} \frac{\exp(- G_i \bar{\epsilon} Gi / 4\gamma^2)}{G_i \bar{\epsilon} Gi} \cdot \exp(iG_i r)
\end{align}
which is obtained by Ewald summation. Secondly, for evaluating the $C$ term, the potential averaged at the atomic position in the relaxed defect supercell in a defined sampling 
region (outside the shpere that is in contact with the Wigner-Seiz cell)
is used instead of the comparing the plane averaged value as in Freysoldt's original method. See Figure 1), 2) and 3) in Kumagai's paper.

\begin{thebibliography}{99}
    \bibitem{Kuwabara2004}
        First Principles Calculation of Defect Formation Energies in Sr and Mg-Doped LaGaO$_3$, Akihide Kuwabara and Isao Tanaka. J. Phys. Chem. B 108, 9168-9172 (2004)
    \bibitem{Freydoldt2009}
        Fully Ab-initio Finite-size Corrections for Charged-Defect Supercell Calculations, Christoph Freysoldt, et. al. PRL 102, 016402 (2019)
    \bibitem{Kumagai2014}
        Electrostatics Based Finite Size Corrections for First-principle Point Defect Calculations, Kumagai, et. al. PRB 89, 195205 (2014) 
\end{thebibliography}


\end{document}