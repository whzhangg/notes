\documentclass{article}
\usepackage{amsmath}
\usepackage[margin=0.8in]{geometry}
\usepackage{verbatim}
\usepackage{graphicx}
\usepackage{hyperref} % \url \href

\newcommand{\pfrac}[2]{\frac{\partial #1}{\partial #2}}
% \renewcommand{\H}{\mathcal{H}}

\begin{document}

\title{Transport in the case of anisotropic parabolic band}
\author{Wenhao}
\date{\today}
\maketitle

\section{Multiband anisotropic transport coefficient}
Here, we follow the W. E. Bies et. al.'s derivation of the 
thermoelectric properties in an anisotropic semiconductor.
\emph{PRB, 65, 085208}. 

The result of the semiclassical transport theory in the Boltzmann 
equation in the relaxation-time approximation gives
\begin{gather}
    \sigma_{ij} = e^2\int d\varepsilon \left( -\pfrac{f_0}{\varepsilon} \right) \Sigma_{ij}(\varepsilon) \\
    T(\sigma \cdot S)_{ij} = e\int d\varepsilon \left(-\pfrac{f_0}{\varepsilon}\right) 
                                \Sigma_{ij}(\varepsilon) (\varepsilon - \mu)     
\end{gather}
and the transport distribution tensor $\Sigma$ is given by:
\begin{equation}
    \Sigma_{ij}(\varepsilon) = \int \frac{2d^3k}{(2\pi)^3} v_i(k) \sum_l \tau_{jl}(k)v_l(k) \delta[\varepsilon - \varepsilon(k)]
\end{equation}
where $\varepsilon$ is the energy of the electronic states, $\mu$ is the chemical potential, 
$v(k)$ is the electronic velocity and $\tau_{ij}(k) = \tau(\varepsilon)U_{ij}$ 
is the relaxation time which is explicitly
an anisotropic tensor. 
Instead of considering $N$ degenerate parabolic valleys as in the derivation of Bies. 
we consider N parabolic valleys with different energy $\varepsilon_n$ and effective mass 
tensor $M_{ij}$ with the electronic dispersion relationship and 
electronic group velocity is given by:
\begin{gather}
    \varepsilon^{(n)}(k) = \varepsilon_n + 
        \frac{\hbar^2}{2} \sum_{ij} (k_i - k_i^{(n)}) M_{ij}^{(n)-1} (k_j - k_j^{(n)}) \\
    v_i^{(n)}(k) = \hbar\sum_j M_{ij}^{(n)-1} (k_j - k_j^{(n)})
\end{gather}
where $\varepsilon_n$ is the energy at the bottom of the parabolic valley and 
$k_{\alpha}^0$ with $\alpha = i,j,k$ is the position of the band maximum. 
Putting the expression of the electronic velocity into the expression for $\Sigma$
and summing over all $N$ transport valleys, we obtain:
\begin{align}
    \Sigma_{ij} (\varepsilon) & = \sum_{n=1}^N \tau(\varepsilon)
        \int \frac{2d^3k}{(2\pi)^3} \hbar^2 
        \sum_{i',j',l} M_{ii'}^{(n)-1} k_{i'} U_{jj'}^{(n)} M_{j'l}^{(n)-1} k_l 
        \delta[\varepsilon - \varepsilon(k + k^{(n)})] \\
        & = \sum_{n=1}^N  \frac{2^{3/2} \tau(\varepsilon - \varepsilon_n) (\varepsilon - \varepsilon_n)^{3/2} }{3\pi^2\hbar^3}
        \sum_l \sqrt{\det \mathbf{M}^{(n)}} U_{jl}^{(n)} M_{li}^{(n)-1} \theta(\varepsilon - \varepsilon_n)
\end{align}
which correspond to Eq.(29) with a change of variable from $\varepsilon \to (\varepsilon - \varepsilon_n)$ 
for different parabolic valleys and A step $\theta$ function is included for each band. 

Following the notation of the paper, we further write:
\begin{equation}
    \Sigma_{ij} (\varepsilon) = \sum_{n=1}^N \mathcal{T}^{(n)}(\varepsilon) A_{ij}^{(n)}
\end{equation}
with a dimensionaless matrix $\mathbf{A}$:
\begin{gather}
    \mathbf{A}^{(n)} = \left( m_0^{-1/2} \sqrt{\det \mathbf{M}^{(n)}} \right) 
        \left( \mathbf{U}^{(n)} \mathbf{M}^{(n)-1} \right)^T \\
    \mathcal{T}^{(n)}(\varepsilon) = \frac{2^{3/2} m_0^{1/2}}{3\pi^2\hbar^3}
    \tau(\varepsilon - \varepsilon_n) (\varepsilon - \varepsilon_n)^{3/2} \theta(\varepsilon - \varepsilon_n)
\end{gather}
In the simplest case, we consider $\mathcal{T} = \mathcal{I}$, a $3\times 3$ identity matrix
correspond to isotropic scattering
and $\mathbf{M}$ diagonal. Then $\mathbf{A}$ is also diagonal and
 $A_{i} = A_{ii}$ will simply be:
\begin{equation}
    A^{(n)}_{i} = \left( m_0^{-1/2} \sqrt{M_1 M_2 M_3} \right) / M_i
\end{equation}
for $i = 1,2,3$, three eigen direction. We note that $\mathbf{A}$ encode all information 
about the anisotropy of the electronic bands.

We write the conductivity tensor as:
\begin{align}
    \mathbf{\sigma} &= e^2\int d\varepsilon \left( -\pfrac{f_0}{\varepsilon} \right) \mathbf{\Sigma}(\varepsilon) \\
            &= e^2 \sum_{n=1}^N \int d\varepsilon \left( -\pfrac{f_0}{\varepsilon} \right) \mathcal{T}^{(n)}(\varepsilon) \mathbf{A}^{(n)} \\
            &= \sum_{n=1}^N \sigma_0^{(n)} \mathbf{A}^{(n)}
\end{align}
The Seebeck tensor is related to $\sigma$ by:
\begin{align}
    \mathbf{(\sigma \cdot S)} &= (e/T) \sum_{n=1}^N \int d\varepsilon 
            \left(-\pfrac{f_0}{\varepsilon}\right)(\varepsilon - \mu) \mathcal{T}^{(n)}(\varepsilon) \mathbf{A}^{(n)} \\
        &= \sum_{n=1}^N \sigma_0^{(n)} S_0^{(n)} \mathbf{A}^{(n)}
\end{align}
This two equations follow Eq.(34) and Eq.(35) in the paper but now with different 
bands have different value of $\mathcal{T}^{(n)}(\varepsilon)$. 

In the case when $\mathbf{A}$ is diagonal, we obtain the Seebeck coefficient:
\begin{equation}
    S_i = \frac{\sum_{n=1}^N \sigma_0^{(n)} S_0^{(n)} A_i^{(n)}}{\sum_{n=1}^N \sigma_0^{(n)} A_i^{(n)}}
\end{equation}


\end{document}