\documentclass{article}
\usepackage{amsmath}
\usepackage[margin=0.8in]{geometry}
\newcommand{\pfrac}[2]{\frac{\partial #1}{\partial #2}}
\newcommand{\fak}{f_{\alpha,k,\uparrow}}
\newcommand{\fbk}{f_{\beta,k',\uparrow}}
\newcommand{\nlq}{n_{\lambda,q}}
\newcommand{\ak}{\alpha,k,\uparrow}
\newcommand{\bk}{\beta,k',\uparrow}
\newcommand{\ql}{\lambda,q}

\begin{document}

\title{Equations for phonon drag}
\author{WH}
\date{\today}
\maketitle

\section{The coupled equations}
Boltzmann equation enable us to reduce the $6N$ phase space of 
a $N$ particles distribution to a one-particle distribution function $f_1$.
In this case, we say that "$f_1$ captures all we really need to know about
a system" 
%\footnote{Does this ignores the correlation and thus is incorrect for correlated system?}.
The most general form of the Boltzmann euqation, for a distribution in equilibrium is:
\begin{equation}
    \pfrac{f_1}{t} = \pfrac{H_1}{r_i} \pfrac{f_1}{p_i} - \pfrac{H_1}{p_i} \pfrac{f_1}{r_i} 
                        + \left(\pfrac{f_1}{t}\right)_{coll} = 0  
\end{equation} 
with the summation implied for cartesian directions $i$. 
For an electron in an electric field $\phi$, we have:
\begin{equation}
    \pfrac{H_1}{r_i}=e \pfrac{\phi}{r_i} ; \pfrac{H_1}{p_i} = v_i
\end{equation}
Including the temperature gredient, we have:
\begin{align}
    v_i \pfrac{f_1}{r_i} - e \pfrac{f_1}{p_i}\pfrac{\phi}{r_i} &=\left(\pfrac{f_1}{t}\right)_{coll} \\
    v_i \pfrac{f_1}{T}\nabla T - e v_i \pfrac{f_1}{\varepsilon}\nabla \phi &=\left(\pfrac{f_1}{t}\right)_{coll} 
\end{align}
with 
\begin{align}
    \pfrac{f_1}{r_i} &= \pfrac{f_1}{T} \pfrac{T}{r_i} \\
    \pfrac{f_1}{p_i} &= \pfrac{f_1}{E} \pfrac{E}{p_i} = \pfrac{f_1}{E} v_i 
\end{align}
and gradient as a vector:
\begin{align}
    \nabla A &= \left( \pfrac{A}{x},\pfrac{A}{y},\pfrac{A}{z} \right)
\end{align}

We separate the collision term of electrons into two parts, one with phonons, one without phonons as:
\begin{equation}
    \left(\pfrac{f_1}{t}\right)_{coll} = -\frac{f_1 - f_1^0}{\tau_e *} + \left(\pfrac{f_1}{t}\right)_{e-p}
\end{equation}
with the local equilibrium of electrons and phonon given by the distribution function, 
the states of electrons and phonons are indicated by $\ak$ and $\ql$:
\begin{align}
    \fak^0 &= \frac{1}{e^{(\varepsilon_{\ak}-\mu)/k_B T} +1} \\
    \nlq^0 &= \frac{1}{e^{\hbar \omega_{\ql}/k_B T} -1}  
\end{align}
for phonon, we also write its Boltzmann equation, since it only response to the temperature gredient:
\begin{equation}
    v \pfrac{n_1}{T} \nabla T = 
         \left(\pfrac{n_1}{t}\right)_{coll} = -\frac{n_1 - n_1^0}{\tau_{ph} *} + \left(\pfrac{n_1}{t}\right)_{e-p}
\end{equation}
for the collision without electron-phonon interaction, the term $\tau_e *$ and $\tau_{ph} *$ can 
be mode dependent and includes
\begin{itemize}
    \item $\tau_e *$ can include electron-impurity sacttering. 
    \item $\tau_{ph} *$ includes phonon-phonon and impurity-phonon.
\end{itemize}.

In summery, we obtained two coupled equations, with band index $\alpha$, $\lambda$ and wave vector
$k$ and $q$:
\begin{align}
    v_{\alpha}(k) \pfrac{\fak}{T} &\nabla T - 
      e v_{\ak} \pfrac{\fak}{\varepsilon}\nabla \phi = 
        -\frac{\fak - \fak^0}{\tau_{\ak}} + \left(\pfrac{\fak}{t}\right)_{e-p} \label{BTE1} \\
    v_{\ql} \pfrac{\nlq}{T} &\nabla T = 
      -\frac{\nlq - \nlq^0}{\tau_{\ql}*} + \left(\pfrac{\nlq}{t}\right)_{e-p} \label{BTE2}
\end{align}

\section{Electron-phonon interaction}
In the Born approximation, the electron feels the extra potential of the ionic 
displacement, and is scattered by this potential. we can define the electron-phonon 
matrix element $g_{\alpha,\beta,\lambda}(k,k',q)$ scattering an electron from $(\alpha, k)$ to $(\beta, k')$ by a phonon 
$(\lambda, q)$. The value of $|g|^2$ in the following text is given by:
\begin{equation}
    g^2 = \frac{\hbar}{2m_0 \omega_{\ql}}|\langle\bk|\partial_{\ql}V|\ak\rangle|^2 
\end{equation}
We classify the scattering even into 4 different kinds and write, with the delta function denoting both the 
momemtum conservation as well as the energy conservation requirement.
\begin{table}[h]
    \centering
    \caption{Four distinct scattering event}
    \begin{tabular}{|l|l|l|l|}
        \hline
        $F_1$ & phonon emission   & $(\ak) \rightarrow (\ql) + (\bk)$ & $\fak(1-\fbk)(\nlq+1)|g|^2\delta(k-k'-q)$ \\ \hline
        $F_2$ & phonon absorption & $(\ak) + (\ql) \rightarrow (\bk)$ & $\fak(1-\fbk)(\nlq)  |g|^2\delta(k+q-k')$ \\ \hline
        $F_3$ & phonon emission   & $(\bk) \rightarrow (\ql) + (\ak)$ & $\fbk(1-\fak)(\nlq+1)|g|^2\delta(k+q-k')$ \\ \hline
        $F_4$ & phonon absorption & $(\bk) + (\ql) \rightarrow (\ak)$ & $\fbk(1-\fak)(\nlq)  |g|^2\delta(k-k'-q)$ \\ \hline
    \end{tabular}
\end{table}

We note that spin is conserved in the scattering event so that the spin of the two electronic state $\ak$ and $\bk$ should 
have the same spin. In other words, the two spin channels are independent to each other.

For an arbitrary electronic state $\ak$, the rate of change of its distribution function 
due to electron-phonon interaction is given by
\begin{equation}
    \left(\pfrac{\fak}{t}\right)_{e-ph} = \frac{1}{N_q}\frac{2\pi}{\hbar}\sum_{\bk,\ql}(-F_1 - F_2 + F_3 + F_4) \label{electron_scattering_rate}
\end{equation}

For phonon, we have, from a single spin channel:
\begin{equation}
    \left(\pfrac{\nlq}{t}\right)_{e-ph} = \frac{1}{2}\frac{1}{N_k}\frac{2\pi}{\hbar}\sum_{\ak,\bk}(F_1 - F_2 + F_3 - F_4)
\end{equation}
We note that the summation over electronic state $\ak$ and $\bk$ are double counted, therefore we have the leading $\frac{1}{2}$ to 
take care of this. But the summation of the $F_1$ term is exactly the same as $F_3$ if we change the summation index. Therefore we 
can reduce the equation to:
\begin{equation}
    \left(\pfrac{\nlq}{t}\right)_{e-ph} = \frac{1}{N_k}\frac{2\pi}{\hbar}\sum_{\ak,\bk}(F_1 - F_2)
\end{equation}
Furthermore, taking account of the spin degeneracy(sum up the spin up and down contribution), we would have
\begin{equation}
    \left(\pfrac{\nlq}{t}\right)_{e-ph} = \frac{2}{N_k}\frac{2\pi}{\hbar}\sum_{\ak,\bk}(F_1 - F_2) \label{phonon_scattering_rate}
\end{equation}

To linearize the equation for both electron part and phonon part, we write the distribution function
\begin{align}
    \fak &= \fak^0 + \Delta \fak \\
    \fbk &= \fbk^0 + \Delta \fbk \\
    \nlq &= \nlq^0 + \Delta \nlq 
\end{align}

We try to find the parts in $\left(\partial\fak / \partial t\right)_{e-ph}$ and $\left(\partial\nlq / \partial t\right)_{e-ph}$ that 
is linear in $\Delta \fak$ and $\Delta \nlq$. To do this, we simply replace the distribution function in equation.\ref{electron_scattering_rate}
and \ref{phonon_scattering_rate} to their $\Delta$ counterpart. As an example, equation.\ref{electron_scattering_rate} linearized in 
terms of $\Delta\fak$ is given by:
\begin{align}
    -F_1 + F_4 &= \left[ -\Delta\fak(1-\fbk)(\nlq+1) + \fbk(-\Delta\fak)\nlq \right]|g|^2\delta(k-k'-q) \notag \\
               &= -(\nlq+1-\fbk)|g|^2\delta(k-k'-q)\Delta\fak \\
    -F_2 + F_3 &= \left[ -\Delta\fak(1-\fbk)(\nlq) + \fbk(-\Delta\fak)(\nlq+1) \right]|g|^2\delta(k+q-k') \notag \\
               &= -(\nlq+\fbk)|g|^2\delta(k+q-k')\Delta\fak
\end{align}
so that when we sum them together, we obtain(note that $\Delta \fak$ can now be pulled outside of the summation)
\begin{equation}
    \left(\pfrac{\fak}{t}\right)_{e-ph} = - \frac{1}{N_q}\frac{2\pi}{\hbar}\sum_{\bk,\ql}\left[ (\nlq+1-\fbk)|g|^2\delta(k-k'-q) + (\nlq+\fbk)|g|^2\delta(k+q-k') \right]  \Delta\fak
\end{equation}

Using the same method, we write the linearized part in terms of other two variation in distribution:
\begin{align}
    \left(\pfrac{\fak}{t}\right)_{e-ph} &= \frac{1}{N_q}\frac{2\pi}{\hbar}\sum_{\bk,\ql}\left[ \left\{ (\fak+\nlq)|g|^2\delta(k-k'-q) + (\nlq+1-\fak)|g|^2\delta(k+q-k') \right\} \Delta\fbk \right]\\
    \left(\pfrac{\fak}{t}\right)_{e-ph} &= \frac{1}{N_q}\frac{2\pi}{\hbar}\sum_{\bk,\ql}\left[ \left\{ (\fbk-\fak)|g|^2\delta(k-k'-q) + (\fbk-\fak)|g|^2\delta(k+q-k')   \right\} \Delta\nlq \right]
\end{align}

The relaxation time due to the electron-phonon interaction is given by:
\begin{equation}
    -\frac{\Delta\fak}{\tau_{\ak}} = \left(\pfrac{\fak}{t}\right)_{e-ph}
\end{equation}
i.e. the linear coefficient of $\Delta\fak$. Therefore, we obtain:
\begin{equation}
    \frac{1}{\tau_{\ak}} = \frac{1}{N_q}\frac{2\pi}{\hbar}\sum_{\bk,\ql}\left[ (\nlq+1-\fbk)|g|^2\delta(k-k'-q) + (\nlq+\fbk)|g|^2\delta(k+q-k') \right]
\end{equation}

Which is the same equation as relaxation time from the self-energy approximation, see equation.8 of the paper \emph{EPW:Electron-phonon coupling, Computer Physics Communication}.
The relaxation time is given by:
\begin{equation}
    \frac{1}{\tau_{\ak}(\omega,T)} = 2 \Sigma''_{\ak}(\omega,T)
\end{equation}
and the change from k-space summation and integration is given by:
\begin{equation}
    \int_{BZ}\frac{dq}{\Omega_{BZ}} \Rightarrow \frac{1}{N_q}\sum_{q}
\end{equation}


Linearizing the phonon scattering rate, we have:
\begin{align}
    \left(\pfrac{\nlq}{t}\right)_{e-ph} = \frac{2}{N_k}\frac{2\pi}{\hbar}\sum_{\ak,\bk} \left[ \fak(1-\fbk)(\nlq+1)|g|^2\delta(k-k'-q) - \fak(1-\fbk)(\nlq)  |g|^2\delta(k+q-k') \right]
\end{align}
Noting that the second part can be rewrite by changing the index $(\ak)\rightarrow(\bk)$ and vice versa:
\begin{equation}
    \fak(1-\fbk)(\nlq)  |g|^2\delta(k+q-k') \Rightarrow \fbk(1-\fak)(\nlq)  |g|^2\delta(k-k'-q)
\end{equation}

We have, to linear part:
\begin{align}
    \left(\pfrac{\nlq}{t}\right)_{e-ph} = \frac{4\pi}{N_k\hbar}\sum_{\ak,\bk} & \left\{ (\nlq+1-\fbk)|g|^2\delta(k-k'-q)\Delta\fak - (\fak+\nlq)|g|^2\delta(k-k'-q)\Delta\fbk \right\} \notag \\
                    +\frac{4\pi}{N_k\hbar}& \left\{ \sum_{\ak,\bk}(\fak-\fbk)|g|^2\delta(k-k'-q) \right\} \Delta\nlq
\end{align}

We are able to write the phonon relaxation:
\begin{equation}
    \frac{1}{\tau_{\ql}} = \frac{4\pi}{N_k\hbar} \sum_{\ak,\bk}(\fbk-\fak)|g|^2\delta(k-k'-q) 
\end{equation}

notice the sign of the relaxation time, we should have $\fbk > \fak$, which is true since from the delta function $\delta(k-k'-q)$, 
state $(\bk)$ should have a lower energy. This equation is comparable to equation.9 of \emph{EPW:Electron-phonon coupling}. 

As a temporary summary, we write down the linearized parts as:
\begin{align}
    \left(\pfrac{\fak}{t}\right)_{e-ph} = - \frac{\Delta \fak}{\tau_{\ak}} 
         +\sum_{\bk,\ql} \left[ \varGamma_{\bk}(\ak,\ql)\Delta\fbk 
           +\varGamma_{\ql}(\ak,\bk)\Delta\nlq \right]  \label{collision1} \\
    \left(\pfrac{\nlq}{t}\right)_{e-p} = - \frac{\Delta\nlq}{\tau_{\ql}}
        + \sum_{\ak,\bk} \left[ \varXi_{\ak}(\bk,\ql)\Delta \fak 
                              + \varXi_{\bk}(\ak,\ql)\Delta\fbk \right] \label{collision2}
\end{align} 
notice here that the electronic part equation.\ref{collision1} is only for single spin channel while 
the second equation for the phonon already take into account the spin degeneracy, the $\uparrow$ in the 
thus only mean a single state, in equation.\ref{collision2}.
With the relaxation time already defined above and the $\varXi$ and $\varGamma$ defined as:
\begin{align}
    \varGamma_{\bk}(\ak,\ql) &= \frac{1}{N_q}\frac{2\pi}{\hbar}\left\{ (\fak+\nlq)|g|^2\delta(k-k'-q) + (\nlq+1-\fak)|g|^2\delta(k+q-k') \right\} \\
    \varGamma_{\ql}{\ak,\bk} &= \frac{1}{N_q}\frac{2\pi}{\hbar}\left\{ (\fbk-\fak)|g|^2\delta(k-k'-q) + (\fbk-\fak)|g|^2\delta(k+q-k')   \right\} \\
    \varXi_{\ak}(\bk,\ql) &= \frac{4\pi}{N_k\hbar} (\nlq+1-\fbk)|g|^2\delta(k-k'-q) \\
    \varXi_{\bk}(\ak,\ql) &= -\frac{4\pi}{N_k\hbar} (\fak+\nlq)|g|^2\delta(k-k'-q)
\end{align}

Putting the terms $\left(\partial\fak / \partial t\right)_{e-ph}$ and $\left(\partial\nlq / \partial t\right)_{e-ph}$ in 
equation.\ref{collision1} and \ref{collision2} 
back into the coupled transport equation.\ref{BTE1} and \ref{BTE2}. We would then obtain the necessary equation 
to solve the phonon drag problem.



%The collision terms of state $\alpha, k$ can be obtained by iterating over the the end states and the 
%phonon that cause the scattering. 
%\begin{align}
%    \left(\pfrac{\fak}{t}\right)_{e-p} =& \frac{2\pi}{\hbar}\sum_{(\beta, k'),(\lambda, q)} \varLambda_{(\alpha, k),(\beta, k'),(\lambda, q)} \\
%    \varLambda_{(\alpha, k),(\beta, k'),(\lambda, q)}= 
%        &( - |g|^2 \nlq\fak(1-\fbk)\delta(k+q-k')     \notag\\
%        &  - |g|^2(\nlq+1)\fak(1-\fbk)\delta(k-k'-q) \notag\\
%        &  + |g|^2\nlq(1-\fak)\fbk\delta(q+k'-k)     \notag\\
%        &  + |g|^2(\nlq+1)(1-\fak)\fbk\delta(k'-k-q) )    
%\end{align}
%For the phonons, we also have:
%\begin{align}
%    \left(\pfrac{\nlq}{t}\right)_{e-p}=& \frac{2\pi}{2\hbar}\sum_{(\alpha, k),(\beta, k')} \varUpsilon_{(\alpha, k),(\beta, k'),(\lambda, q)} \\
%    \varUpsilon_{(\alpha, k),(\beta, k'),(\lambda, q)} = 
%        &( - |g|^2 \nlq\fak(1-\fbk)\delta(k+q-k')    \notag \\
%        &  + |g|^2(\nlq+1)\fak(1-\fbk)\delta(k-k'-q) \notag \\ 
%        &  - |g|^2\nlq(1-\fak)\fbk\delta(q+k'-k)     \notag \\
%        &  + |g|^2(\nlq+1)(1-\fak)\fbk\delta(k'-k-q) )
%\end{align}
%Note that there is a factor $1/2$ for double counting the electron states of $(\alpha, k)$ and $(\beta, k')$.
%
%We express the conservation of both energy and momemtum by the $\delta$ function (selection). Especially, we donate the delta 
%function in such a way that implies the scattering process: the $k$ indexes with $+$ are the incoming states while
%the $k$ indexes with $-$ sign are the outgoing states. For example, $\delta(k-k'-q)$ suggests 
%the process $k\rightarrow k'+q$. 
%
%We also point out here that in $\varLambda$ and $\varUpsilon$ above, the first and fourth
%terms have the same selection function, the second and third terms share another distinct selection function.
%These processes can be seems to be the time reversal process of each other.
%
%Next we linearize the equation by:
%\begin{align}
%    \fak &= \fak^0 + \Delta \fak \\
%    \fbk &= \fbk^0 + \Delta \fbk \\
%    \nlq &= \nlq^0 + \Delta \nlq 
%\end{align}
%We substituting the above into the equation for electron and phonon interaction and keep only upto the linear
%terms. This is done, for example: 
%\begin{align}
%    \nlq\fak(1-\fbk)=& \nlq^0\fak^0+\fak^0\Delta\nlq+\nlq^0\Delta\fak  \notag \\
%                    &+ \nlq^0\fak^0\fbk^0 +\fak^0\fbk^0\Delta\nlq +\nlq^0\fbk^0\Delta\fak +\nlq^0\fak^0\Delta\fbk   
%\end{align}
%collecting the terms, we have:
%\begin{align}
%    \left(\pfrac{\fak}{t}\right)_{e-p} \approx  
%        &-\left[ \sum_{(\beta, k'),(\lambda, q)}\varGamma_{\alpha,k}(\beta, k',\lambda, q) \right] \Delta\fak \notag \\
%        &+\sum_{(\beta, k'),(\lambda, q)} \left[ \varGamma_{\beta,k'}(\alpha,k,\lambda, q)\Delta\fbk \right]
%         +\sum_{(\beta, k'),(\lambda, q)} \left[ \varGamma_{\lambda,q}(\alpha,k,\beta, k')\Delta\nlq \right] \label{collision1} \\
%    \left(\pfrac{\nlq}{t}\right)_{e-p} \approx
%        & \sum_{(\alpha,k),(\beta,k')} \left[ \varXi_{\alpha,k}(\beta,k',\lambda,q)\Delta \fak 
%                                            + \varXi_{\beta,k'}(\alpha,k,\lambda,q)\Delta\fbk \right] \notag \\
%        & - \left[ \sum_{(\alpha,k),(\beta,k')} \varXi_{\lambda,q}(\alpha,k,\beta,k') \right] \Delta\nlq \label{collision2}
%\end{align}
%with the terms defined to be:
%\begin{align}
%    \varGamma_{\alpha,k}(\beta,k',\lambda,q) &= (\nlq^0+\fbk^0)\Pi_{-} + (\nlq^0+1-\fbk^0)\Pi_{+} \\
%    \varGamma_{\beta,k'}(\alpha,k,\lambda,q) &= (\nlq^0+1-\fak^0)\Pi_{-} + (\nlq^0+\fak^0)\Pi_{+} \\
%    \varGamma_{\lambda,q}(\alpha,k,\beta,k') &= (\fbk^0-\fak^0)\Pi_{-} + (\fbk^0-\fak^0)\Pi_{+} \\
%    \varXi_{\alpha,k}(\beta,k',\lambda,q) &= -(\nlq^0+\fbk^0)\Pi_{-} + (\nlq^0+1-\fbk^0)\Pi_{+} \\
%    \varXi_{\beta,k'}(\alpha,k,\lambda,q) &= (\nlq^0+1-\fak^0)\Pi_{-} - (\nlq^0+\fak^0)\Pi_{+} \\
%    \varXi_{\lambda,q}(\alpha,k,\beta,k') &= (\fbk^0-\fak^0)\Pi_{-} - (\fbk^0+1-\fak^0)\Pi_{+} 
%\end{align}
%with $\Pi$ refer to two kind of process:
%\begin{align}
%    \Pi_{-} &= \frac{2\pi}{\hbar} |g(k,k',q)|^2 \delta(k+q-k') \\
%    \Pi_{+} &= \frac{2\pi}{\hbar} |g(k,k',q)|^2 \delta(k-k'-q)
%\end{align}
%We notice that here all the terms of $\varGamma$ and $\varXi$ are given by the equilibrium distribution, thus
%this can be calculated before solving the Boltzmann equation.
%
%In equation.\ref{collision1} and \ref{collision2}, we can take the term that is linear in $\Delta\fak$ and
%$\Delta\nlq$ and absorb them into the mode specific relaxation time. In this case, we define the new
%relaxation time:
%\begin{align}
%    \frac{1}{\tau_{\alpha,k}} &=\frac{1}{\tau_{\alpha,k}*} + 
%            \left[ \sum_{(\beta, k'),(\lambda, q)}\varGamma_{\alpha,k}(\beta, k',\lambda, q) \right] \\
%    \frac{1}{\tau_{\lambda,q}} &=\frac{1}{\tau_{\lambda,q}*} + 
%            \left[ \sum_{(\alpha,k),(\beta,k')} \varXi_{\lambda,q}(\alpha,k,\beta,k') \right]
%\end{align}
%Now the relaxation time includes the scattering of electrons by equilibrium phonon and phonon scattering by 
%equilibrium phonon. 
%Finally, the Boltzmann equation (equation.\ref{BTE1} and \ref{BTE2}) becomes:
%\begin{align}
%    v_{\ak}\pfrac{\fak}{T} \nabla T - e v_{\ak} \pfrac{\fak}{\varepsilon}\nabla \phi = 
%        &-\frac{\fak - \fak^0}{\tau_{\ak}} \notag \\
%        &+\sum_{(\beta, k'),(\lambda, q)} \left[ \varGamma_{\beta,k'}(\alpha,k,\lambda, q)\Delta\fbk + \varGamma_{\lambda,q}(\alpha,k,\beta, k')\Delta\nlq \right] \label{BTE11} \\
%    v_{\ql} \pfrac{\nlq}{T} \nabla T = 
%        &-\frac{\nlq - \nlq^0}{\tau_{\lambda,q}}  \notag \\
%        &+\sum_{(\alpha,k),(\beta,k')} \left[ \varXi_{\alpha,k}(\beta,k',\lambda,q)\Delta \fak 
%             + \varXi_{\beta,k'}(\alpha,k,\lambda,q)\Delta\fbk \right] \label{BTE22}
%\end{align}

\section{Seebeck in the uncoupled case}
We start with the general linear relationship between current flow and 
the field, following mainly the notation of the paper 
\emph{"Transpor coefficients from first-principles calculations, PRB, 2003"}. The 
general linear current response to an applied temperature gradient and a electric field is
given by:
\begin{align}
    J_c&=\sigma E - \zeta \nabla T \\
    J_q&=T\zeta E - \kappa_0 \nabla T 
\end{align} 
with $J_c$ the charge current, $J_q$ the thermal current, $\sigma$ and $\kappa$ can
be recognized to be the electric conductivity and the total 
thermal conductivity. In the matrix form, we have:
\begin{equation}
    \begin{pmatrix} J_c \\ J_q \end{pmatrix} 
    = \begin{pmatrix} \sigma & -\zeta \\ T\zeta & -\kappa_0 \end{pmatrix} 
      \begin{pmatrix} E \\ \nabla T \end{pmatrix}
\end{equation}

Each currents are vectors with element $i,j,k$ correspond to three cartesian directions,
the same is true for the gradient of the field we consider here. Note that
the electric field is given from the electric potential by 
$E = -\nabla \phi$. The 
transport properties $\sigma$, $\zeta$ and $\kappa$ are corresponding tensors 
so that, for example:
\begin{equation}
    J_{c,i} = \sum_j \sigma_{ij} E_j
\end{equation}
according to the above definition, we find the Seebeck coefficients and the Peltier 
coefficients to be:
\begin{align}
    S   &=\frac{\zeta}{\sigma} \\
    \Pi &=\frac{T\zeta}{\sigma} \\
    \pi &= TS
\end{align}

The final equation is the Kelvin equation. The current flow $J_c$ is solely due to 
electrons while the heat flow is from both electrons and phonons. We have:
\begin{align}
    J_{c,i,\uparrow} &= \frac{e}{N_k\Omega}\sum_{\ak} f_{\ak} v_{\ak,i} \label{jce} \\
    J_{q,ele,i,\uparrow} &=  \frac{1}{N_k \Omega}\sum_{\ak} (\varepsilon_{\ak}-\mu) f_{\ak} v_{\ak,i} \label{jqe}\\
    J_{q,phonon,i} &= \frac{1}{N_q \Omega}\sum_{\ql} \hbar \omega_{\ql} n_{\ql} v_{\ql,i} \label{jqp}
\end{align}
where $\Omega$ is the volumn of the unit cell.
Dividing the $N_{k(q)}\Omega$ term is necessary because the these properties are defined as 
"density" and we should note that $N_{k(q)}\Omega$ term is simply the volumn of the 
finite system that we are concerning with. $N_k$ gives the number of unit cells in the 
system.

We can consider two cases:
\begin{itemize}
    \item Starting with $\nabla T =0$ and $E\neq 0 $. This correspond to the Peltier effect.
    \item Starting with $\nabla T \neq 0$ and $E= 0 $. This correspond to the process of 
           Seebeck effect.
\end{itemize}

We start by first deriving the transport equation of decoupled 
equations in the Peltier case, the off-equilibrium electron distribution,
given by (the equilibrium part of the distribution sum up to zero):
\begin{equation}
    \Delta \fak = e \sum_j \left(-\pfrac{\fak^0}{\varepsilon}\right) \tau_{\ak} v_{\ak,j} E_j
\end{equation} 
where we explicitly written down the summation.
the electric current flow is given by equation.\ref{jce} as:
\begin{align}
    J_{c,i} &= 2J_{c,i,\uparrow} = \frac{2e^2}{N_k \Omega} \sum_j \sum_{\ak} \left(-\pfrac{\fak^0}{\varepsilon}\right) \tau_{\ak} v_{\ak,i}v_{\ak,j} E_j \\
    J_{c,i} &= \sum_j \sigma_{ij} E_j
\end{align}
with $J_{c,i}$ now denote the total current of both spin channel by the 
factor of 2 and giving the result for $\sigma$:
\begin{equation}
    \sigma_{ij} = \frac{2e^2}{N_k \Omega} \sum_{\ak} \left(-\pfrac{\fak^0}{\varepsilon}\right) \tau_{\ak} v_{\ak,i}v_{\ak,j} \label{econd}
\end{equation}

Since the electron and phonon are decoupled, phonon do not respond to the electric field. All the 
heat flow is due to electrons, and thus we have, from equation.\ref{jqe}:
\begin{align}
    J_{q,ele,i} &= \frac{2e}{N_k \Omega} \sum_j \sum_{\ak} (\varepsilon_{\ak}-\mu)  \left(-\pfrac{\fak^0}{\varepsilon}\right) \tau_{\ak} v_{\ak,i}v_{\ak,j} E_j \\
    J_{q,i} &= T \sum_j \zeta_{ij} E_j \\
\end{align}
finally giving: 
\begin{equation}
    \zeta_{ij} = \frac{2e}{N_k \Omega T} \sum_{\ak} (\varepsilon_{\ak}-\mu)  \left(-\pfrac{\fak^0}{\varepsilon}\right) \tau_{\ak} v_{\ak,i}v_{\ak,j} \label{zeta_uncouple_peltier}
\end{equation}
the Seebeck coefficients are thus given by: 
\begin{equation}
    S_{ij} = \left[ \frac{2e}{N_k \Omega T} \sum_{\ak} (\varepsilon_{\ak}-\mu)  \left(-\pfrac{\fak^0}{\varepsilon}\right) \tau_{\ak} v_{\ak,i}v_{\ak,j} \right] / \sigma
\end{equation}
in which the division should be interpreted as matrix operation. 

In the Seebeck picture, $E= 0$ and electrons are driven by the thermal gradient $\nabla T $:
\begin{equation}
    \Delta \fak = - \sum_j \left(\pfrac{\fak^0}{T}\right) \tau_{\ak} v_{\ak,j} (\nabla T)_j
\end{equation}
the electric conductivity is still given by equation.\ref{econd}. Now we consider:
\begin{align}
    J_{c,i} &= -\frac{2e}{N_k \Omega} \sum_j \sum_{\ak}\left(\pfrac{\fak^0}{T}\right) \tau_{\ak}v_{\ak,i}v_{\ak,j} (\nabla T)_j \\
    J_{c,i} &= -\sum_j \zeta_{ij} (\nabla T)_j 
\end{align}
finally, we have:
\begin{equation} 
    \zeta_{ij} = \frac{2e}{N_k \Omega}\sum_{\ak}\left(\pfrac{\fak^0}{T}\right) \tau_{\ak}v_{\ak,i}v_{\ak,j} \label{zeta_uncouple_seebeck}
\end{equation}

The two equations of $\zeta$ (\ref{zeta_uncouple_peltier} and \ref{zeta_uncouple_seebeck}) are the same, obeying the 
Kelvin relation. This can be seen, with $\varepsilon' = (\varepsilon_{\ak}-\mu)/k_B T$ by:
\begin{align}
    \pfrac{f_k}{T} &= \pfrac{f_k}{\varepsilon'} \pfrac{\varepsilon'}{T} = \pfrac{f_k}{\varepsilon'} \left( -\frac{\varepsilon_{\ak}-\mu}{k_B T^2} \right) \\
    \pfrac{f_k}{\varepsilon} &= \pfrac{f_k}{\varepsilon'} \pfrac{\varepsilon'}{\varepsilon} = \pfrac{f_k}{\varepsilon'} \left( \frac{1}{k_B T} \right) \\
\end{align}
so that we find:
\begin{equation}
    \pfrac{f_k}{T} = \pfrac{f_k}{\varepsilon} \left( -\frac{\varepsilon_{\ak}-\mu}{T} \right) \label{partial_relation}
\end{equation}
Placing the above relation into equation.\ref{zeta_uncouple_seebeck}, we can recover the results in equation.\ref{zeta_uncouple_peltier}. 
Thus the Seebeck coefficient calculated from the $\zeta$ function and $\sigma$ is the same.

%%%%%%%% SEPARATION %%%%%%%%%%%%%%%%%%
\section{Seebeck in the coupled case}
Now we consider the transport equation when electron and phonon systems are coupled by the
$e-ph$ interaction, as discussed before. The coupled equations are the set \ref{BTE1} and \ref{BTE2}.

We still start with the Peltier case with $\nabla T = 0$. \
Electric field is applied to drive the coupled 
electron-phonon system. Off-equilibrium phonons are only because of the electron-phonon interaction.
We make the following approximation to the coupled transport equation:
\begin{itemize}
    \item In the electron equation.\ref{collision1}, the $\Delta\fbk$ term is small and can be ignored. 
    \item The term due to the non-equilibrium phonon is also small in equation.\ref{collision1} so that $\Delta\nlq$ is also insignificant.
\end{itemize}

The electron transport equation \ref{collision1} are now decoupled from the phonon one. 
We can obtain the electronic part of the heat flow and the elctric conductivity:
\begin{align}
    \sigma_{ij} &= \frac{2e^2}{N_k \Omega} \sum_{\ak} \left(-\pfrac{\fak^0}{\varepsilon}\right) \tau_{\ak} v_{\ak,i}v_{\ak,j} \\
    J_{q,ele,i} &= \frac{2e}{N_k \Omega} \sum_j \sum_{\ak} (\varepsilon_{\ak}-\mu)  \left(-\pfrac{\fak^0}{\varepsilon}\right) \tau_{\ak} v_{\ak,i}v_{\ak,j} E_j \\
\end{align}
for phonons, we have, from equation.\ref{BTE1} and \ref{collision1}:
\begin{align}
    \Delta n_{\ql} &= \tau_{\ql,tot}\sum_{\alpha,k,\beta,k'} \left[ \varXi_{\alpha,k}\Delta \fak + \varXi_{\beta,k'}\Delta\fbk \right]  \\
    &\tau_{\ql,tot} = \tau_{\ql,others}+\tau_{\ql,e-ph}
\end{align}
where the index in function $\varXi$ is omitted for simplicity. The phonon heat flow is thus:
\begin{align}
    J_{q,phonon,i} &= \sum_j \frac{e}{N_q\Omega} \sum_{\ql} \hbar \omega_{\ql} \tau_{\ql} v_{\ql,i} \notag \\ 
     & \sum_{\ak,\bk} \left[ \varXi_{\alpha,k} \left(-\pfrac{\fak^0}{\varepsilon}\right) \tau_{\ak} v_{\ak,j} + \varXi_{\beta,k'}  \left(-\pfrac{\fbk^0}{\varepsilon}\right) \tau_{\bk} v_{\bk,j} \right] E_j
\end{align} 
the total heat flow is given by:
\begin{equation}
    J_{q,i} = J_{q,ele,i} + J_{q,phonon,i}
\end{equation}
corresponding to 
\begin{equation}
    J_{q,i} = T \sum_j \zeta_{ij} E_j
\end{equation}
We obtain the $\zeta$ function:
\begin{align}
    \zeta_{ij} & = \frac{2e}{N_k \Omega T} \sum_{\ak} (\varepsilon_{\ak}-\mu)  \left(-\pfrac{\fak^0}{\varepsilon}\right) \tau_{\ak} v_{\ak,i}v_{\ak,j} \\
      +& \frac{e}{N_q \Omega T} \sum_{\ql} \sum_{\alpha,k,\beta,k'} \hbar \omega_{\ql} \tau_{\ql} v_{\ql,i}\left[ \varXi_{\alpha,k} \left(-\pfrac{\fak^0}{\varepsilon}\right) \tau_{\ak} v_{\ak,j} + \varXi_{\beta,k'}  \left(-\pfrac{\fbk^0}{\varepsilon}\right) \tau_{\bk} v_{\bk,j} \right]
\end{align}

The Seebeck coefficient is thus given by the matrix operation:
\begin{equation}
    S=\zeta / \sigma
\end{equation}

% Coupled, seebeck derivation
From the Seebeck picture, we consider the following:
\begin{itemize}
    \item Term $\Delta\fak$ and $\Delta\fbk$ in the phonon equation.\ref{collision2} are in-significant and thus ignored. 
         Thus the phonons are decoupled from the electron equation.
    \item In the electron equation.\ref{collision1}, the $\Delta\fbk$ term is small and can be ignored.
\end{itemize}
from this picture, we first obtain the off-equilibrium phonon:
\begin{equation}
    \Delta \nlq = - \sum_j \left( \pfrac{\nlq^0}{T} \right) \tau_{\ql} v_{\ql,j} (\nabla T)_j
\end{equation} 

The electrons themself are affected by the temperature gradient, and also the electron phonon interaction. 
We have
\begin{equation}
    v_{\ak} \pfrac{\fak^0}{T} \nabla T  = 
        -\frac{\fak - \fak^0}{\tau_{\alpha,k}}
        +\sum_{\bk,\ql} \varGamma_{\lambda,q}\Delta\nlq
\end{equation}
We have the off-equilibrium electron distribution:
\begin{align}
    \Delta \fak = - \sum_j \tau_{\ak} \left[ \sum_{\beta, k',\lambda, q} \varGamma_{\lambda,q} \left( \pfrac{\nlq^0}{T} \right) \tau_{\ql} v_{\ql,j} + v_{\ak,j} \pfrac{\fak^0}{T} \right] (\nabla T)_j
\end{align}
the electron current is thus given by from equation.\ref{jce} as
\begin{align}
    &J_{c,i} = -\sum_j \zeta_{ij} (\nabla T)_j \\ 
            &= -\frac{2e}{N_k \Omega} \sum_j \left[ \sum_{\ak}\sum_{\bk,\ql} v_{\ak,i} \tau_{\ak} \varGamma_{\lambda,q} \left( \pfrac{\nlq^0}{T} \right) \tau_{\ql} v_{\ql,j} 
                              + \sum_{\ak} \left( \pfrac{\fak^0}{T} \right)\tau_{\ak}v_{\ak,i}v_{\ak,j} \right] (\nabla T)_j
\end{align}
We re-organize and recognize that the $\zeta$ function to be: 
\begin{equation}
    \zeta_{ij} = \frac{2e}{N_k \Omega} \left[ \sum_{\ak} \left( \pfrac{\fak^0}{T} \right)\tau_{\ak}v_{\ak,i}v_{\ak,j} 
                        + \sum_{\ak}\sum_{\beta, k',\lambda, q} v_{\ak,i} \tau_{\ak} \varGamma_{\lambda,q} \left( \pfrac{\nlq^0}{T} \right) \tau_{\ql} v_{\ql,j} \right]
\end{equation}
Seebeck coefficient is given by the $\zeta$ and $\sigma$, as:
\begin{equation}
    S_{ij} = \frac{2e}{N_k \Omega \sigma} \left[ \sum_{\ak} \left( \pfrac{\fak^0}{T} \right)\tau_{\ak}v_{\ak,i}v_{\ak,j} 
                        + \sum_{\ak}\sum_{\beta, k',\lambda, q} v_{\ak,i} \tau_{\ak} \varGamma_{\lambda,q} \left( \pfrac{\nlq^0}{T} \right) \tau_{\ql} v_{\ql,j} \right]
\end{equation}

\subsection*{Rewriting the equation for calculation }
We implement the Seebeck picture in the calculation. In the code, we iterate 
all the q points first, inside each q-point iteration, we iterate over the 
electronic states $\ak$ and $\bk$ of interest. In the $\zeta_{ij}$ function above,
the first part depends straight-forwardly on electronic properties only ($\tau$ given by 
electron-phonon interaction) and is simple to implemented in the transport coefficient 
calculation routine. The second part:
\begin{equation}
    \sum_{\ak}\sum_{\beta, k',\lambda, q} v_{\ak,i} \tau_{\ak} 
            \varGamma_{\lambda,q} \left( \pfrac{\nlq^0}{T} \right) \tau_{\ql} v_{\ql,j}
\end{equation}
requires two part that need to be calculated: $\tau_{\ql}$ and coefficient $\varGamma_{\ql}$. 
We can rewrite this part to:
\begin{equation}
    \sum_{\ak} v_{\ak,i} \tau_{\ak} \left[ \sum_{\ql} \sum_{\bk}  
            \varGamma_{\lambda,q} \left( \pfrac{\nlq^0}{T} \right) \tau_{\ql} v_{\ql,j} \right]
\end{equation}
now the terms in the bracket is summed over $\bk$ and $\ql$, and therefore can be indexed by $\ak$ only. 
The calculation of the bracket term is thus procede almost the same as the calculation of electron life time.

Now, $\zeta$ can be written as:
\begin{equation}
    \zeta_{ij} = \frac{2e}{N_k \Omega} \sum_{\ak}\tau_{\ak}v_{\ak,i} \left[ \left( \pfrac{\fak^0}{T} \right) v_{\ak,j} 
                        + \sum_{\beta, k',\lambda, q}  \varGamma_{\lambda,q} \left( \pfrac{\nlq^0}{T} \right) \tau_{\ql} v_{\ql,j} \right]
\end{equation}
with the two terms in the square bracket is the off-equilibrium electron distribution because of the temperature gradient (diffusive)
and the phonon drag part.

\newpage

\section*{ Usage of atomic units}
Hartree units are named after the physicist Douglas Hartree\footnote{the reference of this note is Wikipedia}, 
in this unit
\emph{the numerical values of the following four fundamental physical constants 
are all unity by definition}. They are:
\begin{itemize}
    \item \textbf{Reduced Planck Constant} $\hbar=1$ (unit of action)
    \item \textbf{Elementary charge} $e=1$ (unit of charge)
    \item \textbf{Bohr radius} $a_0=1$ (unit of length)
    \item \textbf{Electron mass} $m_e=1$ (unit of mass)
\end{itemize}

Each unit in this system can be expressed as a product of powers of four physical
constants \emph{without a multiplying constant}. Therefore, they are 
consistent units, meaning that given any mathematical expression, if all the values 
are in atomic unit, the result will come out as atomic unit without the need for 
conversion.
The derived units in atomic unit system
are converted to SI unit by replacing the value of those constants (the constants
are used as units).
\begin{table}[h]
    \centering
    \caption{Defining constants}
    \begin{tabular}{|l|l|l|}
        \hline 
        \textbf{Symbol} & \textbf{Definition} & \textbf{Value in SI units} \\ \hline
        $\hbar$         & $\hbar$             &$1.054571\times 10^{-34} J\cdot s$ \\ \hline
        $e$             & $e$                 & $1.602176\times 10^{-19} C$  \\ \hline
        $a_0$           & $4\pi\epsilon_0\hbar^2/(m_ee^2)$ & $5.291772\times 10^{-11} m$  \\ \hline
        $m_e$           & $m_e$               & $9.109383\times 10^{-31} kg$ \\ \hline
        $E_h$           & $\hbar^2/(m_ea_0^2) $ & $\hbar^2/(m_e a_0^2)$ \\ \hline
    \end{tabular}
\end{table}

\section{Unit conversion}
As an example, the speed of light in atomic unit of velocity is approximately 137.036,
The atomic unit of velocity is expressed by the constants $a_0 E_h /\hbar$, which is converted 
to SI unit by $a_0 E_h / \hbar = 2.187\times10^6 m/s$, and therefore 
$137.036\times 2.187\times10^6 = 2.99 \times 10^8 m/s$

\begin{table}[h]
    \centering
    \caption{Derived atomic units}
    \begin{tabular}{|l|l|l|}
        \hline
        \textbf{Atomic unit of} & \textbf{Expression} & \textbf{Value in SI} \\ \hline
        Action                  & $\hbar$           & $1.054\times 10^{-34} J\cdot s$ \\ \hline
        Charge                  & $e$               & $1.602\times 10^{-19} C$ \\ \hline
        Charge Density          & $e/a_0^3$         & $1.081\times 10^12 C\cdot m^{-3}$ \\ \hline
        Current                 & $eE_h/\hbar$      & $6.623\times 10^{-3} A$ \\ \hline
        Electric Field          & $E_h/(ea_0)$      & $5.142\times 10^11 V/m $ \\ \hline
        Electric Potential      & $E_h/e$           & $27.211V $ \\ \hline
        Electric Dipole moment  & $ea_0$            & $8.478\times 10^{-30} C\cdot m $ \\ \hline
        Energy                  & $E_h$             & $4.359\times 10^{-18} J $ \\ \hline
        Force                   & $E_h/a_0$         & $8.238\times 10^{-8} N$ \\ \hline
        Length                  & $a_0$             & $5.291\times 10^{-11} m $ \\ \hline
        Magnetic Dipole Moment  & $e\hbar/m_e$      & $1.854\times 10^{-23} J/T$ \\ \hline
        Mass                    & $m_e$             & $9.109\times 10^{-31} kg $ \\ \hline
        Momentum                & $\hbar/a_0$       & $1.992\times 10^{-24} kg\cdot m/s $ \\ \hline
        Permitivity             & $e^2/(a_0E_h)$    & $1.112\times 10^{-10} F/m$ \\ \hline
        Time                    & $\hbar/E_h$       & $2.418\times 10^{-17} s $ \\ \hline   
        Pressure                & $E_h/a_0^3$       & $2.942\times 10^{13} Pa $ \\ \hline
        Velocity                & $a_0E_h/\hbar$    & $2.187\times 10^6 m/s $ \\ \hline
    \end{tabular}
\end{table}

\end{document}